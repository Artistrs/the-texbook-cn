% -*- coding: utf-8 -*-

\input macros

%\beginchapter Appendix E. Example Formats
\beginchapter Appendix E. 版式举例

%Although the plain \TeX\ format of Appendix B is oriented to technical
%reports, it can readily be adapted to quite different applications. Examples of
%three such adaptations are provided in this appendix: (1)~a~format for
%business letters; (2)~a~format for concert programs; (3)~the~format used
%to typeset this book.
尽管附录~B 的 plain \TeX\ 格式是针对技术报告设计的,
稍加修改也可以有完全不同的应用。这个附录给出三种应用的例子:%
(1)~用于商业信件的格式;(2)~用于音乐会节目单的格式;%
(3)~用于排版本书的格式。

%Let's consider ^{business letters} first. Suppose that you want \TeX\ to
%format your ^{correspondence}, and that you have $n$~letters to send. ^^{mail}
%If your computer system contains a file ^|letterformat.tex| like the one
%described later in this appendix, it's easy to do the job by applying \TeX\ to
%a file that looks like this:
\bookmark{2}{+. 商业信件}\baselineskip=14pt
我们先考虑^{商业信件}。假设你想用 \TeX\ 设计版式并发送 $n$~封^{信件}。
^^{mail}如果计算机中包含类似本附录稍后描述的 ^|letterformat.tex| 的文件,
用 \TeX\ 排版类似下面的文件,就可以轻松完成任务:
\begindisplay
\<optional magnification>\cr
^|\input|| letterformat|\cr
\<business letter$_1$>\cr
\noalign{\vskip-1pt}
\qquad\vdots\cr
\<business letter$_n$>\cr
^|\end|\cr
\enddisplay
%Here each of the $n$ business letters has the form
其中这 $n$ 封商业信件中每封都有下面的形式
\begindisplay
\<letterhead>\cr
^|\address|\cr
\<one or more lines of address>\cr
^|\body|\cr
\<one or more paragraphs of text>\cr
^|\closing|\cr
\<one or more lines for salutation and signature>\cr
\<optional annotations>\cr
\<optional postscripts>\cr
^|\endletter|\cr
^|\makelabel|| % omit this if you don't want an address label|\cr
\enddisplay
%The \<letterhead> at the beginning of this construction is usually a
%control sequence like |\rjdletterhead| for letters by R.~J.~D.; each
%letter writer can have a personalized letterhead that is stored with the
%|letterformat| macros. The \<optional annotations> at the end are any number
%of one-line notes preceded by `^|\annotations|'; the \<optional postscripts>
%are any number of paragraphs preceded by `^|\ps|'. When \TeX\ is processing the
%|\address| and the |\closing| and the optional |\annotations|, it produces
%output line-for-line just as the lines appear in the input file; but when
%\TeX\ is processing the |\body| of the letter and the optional |\ps|, it
%chooses line breaks and justifies lines as it normally does when
%typesetting paragraphs in books.
开头的 \<letterhead> 通常是一个类似 |\rjdletterhead| 的控制系列,表示信件来自 R.~J.~D.;
每个写信者可以有自己的信头,它保存在 |letterformat| 的宏中。
结尾的 \<optional annotations> 是以 `^|\annotations|' 开始的任意多个单行附注;
\<optional postscripts> 是以 `^|\ps|' 开始的任意多个段落。
当 \TeX\ 在处理 |\address| 和 |\closing| 和可选的 |\annotations| 时,
它依照输入文件的各行逐行生成输出;
但是当 \TeX\ 在处理信件的 |\body| 和可选的 |\ps| 时,
它像排版书籍段落那样选择断行点并让各行两侧对齐。

%A complete example, together with the resulting output, appears on the
%next two pages. This example starts with `^|\magnification||=|^|\magstep||1|'
%because the letter is rather short. Magnification is usually omitted if the
%letters are long-winded; `|\magnification=|^|\magstephalf|' is
%appropriate when they are medium-size. The same magnification applies to
%all~$n$ letters, so you must run \TeX\ more than once if you want more than
%one magnification.
后面两页给出了一个完整的例子,以及它的输出结果。
这个例子的信件相当简短,因而它以 `^|\magnification||=|^|\magstep||1|' 开头,
如果信件是冗长的,通常就需要去掉这个放大率;
而适合中等长度信件的放大率为 `|\magnification=|^|\magstephalf|'。
同样的放大率将应用到所有 $n$ 封信件,因此如果你想要多个放大率,你必须多次运行 \TeX 。

%\eject
%\begingroup \obeylines
%|\magnification=\magstep1|
%|\input letterformat|
%\bigskip
%|\rjdletterhead % (see the output on the next page)|
%\bigskip
%|\address|
%|Prof.~Brian~K. Reid|^^{Reid}
%|Department of Electrical Engineering|
%|Stanford University|
%|Stanford, CA 94305|
%\bigskip
%|\body|
%|Dear Prof.~Reid:|
%\bigskip
%|I understand that you are having difficulties with|
%|Alka-Seltzer tablets.  Since there are 25~pills|^^{Alka-Seltzer}^^{ties}
%|per bottle, while the manufacturer's directions|
%|recommend ``plop,~plop, fizz,~fizz,'' my colleagues|
%|tell me that you have accumulated a substantial|
%|number of bottles in which there is one tablet|
%|left. % (See the 1978 SCRIBE User Manual, page 90.)|^^{Scribe}
%\bigskip
%|At present I am engaged in research on the potential|
%|applications of isolated analgesics. If you would|
%|be so kind as to donate your Alka-Seltzer collection|
%|to our project, I would be more than happy to send|
%|you preprints of any progress reports that we may|
%|publish concerning this critical problem.|
%\bigskip
%|\closing|
%|Sincerely,|
%|R. J. Drofnats|^^{Drofnats}
%|Professor|
%\bigskip
%|\annotations|
%|RJD/dek|
%|cc: {\sl The \TeX book}|
%\bigskip
%|\ps|
%|P. S. \ If you like, I will check into the|
%|possibility that your donation and the meals that|
%|you have been eating might be tax-deductible, in|^^{IRS}
%|connection with our research.|
%|\endletter|
%|\makelabel|
%\eject\endgroup
\eject
\begingroup \obeylines\baselineskip=12pt
|\magnification=\magstep1|
|\input letterformat|
\bigskip
|\rjdletterhead % (see the output on the next page)|
\bigskip
|\address|
|Prof.~Brian~K. Reid|^^{Reid}
|Department of Electrical Engineering|
|Stanford University|
|Stanford, CA 94305|
\bigskip
|\body|
|Dear Prof.~Reid:|
\bigskip
|I understand that you are having difficulties with|
|Alka-Seltzer tablets.  Since there are 25~pills|^^{Alka-Seltzer}^^{ties}
|per bottle, while the manufacturer's directions|
|recommend ``plop,~plop, fizz,~fizz,'' my colleagues|
|tell me that you have accumulated a substantial|
|number of bottles in which there is one tablet|
|left. % (See the 1978 SCRIBE User Manual, page 90.)|^^{Scribe}
\bigskip
|At present I am engaged in research on the potential|
|applications of isolated analgesics. If you would|
|be so kind as to donate your Alka-Seltzer collection|
|to our project, I would be more than happy to send|
|you preprints of any progress reports that we may|
|publish concerning this critical problem.|
\bigskip
|\closing|
|Sincerely,|
|R. J. Drofnats|^^{Drofnats}
|Professor|
\bigskip
|\annotations|
|RJD/dek|
|cc: {\sl The \TeX book}|
\bigskip
|\ps|
|P. S. \ If you like, I will check into the|
|possibility that your donation and the meals that|
|you have been eating might be tax-deductible, in|^^{IRS}
|connection with our research.|
|\endletter|
|\makelabel|
\eject\endgroup\normalbaselines
%\def\proofcopy(#1){\ifproofmode\raise.5in\hbox{\sevenrm(#1)}\else\hfil\fi}
%\vglue-\topskip \nointerlineskip
%\dimen0=\vsize \advance\dimen0 by-1.2pt \advance\dimen0 by-2.1in
%\vbox{\hrule
%  \hbox{\vrule height \dimen0
%    \hbox to 4.25in{\hss\proofcopy(Output page goes here, reduced 50\%.)\hss}
%    \vrule}}
%\nointerlineskip
%\hrule
%\line{\vrule height 2.1in
%  \hss\proofcopy(Label and stamp go here, reduced 50\%.)\hss\vrule}
%\hrule
%\vskip 0pt plus .001pt minus .001pt % in case of rounding errors
%\eject
\def\proofcopy(#1){\ifproofmode\raise.5in\hbox{\sevenrm(#1)}\else\hfil\fi}
\vglue-\topskip \nointerlineskip
\dimen0=\vsize \advance\dimen0 by-1.2pt \advance\dimen0 by-2.1in
\special{pdf:image width 310pt height 0pt depth 439pt (artwork/letter1a.png)}
\vbox{\hrule
  \hbox{\vrule height \dimen0
    \hbox to 4.25in{\hss\proofcopy(Output page goes here, reduced 50\%.)\hss}
    \vrule}}
\nointerlineskip
\special{pdf:image width 348pt height 0pt depth 153pt (artwork/letter1b.png)}
\hrule
\line{\vrule height 2.1in
  \hss\proofcopy(Label and stamp go here, reduced 50\%.)\hss\vrule}
\hrule
\vskip 0pt plus .001pt minus .001pt % in case of rounding errors
\eject
%If the letter is more than one page long, the addressee, date, and
%page number will appear at the top of subsequent pages. For example,
%the previous letter comes out as follows, if additional paragraphs are
%added to the text:
%\medskip
%\hrule
%\line{\vrule height 3.1127in
%  \hfil\proofcopy(First page, reduced to 28.3\%.)\hfil\vrule\hfil
%  \proofcopy(Second page, reduced to 28.3\%.)\hfil\vrule}
%\hrule
如果信件长度超过一页,收信人地址、日期和页码将出现在后续页面顶部。
假如添加额外段落到信件正文中,之前的例子就变成下面这样:
\medskip
%FIXME: 这里有多余的竖直空白
\vbox to 0pt{\noindent
\hbox{\special{pdf:image width 170pt height 0pt depth 240pt (artwork/letter2a.png)}}\hfil
\hbox{\special{pdf:image width 170pt height 0pt depth 240pt (artwork/letter2b.png)}}\hfil}%
\hrule
\line{\vrule height 3.1127in
  \hfil\proofcopy(First page, reduced to 28.3\%.)\hfil\vrule\hfil
  \proofcopy(Second page, reduced to 28.3\%.)\hfil\vrule}
\hrule
%\bigskip
%\ninepoint
%The macro package |letterformat.tex| that produces this format begins
%with a simple macro that expands to the current ^{date}.
\bigskip
\ninepoint
生成这个格式的 |letterformat.tex| 宏包以一个展开为当前^{日期}的简单宏开始。
\beginlines
|\def|^|\today||{|^|\ifcase|^|\month|^|\or|
|  January\or February\or March\or April\or May\or June\or|
|  July\or August\or September\or October\or November\or December\fi|
|  |^|\space|^|\number|^|\day||, \number|^|\year||}|
\endlines

%Then comes the specification of page layout, which is ``ragged'' at the
%bottom. A rather large ^|\interlinepenalty| is used so that page
%breaks will tend to occur between paragraphs.
接着指定页面为底部不齐平的布局。
还设定相当大的 ^|\interlinepenalty| 值,以让 \TeX\ 尽量在段落之间分页。
\beginlines
^|\raggedbottom|
|\interlinepenalty=1000|
^|\hsize||=6.25truein|
^|\voffset||=24pt|
|\advance|^|\vsize|| by-\voffset|
^|\parindent||=0pt|
^|\parskip||=0pt|
^|\nopagenumbers|
^|\headline||={\ifnum|^|\pageno||>1|
|  \tenrm To \addressee\hfil\today\hfil Page |^|\folio|
|  \else\hfil\fi}|
\endlines

%The contents of a letter are typeset either in ``line mode'' (obeying lines)
%or in ``paragraph mode'' (producing paragraphs in ^{block style}). Control
%sequences |\beginlinemode| and |\beginparmode| are defined to initiate these
%modes; and another control sequence, |\endmode|, is defined and redefined so
%that the current mode will terminate properly:
信件内容以``逐行模式''(保持各行)或``段落模式''(生成^{齐头式}段落)排版。
控制系列 |\beginlinemode| 和 |\beginparmode| 用于初始化这两个模式;
在两者的定义中重定义另外的控制系列 |\endmode|,使得当前模式能够正确终结:
\beginlines
|\def\beginlinemode{\endmode|
|  |^|\begingroup|^|\obeylines||\def\endmode{\par|^|\endgroup||}}|
|\def\beginparmode{\endmode|
|  \begingroup\parskip=\medskipamount \def\endmode{\par\endgroup}}|
|\let\endmode=\par|
|\def\endletter{\endmode\vfill|^|\supereject||}|
\endlines

%One of the chief characteristics of this particular business letter format
%is a parameter called ^|\longindentation|, which is used to indent the
%closing material, the date, and certain aspects of the letterhead. The
%^|\address| macro creates a box that will be used both in the letter and in
%the label on the envelope. If individual lines of the address exceed
%|\longindentation|, they are broken, and hanging indentation is used for any
%material that must be carried over.
这个商业信件格式的主要特征是一个称为 ^|\longindentation| 的参数,
它用于缩进结尾,日期以及信头的某些东西。
宏 ^|\address| 创建一个盒子,它将会在信件里面以及信封标签上用到。
如果地址里面宽度超过 |\longindentation| 的那些行将会被折行,
并且使用悬挂缩进显示后续行。
\beginlines
|\newdimen\longindentation \longindentation=4|^|true||in|
|\newbox\theaddress|
|\def\address{\beginlinemode\getaddress}|
|{\obeylines|^|\gdef||\getaddress #1|
|  #2|
|  {#1\gdef\addressee{#2}%|
|    \global\setbox\theaddress=\vbox|^|\bgroup|^|\raggedright||%|
|    \hsize=\longindentation |^|\everypar||{|^|\hangindent||2em}#2|
|    \def\endmode{\egroup\endgroup |^|\copy||\theaddress |^|\bigskip||}}}|
\endlines
%(Parameter |#2| to |\getaddress| ^^{parameters, delimited}
%will be the contents of the line following |\address|, i.e., the
%name of the addressee.)
( 宏 |\getaddress| 的 |#2| 参数^^{parameters, delimited}是
|\address| 下一行的内容,即收信人地址。)

%The closing macros are careful not to allow a page break anywhere between the
%end of the ^|\body| and the beginning of a ^|\ps|.
结尾宏需要小心处理,使得页面不在 ^|\body| 结束处与 ^|\ps| 开始处之间分开。
\beginlines
|\def\body{\beginparmode}|
|\def\closing{\beginlinemode\getclosing}|
|{\obeylines\gdef\getclosing #1|
|  #2|
|  {#1|^|\nobreak||\bigskip |^|\leftskip||=\longindentation #2|
|    \nobreak\bigskip\bigskip\bigskip % space for signature|
|    \def|
|    {|^|\endgraf||\nobreak}}}|
|\def\annotations{\beginlinemode\def\par{\endgraf\nobreak}\obeylines\par}|
|\def\ps{\beginparmode\nobreak|
|  \interlinepenalty5000\def\par{\endgraf\penalty5000}}|
\endlines

%The remaining portion of |letterformat.tex| deals with ^{letterheads} and
%labels, which of course will be different for different organizations.
%The following macros were used to generate the examples in this
%appendix; they can be modified in more-or-less obvious ways to produce
%suitable letterheads of other kinds. Special fonts are generally
%needed, and they should be loaded at `^|true|' sizes so that they are not
%affected by magnification.  One tiny refinement worth noting here is the
%^|\up| macro, which raises ^{brackets} so that they look better in a
%^{telephone number}.
|letterformat.tex| 的剩余部分处理^{信头}和标签,这当然是因组织而异的。
下列宏用于生成此附录的例子;稍作修改就可以得到适合其他类型的信头。
这里通常要用特殊字体,而且要以 `^|true|' 尺寸载入它们,
以让它们不受放大率影响。这里值得一提的是宏 ^|\up| 这个微小改进,
它提高^{方括号}的位置,使得它们在^{电话号码}中更加美观。
\beginlines
|\def\up#1{|^|\leavevmode|| |^|\raise||.16ex\hbox{#1}}|
|\font\smallheadfont=cmr8 |^|at|| 8truept|
|\font\largeheadfont=cmdunh10 at 14.4truept|
|\font\logofont=manfnt at 14.4truept|
\smallbreak
|\def\rjdletterhead{|
|  \def\sendingaddress{R. J. DROFNATS, F.T.U.G.\par|^^{TeX Users Group}
|    PROFESSOR OF FARM ECOLOGY\par|
|    TEX.RJD @ SU-SCORE.ARPA\par|^^{atsign}
|    \up[415\up]\thinspace 497-4975\par}|
|  \def\returnaddress{R. J. Drofnats, Dept.~of Farm Ecology\par|
|    The University of St.~Anford\par|
|    P. O. Box 1009, Haga Alto, CA 94321 USA}|
|  \letterhead}|
\smallbreak
|\def\letterhead{\pageno=1 \def\addressee{} \univletterhead|
|  {\leftskip=\longindentation|
|    {\baselineskip9truept\smallheadfont\sendingaddress}|
|    \bigskip\bigskip\rm\today\bigskip}}|
\smallbreak
|\def\univletterhead{\vglue-\voffset|
|  \hbox{\hbox to\longindentation{\raise4truemm\hbox{\logofont|
|        \kern2truept X\kern-1.667truept|
|        \lower2truept\hbox{X}\kern-1.667truept X}\hfil|
|      \largeheadfont The University of St.~Anford\hfil}%|
|    \kern-\longindentation|
|    \vbox{\smallheadfont\baselineskip9truept|
|      \leftskip=\longindentation BOX 1009\par HAGA ALTO, CA 94321}}|
|  \vskip2truept\hrule\vskip4truept }|
\smallbreak
|\def|^|\makelabel||{\endletter\hbox{\vrule|
|    \vbox{\hrule \kern6truept|
|      \hbox{\kern6truept\vbox to 2truein{\hsize=\longindentation|
|          \smallheadfont\baselineskip9truept\returnaddress|
|          \vfill\moveright 2truein\copy\theaddress\vfill}%|
|        \kern6truept}\kern6truept\hrule}\vrule}|
|  \pageno=0\vfill\eject}|
\endlines

%\medbreak
%Our second example is a format for ^{concert programs}, to be used in
%^^{music} ^^{programs, for music}
%connection with orchestra performances, recitals, and the like. We shall
%assume that the entire program fits on a single page, and that the
%copy is to be 4~inches wide. Comparatively large type ($12\pt$) will
%normally be used, but there is a provision for $10\pt$ and even ^^{sizes
%of type} $8\pt$ type in case the program includes pieces with a lot of
%subparts (e.g., ^{Bach}'s Mass in B~minor, or ^{Beethoven}'s ^{Diabelli}
%Variations). To select the ^{type size}, a user says ^|\bigtype|,
%^|\medtype|, or ^|\smalltype|, respectively. These macros for ^{size switching}
%are comparatively simple because concert programs don't require any
%mathematics; hence the math fonts don't need to be changed. On the other
%hand, the format does take sharp and flat signs from the ``^{math italic}''
%font, which it calls `|\mus|':
\medbreak
\bookmark{2}{+. 音乐会节目单}
第二个例子是一个^{音乐会节目单}的格式, 它可用于管弦乐表演、独奏音乐会等。
^^{music} ^^{programs, for music}
我们假定整个节目单为 4~英寸宽,且可以印在一页中。
通常使用相对较大的字体($12\pt$),但 $10\pt$ 甚至 $8\pt$ 的字体也可能用到,
特别在节目包含许多部分时(比如^{巴赫}的 B~小调弥撒曲,
或者^{贝多芬}的^{迪亚贝利}变奏曲)。用户可以分别用 ^|\bigtype|、^|\medtype|
或 ^|\smalltype| 选择^{字体尺寸}。这些^{尺寸切换}宏相对来说是很简单的,
因为音乐会节目单不需要任何数学公式;因此数学字体不需要改变。另一方面,
此格式确实从``^{数学意大利字体}''中借用了升半音和降半音符号,
并将该字体称为 `|\mus|':
\beginlines
|\font\twelverm=cmr12|
|\font\twelvebf=cmbx12|
|\font\twelveit=cmti12|
|\font\twelvesl=cmsl12|
|\font\twelvemus=cmmi12|
\smallskip
|\font\eightrm=cmr8|
|\font\eightbf=cmbx8|
|\font\eightit=cmti8|
|\font\eightsl=cmsl8|
|\font\eightmus=cmmi8|
\smallskip
|\def\bigtype{\let|^|\rm||=\twelverm \let|^|\bf||=\twelvebf|
|  \let|^|\it||=\twelveit \let|^|\sl||=\twelvesl \let\mus=\twelvemus|
|  \baselineskip=14pt minus 1pt|
|  \rm}|
|\def\medtype{\let\rm=\tenrm \let\bf=\tenbf|
|  \let\it=\tenit \let\sl=\tensl \let\mus=\teni|
|  \baselineskip=12pt minus 1pt|
|  \rm}|
|\def\smalltype{\let\rm=\eightrm \let\bf=\eightbf|
|  \let\it=\eightit \let\sl=\eightsl \let\mus=\eightmus|
|  \baselineskip=9.5pt minus .75pt|
|  \rm}|
\smallskip
|\hsize=4in|
|\nopagenumbers|
|\bigtype|
\endlines
%Notice the ^{shrinkability} in the ^|\baselineskip| settings. This would be
%undesirable in a book format, because different ^{spacing between lines}
%^^{leading} on different pages would look bad; but in a one-page document
%it helps squeeze the copy to fit the page, in an emergency. \ (There's no
%need for ^{stretchability} in the baselineskip here, because a |\vfill| will
%be used at the bottom of the page.)
注意其中设定了^{可收缩的} ^|\baselineskip|。这在书籍格式中是无法忍受的,
因为不同页面的^{行间距}不同将很难看;
但在这个单页文档中,它将有助于紧急时将内容压缩到一页内。%
(这里的 |\baselineskip| 并不需要有^{可伸长性},
因为在页面底部将会使用 |\vfill|。)

%Musical programs have a specialized vocabulary, and it is desirable to define
%a few control sequences for things that plain \TeX\ doesn't make as convenient
%as they could be for this particular application:
音乐节目单有特定的词汇,为了方便我们可以为此定义一些控制系列:
\beginlines
|\def\(#1){{\rm(}#1\/{\rm)}}|
|\def\sharp{\raise.4ex\hbox{\mus\char"5D}}|
|\def\flat{\raise.2ex\hbox{\mus\char"5B}}|
|\let\,=|^|\thinspace|
\endlines
%The ^|\(| macro produces roman ^{parentheses} in the midst of ^{italicized
%text}; the ^|\sharp| and ^|\flat| macros produce musical signs in the
%current type size. The ^|\,|~macro makes it easy to specify
%the ^{thin space} that is used in constructions like
%`K.\thinspace550' ^^{K\"ochel}^^{Mozart} % his 40th symphony
%and `Op.\thinspace59'. ^^{Dvo\v r\'ak} % his Legends
%\ (Plain \TeX\ has already defined |\,| and |\sharp| and |\flat| in a different
%way; but those definitions apply only to math formulas, so they aren't
%relevant in this application.)
宏 ^|\(| 用于在^{意大利体文本}中间排印罗马体^{圆括号};
宏 ^|\sharp| 和 ^|\flat| 以当前字体大小排印音乐符号。
宏 ^|\,| 使得指定^{细小空白}更加容易,它可用于类似
`K.\thinspace550' ^^{K\"ochel}^^{Mozart} % his 40th symphony
和 `Op.\thinspace59'. ^^{Dvo\v r\'ak} % his Legends
的地方。(Plain \TeX\ 已经以不同方式定义了 |\,| 和 |\sharp| 和 |\flat|;
但这些定义只用于数学公式,所以与这种应用无关。)

%Before discussing the rest of the music macros, let's take a look at a complete
%example. The next two pages show the input and output for a typical concert
%program.
在讨论音乐宏的其他部分之前,我们先来看看完整的例子,
下面两页显示了典型音乐会节目单的输入及输出结果。

%\eject
%\begingroup \parindent=0pt \obeylines
%|\input concert|
%\medskip
%|\tsaologo|
%|\medskip|
%|\centerline{Friday, November 19, 1982, 8:00 p.m.}|
%|\bigskip|
%|\centerline{\bf PROGRAM}|
%|\medskip|
%\medskip
%|\composition{Variations on a Theme by Tchaikovsky}|%
%  ^^{Tchaikovsky, see Cha\u\i...}
%|\composer{Anton S. Arensky (1861--1906)}|^^{Arenski\u\i}
%|\smallskip|
%|{\medtype|
%|\movements{Tema: Moderato\cr|
%|  Var.~I: Un poco pi\`u mosso&Var.~V: Andante\cr|
%|  Var.~II: Allegro non troppo&Var.~VI: Allegro con spirito\cr|
%|  Var.~III: Andantino tranquillo&Var.~VII: Andante con moto\cr|
%|  Var.~IV: Vivace&Coda: Moderato\cr}|
%|}|
%\medskip
%|\bigskip|
%\medskip
%|\composition{Concerto for Horn and Hardart, S.\,27}|
%|\composer{P. D. Q. Bach (1807--1742)?}|^^{Bach, PDQ}
%|\smallskip|
%|\movements{Allegro con brillo\cr|
%|  Tema con variazione \(su una tema differente)\cr|
%|  Menuetto con panna e zucchero\cr}|
%|\medskip|
%|\soloists{Ben Lee User, horn\cr|^^{User}
%|  Peter Schickele, hardart\cr}|^^{Schickele}
%\medskip
%|\bigskip|
%|\centerline{INTERMISSION}|
%|\bigskip|
%\medskip
%|\composition{Symphony No.\,3 in E\flat\ Major\cr|
%|  Op.\,55, ``The Eroica''\cr}|
%|\composer{Ludwig van Beethoven (1770--1827)}|^^{Beethoven}
%|\smallskip|
%|\movements{Allegro con brio\cr|
%|  Marcia funebre: Adagio assai\cr|
%|  Scherzo: Allegro vivace\cr|
%|  Finale: Allegro molto\cr}|
%\medskip
%|\bigskip|
%|\smalltype \noindent|
%|Members of the audience are kindly requested to turn off the|
%|alarms on their digital watches, and to cough only between movements.|
%\medskip
%|\bye|
%\eject\endgroup
%\hbox to\hsize{\hss \vbox{ % \centerline is not a \long macro!
%\font\twelverm=cmr12
%\font\twelvebf=cmbx12
%\font\twelveit=cmti12
%\font\twelvesl=cmsl12
%\font\twelvemus=cmmi12
\eject
\begingroup \parindent=0pt \obeylines \baselineskip=11.5pt
|\input concert|
\medskip
|\tsaologo|
|\medskip|
|\centerline{Friday, November 19, 1982, 8:00 p.m.}|
|\bigskip|
|\centerline{\bf PROGRAM}|
|\medskip|
\medskip
|\composition{Variations on a Theme by Tchaikovsky}|%
  ^^{Tchaikovsky, see Cha\u\i...}
|\composer{Anton S. Arensky (1861--1906)}|^^{Arenski\u\i}
|\smallskip|
|{\medtype|
|\movements{Tema: Moderato\cr|
|  Var.~I: Un poco pi\`u mosso&Var.~V: Andante\cr|
|  Var.~II: Allegro non troppo&Var.~VI: Allegro con spirito\cr|
|  Var.~III: Andantino tranquillo&Var.~VII: Andante con moto\cr|
|  Var.~IV: Vivace&Coda: Moderato\cr}|
|}|
\medskip
|\bigskip|
\medskip
|\composition{Concerto for Horn and Hardart, S.\,27}|
|\composer{P. D. Q. Bach (1807--1742)?}|^^{Bach, PDQ}
|\smallskip|
|\movements{Allegro con brillo\cr|
|  Tema con variazione \(su una tema differente)\cr|
|  Menuetto con panna e zucchero\cr}|
|\medskip|
|\soloists{Ben Lee User, horn\cr|^^{User}
|  Peter Schickele, hardart\cr}|^^{Schickele}
\medskip
|\bigskip|
|\centerline{INTERMISSION}|
|\bigskip|
\medskip
|\composition{Symphony No.\,3 in E\flat\ Major\cr|
|  Op.\,55, ``The Eroica''\cr}|
|\composer{Ludwig van Beethoven (1770--1827)}|^^{Beethoven}
|\smallskip|
|\movements{Allegro con brio\cr|
|  Marcia funebre: Adagio assai\cr|
|  Scherzo: Allegro vivace\cr|
|  Finale: Allegro molto\cr}|
\medskip
|\bigskip|
|\smalltype \noindent|
|Members of the audience are kindly requested to turn off the|
|alarms on their digital watches, and to cough only between movements.|
\medskip
|\bye|
\eject\endgroup
\hbox to\hsize{\hss \vbox{ % \centerline is not a \long macro!
\font\twelverm=cmr12
\font\twelvebf=cmbx12
\font\twelveit=cmti12
\font\twelvesl=cmsl12
\font\twelvemus=cmmi12

\def\bigtype{\let\rm=\twelverm \let\bf=\twelvebf \let\it=\twelveit
  \let\sl=\twelvesl \let\mus=\twelvemus \baselineskip=14pt minus 1pt \rm}
\def\medtype{\let\rm=\tenrm \let\bf=\tenbf \let\it=\tenit
  \let\sl=\tensl \let\mus=\teni \baselineskip=12pt minus 1pt \rm}
\def\smalltype{\let\rm=\eightrm \let\bf=\eightbf \let\it=\eightit
  \let\sl=\eightsl \let\mus=\eighti \baselineskip=9.5pt minus .75pt \rm}

\def\flat{\raise.2ex\hbox{\mus\char"5B}}
\def\sharp{\raise.4ex\hbox{\mus\char"5D}}
\let\,=\thinspace
\def\(#1){{\rm(}#1\/{\rm)}}

\hsize=4in
\bigtype

\def\composition#1{\halign{\bf\quad##\hfil\cr
    \kern-1em#1\crcr}} % use \cr's if more than one line
\def\composer#1{\rightline{\bf#1}}
\def\movements#1{\halign{\quad\it##\hfil&&\qquad\it##\hfil\cr#1\crcr}}
\def\soloists#1{\centerline{\bf\vbox{\halign{##\hfil\cr#1\crcr}}}}

\def\tsaologo{\vbox{\bigtype\bf
    \line{\hrulefill}
    \kern-.5\baselineskip
    \line{\hrulefill\phantom{ THE ST.\,ANFORD ORCHESTRA }\hrulefill}
    \kern-.5\baselineskip
    \line{\hrulefill\hbox{ THE ST.\,ANFORD ORCHESTRA }\hrulefill}
    \kern-.5\baselineskip
    \line{\hrulefill\phantom{ R. J. Drofnats, Conductor }\hrulefill}
    \kern-.5\baselineskip
    \line{\hrulefill\hbox{ R. J. Drofnats, Conductor }\hrulefill}
    }}
\tsaologo
\medskip
\centerline{Friday, November 19, 1982, 8:00 p.m.}
\bigskip
\centerline{\bf PROGRAM}
\medskip

\composition{Variations on a Theme by Tchaikovsky}
\composer{Anton S. Arensky (1861--1906)}
\smallskip
{\medtype
\movements{Tema: Moderato\cr
  Var.~I: Un poco pi\`u mosso&Var.~V: Andante\cr
  Var.~II: Allegro non troppo&Var.~VI: Allegro con spirito\cr
  Var.~III: Andantino tranquillo&Var.~VII: Andante con moto\cr
  Var.~IV: Vivace&Coda: Moderato\cr}
}

\bigskip

\composition{Concerto for Horn and Hardart, S.\,27}
\composer{P. D. Q. Bach (1807--1742)?}
\smallskip
\movements{Allegro con brillo\cr
  Tema con variazione \(su una tema differente)\cr
  Menuetto con panna e zucchero\cr}
\medskip
\soloists{Ben Lee User, horn\cr
  Peter Schickele, hardart\cr}

\bigskip
\centerline{INTERMISSION}
\bigskip

\composition{Symphony No.\,3 in E\flat\ Major\cr
  Op.\,55, ``The Eroica''\cr}
\composer{Ludwig van Beethoven (1770--1827)}
\smallskip
\movements{Allegro con brio\cr
  Marcia funebre: Adagio assai\cr
  Scherzo: Allegro vivace\cr
  Finale: Allegro molto\cr}

\bigskip
\smalltype \noindent
Members of the audience are kindly requested to turn off the
alarms on their digital watches, and to cough only between movements.
}\hss}^^{Drofnats}\vfill\eject

\edef\\{\hskip\the\parindent} % I'm emphasizing this "design element" here!
Most of the macros in |concert.tex| have already been defined. Plain \TeX\
takes care of things like |\centerline| and |\bigskip|, so only
|\composition|, |\composer|, |\movements|, and |\soloists| remain to
be specified:
\beginlines
|\def\composition#1{\halign{\bf\quad##\hfil\cr|
\\|\kern-1em#1\crcr}} % use \cr's if more than one line|
|\def\composer#1{\rightline{\bf#1}}|
|\def\movements#1{\halign{\quad\it##\hfil&&\qquad\it##\hfil\cr#1\crcr}}|
|\def\soloists#1{\centerline{\bf\vbox{\halign{##\hfil\cr#1\crcr}}}}|
\endlines
The |\composition| macro is set up to put the title of the composition on
two or more lines, if needed, but a single line usually suffices. Notice
that ^|\crcr| has been used so that the final ^|\cr| in the argument to
|\composition| is not needed. Similarly, |\movements| might be used to
produce only a single line, and |\soloists| might be used when there
is only one soloist.

There's also a |\tsaologo| macro. It applies only to one particular
orchestra, but the definition is somewhat interesting nonetheless:
\beginlines
|\def\tsaologo{\vbox{\bigtype\bf|
\\^|\line||{|^|\hrulefill||}|
\\|\kern-.5\baselineskip|
\\|\line{\hrulefill|^|\phantom||{ THE ST.\,ANFORD ORCHESTRA }\hrulefill}|
\\|\kern-.5\baselineskip|
\\|\line{\hrulefill\hbox{ THE ST.\,ANFORD ORCHESTRA }\hrulefill}|
\\|\kern-.5\baselineskip|
\\|\line{\hrulefill\phantom{ R. J. Drofnats, Conductor }\hrulefill}|
\\|\kern-.5\baselineskip|
\\|\line{\hrulefill\hbox{ R. J. Drofnats, Conductor }\hrulefill}|
\\|}}|
\endlines

The author ^^{Knuth} has extended these macros to a more elaborate format
that includes special features for listing the members of the orchestra and
for program notes, etc.; in this way it becomes fairly easy to typeset little
booklets for concert patrons. Such extensions need not be discussed further
in this appendix, because they don't illustrate any essentially new ideas.

Notice that the |\composition| and |\movements| and |\soloists| macros do
not include any special provision for vertical ^{spacing}; the user is
supposed to insert ^|\smallskip|, ^|\medskip|, and ^|\bigskip| as
desired. This was done deliberately, because different concert programs
seem to demand different spacing; no automatic scheme actually works very
well in practice, since musical literature is so varied.

%\medbreak
%Let's turn now to the design of a format for an entire book, using this book
%itself as an example. How did the author prepare the computer file
%that generated {\sl The \TeX book\/}?  We have already seen several hundred
%pages of output produced from that file; our goal in the remainder of this
%appendix will be to examine the input that was used behind the scenes.
\medbreak
\bookmark{2}{+. 本书格式}
Let's turn now to the design of a format for an entire book, using this book
itself as an example. How did the author prepare the computer file
that generated {\sl The \TeX book\/}?  We have already seen several hundred
pages of output produced from that file; our goal in the remainder of this
appendix will be to examine the input that was used behind the scenes.

In the first place, the author prepared ^{sample pages} and showed them to
the publisher's book designer. \ (The importance of this step cannot be
overemphasized. There is a danger that authors---who are now able to
typeset their own books with \TeX---will attempt to do their own designs,
^^{author, typesetting by}
without professional help. ^{Book design} is an art that requires considerable
creativity, skill, experience, and taste; it is one of the most important
services that a publisher traditionally provides to an author.)

Sample pages that are used as the basis of a design should show each of
the elements in the book. In this case the elements included chapter titles,
illustrations, subchapter headings, footnotes, displayed formulas,
typewriter type, dangerous bends, exercises, answers, quotations,
tables, numbered lists, bulleted lists, etc.; the author also expressed
a desire for generous margins, so that readers could make marginal notes.

The designer, Herb ^{Caswell}, faced a difficult problem of bringing all
those disparate elements into a consistent framework. He decided to achieve
this by using a uniform indentation of 3~^{picas} for normal paragraph openings
as well as for dangerous bends; and to establish this element of the design
by using it also for all the displayed material, instead of centering the
displays.

He decided to put the page numbers in bold type, out in the margins (where
there was plenty of room, thanks to the author's request for white space);
and he decided to use italic type with caps and lower case for the running
headlines, so that the pages would have a somewhat informal flavor.

He chose 10-point type (on a 12-point base) for the main text, and 9-point type
(on an 11-point base) for the dangerous bends; the typeface was predetermined.
He~chose an ^|\hsize| of 29~picas and a ^|\vsize| of 44~picas. He decided
to give subheadings like `\kern1pt{\manual\char'170}\kern.15em
{\eightbf EXERCISE \bf13.8}' in boldface caps before the statement of each
exercise. He~specified the amount of vertical space before and after such
things as exercises, dangerous-bend paragraphs, and displayed equations.
He decided to devote an entire left-hand page to each chapter illustration.
And so on; each decision influenced the others, so that the final book would
appear to be as coherent and attractive as possible under the circumstances.
After the main portion of the book was designed, he worked out a format for
the ^{front matter} (i.e., the pages that precede page~1); he arranged to
have the same amount of ``^{sinkage}'' (white space) at the top of each page
there, so that the opening pages of the book would look unified and ``open.''

The author hasn't actually followed the designer's specifications in every
detail. For example, nothing about stretching or shrinking of vertical spaces
appeared in the design specs; the author introduced the notion of flexible
glue on his own initiative, based on his observations of cut-and-paste
operations often used in page makeup. If this book has any beauties, they
should be ascribed to Herb Caswell; if it has any blemishes, they should be
ascribed to Don Knuth, who wrote the formatting
macros that we are now about to discuss.

The computer file |manual.tex| that generated {\sl The \TeX book\/} begins
with a copyright notice, and then it says `|\input| |manmac|'. The auxiliary
file ^|manmac.tex| contains the formatting macros, and it begins by
loading 9-point, 8-point, and 6-point~fonts:
\beginlines
|\font\ninerm=cmr9   \font\eightrm=cmr8   \font\sixrm=cmr6|
|\font\ninei=cmmi9   \font\eighti=cmmi8   \font\sixi=cmmi6|
|\font\ninesy=cmsy9  \font\eightsy=cmsy8  \font\sixsy=cmsy6|
|\font\ninebf=cmbx9  \font\eightbf=cmbx8  \font\sixbf=cmbx6|
|\font\ninett=cmtt9  \font\eighttt=cmtt8|
|\font\nineit=cmti9  \font\eightit=cmti8|
|\font\ninesl=cmsl9  \font\eightsl=cmsl8|
\endlines
(These fonts had been ^|\preloaded| in Appendix B\null; now they're officially
loaded.)

The fonts intended for math formulas need to have a nonstandard ^|\skewchar|.
The typewriter fonts are given ^|\hyphenchar||=-1| so that ^{hyphenation}
is inhibited when control sequence names and keywords appear in the text
of a paragraph.
\beginlines
|\skewchar\ninei='177  \skewchar\eighti='177  \skewchar\sixi='177|
|\skewchar\ninesy='60  \skewchar\eightsy='60  \skewchar\sixsy='60|
|\hyphenchar\ninett=-1 \hyphenchar\eighttt=-1 \hyphenchar\tentt=-1|
\endlines

A few more fonts are needed for special purposes:
\beginlines
|\font\tentex=cmtex10               % TeX character set as in Appendix C|
|\font\inchhigh=cminch              % inch-high caps for chapter openings|
|\font\titlefont=cmssdc10 at 40pt   % titles in chapter openings|
|\font\eightss=cmssq8               % quotations in chapter closings|
|\font\eightssi=cmssqi8             % ditto, slanted|
|\font\tenu=cmu10                   % unslanted text italic|
|\font\manual=manfnt                % METAFONT logo and special symbols|
|\font\magnifiedfiverm=cmr5 at 10pt % to demonstrate magnification|
\endlines

Now we come to the ^{size-switching} macros, which are much more
elaborate than they were in the previous example because mathematics
needs to be supported in three different sizes. The format also
provides for a pseudo ``^{small caps}'' (|\sc|); a true caps-and-small-caps
font was not really necessary in the few cases that |\sc| was used.
A~dimension variable ^|\ttglue| is set equal to the desired spacing for
the typewriter-like text that occasionally appears in paragraphs; the
|\tt| fonts have fixed spacing, which doesn't mix well with
variable spacing, hence the macros below use |\ttglue| between words in
appropriate places.
\beginlines
|\catcode`@=11 % we will access private macros of plain TeX (carefully)|
^|\newskip||\ttglue|^^{atsign}
\smallbreak
|\def|^|\tenpoint||{\def|^|\rm||{|^|\fam||0\tenrm}% switch to 10-point type|
|  |^|\textfont||0=\tenrm \scriptfont0=\sevenrm \scriptscriptfont0=\fiverm|
|  \textfont1=\teni  |^|\scriptfont||1=\seveni  \scriptscriptfont1=\fivei|
|  \textfont2=\tensy \scriptfont2=\sevensy |^|\scriptscriptfont||2=\fivesy|
|  \textfont3=\tenex \scriptfont3=\tenex   \scriptscriptfont3=\tenex|
|  \textfont|^|\itfam||=\tenit  \def|^|\it||{\fam\itfam\tenit}%|
|  \textfont|^|\slfam||=\tensl  \def|^|\sl||{\fam\slfam\tensl}%|
|  \textfont|^|\ttfam||=\tentt  \def|^|\tt||{\fam\ttfam\tentt}%|
|  \textfont|^|\bffam||=\tenbf  \scriptfont\bffam=\sevenbf|
|   \scriptscriptfont\bffam=\fivebf  \def|^|\bf||{\fam\bffam\tenbf}%|
|  \tt \ttglue=.5em plus.25em minus.15em|
|  |^|\normalbaselineskip=12pt|
|  \setbox|^|\strutbox||=\hbox{\vrule height8.5pt depth3.5pt width0pt}%|
|  \let|^|\sc||=\eightrm  \let|^|\big||=\tenbig  \normalbaselines\rm}|
\smallbreak
|\def|^|\ninepoint||{\def\rm{\fam0\ninerm}% switch to 9-point type|
|  \textfont0=\ninerm \scriptfont0=\sixrm \scriptscriptfont0=\fiverm|
|  \textfont1=\ninei  \scriptfont1=\sixi  \scriptscriptfont1=\fivei|
|  \textfont2=\ninesy \scriptfont2=\sixsy \scriptscriptfont2=\fivesy|
|  \textfont3=\tenex  \scriptfont3=\tenex \scriptscriptfont3=\tenex|
|  \textfont\itfam=\nineit  \def\it{\fam\itfam\nineit}%|
|  \textfont\slfam=\ninesl  \def\sl{\fam\slfam\ninesl}%|
|  \textfont\ttfam=\ninett  \def\tt{\fam\ttfam\ninett}%|
|  \textfont\bffam=\ninebf  \scriptfont\bffam=\sixbf|
|   \scriptscriptfont\bffam=\fivebf  \def\bf{\fam\bffam\ninebf}%|
|  \tt \ttglue=.5em plus.25em minus.15em|
|  \normalbaselineskip=11pt|
|  \setbox\strutbox=\hbox{\vrule height8pt depth3pt width0pt}%|
|  \let\sc=\sevenrm  \let\big=\ninebig  \normalbaselines\rm}|
\smallbreak
|\def|^|\eightpoint||{\def\rm{\fam0\eightrm}% switch to 8-point type|
|  \textfont0=\eightrm \scriptfont0=\sixrm \scriptscriptfont0=\fiverm|
|  \textfont1=\eighti  \scriptfont1=\sixi  \scriptscriptfont1=\fivei|
|  \textfont2=\eightsy \scriptfont2=\sixsy \scriptscriptfont2=\fivesy|
|  \textfont3=\tenex   \scriptfont3=\tenex \scriptscriptfont3=\tenex|
|  \textfont\itfam=\eightit  \def\it{\fam\itfam\eightit}%|
|  \textfont\slfam=\eightsl  \def\sl{\fam\slfam\eightsl}%|
|  \textfont\ttfam=\eighttt  \def\tt{\fam\ttfam\eighttt}%|
|  \textfont\bffam=\eightbf  \scriptfont\bffam=\sixbf|
|   \scriptscriptfont\bffam=\fivebf  \def\bf{\fam\bffam\eightbf}%|
|  \tt \ttglue=.5em plus.25em minus.15em|
|  \normalbaselineskip=9pt|
|  \setbox\strutbox=\hbox{\vrule height7pt depth2pt width0pt}%|
|  \let\sc=\sixrm  \let\big=\eightbig  \normalbaselines\rm}|
\smallbreak
|\def\tenbig#1{{\hbox{$\left#1\vbox to8.5pt{}\right.\n@space$}}}|
|\def\ninebig#1{{\hbox{$\textfont0=\tenrm\textfont2=\tensy|
|  \left#1\vbox to7.25pt{}\right.\n@space$}}}|
|\def\eightbig#1{{\hbox{$\textfont0=\ninerm\textfont2=\ninesy|
|  \left#1\vbox to6.5pt{}\right.\n@space$}}}|
|\def\tenmath{\tenpoint\fam-1 } % for 10-point math in 9-point territory|
\endlines

Issues of page layout are dealt with next. First, the basics:^^|pc|
\beginlines
^|\newdimen||\pagewidth \newdimen\pageheight \newdimen\ruleht|
^|\hsize||=29pc  |^|\vsize||=44pc  |^|\maxdepth||=2.2pt  |^|\parindent||=3pc|
|\pagewidth=\hsize \pageheight=\vsize \ruleht=.5pt|
^|\abovedisplayskip||=6pt plus 3pt minus 1pt|
^|\belowdisplayskip||=6pt plus 3pt minus 1pt|
^|\abovedisplayshortskip||=0pt plus 3pt|
^|\belowdisplayshortskip||=4pt plus 3pt|
\endlines
(The curious value of\/ |\maxdepth| was chosen only to provide an example
in Chapter~15; there's no deep reason behind it.)

When the author prepared this book, he made notes about what things ought
to go into the index from each page. These notes were shown in small type
on his proofsheets, like the words `^{marginal hacks}' in the right
\insert\margin{\hbox{\marginstyle mar-}}%
\insert\margin{\hbox{\marginstyle ginal}}%
\insert\margin{\hbox{\marginstyle hacks}}%
margin of this page. The |manmac| format uses an insertion class called
^|\margin| to handle such notes.
\beginlines
^|\newinsert||\margin|
|\dimen\margin=\maxdimen % no limit on the number of marginal notes|
|\count\margin=0 \skip\margin=0pt % marginal inserts take up no space|
\endlines

The ^|\footnote| macro of plain \TeX\ needs to be changed because
^{footnotes} are indented and set in 8-point type. Some simplifications have
also been made, since footnotes are used so infrequently in this book.
\beginlines
|\def\footnote#1{\edef\@sf{\spacefactor\the\spacefactor}#1\@sf|
|      \insert\footins\bgroup\eightpoint|
|      \interlinepenalty100 \let\par=|^|\endgraf|
|        \leftskip=0pt \rightskip=0pt|
|        \splittopskip=10pt plus 1pt minus 1pt \floatingpenalty=20000|
|        \smallskip|^|\item||{#1}\bgroup\strut\aftergroup\@foot\let\next}|
|\skip\footins=12pt plus 2pt minus 4pt % space added when footnote exists|
|\dimen\footins=30pc % maximum footnotes per page|
\endlines

The text of ^{running headlines} will be kept in a control sequence
called |\rhead|. Some pages should not have headlines; the |\titlepage|
macro suppresses the headline on the next page that is output.
\beginlines
^|\newif||\iftitle|
|\def\titlepage{\global\titletrue} % for pages without headlines|
|\def\rhead{} % \rhead contains the running headline|
\smallskip
|\def\leftheadline{\hbox to \pagewidth{%|
|    \vbox to 10pt{}% strut to position the baseline|^^{strut}
|    |^|\llap||{\tenbf|^|\folio|^|\kern||1pc}% folio to left of text|
|    \tenit\rhead\hfil}} % running head flush left|
|\def\rightheadline{\hbox to \pagewidth{\vbox to 10pt{}%|
|    \hfil\tenit\rhead\/% running head flush right|
|    |^|\rlap||{\kern1pc\tenbf\folio}}} % folio to right of text|
\endlines

Pages are shipped to the output by the |\onepageout| macro, which
attaches headlines, marginal notes, and/or footnotes, as appropriate.
Special ^{registration marks} are typeset at the top of title pages, so
that the pages will line up properly on printing plates that are made
photographically from \TeX's ``camera-ready'' output. ^^{camera alignment}
A small page number is also printed next to the corner markings; such
auxiliary information will, of course, be erased before the pages are
actually printed.
\beginlines
|\def\onepageout#1{\shipout\vbox{ % here we define one page of output|
|    |^|\offinterlineskip|| % butt the boxes together|
|    \vbox to 3pc{ % this part goes on top of the 44pc pages|
|      \iftitle \global\titlefalse \setcornerrules|
|      \else|^|\ifodd||\pageno\rightheadline\else\leftheadline\fi\fi \vfill}|
|    \vbox to \pageheight{|
|      |^|\ifvoid||\margin\else % marginal info is present|
|       |^|\rlap||{\kern31pc\vbox to0pt{\kern4pt\box\margin \vss}}\fi|
|      #1 % now insert the main information|
|      \ifvoid\footins\else % footnote info is present|
|       \vskip\skip\footins \kern-3pt|
|       \hrule height\ruleht width\pagewidth \kern-\ruleht \kern3pt|
|       \unvbox\footins\fi|
|      \boxmaxdepth=\maxdepth}}|
|  |^|\advancepageno||}|
\smallbreak
|\def\setcornerrules{\hbox to \pagewidth{% for camera alignment|
|    \vrule width 1pc height\ruleht \hfil \vrule width 1pc}|
|  \hbox to \pagewidth{|^|\llap||{\sevenrm(page \folio)\kern1pc}%|
|    \vrule height1pc width\ruleht depth0pt|
|    \hfil \vrule width\ruleht depth0pt}}|
\smallbreak
^|\output||{\onepageout{|^|\unvbox||255}}|
\endlines

A different ^{output routine} is needed for Appendix~I (the index), because
most of that appendix appears in ^{two-column format}. Instead of handling
double columns with an `|\lr|' switch, as discussed in Chapter~23,
|manmac| does the job with ^|\vsplit|, after collecting more than
enough material to fill a page. This approach makes it comparatively
easy to ^{balance the columns} on the last page of the index. A~more
difficult approach would be necessary if the index contained
insertions (e.g., footnotes); fortunately, it doesn't.
Furthermore, there is no need to use ^|\mark| as suggested in the
index example of Chapter~23, since the entries in Appendix~I tend to
be quite short. The only real complication that |manmac| faces
is the fact that Appendix~I begins and ends with single-column format;
partial pages need to be juggled carefully as the format changes back and forth.
\beginlines
^|\newbox||\partialpage|
|\def\begindoublecolumns{\begingroup|
|  \output={\global\setbox\partialpage=\vbox{\unvbox255\bigskip}}\eject|
|  \output={\doublecolumnout} |^|\hsize||=14pc |^|\vsize||=89pc}|
|\def\enddoublecolumns{\output={\balancecolumns}\eject|
|  \endgroup \pagegoal=\vsize}|
\smallbreak
|\def\doublecolumnout{\splittopskip=\topskip |^|\splitmaxdepth||=\maxdepth|
|  \dimen@=44pc \advance\dimen@ by-\ht\partialpage|
|  \setbox0=\vsplit255 to\dimen@ \setbox2=\vsplit255 to\dimen@|
|  \onepageout\pagesofar \unvbox255 \penalty|^|\outputpenalty||}|
|\def\pagesofar{\unvbox\partialpage|
|  |^|\wd||0=\hsize \wd2=\hsize \hbox to\pagewidth{\box0\hfil\box2}}|
|\def\balancecolumns{\setbox0=\vbox{\unvbox255} \dimen@=\ht0|
|  \advance\dimen@ by\topskip \advance\dimen@ by-\baselineskip|
|  |^|\divide||\dimen@ by2 |^|\splittopskip||=\topskip|
|  {\vbadness=10000 |^|\loop|| \global\setbox3=\copy0|
|    \global\setbox1=\vsplit3 to\dimen@|
|    |^|\ifdim|^|\ht||3>\dimen@ \global\advance\dimen@ by1pt \repeat}|
|  \setbox0=\vbox to\dimen@{\unvbox1} \setbox2=\vbox to\dimen@{\unvbox3}|
|  \pagesofar}|
\endlines
The balancing act sets ^|\vbadness| infinite while it is searching for
a suitable column height, so that ^{underfull} vboxes won't be reported
unless the actual columns are bad after balancing. The columns in Appendix~I
have a lot of stretchability, since there's a ^|\parskip|
of |0pt| |plus|~|.8pt| between adjacent entries, and since there is room for
more than 50 lines per column; therefore the |manmac| balancing routine tries
to make both the top and bottom baselines agree at the end of the index.
In applications where the glue is not so flexible it would be more
appropriate to let the right-hand column be a little short; the best
way to do this is probably to replace the command `|\unvbox3|' by
`|\dimen2=|^|\dp||3| |\unvbox3| |\kern-\dimen2| ^|\vfil|'.

The next macros are concerned with chapter formatting. Each chapter in the
manuscript file starts out with the macro ^|\beginchapter|; it ends
with ^|\endchapter| and two ^{quotations}, ^^{epigraphs} followed
by ^|\eject|. For example, Chapter~15 was generated by \TeX\ commands
that look like this in the file |manual.tex|:
\beginlines
|\beginchapter Chapter 15. How \TeX\ Makes\\Lines into Pages|
\medskip
|\TeX\ attempts to choose desirable places to divide your document into|
\qquad\dots\quad (about 1100 lines of the manuscript are omitted here)
|break. \ (Deep breath.) \ You got that?|
\medskip
|\endchapter|
\medskip
|Since it is impossible to foresee how [footnotes] will happen to come out|
|in the make-up, it is |%
  |impracticable to number them from 1 up on each page.|\kern-1em
|The best way is to number them consecutively throughout an article|
|or by chapters in a book.|
|\author UNIVERSITY OF CHICAGO PRESS, {\sl Manual of Style\/} (1910)|
\medskip^^{CHICAGO}^^{KNUTH}
|\bigskip|
\medskip
|Don't use footnotes in your books, Don.|
|\author JILL KNUTH (1962)|
\medskip
|\eject|
\endlines
The `|\\|' in the title line specifies a line break to be used on the
left-hand title page that faces the beginning of the chapter. Most of the
|\beginchapter| macro is devoted to preparing that title page; the
^|\TeX| logo needs somewhat different spacing when it is typeset in
|\titlefont|, and the |\inchhigh| digits need to be brought closer
together in order to look right in a title.
\beginlines
^|\newcount||\exno % for the number of exercises in the current chapter|
|\newcount\subsecno % for the number of subsections in the current chapter|
\smallbreak
^|\outer||\def\beginchapter#1 #2#3. #4\par{\def\chapno{#2#3}|
|  \global\exno=0 \subsecno=0|
|  \ifodd\pageno|
|   |^|\errmessage||{You had too much text on that last page; I'm backing up}|
|   \advance\pageno by-1 \fi|
|  \def\\{ } % \\'s in the title will be treated as spaces|
|  |^|\message||{#1 #2#3:} % show the chapter title on the terminal|
|  |^|\xdef||\rhead{#1 #2#3: #4\unskip} % establish a new running headline|
|  {\def\TeX{T\kern-.2em\lower.5ex\hbox{E}\kern-.06em X}|
|    \def\\{#3}|
|    |^|\ifx||\empty\\ \rightline{\inchhigh #2\kern-.04em}|
|    \else\rightline{\inchhigh #2\kern-.06em#3\kern-.04em}\fi|
|    \vskip1.75pc \baselineskip=36pt \lineskiplimit=1pt \lineskip=12pt|
|    \let\\=|^|\cr|| % now the \\'s are line dividers|
|    \halign{\line{\titlefont\hfil##}\\#4\unskip\\}|
|    \titlepage\vfill\eject} % output the chapter title page|
|  \tenpoint\noindent\ignorespaces} % the first paragraph is not indented|
\endlines

An extra page is ejected at the end of a chapter, if necessary, so
that the closing quotations will occur on a right-hand page. \ (The
logic for doing this is not perfect, but it doesn't need to be, because
it fails only when the chapter has to be shortened or lengthened anyway;
book preparation with \TeX, as with type, encourages interaction between
humans and machines.) \
The lines of the quotations are set ^{flush right} by using
^|\obeylines| together with a stretchable ^|\leftskip|:
\beginlines
|\outer\def\endchapter{\ifodd\pageno \else\vfill|^|\eject||\null\fi|
|  \begingroup\bigskip\vfill % beginning of the quotes|
|  \def\eject{|^|\endgroup||\eject} % ending of the quotes|
|  \def\par{\ifhmode\/\endgraf\fi}\obeylines|
|  \def|^|\TeX||{T\kern-.2em\lower.5ex\hbox{E}X}|
|  \eightpoint \let\tt=\ninett \baselineskip=10pt |^|\interlinepenalty||=10000|
|  \leftskip=0pt plus 40pc minus \parindent |^|\parfillskip||=0pt|
|  \let|^|\rm||=\eightss \let|^|\sl||=\eightssi \everypar{\sl}}|
|\def\author#1(#2){\smallskip\noindent\rm--- #1|^|\unskip|^|\enspace||(#2)}|
\endlines

We come now to what goes on inside the chapters themselves. Dangerous and
doubly dangerous bends are specified by typing `^|\danger|' or `^|\ddanger|'
just before a paragraph that is supposed to display a warning symbol:
\beginlines
|\def\dbend{{\manual\char127}} % "dangerous bend" sign|^^{dangerous bend}
|\def\d@nger{\medbreak|^|\begingroup|^|\clubpenalty||=10000|
|  \def\par{|^|\endgraf|^|\endgroup|^|\medbreak||} |%
  ^|\noindent|^|\hang|^|\hangafter||=-2|
|  \hbox to0pt{\hskip-\hangindent\dbend\hfill}|^|\ninepoint||}|
^|\outer||\def\danger{\d@nger}|
|\def\dd@nger{\medbreak\begingroup\clubpenalty=10000|
|  \def\par{\endgraf\endgroup\medbreak} \noindent\hang\hangafter=-2|
|  \hbox to0pt{\hskip-\hangindent\dbend\kern1pt\dbend\hfill}\ninepoint}|
|\outer\def\ddanger{\dd@nger}|
|\def\enddanger{\endgraf\endgroup} % omits the \medbreak|
\endlines
(It's necessary to type `|\enddanger|' at the end of a dangerous bend
only in rare cases that a medium space is not desired after the paragraph;
e.g., `|\smallskip|^|\item|' might be used to give an itemized list within
the scope of the dangerous bend sign.)

A few chapters and appendices of this book (e.g., Chapter~18 and Appendix~B)
are divided into numbered subsections. Such subsections are specified
in the manuscript by typing, for example,
\begintt
\subsection Allocation of registers.
\endtt
Appendix A is subdivided in another way, by paragraphs that have answer numbers:
\beginlines
|\outer\def\subsection#1. {\medbreak\advance\subsecno by 1|
|  \noindent{\it \the\subsecno.\enspace#1.\enspace}}|
|\def\ansno#1.#2:{\medbreak\noindent|
|  \hbox to\parindent{\bf\hss#1.#2.\enspace}\ignorespaces}|
\endlines
We will see below that the manuscript doesn't actually specify an |\ansno|
directly; each call of\/ |\ansno| is generated automatically by the
|\answer| macro.

Appendix H points out {\sl The \TeX book\/} calls for
three hyphenation exceptions:
\beginlines
^|\hyphenation||{man-u-script man-u-scripts ap-pen-dix}|
\endlines

A few macros in |manmac| provide special constructions that are occasionally
needed in paragraphs: |\MF| for ^^{METAFONT}`\MF', |\AmSTeX| for ^^{AmSTeX}
`\AmSTeX', ^|\bull| for ^^{square bullet} `\bull', |\dn| and |\up| for
`\dn' and `\up', |\|\| and |\]| for `\|' and ^^{visible space} `\]'.  ^^{the
visible space symbol} To typeset
\begindisplay
$3\pt$ of \<stuff>, $\oct{105}=69$, $\hex{69}=105$, \cstok{wow}
\enddisplay
the manuscript says ^^{lxix}^^{octal constant}^^{hexadecimal constant}
^^{angle brackets} % cf. The Hacker's Dictionary
\begintt
$3\pt$ of \<stuff>, $\oct{105}=69$, $\hex{69}=105$, \cstok{wow}
\endtt
using the macros |\pt|, |\<|, |\oct|, |\hex|, and |\cstok|.
\beginlines
|\def\MF{{\manual META}\-{\manual FONT}}|
|\def\AmSTeX{$\cal A\kern-.1667em\lower.5ex\hbox{$\cal M$}\kern-.075em|
|  S$-\TeX}|
|\def\bull{\vrule height .9ex width .8ex depth -.1ex } % square bullet|
|\def\SS{{\it SS}} % scriptscript style|
|\def\dn{\leavevmode\hbox{\tt\char'14}} % downward arrow|
|\def\up{|^|\leavevmode||\hbox{\tt\char'13}} % upward arrow|
|\def\|\||{\leavevmode\hbox{\tt\char`\|\||}} % vertical line|
|\def\]{\leavevmode\hbox{\tt\char`\ }} % visible space|
\smallbreak
|\def\pt{\,{\rm pt}} % units of points, in math formulas|
|\def\<#1>{\leavevmode\hbox{$\langle$#1\/$\rangle$}} % syntactic quantity|
|\def\oct#1{\hbox{\rm\'{}\kern-.2em\it#1\/\kern.05em}} % octal constant|
|\def\hex#1{\hbox{\rm\H{}\tt#1}} % hexadecimal constant|
|\def\cstok#1{\leavevmode\thinspace\hbox{|^|\vrule||\vtop{\vbox{\hrule\kern1pt|
|        \hbox{\vphantom{\tt/}\thinspace{\tt#1}\thinspace}}|
|      \kern1pt|^|\hrule||}\vrule}\thinspace} % control sequence token|
\endlines

^{Displays} in this book are usually indented rather than centered,
and they usually involve text rather than mathematics. The
|manmac| format makes such displays convenient by introducing
two macros called |\begindisplay| and |\enddisplay|; there's
also a pair of macros |\begintt| and |\endtt| for displays that
are entirely in ^{typewriter type}. The latter displays are copied
^{verbatim} from the manuscript file, without interpreting
symbols like |\| or |$| in any special way. For example,
part of the paragraph above was typed as follows:
\begintt
... To typeset
\begindisplay
$3\pt$ of \<stuff>, $\oct{105}=69$, $\hex{69}=105$, \cstok{wow}
\enddisplay
the manuscript says
\begintt
$3\pt$ of \<stuff>, $\oct{105}=69$, $\hex{69}=105$, \cstok{wow}
|char`|\endtt
using the macros ||\pt||, ||\<||, ||\oct||, ||\hex||, and ||\cstok||.
\endtt
(The last line of this example illustrates the fact
that verbatim typewriter text can be obtained within a
paragraph by using vertical lines as brackets.) \ The |\begindisplay|
macro is actually more general than you might expect from this example;
it allows multiline displays, with ^|\cr| following each line, and
it also allows local definitions (which apply only within the display) to
be specified immediately after |\begindisplay|.
\beginlines
|\outer\def\begindisplay{\obeylines\startdisplay}|
|{\obeylines\gdef\startdisplay#1|
|  {\catcode`\^^M=5$$#1\halign|^|\bgroup||\indent##\hfil&&\qquad##\hfil\cr}}|
|\outer\def\enddisplay{|^|\crcr|^|\egroup||$$}|^^{dollardollar}
\smallbreak
^|\chardef|^|\other||=12|
|\def\ttverbatim{\begingroup \catcode`\\=\other \catcode`\{=\other|
|  \catcode`\}=\other \catcode`\$=\other \catcode`\&=\other|
|  \catcode`\#=\other \catcode`\%=\other \catcode`\~=\other|
|  \catcode`\_=\other \catcode`\^=\other|
|  \obeyspaces \obeylines \tt}|
|{|^|\obeyspaces||\gdef {\ }} % \obeyspaces now gives \ , not \space|
\smallbreak
|\outer\def\begintt{$$\let\par=\endgraf \ttverbatim \parskip=0pt|
|  \catcode`\|\||=0 |^|\rightskip||=-5pc \ttfinish}|
|{\catcode`\|\||=0 |\||catcode`|\||\=\other % |\|%
  | is temporary escape character|
|  |\||obeylines % end of line is active|
|  |\||gdef|\||ttfinish#1^^M#2\endtt{#1|\||vbox{#2}|\||endgroup$$}}|
\smallbreak
^|\catcode||`\|\||=|^|\active|
|{\obeylines\gdef|\||{\ttverbatim\spaceskip=\ttglue\let^^M=\ \let|\|%
  |=\endgroup}}|\kern-.3pt
\endlines
These macros are more subtle than the others in this appendix, and they
deserve careful study because they illustrate how to disable \TeX's
normal formatting. The `\|' character is normally active (category~13)
in |manmac| format, and its appearance causes the |\ttverbatim| macro to
make all of the other unusual characters into normal symbols (category~12).
However, within the scope of\/ |\begintt...\endtt| a vertical line is an
^{escape character} (category~0); this permits an escape out of verbatim mode.

The |\begintt| macro assumes that a comparatively small amount of
text will be displayed; the verbatim lines are put into a vbox, so
that they cannot be broken between pages. A different approach has
been used for most of the typewriter copy in this appendix and in
Appendix~B\null: Material that is quoted from a format file is
delimited by |\beginlines| and |\endlines|, between which it
is possible to give commands like `^|\smallbreak|' to help with
spacing and page breaking. The |\beginlines| and |\endlines| macros
also insert rules, fore and aft:
\beginlines
|\def\beginlines{\par\begingroup\nobreak\medskip\parindent=0pt|
\nobreak
|  |^|\hrule||\kern1pt\nobreak \obeylines |^|\everypar||{\strut}}|
|\def\endlines{\kern1pt\hrule\endgroup\medbreak\noindent}|
\endlines
For example, the previous three lines were typeset by the specification
\begintt
\beginlines
||\def\beginlines{\par\begingroup\nobreak\medskip\parindent=0pt||
\nobreak
||  \hrule\kern1pt\nobreak \obeylines \everypar{\strut}}||
||\def\endlines{\kern1pt\hrule\endgroup\medbreak\noindent}||
\endlines
\endtt
A ^{strut} is placed in each line so that the rules will be positioned
properly. The |manmac| format also has macros |\beginmathdemo...\endmathdemo|
that were used to produce examples of mathematics in Chapters 16--19,
|\beginsyntax...\endsyntax| for the formal syntax in Chapters 24--26,
|\beginchart...\endchart| for the font tables in Appendices C and~F\null, etc.;
those macros are comparatively simple and they need not be shown here.

Exercises are specified by an |\exercise| macro; for example, the first
exercise in Chapter~1 was generated by the following lines in the
manuscript:
\begintt
\exercise After you have mastered the material in this book,
what will you be: A \TeX pert, or a \TeX nician?
\answer A \TeX nician (underpaid); sometimes also called
a \TeX acker.
\endtt
Notice that the |\answer| is given immediately after each exercise;
that makes it easy to insert new exercises or to delete old ones,
without keeping track of exercise numbers. Exercises that are dangerous
or doubly dangerous are introduced by the macros |\dangerexercise|
and~|\ddangerexercise|.
\beginlines
|\outer\def\exercise{\medbreak \global\advance\exno by 1|
|  \noindent|^|\llap||{\manual\char'170\rm\kern.15em}% triangle in margin|
|  {\ninebf EXERCISE \bf\chapno.\the\exno}\par\nobreak\noindent}|
|\def\dexercise{\global\advance\exno by 1|
|  \llap{\manual\char'170\rm\kern.15em}% triangle in indented space|
|  {\eightbf EXERCISE \bf\chapno.\the\exno}\hfil\break}|
|\outer\def\dangerexercise{\d@nger \dexercise}|
|\outer\def\ddangerexercise{\dd@nger \dexercise}|
\endlines
(The last two lines use |\d@nger| and |\dd@nger|, which are non-|\outer|
equivalents of\/ |\danger| and |\ddanger|; such duplication is necessary
because control sequences of type ^|\outer| cannot appear within a |\def|.)

The |\answer| macro copies an answer into a file called
|answers.tex|; then Appendix~A reads this
file by saying `^|\immediate|^|\closeout||\ans| |\ninepoint|
^|\input| |answers|'. Each individual answer ends with a blank line;
thus, |\par| must be used between the paragraphs of a long answer.
\beginlines
^|\newwrite||\ans|
|\immediate|^|\openout||\ans=answers % file for answers to exercises|
|\outer\def\answer{\par\medbreak|
|  \immediate|^|\write||\ans{}|
|  \immediate\write\ans{\string\ansno\chapno.|^|\the||\exno:}|
|  \copytoblankline}|
|\def\copytoblankline{\begingroup\setupcopy\copyans}|
|\def\setupcopy{\def\do##1{\catcode`##1=\other}\dospecials|
|  \catcode`\|\||=\other \obeylines}|
|{\obeylines \gdef\copyans#1|
|  {\def\next{#1}%|
|  \ifx\next\empty\let\next=\endgroup %|
|  \else\immediate\write\ans{\next} \let\next=\copyans\fi\next}}|
\endlines
Notice the use of\/ ^|\dospecials| here, to set up the ^{verbatim copying}.
The |\ttverbatim| macro could have invoked |\dospecials| in the same way;
^^{efficiency}
but |\ttverbatim| is used quite frequently, so it was streamlined for speed.

The remaining macros in |manmac| format are designed to help in producing
a good ^{index}. When a paragraph contains a word or group of words that
deserve to be indexed, the manuscript indicates this by inserting
|^{...}|; for example, the first sentence of the present paragraph
actually ends with `|a good ^{index}|'. This caused an appropriate
entry to be written onto a file |index.tex| when \TeX\ was typesetting
the page; it also put the word `{\sevenrm index}' into the margin
of the proofsheets, so that the author could remember what
had been marked for indexing without looking into the manuscript file.
Indexing with the |^{...}| notation doesn't change \TeX's behavior in
any essential way; thus, the word `index' appears
in the text as well as in the index. ^^{strut}
\beginlines
^|\newwrite||\inx|
^|\immediate|^|\openout||\inx=index % file for index reminders|
|\def\marginstyle{\sevenrm % marginal notes are in 7-point type|
|  \vrule height6pt depth2pt width0pt } % a strut for \insert\margin|
\endlines

Sometimes it is desirable to index words that don't actually appear on the
page; the notation |^^{...}| stands for a ``silent'' index entry, and
spaces are ignored after the closing `|}|' in such a case. For example,
Appendix~I lists page~1 under `beauty', even though page~1 contains only
the word `beautiful'; the manuscript achieves this by saying
`|beautiful ^^{beauty}|'.  \ (The author felt that it was important to
index `beauty' because he had already indexed `truth'.)

It's not difficult to make |^| into an active character that produces
such index entries, while still retaining its use for superscripts in
math formulas, because ^|\ifmmode| can be used to test whether a
control sequence is being used in math mode. However, |manmac|'s use
of\/ |^| as an active character means that |^^M| cannot be used to refer
^^{hat as an active character} ^^{hat hat}
to a \<return> character. Fortunately the |^^M| notation isn't needed
except when the formatting macros themselves are being defined.

The following macros set things up so that |^| and |^^| are respectively
converted to |\silentfalse\xref| and |\silenttrue\xref|, outside of
math mode:
\beginlines
^|\newif||\ifsilent|
|\def\specialhat{\ifmmode\def\next{^}\else\let\next=\beginxref\fi\next}|
|\def\beginxref{|^|\futurelet||\next\beginxrefswitch}|
|\def\beginxrefswitch{\ifx\next\specialhat\let\next=\silentxref|
|  \else\silentfalse\let\next=\xref\fi \next}|
|\catcode`\^=\active \let ^=\specialhat|
|\def\silentxref^{\silenttrue\xref}|
\endlines

Entries in the index aren't always words in roman type; they might require
special typesetting conventions. For example, there are
hundreds of items in Appendix~I that are preceded by a backslash and
set in typewriter type. The |manmac| format makes it easy to produce
such entries by typing, e.g., `|^|\||\immediate|\|' instead of
`|^{|\||\immediate|\||}|'. In this case the backslash is not written
onto the |index| file, because it would interfere with alphabetization
of the entries; a code number is written out so that the backslash can
be reinstated after the index has been sorted. The code number also
is used to put the entry in typewriter type.

The indexing macros of |manmac| produce entries of four kinds, which are
assigned to codes 0,~1, 2, and~3. Code~0 applies when the argument is enclosed
in braces, e.g., `|^{word}|'\thinspace; code~1 applies when the argument
is enclosed in vertical lines and there's no backslash, e.g.,
`|^|\||plus|\|'\thinspace; code~2 is similar but with a backslash, e.g.,
`|^|\||\par|\|'\thinspace; code~3 applies when the argument is enclosed in
angle brackets, e.g., `|^\<stuff>|'.  The four example entries in the previous
sentence will be written on file |index.tex| in the form\strut
\begintt
word !0 123.
plus !1 123.
par !2 123.
stuff !3 123.
|kern-3pt
\endtt
if they appear on page 123 of the book.
\beginlines
|\chardef\bslash=`\\ % \bslash makes a backslash (in tt fonts)|
|\def\xref{\futurelet\next\xrefswitch} % branch on the next character|
|\def\xrefswitch{\begingroup\ifx\next|\||\aftergroup\vxref|
|  \else\ifx\next\<\aftergroup\anglexref|
|    \else\aftergroup\normalxref \fi\fi \endgroup}|
|\def\vxref|\||{\catcode`\\=\active \futurelet\next\vxrefswitch}|
|\def\vxrefswitch#1|\||{\catcode`\\=0|
|  \ifx\next\empty\def\xreftype{2}%|
|    \def\next{{\tt\bslash\text}}% code 2, |\||\arg|\|
|  \else\def\xreftype{1}\def\next{{\tt\text}}\fi % code 1, |\||arg|\|
|  \edef\text{#1}\makexref}|
|{|^|\catcode||`\|\||=0 \catcode`\\=\active |\||gdef\{}}|
|\def\anglexref\<#1>{\def\xreftype{3}\def\text{#1}%|
|  \def\next{\<\text>}\makexref} % code 3, \<arg>|
|\def\normalxref#1{\def\xreftype{0}\def\text{#1}\let\next=\text\makexref}|
\endlines

Indexing is suppressed unless the |proofmode| switch is set to true,
since material is gathered for the index only during trial runs---not
on the triumphant occasion when the book is finally being printed.
\beginlines
|\newif\ifproofmode|
|\proofmodetrue % this should be false when making camera-ready copy|
|\def\makexref{\ifproofmode\insert\margin{\hbox{\marginstyle\text}}%|
|   |^|\xdef||\writeit{|^|\write||\inx{\text\space!\xreftype\space|
|     |^|\noexpand|^|\number||\pageno.}}\writeit|
|   \else\ifhmode\kern0pt \fi\fi|
|  \ifsilent|^|\ignorespaces||\else\next\fi}|
\endlines
(The ^|\insert| suppresses ^{hyphenation} when proofs are being checked;
a ^|\kern||0pt| is therefore emitted to provide consistent behavior
in the other case.)

The material that accumulates on file |index.tex| gives a good first
approximation to an index, but it doesn't contain enough information to do
the whole job; a topic often occurs on several pages, but only the first
of those pages is typically listed in the file.  The author ^^{Knuth}
prefers not to generate indexes automatically; he likes to reread his
books as he checks the cross-references, thereby having the opportunity to
rethink everything and to catch miscellaneous errors before it is too
late. As a result, his books tend to be delayed, but the indexes tend to
be pretty good.  Therefore he designed the indexing scheme of |manmac| to
provide only the clues needed to make a real index. On the other hand,
it would be possible to extend the macros above and to obtain a comprehensive
system that generates an excellent index with no subsequent human intervention;
see, for example, ``An indexing facility for \TeX'' by Terry ^{Winograd}
and Bill ^{Paxton}, in {\sl ^{TUGboat}\/ \bf1} (1980), A1--A12.

The |manmac| macros have now been fully presented; we shall close this
appendix by presenting one more example of their use. Chapter~27 mentions
the desirability of creating a long book in small parts, by using a
``^{galley}'' file. The author adopted that strategy for {\sl The \TeX book},
entering each chapter into a small file |galley.tex| that looked like this:
\beginlines
|\input manmac|
|\tenpoint|
|\pageno=800|
|\def\rhead{Experimental Pages for The \TeX book}|
|\def\chapno{ X}|
|{\catcode`\%=12 \immediate\write\ans{% Answers for galley proofs:}}|
\qquad\vdots
\<new text being tested, usually an entire chapter>
\qquad\vdots
| |
|% that blank line will stop an unfinished \answer|
|\immediate\closeout\ans|
|\vfill\eject|
|\ninepoint \input answers % typeset the new answers, if any|
|\bye|
\endlines

\endchapter

It is much easier to use macros than to define them.
$\ldots$
The use of macro libraries, in fact, mirrors almost exactly
the use of subroutine libraries for programming languages.
There are the same levels of specialization,
from publicly shared subroutines
to special subroutines within a single program,
and there is the same need for a programmer
with particular skills to define the subroutines.
\author PETER ^{BROWN}, {\sl Macro Processors\/} (1974) % p10

\bigskip

The ^{epigraph} is among the most delightful of scholarly habits.
Donald ^{Knuth}'s work on fundamental algorithms would be
just as important if he hadn't begun with a quotation
from Betty ^{Crocker}, but not so enjoyable.
Part of the fun of an epigraph is turning a source to an unexpected use.
\author MARY-CLAIRE ^{VAN LEUNEN}, {\sl A Handbook for Scholars\/} (1978)
 % page 53. [But it was McCall's, not Betty Crocker]

%\eject
\eject\byebye
