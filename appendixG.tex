% -*- coding: utf-8 -*-

\input macros

%\beginchapter Appendix G. Generating Boxes\\from Formulas
\beginchapter Appendix G. 公式的盒子

\ninepoint
People who define new math fonts and/or macros sometimes need to know
exactly how \TeX\ manipulates the constituents of formulas. The purpose
of this appendix is to explain the precise positioning rules by which
\TeX\ converts a math list into a horizontal list. \ (It is a good idea
to review the introduction to ^{math lists} in Chapter~17 before
reading further; ``double dangerous bends'' are implied throughout
this appendix.)

\TeX\ relies on lots of parameters when it typesets formulas, and you have
the option of changing any or all of them. But of course you will want to
know what each parameter means, before you change it. Therefore each rule
below is numbered, and a table appears at the end to show which rules
depend on which parameters.

The most important parameters appear in the ^{symbol fonts} (family~2) and
the ^{extension fonts} (family~3). \TeX\ will not typeset a formula unless
^|\textfont||2|, ^|\scriptfont||2|, and ^|\scriptscriptfont||2| each contain
at least 22~^|\fontdimen| parameters. For brevity we shall call these parameters
$\sigma_1$ to $\sigma_{22}$, where the parameter is taken from
|\textfont2| if the current style is display or text ($D$ or $D'$ or $T$
or $T'$), from |\scriptfont2| if the current style is $S$ or $S'$, and from
|\scriptscriptfont2| otherwise. Similarly, the three fonts in family~3
must each have at least 13~|\fontdimen| parameters, and we will denote
them by $\xi_1$ to $\xi_{13}$. The notation $\xi_9$, for example, stands for
the ninth parameter of\/ |\scriptfont3|, if \TeX\ is typesetting something
in |\scriptstyle|.

A math list is a sequence of items of the various kinds listed in Chapter~17,
and \TeX\ typesets a formula by converting a math list to a horizontal
list. When such typesetting begins, \TeX\ has two other pieces of
information in addition to the math list itself. \ (a)~The starting style
tells what style should be used for the math list, unless another style
is specified by a style item. For example, the starting style for a
displayed formula is $D$, but for an equation in the text or an equation
number it is $T$; and for a subformula it can be any one of the eight
^{styles} defined in Chapter~17. We shall use $C$ to stand for the current
style, and we shall say that the math list is being typeset in style~$C$.
\ (b)~The typesetting is done either with or without penalties. Formulas
in the text of a paragraph are converted to horizontal lists in which
additional penalty items are inserted after binary operations and relations,
in order to aid in line breaking. Such penalties are not inserted in
other cases, because they would serve no useful function.

The eight styles are considered to be $D>D'>T>T'>S>S'>\it SS>SS'$, in
decreasing order. Thus, $C\le S$ means that the current style is $S$,
$S'$, $\it SS$, or~$\it SS'$.  Style $C'$ means the current style with a
prime added if one isn't there; for example, we have $C'=T'$ if and only
if $C=T$ or $C=T'$.  Style~$C\mathord\uparrow$ is the superscript style
for $C$; this means style~$S$ if $C$~is $D$ or~$T$, style~$S'$ if $C$~is
$D'$ or $T'$, style~$\it SS$ if $C$~is $S$ or $\it SS$, and style~$\it
SS'$ if $C$~is $S'$ or $\it SS'$.  Finally, style~$C\mathord\downarrow$ is
the subscript style, which is~$(C\mathord\uparrow)'$.

Chapter 17 stated that the most important components of math lists are
called atoms, and that each atom has three fields called its nucleus,
subscript, and superscript.  We frequently need to execute a subroutine
called ``Set box~$x$ to the so-and-so field in style such-and-such.'' This
means (a)~if the specified field is empty, $x$~is set equal to a null box;
(b)~if the field contains a symbol, $x$~is set to an hbox containing that
symbol in the appropriate size, and the ^{italic correction} for the
character is included in the width of the box; (c)~if the field contains a
math list or horizontal list, $x$~is set to an hbox containing the result
of typesetting that list with the specified starting style. In case~(c),
the glue is set with no stretching or shrinking, and an additional level
of hboxing is omitted if it turns out to be redundant.

Another subroutine sets box $x$ to a specified variable ^{delimiter},
having a specified minimum height plus depth.  This means that a search is
conducted as follows: The delimiter is defined by two symbols, a ``small
character''~$a$ in family~$f$ and a ``large character''~$b$ in family~$g$.
The search looks first at character $a$ in scriptscriptfont~$f$, if $C\le
\it SS$; then it looks at $a$ in scriptfont~$f$, if $C\le S$; then it looks at
$a$ in textfont~$f$.  If nothing suitable is found from $a$ and~$f$, the
larger alternative $b$ and $g$ is examined in the same way.  Either
$(a,f)$ or $(b,g)$ may be $(0,0)$, which means that the corresponding part
of the search is to be bypassed.  When looking at a character in a
font, the search stops immediately if that character has sufficient height
plus depth, or if the character is ^{extensible}; furthermore, if the
character does not stop the search, but if it has a ^{successor} in the
font, the successor is looked at next. \ (See the \MF\ manual or the
^^{METAFONT}
system documentation of |tfm| files for further information about
successors and extensible characters.) \ If the search runs all the way to
completion without finding a suitable character, the one with greatest
height plus depth is chosen. If no characters at all were found (either
because $a=f=b=g=0$ or because the characters did not exist in the fonts),
$x$~is set to an empty box whose width is ^|\nulldelimiterspace|.  If an
extensible character was found, $x$~is set to a vbox containing enough
pieces to build up a character of sufficient size; the height of this vbox
is the height of the topmost piece, and the width is the width of the
repeatable piece. ^^{built-up characters} Otherwise
$x$~is set to an hbox containing the character that was found; the italic
correction of the character is included in the width of this box.

There's also a subroutine that ``reboxes'' a given box to a given~width.
If the box doesn't already have the desired width, \TeX\ unpackages it
(unless it was a vbox), then adds a kern for an italic correction if one was
implied, and inserts ^|\hss| glue at both left and right; the resulting
horizontal list is packaged into an hbox.  This process is used, for
example, to give a common width to the numerator and denominator of a
fraction; it centers whichever is smaller, unless infinite glue is present
in addition to the newly added |\hss|.

If $x$ is a box, we shall use the abbreviations $h(x)$, $d(x)$, and $w(x)$ for
its height, depth, and width, respectively.

Here now are the rules for typesetting a given math list in starting
style~$C$. The process applies from left to right, translating each
item in turn. Two passes are made over the list;
most of the work is done by the first pass, which compiles individual
translations of the math items. We shall consider this part of the
task first:

\def\rule#1.{\smallskip\textindent{\bf#1.}\ignorespaces}
\rule 1. If the current item is a rule or discretionary or penalty or
``whatsit'' or boundary item, simply leave it unchanged and move to the
next item.

\rule 2. If the current item is glue or a kern, translate it as follows:
If it is glue from ^|\nonscript|, check if the immediately following item
is glue or a kern; and if so, remove that item if $C\le S$. Otherwise, if the
current item is from ^|\mskip| or ^|\mkern|, convert from |mu| to absolute
units by multiplying each finite dimension by ${1\over18}\sigma_6$. Then
move on to the next item.

\rule 3. If the current item is a style change, set $C$ to the specified
style. Delete the current item from the list and move on to the next.

\rule 4. If the current item is a four-way choice, it contains four math
lists for the four main styles. Replace it by the math list that corresponds
to the current style $C$, then move to the first unprocessed item.

\rule 5. If the current item is a Bin atom, and if this was the first atom
in the list, or if the most recent previous atom was Bin, Op, Rel, Open,
or Punct, change the current Bin to Ord and continue with Rule~14.
Otherwise continue with Rule~17.

\rule 6. If the current item is a Rel or Close or Punct atom, and if the most
recent previous atom was Bin, change that previous Bin to Ord. Continue
with Rule~17.

\rule 7. If the current item is an Open or Inner atom, go directly to Rule~17.

\rule 8. If the current item is a Vcent atom (from ^|\vcenter|), let its
nucleus be a vbox of height-plus-depth~$v$. Change the height to
${1\over2}v+a$ and the depth to ${1\over2}v-a$, where $a$~is the
axis height, $\sigma_{22}$. Change this atom to type Ord and continue
with Rule~17.

\rule 9. If the current item is an Over atom (from ^|\overline|), set
box~$x$ to the nucleus in style $C'$. Then replace the nucleus by a vbox
containing kern~$\theta$, hrule of~height~$\theta$, kern~$3\theta$,
and box~$x$, from top to bottom, where $\theta=\xi_8$ is the ^{default
rule thickness}. \ (This puts a rule over the nucleus, with $3\theta$ clearance,
and with $\theta$~units of extra white space assumed to be present above
the rule.) \ Continue with Rule~16.

\rule 10. If the current item is an Under atom (from ^|\underline|), set
box~$x$ to the nucleus in style $C$. Then replace the nucleus by a vtop
made from box~$x$, kern~$3\theta$, and hrule of~height~$\theta$,
where $\theta=\xi_8$ is the default rule thickness; and add $\theta$ to the
depth of the box. \ (This puts a rule under the nucleus, with $3\theta$
clearance, and with $\theta$~units of extra white space assumed to be
present below the rule.) \ Continue with Rule~16.

\rule 11. If the current item is a Rad atom (from ^|\radical|, e.g.,
^|\sqrt|), set box~$x$ to the nucleus in style $C'$. Let $\theta=\xi_8$;
and let $\varphi=\sigma_5$ if $C>T$, otherwise $\varphi=\theta$.
Set $\psi=\theta+{1\over4}\vert\varphi\vert$; this is the minimum
clearance that will be allowed between box~$x$ and the rule that will go
above it. Set box~$y$ to a variable delimiter for this radical atom, having
height plus depth $h(x)+d(x)+\psi+\theta$ or more. Then set $\theta\leftarrow
h(y)$; this is the thickness of the rule to be used in the radical
construction. \ (Note that the font designer specifies the thickness of
the rule by making it the height of the radical character; the baseline
of the character should be precisely at the bottom of the rule.) \
If $d(y)>h(x)+d(x)+\psi$, increase $\psi$ by half of the excess; i.e.,
set $\psi\leftarrow{1\over2}\bigl(\psi+d(y)-h(x)-d(x)\bigr)$. Construct
a vbox consisting of kern~$\theta$, hrule of~height~$\theta$, kern~$\psi$,
and box~$x$, from top to bottom. The nucleus of the radical atom is now
replaced by box~$y$ raised by $h(x)+\psi$, followed by the new vbox.
Continue with Rule~16.

\rule 12. If the current item is an Acc atom (from ^|\mathaccent|), just
go to Rule~16 if the accent character doesn't exist in the current size.
Otherwise set box~$x$ to the nucleus in style $C'$, and set $u$ to the
width of this box.  If the nucleus is not a single character, let $s=0$;
otherwise set $s$ to the kern amount for the nucleus followed by the
^|\skewchar| of its font.  If the accent character has a successor in its
font whose width is $\le u$, change it to the successor and repeat this
sentence.  Now set $\delta\leftarrow \min\bigl(h(x),\chi\bigr)$, where
$\chi$ is |\fontdimen5| (the ^{x-height}) in the accent font.  If the
nucleus is a single character, replace box~$x$ by a box containing the
nucleus together with the superscript and subscript of the Acc atom, in
style~$C$, and make the sub/superscripts of the Acc atom empty; also
increase $\delta$ by the difference between the new and old values
of~$h(x)$. Put the accent into a new box~$y$, including the italic
correction.  Let~$z$ be a vbox consisting of: box~$y$ moved right
$s+{1\over2}\bigl(u-w(y)\bigr)$, kern $-\delta$, and box~$x$. If
$h(z)<h(x)$, add a kern of $h(x)-h(z)$ above box~$y$ and set $h(z)\gets
h(x)$. Finally set $w(z)\gets w(x)$, replace the nucleus of the Acc atom
by box~$z$, and continue with Rule~16.

\rule 13. If the current item is an Op atom, mark this atom as having
limits if it has been marked with ^|\limits|, or if it has been marked
with ^|\displaylimits| and $C>T$.  If the nucleus is not a symbol, set
$\delta\leftarrow0$ and go to Rule~13a. Otherwise if $C>T$ and if the
nucleus symbol has a successor in its font, move to the successor. \ (This
is where operators like $\sum$ and $\int$ change to a larger size in
display styles.) \ Put the symbol into a new box~$x$, in the current size,
and set $\delta$ to the italic correction for the character; include $\delta$
in the width of box~$x$ if and only if limits are to be set or
there is no subscript. Shift box~$x$ down by ${1\over2}\bigl(h(x)-d(x)\bigr)
-a$, where $a=\sigma_{22}$, so that the operator character is centered
vertically on the axis; this shifted box becomes the nucleus of the Op atom.

\rule 13a. If limits are not to be typeset for this Op atom, go to
Rule~17; otherwise the limits are attached as follows:
Set box~$x$ to the superscript field in style $C\mathord\uparrow$; set box~$y$
to the nucleus field in style~$C$; and set box~$z$ to the subscript field
in style $C\mathord\downarrow$. Rebox all three of these boxes to width
$\max\bigl(w(x),w(y),w(z)\bigr)$.  If the superscript field was not empty,
attach box~$x$ above box~$y$, separated by a kern of size
$\max\bigl(\xi_9,\xi_{11}-d(x)\bigr)$, and shift box~$x$ right by
${1\over2}\delta$; also put a kern of size~$\xi_{13}$ above box~$x$.
If the subscript field was not empty,
attach box~$z$ below box~$y$, separated by a kern of size
$\max\bigl(\xi_{10},\xi_{12}-h(z)\bigr)$, and shift box~$z$ left by
${1\over2}\delta$; also put a kern of size~$\xi_{13}$ below box~$z$.
The resulting vbox becomes the nucleus of the current Op atom; move to
the next item.

\rule 14. If the current item is an Ord atom, go directly to Rule~17 unless
all of the following are true: The nucleus is a symbol; the subscript
and superscript are both empty; the very next item in the math list is an
atom of type Ord, Op, Bin, Rel, Open, Close, or Punct; and the nucleus of the
next item is a symbol whose family is the same as the family in the present
Ord atom. In such cases the present symbol is marked as a text symbol.
If the font information shows a ligature between this symbol and the
following one, using the specified family and the current size, then
insert the ligature character and continue as specified by the font;
in this process, two characters may collapse into a single Ord
text symbol, and/or new Ord text characters may appear. If the font information
shows a kern between the current symbol and the next, insert a kern item
following the current atom.
As soon as an Ord atom has been fully processed for ligatures and kerns,
go to Rule~17.

\rule 15. If the current item is a generalized fraction (and it had better
be, because that's the only possibility left if Rules 1--14 don't apply),
let $\theta$ be the thickness of the bar line and let $(\lambda,\rho)$ be
the left and right delimiters. \ If this fraction was generated by
^|\over| or ^|\overwithdelims|, then $\theta=\xi_8$; if it was generated by
^|\atop| or ^|\atopwithdelims|, $\theta=0$; otherwise it was generated by
^|\above| or ^|\abovewithdelims|, and a specific value of~$\theta$ was
given at that time. The values of $\lambda$ and $\rho$ are null unless
the fraction is ``with delims.''

\rule 15a. Put the numerator into box~$x$, using style $T$ or $T'$ if
$C$~is $D$ or $D'$, otherwise using style $C\mathord\uparrow$.
Put the denominator into box~$z$, using style $T'$ if
$C>T$, otherwise using $C\mathord\downarrow$.
If $w(x)<w(z)$, rebox $x$ to width~$w(z)$;
if $w(z)<w(x)$, rebox $z$ to width~$w(x)$.

\rule 15b. If $C>T$, set $u\leftarrow\sigma_8$ and $v\leftarrow\sigma_{11}$.
Otherwise set $u\leftarrow\sigma_9$ or $\sigma_{10}$, according as $\theta\ne0$
or $\theta=0$, and set $v\leftarrow\sigma_{12}$. \ (The fraction will be
typeset with its numerator shifted up by an amount~$u$ with respect to
the current baseline, and with the denominator shifted down by~$v$,
unless the boxes are unusually large.)

\rule 15c. If $\theta=0$ (|\atop|), the numerator and denominator are
combined as follows: Set $\varphi\leftarrow7\xi_8$ or
$3\xi_8$, according as $C>T$ or $C\le T$; $\varphi$ is the minimum
clearance that will be tolerated between numerator and denominator.
Let $\psi=\bigl(u-d(x)\bigr)-\bigl(h(z)-v\bigr)$ be the actual clearance
that would be obtained with the current values of $u$ and~$v$; if
$\psi<\varphi$, add ${1\over2}(\varphi-\psi)$ to both $u$ and~$v$.
Then construct a vbox of height $h(x)+u$ and depth $d(z)+v$, consisting
of box~$x$ followed by an appropriate kern followed by box~$z$.

\rule 15d. If $\theta\ne0$ (|\over|), the numerator and denominator are
combined as follows: Set $\varphi\leftarrow3\theta$ or $\theta$, according
as $C>T$ or $C\le T$; $\varphi$ is the minimum clearance that will be
tolerated between numerator or denominator and the bar line.  Let
$a=\sigma_{22}$ be the current axis height; the middle of the bar line
will be placed at this height.  If
$\bigl(u-d(x)\bigr)-(a+{1\over2}\theta)<\varphi$, increase~$u$ by the
difference between these quantities; and if $(a-{1\over2}\theta)-
\bigl(h(z)-v\bigr)<\varphi$, increase~$v$ by the difference. Finally
construct a vbox of height $h(x)+u$ and depth $d(z)+v$, consisting of
box~$x$ followed by a kern followed by an hrule of height~$\theta$
followed by another kern followed by box~$z$, where the kerns are figured
so that the bottom of the hrule occurs at $a-{1\over2}\theta$ above the
baseline.

\rule 15e. Enclose the vbox that was constructed in Rule 15c or 15d by
delimiters whose height plus depth is at least $\sigma_{20}$, if $C>T$, and at
least $\sigma_{21}$ otherwise. Shift the delimiters up or down so that they are
vertically centered with respect to the axis. Replace the generalized
fraction by an Inner atom whose nucleus is the resulting sequence of three boxes
(left delimiter, vbox, right delimiter).

\bigbreak\noindent
Rules 1--15 account for the preliminary processing of math list items;
but we still haven't specified how subscripts and superscripts are to be
typeset. Therefore some of those rules lead to the following post-process:

\rule 16. Change the current item to an Ord atom, and continue with Rule~17.

\rule 17. If the nucleus of the current item is a math list, replace it by
a box obtained by typesetting that list in the current style.  Then if the
nucleus is not simply a symbol, go on to Rule~18.  Otherwise we are in the
common case that a math symbol is to be translated to its horizontal-list
equivalent: Convert the symbol to a character box for the specified family
in the current size. If the symbol was not marked by Rule~14 above as a
text symbol, or if\/ |\fontdimen| parameter number~2 of its font is zero, set
$\delta$ to the italic correction; otherwise set $\delta$ to zero. If
$\delta$ is nonzero and if the subscript field of the current atom is
empty, insert a kern of width~$\delta$ after the character box, and set
$\delta$ to zero. Continue with Rule~18.

\rule 18. (The remaining task for the current atom is to attach a possible
subscript and superscript.) \ If both subscript and superscript fields
are empty, move to the next item. Otherwise continue with the following
subrules:

\rule 18a. If the translation of the nucleus is a character box, possibly
followed by a kern, set $u$ and~$v$ equal to zero; otherwise set
$u\leftarrow h-q$ and $v\leftarrow d+r$, where $h$ and~$d$ are the height
and depth of the translated nucleus, and where $q$ and~$r$ are the values
of $\sigma_{18}$ and $\sigma_{19}$ in the font corresponding to styles
$C\mathord\uparrow$ and~$C\mathord\downarrow$.  \ (The quantities $u$
and~$v$ represent minimum amounts by which the superscript and subscript
will be shifted up and down; these preliminary values of $u$ and~$v$ may
be increased later.)

\rule 18b. If the superscript field is empty (so that there is a subscript
only), set box~$x$ to the subscript in style~$C\mathord\downarrow$, and add
^|\scriptspace| to $w(x)$. Append this box to the translation of the
current item, shifting it down by $\max\bigl(v,\sigma_{16},h(x)-{4\over5}
\vert\sigma_5\vert\bigr)$, and move to the next item. \ (The idea is
to make sure that the subscript is shifted by at least~$v$ and by at
least $\sigma_{16}$; furthermore, the top of the subscript should not extend
above $4\over5$~of the current x-height.)

\rule 18c. Set box~$x$ to the superscript field in style~$C\mathord\uparrow$,
and add |\scriptspace| to~$w(x)$. Then set $u\leftarrow\max\bigl(u,p,
d(x)+{1\over4}\vert\sigma_5\vert\bigr)$, where $p=\sigma_{13}$ if $C=D$,
$p=\sigma_{15}$ if $C=C'$, and $p=\sigma_{14}$ otherwise; this gives a
tentative position for the superscript.

\rule 18d. If the subscript field is empty (so that there is a
superscript only), append box~$x$ to the translation of the current atom,
shifting it up by $u$, and move to the next item. Otherwise (i.e.,
both subscript and superscript are present), set box~$y$ to the
subscript in style~$C\mathord\downarrow$, add |\scriptspace| to~$w(y)$,
and set $v\leftarrow\max(v,\sigma_{17})$.

\rule 18e. (The remaining task is to position a joint
subscript/superscript combination.) \ Let $\theta=\xi_8$ be the default
rule thickness.  If $\bigl(u-d(x)\bigr)-\bigl(h(y)-v\bigr)\ge4\theta$, go
to Rule~18f. \ (This means that the white space between subscript and
superscript is at least $4\theta$.) \ Otherwise reset $v$ so that
$\bigl(u-d(x)\bigr)-\bigl(h(y)-v\bigr)=4\theta$. Let $\psi={4\over5}
\vert\sigma_5\vert-\bigl(u-d(x)\bigr)$. If $\psi>0$, increase $u$
by~$\psi$ and decrease $v$ by~$\psi$. \ (This means that the bottom of the
superscript will be at least as high above the baseline as $4\over5$ of
the x-height.)

\rule 18f. Finally, let $\delta$ be zero unless it was set to a nonzero
value by Rules 13 or 17. \ (This is the amount of horizontal displacement
between subscript and superscript.) Make a vbox of height $h(x)+u$ and
depth $d(y)+v$, consisting of box~$x$ shifted right by~$\delta$, followed
by an appropriate kern, followed by box~$y$. Append this vbox to the
translation of the current item and move to the next.

\bigbreak\noindent
After the entire math list has been processed by Rules 1--18, \TeX\ looks
at the last atom (if there was one), and changes its type from
Bin to Ord (if it was of type Bin). Then the following rule is performed:

\rule 19. If the math list begins and ends with boundary items, compute
the maximum height~$h$ and depth~$d$ of the boxes in the translation of
the math list that was made on the first pass, taking into account the
fact that some boxes may be raised or lowered. Let $a=\sigma_{22}$ be the
axis height, and let $\delta=\max(h-a,d+a)$ be the amount by which the
formula extends away from the axis. Replace the boundary items by
delimiters whose height plus depth is at least $\max(\lfloor\delta/500\rfloor
f,2\delta-l)$, where $f$ is the ^|\delimiterfactor|
and $l$ is the ^|\delimitershortfall|.  Shift the delimiters up or down so
that they are vertically centered with respect to the axis. Change the left
boundary item to an Open atom and the right boundary item to
a Close atom. \ (All of the calculations in this step are done with
$C$ equal to the starting style of the math list; style items in the
middle of the list do not affect the style of the right boundary item.)

\medbreak \rule 20. Rules 1--19 convert the math list into a sequence of
items in which the only remaining atoms are of types Ord, Op, Bin, Rel,
Open, Close, Punct, and Inner. After that conversion is complete, a second pass
is made through the entire list, replacing all of the atoms by the boxes
and kerns in their translations. Furthermore, additional inter-element
spacing is inserted just before each atom except the first, based on the
type of that atom and the preceding one.  Inter-element spacing is defined
by the three parameters ^|\thinmuskip|, ^|\medmuskip|, and
^|\thickmuskip|; the |mu| units are converted to absolute units as in
Rule~2 above.  Chapter~18 has a chart that defines the inter-element
spacing, some of which is ^|\nonscript|, i.e., it is inserted only in
styles $>S$.  The list might also contain style items, which are removed
during the second pass; they are used to change the current style just as
in the the first pass, so that both passes have the same value of~$C$ when
they work on any particular atom.

\rule 21. Besides the inter-element spacing, penalties are placed after
the translation of each atom of type Bin or Rel, if the math list was part
of a paragraph.  The penalty after a Bin is ^|\binoppenalty|, and the
penalty after a Rel is ^|\relpenalty|. However, the penalty is not
inserted after the final item in the entire list, or if it has a numeric
value~$\ge10000$, or if the very next item in the list is already a
penalty item, or after a Rel atom that is immediately followed by another
Rel atom.

\rule 22. After all of the preceding actions have been performed, the math
list has been totally converted to a horizontal list. If the result is
being inserted into a larger horizontal list, in horizontal mode or
restricted horizontal mode, it is enclosed by ``^{math-on}'' and
``^{math-off}'' items that each record the current value of\/
^|\mathsurround|.  Or if this list is a displayed formula, it
is processed further as explained in Chapter~19.

\medbreak
\def\\{\kern1pt}
\noindent{\bf Summary of parameter usage.} \ Here is the promised index
that refers to everything affected by the mysterious parameters
in the symbol fonts. Careful study of the rules allows you to get the
best results by appropriately setting the parameters for new fonts
that you may wish to use in mathematical typesetting. Each font parameter
has an external name that is used in supporting software packages;
for example, $\sigma_{14}$ is generally referred to as `sup2' and $\xi_8$ as
`default\_\\ rule\_thickness'. These external names are indicated in the table.
$$\vbox{
\halign to\hsize{$#\hfil$\tabskip10pt plus 10pt minus 3pt&
#\hfil&$\rm#\hfil$\quad&
$#\hfil$&#\hfil&$\rm#\hfil$\tabskip0pt\cr
\omit\span\omit\it Parameter&\omit\it Used in\hfil&
\omit\span\omit\it Parameter&\omit\it Used in\hfil\cr
\noalign{\medskip\hrule\smallskip}
\sigma_2&space&17&
\sigma_{17}&sub2&18d\cr
\sigma_5&x\_\\height&11, 18b, 18c, 18e&
\sigma_{18}&sup\_drop&18a\cr
\sigma_6&quad&2, 20&
\sigma_{19}&sub\_drop&18a\cr
\sigma_8&num1&15b&
\sigma_{20}&delim1&15e\cr
\sigma_9&num2&15b&
\sigma_{21}&delim2&15e\cr
\sigma_{10}&num3&15b&
\sigma_{22}&axis\_\\height&8, 13, 15d, 19\cr
\sigma_{11}&denom1&15b&
\xi_8&default\_\\rule\_thickness&9, 10, 11, 15, 15c, 18e\cr
\sigma_{12}&denom2&15b&
\xi_9&big\_op\_\\spacing1&13a\cr
\sigma_{13}&sup1&18c&
\xi_{10}&big\_op\_\\spacing2&13a\cr
\sigma_{14}&sup2&18c&
\xi_{11}&big\_op\_\\spacing3&13a\cr
\sigma_{15}&sup3&18c&
\xi_{12}&big\_op\_\\spacing4&13a\cr
\sigma_{16}&sub1&18b&
\xi_{13}&big\_op\_\\spacing5&13a\cr
\noalign{\smallskip\hrule\medskip}}}$$
Besides the symbol and extension fonts (families 2 and~3), the
rules above also refer to parameters in other families:
Rule~17 uses |\fontdimen| parameter~2 (space) to determine whether to insert
an italic correction between adjacent letters, and Rule~12 uses parameter~5
(x\_\\height) to position an accent character. Several non-font parameters
also affect mathematical typesetting:
dimension parameters
 \hbox{|\delimitershortfall|} (Rule~19),
 \hbox{|\nulldelimiterspace|} (in
the construction of variable delimiters for Rules 11, 15e,~19),
 \hbox{|\mathsurround|} (Rule~22),
and
 \hbox{|\scriptspace|} (Rules 18bcd);
integer parameters
 \hbox{|\delimiterfactor|} (Rule~19),
 \hbox{|\binoppenalty|} (Rule~21),
and
 \hbox{|\relpenalty|} (Rule~21);
muglue parameters
 \hbox{|\thinmuskip|},
 \hbox{|\medmuskip|,} and
 \hbox{|\thickmuskip|}
(Rule~20).

\endchapter

Woe to the author who always wants to teach!\/
% Mais malheur a` l'auteur qui veut toujours instruire!
The secret of being a bore is to tell everything.
% Le secret d'ennuyer est celui de tout dire.
\author ^{VOLTAIRE}, {\sl De la Nature de l'Homme\/} (1737)
% 6th discours en vers sur l'homme

\bigskip

Very few Compositors are fond of Algebra,
and rather chuse to be employed upon plain work.
\author PHILIP ^{LUCKOMBE}, {\sl The History and Art of Printing\/} (1770)
  % page 472

%\eject
\eject\byebye
