% -*- coding: utf-8 -*-

\input macros

%\beginchapter Chapter 5. Grouping
\beginchapter Chapter 5. 编组

\origpageno=19

%Every once in a while it is necessary to treat part of a manuscript as a
%unit, so you need to indicate somehow where that part begins and where it
%ends. For this purpose \TeX\ gives special interpretation to two
%``^{grouping characters},'' which (like the escape character) are
%treated differently from the normal symbols that you type. We assume in
%this manual that |{| and |}| are the grouping characters, since they
%are the ones used in plain \TeX. ^^{curly braces, see braces}
\1不时地需要把文稿的一部分当成一个整体来处理,所以你应该声明此部分的开头和结尾。%
为此,\TeX\ 给两个``编组字符''赋予了特殊含义,它(象转义符一样)与所键入的正常%
符号的处理方法不同。%
在本手册中我们设定 |{| 和 |}| 是编组字符,因为在 plain \TeX\ 中如此。

%We saw examples of grouping in the previous chapter, where it was mentioned
%that font changes inside a group do not affect the fonts in force outside.
%The same principle applies to almost anything else that is defined inside
%a group, as we will see later; for example, if you define a control sequence
%within some group, that definition will disappear when the group ends.
%In this way you can conveniently instruct \TeX\ to do something unusual,
%by changing its normal conventions temporarily inside of a group; since
%the changes are invisible from outside the group, there is no need to worry
%about messing up the rest of a manuscript by forgetting to restore the
%normal conventions when the unusual construction has been finished.
%Computer scientists have a name for this aspect of grouping, because it's
%an important aspect of programming languages in general; they call it
%``^{block structure},'' and definitions that are in force only within
%a group are said to be ``^{local}'' to that group.
在前一章,我们已经遇见了编组的例子,那里用到的是在组中的字体变化不影响外面的字\hbox{体。}%
就象外面后面要学习的一样,同一原理可以应用到定义在组中的几乎所有内容;
例如,如果你在某些组中定义了一个控制系列,当组结束时,那些定义就消失了。%
用这种方法,通过在组中临时改变其正常规则,可简便地告诉 \TeX\ 完成特殊的工作;
因为这种变化在组外部将消失了,所以当特殊工作结束后,不必担心文稿后面的部分怎样%
才能恢复正常规则。%
因为编组一般在编程语言方面是很重要的,所以计算机方面上它有一个名字,
即``模块结构'', 只在组中起作用的定义相对于那个组是``局部的''。

%You might want to use grouping even when you don't care about block
%structure, just to have better control over spacing. For example, let's
%consider once more the control sequence ^|\TeX| that produces
%the logo `\TeX' in this manual: We observed in Chapter~3 that a blank space
%after this control sequence will be gobbled up unless one types
%`\hbox{|\TeX\ |}', yet it is a mistake to say `|\TeX\|' when the following
%character is not a blank space. Well, in {\sl all\/} cases it would be
%correct to specify the simple group
%\begintt
%{\TeX}
%\endtt
%whether or not the following character is a ^{space}, because the |}| stops
%\TeX\ from absorbing an optional space into |\TeX|. This might come in
%handy when you're using a text editor (e.g., when replacing all occurrences
%of a particular word by a control sequence). Another thing you could do is
%type
%\begintt
%\TeX{}
%\endtt
%using an {\sl empty\/} group for the same purpose: The `|{}|' here is a
%group of no characters, so it produces no output, but it does have the
%effect of stopping \TeX\ from skipping blanks.
%^^{empty group} ^^{lbrace rbrace}
即使你不关心模块结构,就是为了更好地得到间距,你可能也要利用编组。%
例如,让外面再讨论本手册中得到标识符`\TeX'的控制系列 |\TeX|:
在第三章我们看到,如果不键入`\hbox{|\TeX\ |}', 控制系列后面的空格会被吃掉,
但是当后面跟的是非空格字符时,键入`|\TeX\|'又出现错误。%
噢,在{\KT{10}所有}情况下,给出一个简单的组:
\begintt
{\TeX}
\endtt
总是正确的,不管跟着的字符是不是空格,
因为 |}| 使 \TeX\ 不能把可能的空格吃\hbox{掉。}%
当你正在使用文本编辑器时这更方便(比如,当用控制下列代替所有出现的特殊单词时)。%
你也可用另一种方法,就是键入
\begintt
\TeX{}
\endtt
用{\KT{10}空}组来达到同样的目的:
这里的`|{}|'是没有字符的组,所有它没有输出结果,但是它的确有阻止 \TeX\ 吃掉空格的%
效果。

%\exercise Sometimes you run into a rare word like `shelfful' that looks
%better as `shelf{}ful' without the `ff' ^{ligature}. How can you fool
%\TeX\ into thinking that there aren't two consecutive f's in such a word?
%\answer |{shelf}ful| or |shelf{}ful|, etc.; or even |shelf\/ful|, which
%yields a shelf\/ful instead of a shelf{\kern0pt}ful.
%In fact, the latter idea---to
%insert an ^{italic correction}---is preferable because \TeX\ will ^^|\/|
%reinsert the ff ligature by itself after ^{hyphenating} |shelf{}ful|. \
%(Appendix~H points out that ligatures are put into a hyphenated word that
%contains no ``^{explicit kerns},'' and an italic correction is an
%explicit kern.) \ But the italic correction may be too much (especially in an
%italic font); |shelf{|^|\kern||0pt}ful| is often best.
\exercise 有时候,会遇到一个类似 `shelfful' 的罕见单词,不用 `ff' ^{连写}而用%
`shelf{}ful' 看起来更好。想想怎样愚弄一下 \TeX ,让它认为这样的单词中的两个 f~%
不是紧接着的。
\answer |{shelf}ful| 或 |shelf{}ful| 等;甚或是 |shelf\/ful|,
利用它将得到 shelf\/ful 而不是 shelf{\kern0pt}ful。
实际上,后者的想法——插入一个^{倾斜校正}——更合适,^^|\/|
因为 \TeX\ 在对 |shelf{}ful| ^{连字化}后将重新插入 ff 连写。%
(附录 H 指出连写将被放入不含``^{显式紧排}''的已连字单词中,
而倾斜校正就是一个显式紧排。)但是倾斜校正也许太多了(尤其在意大利体中);
|shelf{|^|\kern||0pt}ful| 通常是最好的。%
【译注:前面的“连字化”和“已连字”应该分别理解为“尝试连字化”和“已尝试连字化”。
在给一个段落断行时,\TeX\ 首先尝试不连字的方式,如果失败再尝试带连字的方式。
在第二轮尝试中每个单词的可能断行点的 |{}| 将被丢弃,因此连写可能会被重新生成。
不过段落的第一个单词是个例外,因为 \TeX\ 从不尝试对第一个单词连字化。
另外,在 Lua\TeX 中,这种添加 |{}| 以阻止连写的做法始终是无效的。】

%\dangerexercise Explain how to get three blank spaces in a row without
%using `|\|\]'.^^{control space}
%\answer `\]|{|\]|}|\]' or `\]|{}|\]|{}|\]', etc. Plain \TeX\ also has a
%^|\space| macro, so you can type |\space\space\space|.  \ (These aren't
%strictly equivalent to `|\|\]|\|\]|\|\]', since they adjust the spaces by
%the current ``^{space factor},'' as explained later.)
\dangerexercise 不用`|\|\]', 看看怎样在一排中得到 3 个空格?
\answer `\]|{|\]|}|\]' 或 `\]|{}|\]|{}|\]' 等。Plain \TeX\ 中还有
^|\space| 宏,从而你也可以键入 |\space\space\space|。
(这些并不严格等同于 `|\|\]|\|\]|\|\]',
因为它们将根据当前的``^{间距因子}''调整空白的大小,这会在稍后说明。)

%But \TeX\ also uses grouping for another, quite different, purpose, namely
%to determine how much of your text is to be governed by certain control
%sequences. For example, if you want to center something on a line you can type
%\begintt
%\centerline{This information should be centered.}
%\endtt
%using the control sequence ^|\centerline| defined in plain \TeX\ format.
\1但是 \TeX\ 也在另外非常不同的方面使用编组,即规定特定的控制系列作用在哪些文本上。%
例如,你要把一行中的内容居中,利用定义在 plain \TeX\ 中的控制系列 |\centerline|,
你可以键入
\begintt
\centerline{This information should be centered.}
\endtt

%Grouping is used in quite a few of \TeX's more intricate instructions; and
%it's possible to have groups within groups within groups, as you can see
%by glancing at Appendix~B\null. Complex grouping is generally unnecessary,
%however, in ordinary manuscripts, so you needn't worry about it. Just
%don't forget to finish each group that you've started, because a lost
%`|}|' might cause trouble.
编组还出现在 \TeX\ 的很少几个更复杂的指令中;
当浏览附录 B 时你就会看到,可以组中套组。%
但是在普通文稿中,一般不需要复杂的编组,所以你不必担心。%
只是别忘了把每个开始的组要结束掉,因为忘记`|}|'可能会出问题。

%Here's an example of two groups, one ^{nested} inside the other:
%\begintt
%\centerline{This information should be {\it centered}.}
%\endtt
%As you might expect, \TeX\ will produce a centered line that also contains
%italics:
%$$\hbox{This information should be {\it centered}.}$$
%But let's look at the example more closely: `|\centerline|' appears outside
%the curly braces, while `|\it|' appears inside. Why are the two cases
%different? And how can a beginner learn to remember which is which?
%Answer: |\centerline| is a control sequence that applies only to the very next
%thing that follows, so you want to put braces around the text that is to
%be centered (unless that text consists of a single symbol or control sequence).
%For example, to center the \TeX\ logo on a line, it would suffice to
%type `|\centerline\TeX|', but to center the phrase `\TeX\ has groups' you
%need braces: `|\centerline{\TeX\ has groups}|'. On the other hand, |\it| is
%a control sequence that simply means ``change the current font''; it acts
%without looking ahead, so it affects {\sl everything\/} that follows, at
%least potentially. The braces surround |\it| in order to confine the font
%change to a local region.
这里有一个两个组的例子,一个嵌套在另一个中间:
\begintt
\centerline{This information should be {\it centered}.}
\endtt
正如所预期的一样,\TeX\ 产生了一个居中的行,其中包含 italic 字体:
$$\hbox{This information should be {\it centered}.}$$
但是,让我们更详细地讨论这个例子:
`|\centerline|'出现在大括号外面,而`|\it|'出现在里面。%
为什么这两种情况不同呢?
初学者怎样才能弄清谁是谁呢?
答案是:|centerline| 这个控制系列是只作用在其后接下来的内容上,
所以你要把居中的内容放在括号中(除非其内容仅仅是一个符号或一个控制系列)。%
例如,要把标识符 \TeX\ 居中,键入 `|\centerline\TeX|' 就可以了,
但是要居中短语`\TeX\ has groups', 就需要括号:`|\centerline{\TeX\ has groups}|'。%
另一方面,|\it| 这个控制系列简单表示``改变当前字体'';
它对前面没有作用,所以它——至少暗中——对其后的{\KT{10}所有}内容起作用。%
括住 |\it| 的括号是为了把字体改变的作用限制在局部区域。

%In other words, the two sets of braces in this example actually have different
%functions: One serves to treat several words of the text as if they
%were a single object, while the other provides local block structure.
换句话说,本例中的两组括号的作用不同:
一个是把文本的多个单词看作一个单一对象,而另一个提供了局部模块结构。

%\exercise What do you think happens if you type the following:
%\begintt
%\centerline{This information should be {centered}.}
%\centerline So should this.
%\endtt
%\answer In the first case, you get the same result as if the innermost
%braces had not appeared at all, because you haven't used the grouping to
%change fonts or to control spacing or anything. \TeX\ doesn't mind if you
%want to waste your time making groups for no particular reason.
%But in the second case, the necessary braces were forgotten. You get the
%letter `S' centered on a line by itself, followed by a paragraph that
%begins with `o should this.' on the next line.
\exercise 如果你键入下列内容,会得到什么结果?
\begintt
\centerline{This information should be {centered}.}
\centerline So should this.
\endtt
\answer 第一种情形中,你得到的结果与内部花括号不存在时的一样,
因为你没有利用该编组改变字体或控制间距或做其他事情。
\TeX\ 不在乎你浪费时间制造没有用处的编组。
但是在第二种情形中,必需的花括号被遗漏了。
你将得到在单独一行居中的字母 `S',
在下一行跟着一个以 `o should this.' 开始的段落。

%\exercise And how about this one?
%\begintt
%\centerline{This information should be \it centered.}
%\endtt
%\answer You get the same result as if another pair of braces were present
%around `|\it centered|', except that the period is typeset from the
%italic font. \ (Both periods look about the same.) \ The |\it| font
%will not remain in force after the |\centerline|, but this is
%something of a coincidence: \TeX\ uses the braces to determine what
%text is to be centered, but then it removes the braces. The
%|\centerline| operation, as defined in Appendix~B\null, puts the
%resulting braceless text inside {\sl another\/} group; and that's why
%|\it| disappears after |\centerline|. \ (If you don't understand this,
%just don't risk leaving out braces in tricky situations, and you'll be OK.)
\exercise 下面这个呢?
\begintt
\centerline{This information should be \it centered.}
\endtt
\answer 你得到的结果与 `|\it centered|' 被包含在另一对花括号时的一样,
除了此时句号也用意大利体排版。(这两种句号看起来几乎一样。)
|\it| 字体在 |\centerline| 之后将不再生效,但这只是一个巧合:
\TeX\ 用花括号确定要居中显示的文本,而后将删除花括号。
在附录 B 定义的 |\centerline|操作,
将不带花括号的结果文本放入{\sl 另一个}编组中;
这就是 |\it| 在 |\centerline| 之后失效的原因。%
(如果你不能理解此事,在棘手的情况下就不要冒险去掉花括号,
这样你就不会有问题。)

%\smallskip
%\dangerexercise Define a control sequence |\ital| so that a user could type
%`|\ital{text}|' instead of `|{\it text\/}|'. Discuss the pros and cons of
%|\ital| versus |\it|.
%\answer |\def\ital#1{{\it#1\/}}|. \ Pro:~Users might find this easier to
%learn, because it works more like |\centerline| and they don't have to
%remember to make the italic correction. \ Con:~To avoid the italic correction
%just before a {\it comma} or {\it period}, users should probably be taught
%another control sequence; for example, with
%\begintt
%\def\nocorr{\kern0pt }
%\endtt
%a user could type `|\ital{comma} or \ital{period\nocorr},|'. The alternative
%of putting a period or comma in italics, to avoid the italic correction,
%doesn't look as good. A long sequence of italics would be inefficient for
%\TeX, since the entire text for the argument to |\ital| must be read into
%memory only to be scanned again.
\smallskip
\dangerexercise 定义一个控制系列 |\ital|, 使得用户用键入`|\ital{text}|'来%
代替`|{\it text\/}|'。与 |\it| 相比,看看 |\ital| 有什么优缺点。
\answer |\def\ital#1{{\it#1\/}}|。\
优点:用户会发现这简单易学,因为它用起来很像 |\centerline|,且无须记得要作倾斜校正。\
缺点:要避免在{\it 逗号}或{\it 句号}之前的倾斜校正,用户也许需要学会另一个控制系列;
比如利用
\begintt
\def\nocorr{\kern0pt }
\endtt
用户可以键入 `|\ital{comma} or \ital{period\nocorr},|'。
另一种方法是将句号或逗号也放入意大利体文本中,以避免倾斜校正,
但这看起来不够好。一大段的意大利体文本将降低 \TeX\ 的效率,
因为 |\ital| 的整个参量文本必须先读入内存才能进行再次扫描 。

%\ddanger Subsequent chapters describe many primitive operations of \TeX\ for
%which the locality of grouping is important. For example, if one of \TeX's
%internal parameters is changed within a group, the previous contents of
%that parameter will be restored when the group ends. Sometimes, however,
%it's desirable to make a definition that transcends its current group. This
%effect can be obtained by prefixing `^|\global|' to the definition. For
%example, \TeX\ keeps the current page number in a register called~|\count0|,
%and the routine that outputs a page wants to increase the
%^{page number}. ^{Output routines} are always protected by enclosing them
%in groups, so that they do not inadvertently mess up the rest of \TeX; but
%the change to |\count0| would disappear if it were kept local to the
%output group. The command
%\begintt
%\global\advance\count0 by 1
%\endtt
%^^|\advance|
%solves the problem; it increases |\count0| and makes this value stick around
%at the end of the output routine. In general, |\global| makes the immediately
%following definition pertain to all existing groups, not just to the
%innermost one.
\ddanger \1后面的章节规定了许多 \TeX\ 的原始操作,其中编组的局部性很重要。%
例如,当 \TeX\ 的一个内部参数在组中变化了,当组结束时,此参数会恢复为原先的内容。%
但是有时候,希望一个定义能不限在当前组中。%
这可以通过在定义上加前缀`|\global|'而实现。%
例如,\TeX\ 把当前的页码放在叫做 |\count0| 的寄存器中,
并且输出一页的程序要增大页数。%
总是通过把输出程序放在组中来保护它们,使得它们不会混淆 \TeX\ 的其它内容;
但是如果对输出组而言 |\count0| 是局部的,那么产生的变化就会被掩盖。%
命令
\begintt
\global\advance\count0 by 1
\endtt
就解决了这个问题;它增大 |count0|, 并且使其值在输出程序结束时保留下来。
一般地,|\global| 使其后紧接的定义适用于所有存在的组,而不仅仅是最内层的组。

%\ddangerexercise If you think you understand local and global definitions,
%here's a little test to make sure: Suppose |\c| stands for `|\count1=|',
%|\g| stands for `|\global\count1=|', and |\s| stands for
%`|\showthe\count1|'. What values will be shown?
%\begintt
%{\c1\s\g2{\s\c3\s\g4\s\c5\s}\s\c6\s}\s
%\endtt
%\answer |{1 {2 3 4 5} 4 6} 4|.
\ddangerexercise 如果你认为自己掌握了局部和整体的定义,这里有一个小测验:
设 |\c| 代表`|\count1=|', |\g| 代表`|\global\count1=|', |\s| 代表%
`|\showthe\count1|'。下面的结果是什么?
\begintt
{\c1\s\g2{\s\c3\s\g4\s\c5\s}\s\c6\s}\s
\endtt
\answer |{1 {2 3 4 5} 4 6} 4|.

%\ddanger Another way to obtain block structure with \TeX\ is to use the
%primitives ^|\begingroup| and ^|\endgroup|. These control sequences make it
%easy to begin a group within one control sequence and end it within
%another. The text that \TeX\ actually executes, after control sequences
%have been expanded, must have properly ^{nested groups}, i.e., groups that
%don't overlap. For example,
%\begintt
%{ \begingroup } \endgroup
%\endtt
%is not legitimate.
\ddanger 利用 \TeX\ 得到模块结构的另一种方法是用原始的 |\begingroup| 和 |\endgroup|。%
这些控制系列很容易在一个控制系列中开始一个组,并且在另一个控制系列中结束组。%
在控制系列展开后,\TeX\ 实际执行的文本必须是真正的嵌套组,即,组不能交叉。%
例如,
\begintt
{ \begingroup } \endgroup
\endtt
是不合法的。

%\ddangerexercise Define control sequences |\beginthe|\<block name> and
%|\endthe|\<block name> that provide a ``named'' block structure. In other
%words,
%\begintt
%\beginthe{beguine}\beginthe{waltz}\endthe{waltz}\endthe{beguine}
%\endtt
%should be permissible, but not
%\begintt
%\beginthe{beguine}\beginthe{waltz}\endthe{beguine}\endthe{waltz}.
%\endtt
%\answer |\def\beginthe#1{\begingroup\def\blockname{#1}}|\parbreak
%|\def\endthe#1{\def\test{#1}%|\parbreak
%|  \ifx\test\blockname\endgroup|\parbreak
%|  \else\errmessage{You should have said|\parbreak
%|    \string\endthe{\blockname}}\fi}|
\ddangerexercise 定义一个控制系列 |\beginthe|\<block name> 和 %
|\endthe|\<block name>, 它给出一个``命名''模块结构。
换句话说,
\begintt
\beginthe{beguine}\beginthe{waltz}\endthe{waltz}\endthe{beguine}
\endtt
是允许的,但是不允许
\begintt
\beginthe{beguine}\beginthe{waltz}\endthe{beguine}\endthe{waltz}.
\endtt
\answer |\def\beginthe#1{\begingroup\def\blockname{#1}}|\parbreak
|\def\endthe#1{\def\test{#1}%|\parbreak
|  \ifx\test\blockname\endgroup|\parbreak
|  \else\errmessage{You should have said|\parbreak
|    \string\endthe{\blockname}}\fi}|

\endchapter

I have had recourse to varieties of type,
and to braces.
\author JAMES ^{MUIRHEAD}, {\sl The Institutes of Gaius\/} (1880) % p. xii
% he actually said braces for what we call brackets

\bigskip

An encounter group is a gathering, for a few hours or a few days,
of twelve or eighteen personable, responsible, certifiably normal
and temporarily smelly people.
\author JANE ^{HOWARD}, {\sl Please Touch\/} (1970)

\vfill\eject\byebye
