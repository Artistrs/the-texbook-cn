% -*- coding: utf-8 -*-

\input macros

%\beginchapter Chapter 10. Dimensions
\beginchapter Chapter 10. 尺寸

\origpageno=57

%Sometimes you want to tell \TeX\ how big to make a space, or how wide to
%make a line. For example, the short story of Chapter~6 used the instruction
%`|\vskip .5cm|' to skip vertically by half a centimeter, and we also
%said `|\hsize=4in|' to specify a horizontal size of 4~inches. It's time now
%to consider the various ways such ^{dimensions} can be communicated to \TeX.
\1有时候,需要告诉 \TeX\ 产生多大的间距,或者生成一条多长的直线。%
例如,第六章简短的 stroy 中,使用了命令`|\vskip .5cm|'来在垂直方向跳过半厘米,
并且用`|\hsize=4in|'给出了宽度为 4 英寸的水平栏。%
现在,应该讨论一下把这些尺寸要求告诉 \TeX\ 的各种方法了。

%``^{Points}'' and ``^{picas}'' are the traditional units of measure for
%printers and compositors in English-speaking countries, so \TeX\
%understands points and picas. \TeX\ also understands inches and metric
%units, as well as the continental European versions of points and picas.
%Each unit of measure is given a two-letter abbreviation, as follows:
%^^{units of measure, table}
%$$\halign{\indent\tt#&\quad#\hfil\cr
%pt&point (baselines in this manual are $12\pt$ apart)\cr
%pc&pica ($\rm1\,pc=12\,pt$)\cr
%in&inch ($\rm1\,in=72.27\,pt$)\cr
%bp&big point ($\rm72\,bp=1\,in$)\cr
%cm&centimeter ($\rm2.54\,cm=1\,in$)\cr
%mm&millimeter ($\rm10\,mm=1\,cm$)\cr
%dd&didot point ($\rm1157\,dd=1238\,pt$)\cr
%cc&cicero ($\rm1\,cc=12\,dd$)\cr
%sp&scaled point ($\rm65536\,sp=1\,pt$)\cr}$$
%^^|pt|^^{point}
%^^|pc|^^{pica}
%^^|in|^^{inch}
%^^|bp|^^{big point}
%^^|cm|^^{centimeter}
%^^|mm|^^{millimeter}
%^^|dd|^^{didot point}^^{Didot, F. A.}
%^^|cc|^^{cicero}
%^^|sp|^^{scaled point}
%The output of \TeX\ is firmly grounded in the metric system, using the
%conversion factors shown here as exact ratios.
在使用英语的国家,``{points}''和``{picas}''是打印机和排版界的传统的度量单位,
所有 \TeX\ 认识 points 和 picas。%
\TeX\ 还认识英寸制和米制,以及欧洲大陆使用的 points 和 picas。%
每个度量单位用两个字母的缩写表示如下:
$$\halign{\indent\tt#&\quad#\hfil\cr
pt&point ({\hbox{本手册的基线之间距离是}}~$12\pt$)\cr
pc&pica ($\rm1\,pc=12\,pt$)\cr
in&inch ($\rm1\,in=72.27\,pt$)\cr
bp&big point ($\rm72\,bp=1\,in$)\cr
cm&centimeter ($\rm2.54\,cm=1\,in$)\cr
mm&millimeter ($\rm10\,mm=1\,cm$)\cr
dd&didot point ($\rm1157\,dd=1238\,pt$)\cr
cc&cicero ($\rm1\,cc=12\,dd$)\cr
sp&scaled point ($\rm65536\,sp=1\,pt$)\cr}$$
利用列在这里的转换因子作为精确的比例,\TeX\ 严格按照米制系统输出。

%\exercise How many points are there in 254 centimeters?
%\answer Exactly $7227\pt$.
\exercise 在 254 厘米中有多少 points?
\answer 正好 $7227\pt$。

%When you want to express some physical dimension to \TeX, type it as
%\begindisplay
%\<optional sign>\<number>\<unit of measure>\cr
%\noalign{\hbox{or}}
%\<optional sign>\<digit string>|.|\<digit string>\<unit of measure>\cr
%\enddisplay
%where an ^\<optional sign> is either a `|+|' or a `|-|' or nothing at all,
%and where a ^\<digit string> consists of zero or more consecutive
%decimal digits. The `|.|'\ can also be a `|,|'\null.
%For example, here are six typical dimensions:
%$$\halign{\indent#\hfil&\hskip 6em#\hfil\cr
%|3 in|&|29 pc|\cr
%|-.013837in|&|+ 42,1 dd|\cr
%|0.mm|&|123456789sp|\cr}$$
%A plus sign is redundant, but some people occasionally like extra
%redundancy once in a~while. Blank spaces are optional before the signs and the
%numbers and the units of measure, and you can also put an optional space
%after the dimension; but you should not put spaces within the digits
%of a number or between the letters of the unit of measure.
当你要告诉 \TeX\ 某些物理尺寸时,如下输入:
\begindisplay
\<optional sign>\<number>\<unit of measure>\cr
\noalign{\hbox{或者}}
\<optional sign>\<digit string>|.|\<digit string>\<unit of measure>\cr
\enddisplay
其中,\<optional sign> 可以是`|+|'或`|-|', 也可以什么也没有,
\<digit string> 由零或更多的连续的小数组成。%
`|.|'也可以是`|,|'。%
例如,下面是六个典型的尺寸:
$$\halign{\indent#\hfil&\hskip 6em#\hfil\cr
|3 in|&|29 pc|\cr
|-.013837in|&|+ 42,1 dd|\cr
|0.mm|&|123456789sp|\cr}$$
正号是多余的,多少有些人喜欢偶尔画蛇添足。%
在正负号,数字和测量单位前面的空格可有可无,
也可以在尺寸后面放一个空格;
但是在数字的数中间,或者在测量单位的字母之间不能有空格。

%\exercise Arrange those six ``typical dimensions'' into order,
%from smallest to largest.
%\answer $\rm-.013837\,in$, $\rm0.\,mm$, $\rm+42.1\,dd$, $\rm3\,in$,
%$\rm29\,pc$, $\rm123456789\,sp$.
%\ (The lines of text in this manual are 29~picas wide.)
\exercise 把上面的六个``典型的尺寸''从小到大排序。
\answer $\rm-.013837\,in$,$\rm0.\,mm$,$\rm+42.1\,dd$,$\rm3\,in$,
$\rm29\,pc$,$\rm123456789\,sp$。%
(本手册的文本行的宽度为29~picas。)

%\dangerexercise Two of the following three dimensions are legitimate
%according to \TeX's rules. Which two are they? What do they mean?
%Why is the other one incorrect?
%\begintt
%'.77pt
%"Ccc
%-,sp
%\endtt
%\answer The first is not allowed, since octal notation cannot be used with
%a decimal point. The second is, however, legal, since a \<number> can be
%hexadecimal according to the rule mentioned in Chapter~8; it means
%$\rm12\,cc$, which is $\rm144\,dd\approx154.08124\,pt$. The third is also
%accepted, since a \<digit string> can be empty; it is a complicated
%way to say $\rm0\,sp$.
\dangerexercise \1下面的三个尺寸有两个是符合 \TeX\ 的规则的。%
它们是哪两个?意思各是什么?
\begintt
'.77pt
"Ccc
-,sp
\endtt
\answer 第一个是不允许的,因为八进制表示法不能用在小数点之前。
第二个却是允许的,因为根据第 8 章介绍的规则,\<number> 可以为十六进制数;
它表示 $\rm12\,cc$,即 $\rm144\,dd\approx154.08124\,pt$。
第三个也是允许的,因为 \<digit string> 可以为空;
它是 $\rm0\,sp$ 的复杂说法。

%\smallskip
%The following ``rulers'' have been typeset by \TeX\ so that you can get
%some idea of how different units compare to each other. If no distortion
%has been introduced during the camera work and printing processes that
%have taken place after \TeX\ did its work, these rulers are highly accurate.
%$$ \abovedisplayskip 15pt plus 4pt minus 4pt
%\belowdisplayskip 15pt plus 4pt minus 4pt
%\vbox{
%\def\1{\vrule height 0pt depth 2pt}
%\def\2{\vrule height 0pt depth 4pt}
%\def\3{\vrule height 0pt depth 6pt}
%\def\4{\vrule height 0pt depth 8pt}
%\def\ruler#1#2#3{\leftline{$\vcenter{\hrule\hbox{\4#1}}\,\,\rm#2\,{#3}$}}
%\def\\#1{\hbox to .125in{\hfil#1}}
%\def\8{\\\1\\\2\\\1\\\3\\\1\\\2\\\1\\\4}
%\ruler{\8\8\8\8}4{in}
%\vskip 18pt
%\def\\#1{\hbox to 10pt{\hfil#1}}
%\def\8{\\\1\\\1\\\1\\\1\\\2\\\1\\\1\\\1\\\1\\\4}
%\ruler{\8\8\8}{300}{pt}
%\vskip 18pt
%\def\\#1{\hbox to 10dd{\hfil#1}}
%\def\8{\\\1\\\1\\\1\\\1\\\2\\\1\\\1\\\1\\\1\\\4}
%\ruler{\8\8\8}{300}{dd}
%\vskip 18pt
%\def\\#1{\hbox to 5mm{\hfil#1}}
%\def\8{\\\2\\\4}
%\ruler{\8\8\8\8\8\8\8\8\8\8}{10}{cm}
%\vskip 6pt}$$
\smallskip
下面的``标尺''已经被 \TeX\ 画出来了,这样你可以通过互相比较它们来%
体会一下它们的差\hbox{别。}%
如果 \TeX\ 输出结束后在打印过程中没有发生变形,这些标尺是非常精确的。%
$$ \abovedisplayskip 15pt plus 4pt minus 4pt
\belowdisplayskip 15pt plus 4pt minus 4pt
\vbox{
\def\1{\vrule height 0pt depth 2pt}
\def\2{\vrule height 0pt depth 4pt}
\def\3{\vrule height 0pt depth 6pt}
\def\4{\vrule height 0pt depth 8pt}
\def\ruler#1#2#3{\leftline{$\vcenter{\hrule\hbox{\4#1}}\,\,\rm#2\,{#3}$}}
\def\\#1{\hbox to .125in{\hfil#1}}
\def\8{\\\1\\\2\\\1\\\3\\\1\\\2\\\1\\\4}
\ruler{\8\8\8\8}4{in}
\vskip 18pt
\def\\#1{\hbox to 10pt{\hfil#1}}
\def\8{\\\1\\\1\\\1\\\1\\\2\\\1\\\1\\\1\\\1\\\4}
\ruler{\8\8\8}{300}{pt}
\vskip 18pt
\def\\#1{\hbox to 10dd{\hfil#1}}
\def\8{\\\1\\\1\\\1\\\1\\\2\\\1\\\1\\\1\\\1\\\4}
\ruler{\8\8\8}{300}{dd}
\vskip 18pt
\def\\#1{\hbox to 5mm{\hfil#1}}
\def\8{\\\2\\\4}
\ruler{\8\8\8\8\8\8\8\8\8\8}{10}{cm}
\vskip 6pt}$$

%\dangerexercise (To be worked after you know about boxes and glue and have
%read Chapter~21.) \ Explain how to typeset such a $\rm10\,cm$ ^{ruler},
%using \TeX.
%\answer {\obeylines|\def\tick#1{\vrule height 0pt depth #1pt}|
%|\def\\{\hbox to 1cm{\hfil\tick4\hfil\tick8}}|
%|\vbox{\hrule\hbox{\tick8\\\\\\\\\\\\\\\\\\\\}}|
%\noindent(You might also try putting ticks at every millimeter, in order %
%to see how good your system is; %
%some output devices can't handle 101~rules all at once.)}
\dangerexercise (在学会盒子和粘连以及看完第21章后再做。)
看看怎样用 \TeX\ 画出 $\rm10\,cm$ 的这样的标尺。
\answer {\obeylines|\def\tick#1{\vrule height 0pt depth #1pt}|
|\def\\{\hbox to 1cm{\hfil\tick4\hfil\tick8}}|
|\vbox{\hrule\hbox{\tick8\\\\\\\\\\\\\\\\\\\\}}|
\noindent (你也可以试着在各毫米处都标上刻度线,以看看你的系统的表现;%
有些输出设备不能同时处理全部 101 个标尺。)}

%\danger \TeX\ represents all dimensions internally as an integer multiple
%of the tiny units called sp. Since the wavelength of visible light is
%approximately $\rm100\,sp$, % in fact: violet=75sp, red=135sp!
%rounding errors of a few sp make no difference to the eye.
%However, \TeX\ does all of its arithmetic very carefully so that
%identical results will be obtained on different computers. Different
%implementations of \TeX\ will produce the same line breaks and the same
%page breaks when presented with the same document, because the integer
%arithmetic will be the same.
%^^{machine-independence} ^^{rounding}
\danger 在 \TeX\ 内部把所有尺寸表示为一个叫做 sp 的小单位的整数倍。%
因为可见光的波长近似等于 $\rm100\,sp$, 几个 sp 的误差眼睛是看不出来的。%
但是,\TeX\ 非常仔细地进行计算,使得在不同计算机上得到的结果相同。%
对同一文档,\TeX\ 的不同运行环境却能得到同样的断行和分页,
因为它所应用的整数算法是\hbox{相同的。}

%\danger The units have been defined here so that precise conversion to~sp
%is~efficient on a wide variety of machines. In order to achieve this,
%\TeX's ``pt'' has been made slightly larger than the official printer's
%point, which was defined to equal exactly $\rm.013837\,in$ by the American
%Typefounders Association in~1886 [cf.~National Bureau of Standards
%Circular~570 (1956)]. In fact, one classical point is exactly
%$.99999999\pt$, so the ``error'' is essentially one part in $10^8$.
%This is more than two orders of magnitude less than the amount by which
%the inch itself changed during 1959, when it shrank to $\rm2.54\,cm$ from
%its former value of $\rm(1/0.3937)\,cm$; so there is no point in worrying
%about the difference. The new definition $\rm72.27\,pt=1\,in$ is not only
%better for calculation, it~is also easier to remember.
\danger 在这里,给定单位是为了转换为 sp 时在各种计算机上的效率都很高。%
为此,\TeX\ 的``pt''比官方打印机的 point 略大,
在 1886 年,American
Typefounders Association 把point 精确地定义为 $\rm.013837\,in$%
[参见 National Bureau of Standards Circular~570 (1956)]。%
实际上,一个传统的 point 精确地等于 $.99999999\pt$, 因此``误差''最大在 $10^8$ 分%
之几。%
这个差比 1959 年英寸从先前的 $\rm(1/0.3937)\,cm$ 缩到 $\rm2.54\,cm$ 时减小的%
量小两个量级;所以不必在意这个差别。%
新定义 $\rm72.27\,pt=1\,in$ 不仅好计算,而且容易\hbox{记住。}

%\danger \TeX\ will not deal with dimensions whose absolute value is
%$\rm2^{30}\,sp$ or more. In other words, the ^{maximum legal dimension} is
%slightly less than $16384\pt$. This is a distance of about 18.892 feet
%(5.7583 meters), so it won't cramp your style.
\danger \TeX\ 不能处理绝对值大于等于$\rm2^{30}\,sp$的尺寸。%
换句话说,最大的可能的尺寸比 $16384\pt$ 略小。%
这大概是 18.892 feet(5.7583 米), 所以它是不会束缚你的。

%In a language manual like this it is convenient to use ``^{angle brackets}''
%in abbreviations for various constructions like \<number> and \<optional
%sign> and \<digit string>. Henceforth we shall use the term ^\<dimen> to
%stand for a legitimate \TeX\ dimension. For example,
%\begindisplay
%|\hsize=|\<dimen>
%\enddisplay
%will be the general way to define the column width that \TeX\ is supposed
%to use. The idea is that \<dimen> can be replaced by any quantity like
%`|4in|' that satisfies \TeX's grammatical rules for dimensions;
%abbreviations in angle brackets make it easy to state such laws of grammar.
\1在类似本书这样的编程的手册中,对像 \<number>、\<optional sign> 和
\<digit string> 这些指令使用``角括号''表示缩写是很方便的。
因此,我们将用 \<dimen> 表示 \TeX\ 允许的尺寸。例如,
\begindisplay
|\hsize=|\<dimen>
\enddisplay
就是 \TeX\ 设置的定义栏宽度的一般方法。意思就是,
\<dimen> 可以用任何像 `|4in|' 这样符合 \TeX\ 语法规则的尺寸的量来代替;
角括号中的缩写把这样的语法规律叙述得更清楚。

%When a dimension is zero, you have to specify a unit of measure even
%though the unit is irrelevant. Don't just say `|0|'\thinspace; say `|0pt|' or
%`|0in|' or something.
当尺寸为零时,你必须给出一个测量单位,即使它没什么关系。%
不能只使用`|0|'\thinspace; 应该使用`|0pt|', `|0in|'或其它东西。

%\smallbreak
%The 10-point size of type that you are now reading is normal in textbooks,
%but you probably will often find yourself wanting a larger font. Plain \TeX\
%makes it easy to do this by providing {\magnifiedfiverm ^{magnif{}ied
%output}.} If you say
%\begintt
%\magnification=1200
%\endtt
%at the beginning of your manuscript, everything will be enlarged by 20\%;
%i.e., it will come out at 1.2 times the normal size. Similarly,
%`|\magnification=2000|' doubles everything; this actually quadruples the area of
%each letter, since heights and widths are both doubled. To magnify a
%document by the factor $f$, you say ^|\magnification||=|\<number>, where
%the \<number> is 1000~times~$f$. This instruction must be given before the
%first page of output has been completed. You cannot apply two different
%magnifications to the same document.
\smallbreak
在本手册中你所看到的是 10-point 的大小,但是可能你会经常遇到要使用更大的字体的时候。
Plain \TeX\ 通过放大输出来很容易地实现它。如果你在文稿的开头规定
\begintt
\magnification=1200
\endtt
那么所有的内容都被放大 20\%;即变成正常尺寸的 1.2 倍大小。
类似地,`|\magnification=2000|' 把所有内容放大一倍;
这实际上把每个字符的面积变成四倍,因为高度和宽度都加倍了。
为了按照因子 $f$ 来放大文档,你可以规定 |\magnification||=|\<number>,
其中 \<number> 是 1000 乘以 $f$。这个指令必须在第一页输出完成之前给出。
在同一文档不能使用两个不同的放大率。

%Magnification has obvious advantages: You'll have less ^{eyestrain} when
%you're ^{proofreading}; you can easily make ^{transparencies} ^^{slides}
%for lectures; and you can photo-reduce magnified output, in order to minimize
%the deficiencies of a ^{low-resolution printer}. Conversely, you might
%even want `|\magnification=500|' in order to create a ^{pocket-size}
%version of some book. ^^{squint print} But there's a slight catch:
%You can't use magnification unless your printing device happens to have the
%fonts that you need at the magnification you desire. In other words, you need
%to find out what sizes are available before you can magnify. Most
%installations of \TeX\ make it possible to print all the fonts of plain
%\TeX\ if you magnify by ^|\magstep||0|,~|1|, |2|,~|3|, and perhaps~|4| or
%even~|5| (see Chapter~4); but the use of large fonts can be expensive
%because a lot of system memory space is often required to store the shapes.
放大功能有一个明显的好处:
当你在校对时,眼睛不会感到太累;
可以简单地得到演讲用的透明胶片;
还可以弥补低分辨率打印机对输出图像的造成的损失。%
反过来,为了把某些书变成口袋大小,你可能要用`|\magnification=500|'。%
但是有个条件:
如果你的打印设备凑巧没有你所要放大的字体,那么就不能使用放大功能。%
只要你按照 |\magstep||0|,~|1|, |2|,~|3|, 甚至是 |4| 和 |5|(见第四章)来放大,
那么可能大部分安装好的 \TeX\ 都能打印所有 plain \TeX\ 的字体;
但是使用大字体可能花费比较大,因为通常需要许多内存空间来储存字形。

%\exercise Try printing the short story of Chapter 6 at 1.2, 1.44, and 1.728
%times the normal size. What should you type to get \TeX\ to do this?
%\answer For example, say `|\magnification=\magstep1 \input story \end|'
%to get magnification 1200; |\magstep2| and |\magstep3| are 1440 and 1728.
%Three separate runs are needed, since there can be at most one
%magnification per job. The output may look funny if the fonts don't exist
%at the stated magnifications.
\exercise 试试按照正常尺寸的 1.2,1.44 和 1.728 倍来输出第 6 章的 story。%
应该怎样做才可以?
\answer 例如,键入 `|\magnification=\magstep1 \input story \end|'
就可以得到 1200 的放大因子;而|\magstep2| 和 |\magstep3| 分别为 1440 和 1728。
你需要分别运行三次,因为每个任务只能有一个放大率。
如果指定放大率的字体不存在,输出结果看起来将很可笑。

%\danger When you say |\magnification=2000|, an operation like
%`|\vskip.5cm|' will actually skip $\rm1.0\,cm$ of space in the final
%document. If you want to specify a dimension in terms of the final size,
%\TeX\ allows you to say `^|true|' just before |pt|, |pc|, |in|, |bp|,
%|cm|, |mm|, |dd|, |cc|, and |sp|.  This unmagnifies the units, so that the
%subsequent magnification will cancel out. For example, `|\vskip.5truecm|'
%is equivalent to `|\vskip.25cm|' if you have previously said
%`|\magnification=2000|'. Plain \TeX\ uses this feature in the
%|\magnification| command itself: Appendix~B includes the instruction
%\begintt
%\hsize = 6.5 true in
%\endtt
%just after a new magnification has taken effect. This adjusts the line width
%so that the material on each page will be $6{1\over2}$ inches wide when it
%is finally printed, regardless of the magnification factor.
%There will be an inch of margin at both left and right,
%assuming that the paper is $8{1\over2}$ inches wide.
\danger 当你规定了 |\magnification=2000| 后,象`|\vskip.5cm|'这样的结果实际上%
在最后的文档中是跳过了 $\rm1.0\,cm$ 的空白。%
如果你要给出一个不变的尺寸,\TeX\ 允许在 |pt|, |pc|, |in|, |bp|,
|cm|, |mm|, |dd|, |cc| 和 |sp| 紧前面加上`|true|'。%
\1这是不被放大的尺寸,
使得后面的放大被取消。%
例如,如果你在前面规定了`|\magnification=2000|', 那么`|\vskip.5truecm|'%
等价于`|\vskip.25cm|'。%
Plain \TeX\ 在 |\magnification| 命令自己中使用了这个特性:
附录 B 就在新的放大指令起作用后,包括了指令
\begintt
\hsize = 6.5 true in
\endtt
它调整行的宽度,使得最后输出时,在每页的内容都是 $6{1\over2}$ 英寸宽,
而与放大因子无关。%
设定纸的宽度为 $8{1\over2}$ 英寸,那么在左右各有 1 英寸的边界。

%\danger If you use no `|true|' dimensions, \TeX's internal computations are not
%affected by the presence or absence of magnification; line breaks and page
%breaks will be the same, and the ^|dvi| file will change in only two places.
%\TeX\ simply tells the printing routine that you want a certain magnification,
%and the printing routine will do the actual enlargement when it reads the
%|dvi| file.
\danger 如果你没有使用`|true|'尺寸,那么 \TeX\ 的内部运算不会因为放大的出现或%
不存在而受到影响;
断行和分页是一样的,并且 |dvi| 文件只改变两个地方。%
\TeX\ 将直接告诉打印机你想要的放大率,当读入 |dvi| 文件时,
打印机将给出实际的放大输出。

%\dangerexercise Chapter 4 mentions that fonts of different magnifications
%can be used in the same job, by loading them `^|at|' different sizes.
%Explain what fonts will be used when you give the commands
%^^{magnified fonts} ^^|scaled|
%\begintt
%\magnification=\magstep1
%\font\first=cmr10 scaled\magstep1
%\font\second=cmr10 at 12truept
%\endtt
%\answer Magnification is by a factor of 1.2. Since font |\first| is |cmr10|
%at $12\pt$, it will be |cmr10| at $14.4\pt$ after magnification;
%font |\second| will be |cmr10| at $12\pt$. \ (\TeX\ changes
%`|12truept|' into `|10pt|', and the final output magnifies it back to
%$12\pt$.)
\dangerexercise 第 4 章提到,通过 `|at|' 不同大小载入字体,
不同放大率的字体可以在同一文档中使用。%
当你给出下列命令时,看看使用的是什么字体:
\begintt
\magnification=\magstep1
\font\first=cmr10 scaled\magstep1
\font\second=cmr10 at 12truept
\endtt
\answer 放大率为 1.2。字体 |\first| 为 |cmr10|
at $12\pt$,在放大后它将变为 |cmr10| at $14.4\pt$;
而字体 |\second| 将变为 |cmr10| at $12\pt$。%
(\TeX\ 将 `|12truept|' 改成 `|10pt|',
因此在最终输出中它将变回 $12\pt$。)

%\ddanger Magnification is actually governed by \TeX's ^|\mag| primitive,
%which is an integer parameter that should be positive and at~most~32768.
%The value of\/ |\mag| is examined in three cases: (1)~just before the
%first page is shipped to the |dvi| file; (2)~when computing a |true|
%dimension; (3)~when the |dvi| file is being closed. Alternatively,
%some implementations of \TeX\ produce non-|dvi| output; they examine
%|\mag| in case~(2) and also when shipping out each page. Since each
%document has only one magnification, the value of\/ |\mag| must not change
%after it has first been examined.
\ddanger 放大功能实际上是由 \TeX\ 的原始控制系列 |\mag| 所控制,
它是一个整数参数,应当是正数且最大为 32768。%
|\mag| 的值在三种情况下被用到:
(1) 在第一页被输出到 |dvi| 文件之前;
(2) 当计算一个 |true| 尺寸时;
(3) 当 |dvi| 文件结束时。%
另外,有些 \TeX\ 的输出不是 |dvi| 文件;
这时在情况(2)下并且当输出到每页时要用到 |\mag|。%
因为每个文档只有一个放大率,所以一旦 |\mag| 的值第一次被用到后,
就不能\hbox{改变了。}

%\danger \TeX\ also recognizes two units of measure that are relative
%rather than absolute; i.e., they depend on the current context:
%\begindisplay
%^|em| is the width of a ``^{quad}'' in the current font;\cr
%^|ex| is the ``^{x-height}'' of the current font.\cr
%\enddisplay
%Each font defines its own em and ex values. In olden days, an ``em'' was
%the width of an `M', but this is no longer true; ems are simply arbitrary
%units that come with a font, and so are exes.  The Computer Modern fonts
%have the property that an em-dash is one em wide, each of the ^{digits} 0
%to~9 is half an em wide, and lowercase `x' is one ex high; but these are
%not hard-and-fast rules for all fonts.
%The |\rm| font (^|cmr10|) of plain \TeX\ has $\rm1\,em=10\,pt$
%and $\rm1\,ex\approx4.3\,pt$; the |\bf| font (^|cmbx10|) has
%$\rm1\,em=11.5\,pt$ and $\rm1\,ex\approx4.44\,pt$; and the |\tt| font
%(^|cmtt10|) has $\rm1\,em=10.5\,pt$ and $\rm1\,ex\approx4.3\,pt$. All of
%these are ``10-point'' fonts, yet they have different em and ex values.
%It~is generally best to use |em| for horizontal measurements and |ex| for
%vertical measurements that depend on the current font.
\danger \TeX\ 还有两个相对而不是绝对的测量单位;
即,它们与当前的上下文有关:
\begindisplay
|em| 是当前字体的一个``{quad}''的宽度;\cr
|ex| 是当前字体的``{x-height}''。\cr
\enddisplay
每个字体都有它自己的 em 和 ex 值。%
在过去,``em''是`M'的宽度,但是现在不是了;
em 是直接来自字体的任意单位,ex 也是。%
对 Computer Modern 字体,em 破折号的宽度是一个 em,
每个数字 0 到 9 的宽度是半个 em, 小写`x'的高度是一个 ex;
但是不是对所有字体都这样。%
Plain \TeX\ 的 |\rm| 字体(|cmr10|)为 $\rm1\,em=10\,pt$, $\rm1\,ex\approx4.3\,pt$;
|\bf| 字体(|cmbx10|)为 $\rm1\,em=11.5\,pt$ 和 $\rm1\,ex\approx4.44\,pt$;
|\tt| 字体(|cmtt10|)为 $\rm1\,em=10.5\,pt$ 和 $\rm1\,ex\approx4.3\,pt$。%
它们都是``10-point''字体,但是 em 和 ex 值却不同。%
对当前字体的水平距离最好用 |em|, 对垂直距离最好用 |ex|。

%\danger A \<dimen> can also refer to \TeX's internal registers or parameters.
%We shall discuss registers later, and a complete definition of everything that
%a ^\<dimen> can be will be given in Chapter~24. For now it will suffice to
%give some hints about what is to come:
%`|\hsize|' stands for the current horizontal line size,
%and `|.5\hsize|' is half that amount;
%`|2\wd3|' denotes twice the width of register~|\box3|;
%`|-\dimen100|' is the negative of register~|\dimen100|.
\danger \1\<dimen> 也能用 \TeX\ 的内部寄存器或参数。%
我们在后面将讨论寄存器,\<dimen> 的所有内容的完整定义在第二十四章给出。%
`|\hsize|'是当前行的宽度,那么`|.5\hsize|'是那个量的一半;
`|2\wd3|'表示寄存器 |\box3| 的两倍;
`|-\dimen100|'是寄存器 |\dimen100| 的负值。

%\ddanger Notice that the unit names in dimensions
%are not preceded by backslashes. The same is true of other so-called
%^{keywords} of the \TeX\ language. Keywords can be given in uppercase letters
%or in a mixture of upper and lower case; e.g., `|Pt|' is equivalent to `|pt|'.
%The category codes of these letters are irrelevant; you may, for example,
%be using a |p| of category~12 (other) that was generated by expanding
%`|\the\hsize|' as explained in Chapter~20.
%\TeX\ gives a special interpretation to keywords only when they
%appear in certain very restricted contexts. For example, `|pt|' is a
%keyword only when it appears after a number in a \<dimen>;
%`|at|' is a keyword only when it appears after the external name of a
%font in a |\font| declaration.
%Here is a complete list of \TeX's keywords, in case you are wondering about
%the full set: |at|, |bp|, |by|, |cc|, |cm|, |dd|, |depth|, |em|, |ex|,
%|fil|, |height|, |in|, |l|, |minus|, |mm|, |mu|, |pc|, |plus|,
%|pt|, |scaled|, |sp|, |spread|, |to|, |true|, |width|. ^^{reserved words}
%\ (See Appendix~I for references to the contexts in which each of these is
%recognized as a keyword.)
\ddanger 注意,在尺寸中,单位的名称前面没有反斜线。%
\TeX\ 语言的其它所谓关键词也是如此。%
关键词可以用大写或大写小写混合起来;
比如,`|Pt|'等价于`|pt|'。%
这些字母的类代码无关;
例如你可以正在用第 12 类(其它字符)的 |p|, 正如第二十章讨论的那样所展开的%
`|\the\hsize|'生成。%
只要当关键词出现在某些非常有限的上下文中时,\TeX\ 才给出特殊解释。%
例如,只有当出现在 \<dimen> 中的数字后面时`|pt|'才是关键词;
只有当出现在 |\font| 声明中的字体外部名字后面时,`|at|'才是关键词。%
如果你想知道所有的关键词,下面就是 \TeX\ 的全部关键词的列表:
|at|, |bp|, |by|, |cc|, |cm|, |dd|, |depth|, |em|, |ex|,
|fil|, |height|, |in|, |l|, |minus|, |mm|, |mu|, |pc|, |plus|,
|pt|, |scaled|, |sp|, |spread|, |to|, |true|, |width|。%
(至于在哪些情况下它们被看作关键词,见附录 I 给出的参考。)

\endchapter

The methods that have hitherto been taken
to discover the measure of the Roman foot,
will, upon examination, be found so unsatisfactory, that
it is no wonder the learned are not yet agreed on that point.
$\ldots$
9 London inches are equal to 8,447 Paris inches.
\author MATTHEW ^{RAPER}, in {\sl Philosophical Transactions\/} (1760)
% ``An Enquiry into the Measure of the {\sl Roman\/} Foot,''
% {\sl Philos.\ Trans.\ \bf51} (1760), 774--823.

\bigskip

%\checkequals\sesame\pageno %
Without the letter U,
units would be nits.
\author ^{SESAME STREET}{^^{Children's Television Workshop}} (1970)

\vfill\eject\byebye
