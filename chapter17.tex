% -*- coding: utf-8 -*-

\input macros

%\beginchapter Chapter 17. More about Math
\beginchapter Chapter 17. 数学排版进阶

\origpageno=139

%Another thing mathematicians like to do is make fractions---and they
%like to build symbols up on top of each other in a variety of different ways:
%\begindisplay
%$\displaystyle
%{1\over2}\qquad{\rm and}\qquad{n+1\over3}\qquad{\rm and}\qquad
%{n+1\choose3}\qquad{\rm and}\qquad\sum_{n=1}^3 Z_n^2\,.$
%\enddisplay
%You can get these four formulas as displayed equations by typing
%`|$$1\over2$$|' and
%`|$$n+1\over3$$|' and
%`|$$n+1\choose3$$|' and
%`|$$\sum_{n=1}^3 Z_n^2$$|';
%we shall study the simple rules for such constructions in this chapter.
%^^|\sum|^^|\choose|
\1数学家爱做的另一件事就是构造分数——并且喜欢把用各种各样的方法%
把符号放在其它符号上面:
\begindisplay
$\displaystyle
{1\over2}\qquad{\hbox{\ST{10}和}}\qquad{n+1\over3}\qquad{\hbox{\ST{10}和}}\qquad
{n+1\choose3}\qquad{\hbox{\ST{10}和}}\qquad\sum_{n=1}^3 Z_n^2\,.$
\enddisplay
要把它们变成陈列方程,你可以输入 `|$$1\over2$$|' 和 `|$$n+1\over3$$|'
和 `|$$n+1\choose3$$|' 和 `|$$\sum_{n=1}^3 Z_n^2$$|';
本章我们将讨论这些构造的简单规则。

%First let's look at ^{fractions}, which use the `^|\over|' notation. The
%control sequence |\over| applies to everything in the formula unless you
%use braces to enclose it in a specific subformula; in the latter
%^^{stacked fractions, see over}
%case, |\over| applies to everything in that subformula.
%\begindisplaymathdemo
%\it Input&\it Output\cr
%\noalign{\vskip-3pt}
%|$$x+y^2\over k+1$$|&x+y^2\over k+1\cr
%\noalign{\vskip2pt}
%|$${x+y^2\over k}+1$$|&{x+y^2\over k}+1\cr
%\noalign{\vskip-1pt}
%|$$x+{y^2\over k}+1$$|&x+{y^2\over k}+1\cr
%\noalign{\vskip-1pt}
%|$$x+{y^2\over k+1}$$|&x+{y^2\over k+1}\cr
%\noalign{\vskip-3pt}
%|$$x+y^{2\over k+1}$$|&x+y^{2\over k+1}\cr
%\endmathdemo
%You aren't allowed to use |\over| twice in the same subformula; instead of
%typing something like
%`|a \over b \over 2|', you must specify what goes over what:
%\begindisplaymathdemo
%\noalign{\vskip3pt}
%|$${a\over b}\over 2$$|&{a\over b}\over 2\cr
%|$$a\over{b\over 2}$$|&a\over{b\over 2}\cr
%\endmathdemo
%Unfortunately, both of these alternatives look pretty awful. Mathematicians
%tend to ``overuse'' |\over| when they first begin to typeset their own work
%on a system like \TeX. A good typist or copy editor will convert fractions
%to a ``^{slashed form},'' whenever a built-up construction would be too
%small or too crowded. For example, the last two cases should be treated
%as follows:
%\begindisplaymathdemo
%\noalign{\vskip3pt}
%|$$a/b \over 2$$|&a/b \over 2\cr
%|$$a \over b/2$$|&a \over b/2\cr
%\endmathdemo
%Conversion to slashed form takes a little bit of mathematical knowhow, since
%^{parentheses} sometimes need to be inserted in order to preserve the meaning
%of the formula. Besides substituting `|/|' for~`|\over|', the two parts
%of the fraction should be put in parentheses unless they are single
%symbols; for example, $a\over b$~becomes simply~$a/b$, but
%$a+1\over b$ becomes $(a+1)/b$, and $a+1\over b+1$ becomes
%${(a+1)/(b+1)}$. Furthermore, the entire fraction should generally
%be enclosed in parentheses if it appears next to something else;
%for example, ${a\over b}x$ becomes $(a/b)x$. If you are a typist without
%mathematical training, it's best to ask the author of the manuscript
%for help, in doubtful cases; you might also tactfully suggest that
%unsightly fractions be avoided altogether in future manuscripts.
首先我们讨论分数,它用到命令`|\over|'。%
控制系列 |\over| 要作用到公式中的所有内容,除非你把它用大括号封装在一个规定%
的子公式中;
在后一种情况下,|\over| 只作用于那个子公式的所有内容。
\begindisplaymathdemo
{\KT{10}输入}&{\hbox{\KT{10}输出}}\cr
\noalign{\vskip-3pt}
|$$x+y^2\over k+1$$|&x+y^2\over k+1\cr
\noalign{\vskip2pt}
|$${x+y^2\over k}+1$$|&{x+y^2\over k}+1\cr
\noalign{\vskip-1pt}
|$$x+{y^2\over k}+1$$|&x+{y^2\over k}+1\cr
\noalign{\vskip-1pt}
|$$x+{y^2\over k+1}$$|&x+{y^2\over k+1}\cr
\noalign{\vskip-3pt}
|$$x+y^{2\over k+1}$$|&x+y^{2\over k+1}\cr
\endmathdemo
不允许在同一子公式中使用两次 |\over|;
不能输入象`|a \over b \over 2|'这样的内容,必须给出 over 作用的范围:
\begindisplaymathdemo
\noalign{\vskip3pt}
|$${a\over b}\over 2$$|&{a\over b}\over 2\cr
|$$a\over{b\over 2}$$|&a\over{b\over 2}\cr
\endmathdemo
不幸的是,这两种方法看起来都很别扭。%
在数学家开始用 \TeX\ 排版时,总爱``过度使用''~|over|。%
只要当所构建的东西太小或太拥挤时,好的排版者或编辑就把分数变成``除式''。%
例如,最后两种情况可以写作:
\begindisplaymathdemo
\noalign{\vskip3pt}
|$$a/b \over 2$$|&a/b \over 2\cr
|$$a \over b/2$$|&a \over b/2\cr
\endmathdemo
转换到除式需要一点数学常识,
因为为了不改变公式的意思,有时候要插入圆括号。%
\1除了用`|/|'代替`|\over|'外,分数的分子和分母应该放在括号中,除非它们是单个字符;
例如,~$a\over b$ 就直接变成 $a/b$,
但是 $a+1\over b$ 要变成 $(a+1)/b$, $a+1\over b+1$ 要变成 ${(a+1)/(b+1)}$。%
还有,如果分数的前面有东西,那么整个分数应放在括号中;
例如,${a\over b}x$ 变成 $(a/b)x$。%
作为没有数学常识的排版者,在不清楚时应该询问作者;
也可以巧妙地建议在以后的文稿中尽量不出现不好看的分数。

%\exercise What's a better way to render the formula $x+y^{2\over k+1}$?
%\answer $x+y^{2/(k+1)}$\quad(|$x+y^{2/(k+1)}$|).
\exercise 怎样更好地排版出公式$x+y^{2\over k+1}$?
\answer $x+y^{2/(k+1)}$(|$x+y^{2/(k+1)}$|)。

%\exercise Convert `${a+1\over b+1}x$' to slashed form.
%\answer $((a+1)/(b+1))x$\quad(|$((a+1)/(b+1))x$|).
\exercise 把`${a+1\over b+1}x$'转换为除式。
\answer $((a+1)/(b+1))x$(|$((a+1)/(b+1))x$|)。

%\exercise What surprise did B. L. ^{User} get when he typed `|$$x = (y^2\over
%k+1)$$|'\thinspace?
%\answer He got the displayed formula$$x=(y^2\over k+1)$$ because he forgot
%that an unconfined |\over| applies to everything.  \ (He should probably
%have typed `|$$x=\left(y^2\over k+1\right)$$|', using ideas that will be
%presented later in this chapter; this not only makes the parentheses
%larger, it keeps the `$x=$' out of the fraction, because |\left| and
%|\right| introduce subformulas.)
\exercise 当^{用户笨笨}输入`|$$x = (y^2\over k+1)$$|'时,他将得到什么?
\answer 因为忘记了无约束的 |\over| 将应用到整个公式,
他将得到陈列公式 $$x=(y^2\over k+1)$$
(利用本章后面将介绍的思想,
他也许应该键入 `|$$x=\left(y^2\over k+1\right)$$|';这不仅让圆括号变大,
而且也让 `$x=$' 离开该分式,因为 |\left| 和 |\right| 引入一个子公式。)

%\def\cents{\hbox{\rm\rlap/c}}
%\exercise How can you make `$7{1\over2}\cents$'? \ (Assume that
%the control sequence |\cents| yields~`$\cents$'.)^^{money}^^{cents}
%\answer `|$7{1\over2}\cents$|' or `|7$1\over2$\cents|'. \ (Incidentally,
%the definition used here was |\def\cents{\hbox{\rm\rlap/c}}|.)
%^^|\rlap|^^|\cents|
\def\cents{\hbox{\rm\rlap/c}}
\exercise 怎样得到`$7{1\over2}\cents$'?%
(假定控制系列 |\cents| 得到的就是`$\cents$'。)
\answer `|$7{1\over2}\cents$|' 或 `|7$1\over2$\cents|'。%
(顺便说一下,这里用到的定义是 |\def\cents{\hbox{\rm\rlap/c}}|。)
^^|\rlap|^^|\cents|

%The examples above show that letters and other symbols sometimes get
%smaller when they appear in fractions, just as they get smaller when they
%are used as exponents. It's about time that we studied \TeX's method for
%choosing the sizes of things. \TeX\ actually has eight different
%^{styles} in which it can treat formulas, namely
%$$\halign{\indent#\hfil\quad&#\hfil\cr
%display style&(for formulas displayed on lines by themselves)\cr
%text style&(for formulas embedded in the text)\cr
%script style&(for formulas used as superscripts or subscripts)\cr
%scriptscript style&(for second-order superscripts or subscripts)\cr}$$
%^^{display style}^^{text style}^^{script style}^^{scriptscript style}
%and four other ``^{cramped}'' styles that are almost the same except that
%exponents aren't raised quite so much. For brevity we shall refer to the
%eight styles as
%\begindisplay
%$\displaystyle D,\ D',\ T,\ T',\ S,\ S',\ \SS,\ \SS',$
%\enddisplay
%where $D$ is display style, $D'$ is cramped display style, $T$~is text style,
%etc. \TeX\ also uses three different ^{sizes of type for mathematics};
%they are called ^{text size}, ^{script size}, and ^{scriptscript size}.
上面的例子表明,当字母和其它符号出现在分数中时,有时候会变得更小,
就象把它们放在指数上那么小。%
现在我们应该讨论一下 \TeX\ 怎样来选择符号的大小。%
当处理公式时, \TeX\ 实际上有八种不同的样式,即,
$$\halign{\indent#\hfil\quad&#\hfil\cr
陈列样式&(用在行中单独的陈列公式中)\cr
文本样式&(用在嵌入文本的公式中)\cr
标号样式&(用于公式的上下标)\cr
小标号样式&(用于公式的二阶上下标)\cr}$$
以及四种其它的``近似''样式,它们与上面四种几乎一样,只是指数升高得不那么多。%
为了简化,我们用
\begindisplay
$\displaystyle D,\ D',\ T,\ T',\ S,\ S',\ \SS,\ \SS',$
\enddisplay
来表示这八种样式,
其中 $D$ 是陈列样式,~$D'$ 是近似的陈列样式,
~$T$ 是文本样式等等。%
 \TeX\ 还使用数学字体的三种不同大小,分别叫做文本尺寸,标号尺寸,小标号尺寸。

%The normal way to typeset a formula with \TeX\ is to enclose it in dollar
%signs |$|$\,\ldots\,$|$|; this yields the formula in text style
%(style~$T$). Or you can enclose it in double dollar signs |$$|$\,\ldots\,$|$$|;
%this displays the formula in display style (style~$D$). The subformulas of
%a formula might, of course, be in different styles. Once you know
%the style, you can determine the size of type that \TeX\ will use:
%$$\everycr{\noalign{\penalty10000}}
%\halign{\indent#\hfil\qquad&#\hfil&\quad#\llap(like this)\hfil\cr
%If a letter is in style&then it will be set in\cr
%\noalign{\vskip 2pt}
%$D,D',T,T'$&text size&\cr
%$S,S'$&script size&\sevenrm\cr
%$\SS,\SS'$&scriptscript size&\fiverm\cr}$$
%There is no ``$\it SSS$'' style or ``scriptscriptscript'' size; such tiny
%symbols would be even less readable than the scriptscript ones. Therefore
%\TeX\ stays with scriptscript size as the minimum:
%$$\halign{\indent\hbox to 1.3in{#\hfil}&\hbox to 1.2in{#\hfil}&#\hfil\cr
%In a formula&the superscript&and the subscript\cr
%of style&style is&style is\cr
%\noalign{\vskip 2pt}
%$D,T$&$S$&$S'$\cr
%$D',T'$&$S'$&$S'$\cr
%$S,\SS$&$\SS$&$\SS'$\cr
%$S',\SS'$&$\SS'$&$\SS'$\cr}$$
%For example, if |x^{a_b}| is to be typeset in style $D$, then |a_b| will
%be set in style~$S$, and {\tt b}~in style~$\SS'$; the result is
%`$\displaystyle x^{a_b}$'.
用 \TeX\ 排版公式的正常方法是把它封装在符号 |$|$\,\ldots\,$|$| 中;
这样得到的是文本样式(样式 $T$)。或者封装在符号 |$$|$\,\ldots\,$|$$| 中,
这样得到的是陈列样式(样式 $D$)。当然公式的子公式使用的可能是不同的样式。
一旦知道了样式,就可以确定 \TeX\ 要用的字体的大小:
$$\everycr{\noalign{\penalty10000}}
\halign{\indent#\hfil\qquad&#\hfil&\quad#\llap(like this)\hfil\cr
如果字母的样式是&那么设定的大小为\cr
\noalign{\vskip 2pt}
$D,D',T,T'$&文本尺寸&\cr
$S,S'$&标号尺寸&\sevenrm\cr
$\SS,\SS'$&小标号尺寸&\fiverm\cr}$$
\1没有 ``$\it SSS$'' 样式或者``小小标号''尺寸;
这样小的符号比小标号样式更难看清。因此 \TeX\ 把小标号尺寸作为最小的:
$$\halign{\indent\hbox to 1.3in{#\hfil}&\hbox to 1.2in{#\hfil}&#\hfil\cr
公式的样式&上标的样式&下标的样式\cr
\noalign{\vskip 2pt}
$D,T$&$S$&$S'$\cr
$D',T'$&$S'$&$S'$\cr
$S,\SS$&$\SS$&$\SS'$\cr
$S',\SS'$&$\SS'$&$\SS'$\cr}$$
例如,如果 |x^{a_b}| 用样式 $D$ 排版,那么 |a_b| 就用样式 $S$,
而 {\tt b} 就用样式 $\SS'$;结果为:
`$\displaystyle x^{a_b}$'。

%So far we haven't seen any difference between styles $D$ and $T$. Actually
%there is a slight difference in the positioning of exponents, although
%script size is used in each case: You get
%$\displaystyle x^2$~in $D$~style and $x^2$~in $T$~style and \vbox to 0pt{
%\vss\hbox{$\displaystyle{\atop x^2}$}\kern0pt}~in $D'$ or $T'$~style---do
%you see the difference? But there is a big distinction between $D$ style and
%$T$ style when it comes to fractions:
%$$\halign{\indent\hbox to 1.3in{#\hfil}&\hbox to 1.2in{#\hfil}&#\hfil\cr
%In a formula&the style of the&and the style of the\cr
%$\alpha$|\over|$\,\beta$ of style&numerator $\alpha$ is&denominator
%$\beta$ is\cr
%\noalign{\vskip 2pt}
%$D$&$T$&$T'$\cr
%$D'$&$T'$&$T'$\cr
%$T$&$S$&$S'$\cr
%$T'$&$S'$&$S'$\cr
%$S,\SS$&$\SS$&$\SS'$\cr
%$S',\SS'$&$\SS'$&$\SS'$\cr}$$
%^^{numerator}^^{denominator}
%Thus if you type `|$1\over2$|' (in a text) you get $1\over2$, namely style
%$S$ over style~$S'$; but if you type
%`|$$1\over2$$|' you get $$1\over2$$ (a displayed formula), which is style
%$T$ over style $T'$.
现在我们还没发现样式 $D$ 和样式 $T$ 之间的区别。
实际上,虽然两种情况下指数都用标号尺寸,但是其位置有点不同:
看看样式 $D$ 中的 $\displaystyle x^2$ 和 $T$ 中的 $x^2$ 以及
$D'$ 或 $T'$ 中的 \vbox to 0pt{
\vss\hbox{$\displaystyle{\atop x^2}$}\kern0pt} 就明白了。
但是当处理分数时,样式 $D$ 和样式 $T$ 之间有一个明显的差别:
$$\halign{\indent\hbox to 1.8in{#\hfil}&\hbox to 1.2in{#\hfil}&#\hfil\cr
公式 $\alpha$|\over|$\,\beta$ 的样式&分子的样式&分母的样式\cr
\noalign{\vskip 2pt}
$D$&$T$&$T'$\cr
$D'$&$T'$&$T'$\cr
$T$&$S$&$S'$\cr
$T'$&$S'$&$S'$\cr
$S,\SS$&$\SS$&$\SS'$\cr
$S',\SS'$&$\SS'$&$\SS'$\cr}$$
因此,如果在文本中输入 `|$1\over2$|' 那么得到的是 $1\over2$,
即样式 $S$ 在样式 $S'$ 之上;但是如果输入 `|$$1\over2$$|' 就得到陈列公式
$$1\over2$$
其中样式 $T$ 在样式 $T'$ 之上。

%\danger While we're at it, we might as well finish the style rules:
%^|\underline| does not change the style. ^{Math accents}, and the operations
%^|\sqrt| and ^|\overline|, change uncramped styles to their cramped
%counterparts; for example, $D$ changes to $D'$, but $D'$ stays as it was.
\danger 在这里我们最好还是给出全部的样式规则:
|\underline| 不改变样式。数学重音和 |\sqrt| 运算与 |\overline|
把非近似的样式变成相应的近似样式;例如,$D$ 变成 $D'$,但是 $D'$ 保持不变。

%\dangerexercise State the style and size of each part of the formula
%$\displaystyle \sqrt{p_2^{e'}}$, assuming that the formula itself is in
%style~$D$.
%\answer Style $D'$ is used for the subformula $p_2^{e'}$, hence style~$S'$
%is used for the superscript~$e'$ and the subscript~2, and style~$\SS'$
%is used for the supersuperscript prime. The square root sign and the $p$
%appear in text size; the 2 and the~$e$ appear in script size; and the
%$\prime$ is in scriptscript size.
\dangerexercise 假定公式 $\displaystyle \sqrt{p_2^{e'}}$ 的样式是 $D$,
给出其每一部分的样式和大小。
\answer 子公式 $p_2^{e'}$ 的样式为 $D'$,从而上标 $e'$ 和下标 2 的样式为 $S'$,
而上上标撇号的样式为 $\SS'$。平方根符号和 $p$ 以文本尺寸出现,
2 和 $e$ 以标号尺寸出现,而 $\prime$ 以小标号尺寸出现。

%Suppose you don't like the style that \TeX\ selects by its automatic style
%rules.  Then you can specify the style you want by typing ^|\displaystyle|
%or ^|\textstyle| or ^|\scriptstyle| or ^|\scriptscriptstyle|; the style
%that you select will apply until the end of the formula or subformula, or
%until you select another style.  For example,
%`|$$n+\scriptstyle n+\scriptscriptstyle n.$$|' produces the display
%$$n+\scriptstyle n+\scriptscriptstyle n.$$
%This is a rather silly example, but it does show
%that the plus signs get smaller too, as the style changes. \TeX\ puts no
%space around + signs in script styles.
如果不喜欢 \TeX\ 自动选择的样式,那么可规定你所要的样式,只要输入
|\displaystyle|、|\textstyle|、|\scriptstyle| 或者 |\scriptscriptstyle|;
所选定的样式将应用到公式或子公式结束,或者直到你给出另一种样式。
\1例如,`|$$n+\scriptstyle n+\scriptscriptstyle n.$$|' 得到陈列公式
$$n+\scriptstyle n+\scriptscriptstyle n.$$
这个例子比较蠢,但是可以看到,随着样式的改变,加号也变得更小了。
在标号样式中,\TeX\ 不在 + 号两边添加间距。

%Here's a more useful example of style changes: Sometimes you need to
%typeset a ``^{continued fraction}'' made up of many other fractions,
%all of which are supposed to be in display style:
%$$a_0+{1\over\displaystyle a_1+
%          {\strut 1\over\displaystyle a_2+
%            {\strut 1\over\displaystyle a_3+
%              {\strut 1\over a_4}}}}$$
%In order to get this effect, the idea is to type
%\begintt
%$$a_0+{1\over\displaystyle a_1+
%        {\strut 1\over\displaystyle a_2+
%          {\strut 1\over\displaystyle a_3+
%            {\strut 1\over a_4}}}}$$
%\endtt
%(The control sequence ^|\strut| has been used to make the denominators
%taller; this is a refinement that will be discussed in
%Chapter~18. Our concern now is with the style commands.) \
%Without the appearances of\/ |\strut| and |\displaystyle| in this formula,
%the result would be completely different:
%$$a_0+{1\over a_1+{1\over
%      a_2+{1\over a_3+{1\over a_4}}}}$$
下面是利用样式变化的一个更好的例子:
有时候需要输入``连分数'', 由许多其它分数组成,
所有的都被假定用陈列样式:
$$a_0+{1\over\displaystyle a_1+
          {\strut 1\over\displaystyle a_2+
            {\strut 1\over\displaystyle a_3+
              {\strut 1\over a_4}}}}$$
为了得到这种效果,要输入的是:
\begintt
$$a_0+{1\over\displaystyle a_1+
        {\strut 1\over\displaystyle a_2+
          {\strut 1\over\displaystyle a_3+
            {\strut 1\over a_4}}}}$$
\endtt
控制系列 |\strut| 被用来使分母更高;这是第十八章要讨论的微调。%
我们现在关心的是样式命令。)
如果在此公式中不出现 |\strut| 和 |\displaystyle|, 那么结果将完全不同:
$$a_0+{1\over a_1+{1\over
      a_2+{1\over a_3+{1\over a_4}}}}$$

%\danger These examples show that the numerator and denominator of a fraction
%are generally centered with respect to each other. If you prefer to have
%the numerator or denominator appear ^{flush left}, put `^|\hfill|' after
%it; or if you prefer ^{flush right}, put `|\hfill|' at the left. For
%example, if the first three appearances of `|1\over|' in the previous
%example are replaced by `|1\hfill\over|', you get the display
%$$a_0+{1\hfill\over\displaystyle a_1+
%          {\strut1\hfill\over\displaystyle a_2+
%            {\strut1\hfill\over\displaystyle a_3+
%              {\strut1\over a_4}}}}$$
%(a format for continued fractions that many authors prefer). This works
%because |\hfill| stretches at a faster rate than the glue that is
%actually used internally by \TeX\ when it centers the numerators
%and denominators.
\danger 这些例子表明,分数中的分子和分母一般都相对居中。%
如果要分子或分母左对齐,就在其后加上`|\hfill|';
如果要右对齐,就在左边加上`|\hfill|'。%
例如,如果前一个例子中前三个`|1\over|'用`|1\hfill\over|'代替,
得到的陈列公式为:
$$a_0+{1\hfill\over\displaystyle a_1+
          {\strut1\hfill\over\displaystyle a_2+
            {\strut1\hfill\over\displaystyle a_3+
              {\strut1\over a_4}}}}$$
(这种连分数的格式可能是许多作者想要的)。
之所以能这样的原因是,|\hfill| 的伸长能力比分子和分母居中时在内部实际%
所使用的粘连更大。

%\TeX\ has another operation `^|\atop|', which is like |\over| except that
%it leaves out the fraction line:
%\begindisplaymathdemo
%|$$x\atop y+2$$|&x\atop y+2\cr
%\endmathdemo
%The plain \TeX\ format in Appendix B also defines `^|\choose|', which is
%like |\atop| but it encloses the result in parentheses:
%\begindisplaymathdemo
%|$$n\choose k$$|&n\choose k\cr
%\endmathdemo
%It is called |\choose| because it's
%a common notation for the so-called ^{binomial coefficient}
%that tells how many ways there are to choose $k$~things out of $n$~things.
\1\TeX\ 有另外一个命令`|\atop|', 它象 |\over|, 但是没有分数中的横线:
\begindisplaymathdemo
|$$x\atop y+2$$|&x\atop y+2\cr
\endmathdemo
在附录 B 的 plain \TeX\ 格式中还定义了`|\choose|',
它象 |\atop|, 但是把结果封装在括号中了:
\begindisplaymathdemo
|$$n\choose k$$|&n\choose k\cr
\endmathdemo
之所以叫做 |\choose| 是因为它就是所谓二项式系数的通用符号,
给出了从 $n$ 个中间取 $k$ 个有多少种方法。

%You can't mix |\over| and |\atop| and |\choose| with each other.
%For example, `|$$n \choose k \over 2$$|' is illegal; you must use
%grouping, to get either `|$${n\choose k}\over2$$|' or
%`|$$n\choose{k\over2}$$|', i.e.,
%\begindisplay
%$\displaystyle{{n\choose k}\over2}\qquad{\rm or}\qquad {n\choose{k\over2}}.$
%\enddisplay
%The latter formula, incidentally, would look better as
%`|$$n\choose k/2$$|' or `|$$n\choose{1\over2}k$$|', yielding
%\begindisplay
%$\displaystyle{n\choose k/2}\qquad{\rm or}\qquad{n\choose{1\over2}k}.$
%\enddisplay
不要把 |\over| 和 |\atop| 和 |\choose| 混合使用。%
例如,`|$$n \choose k \over 2$$|'是不对的;
你必须使用编组,输入`|$${n\choose k}\over2$$|'或者`|$$n\choose{k\over2}$$|',
即,
\begindisplay
$\displaystyle{{n\choose k}\over2}\qquad{\hbox{\ST{10}或}}\qquad {n\choose{k\over2}}.$
\enddisplay
顺便说一下,要让后一个公式看来更美观,你可以写成 `|$$n\choose k/2$$|' 或
`|$$n\choose{1\over2}k$$|',此时结果是
\begindisplay
$\displaystyle{n\choose k/2}\qquad{\rm or}\qquad{n\choose{1\over2}k}.$
\enddisplay

%\medskip
%\exercise As alternatives to $\displaystyle{{n\choose k}\over2}$,
%discuss how you could obtain the two displays
%\begindisplay\abovedisplayskip=0pt\belowdisplayskip=0pt
%$\displaystyle
%{1\over2}{n\choose k}
%\qquad{\rm and}\qquad
%{\displaystyle{n\choose k}\over2}.$
%\enddisplay
%\answer |$${1\over2}{n\choose k}$$|;
%|$$\displaystyle{n\choose k}\over2$$|.
%All of these braces are necessary.
\medskip
\exercise 作为 $\displaystyle{{n\choose k}\over2}$ 的另外写法,
看看怎样才能得到两个陈列公式
\begindisplay\abovedisplayskip=0pt\belowdisplayskip=0pt
$\displaystyle
{1\over2}{n\choose k}
\qquad\hbox{和}\qquad
{\displaystyle{n\choose k}\over2}.$
\enddisplay
\answer |$${1\over2}{n\choose k}$$|;
|$$\displaystyle{n\choose k}\over2$$|。
所有这些花括号都是必需的。

%\bigbreak
%\exercise Explain how to specify the displayed formula
%$${p \choose 2}x^2 y^{p-2} - {1 \over 1-x}{1 \over 1-x^2}.$$
%\answer |$${p \choose 2} x^2 y^{p-2} - {1 \over 1-x}{1 \over 1-x^2}.$$|
\bigbreak
\exercise 看看怎样得到陈列公式
$${p \choose 2}x^2 y^{p-2} - {1 \over 1-x}{1 \over 1-x^2}.$$
\answer |$${p \choose 2} x^2 y^{p-2} - {1 \over 1-x}{1 \over 1-x^2}.$$|

%\danger \TeX\ has a generalized version of\/ |\over| and |\atop| in which you
%specify the exact thickness of the line rule by typing
%`^|\above|\<dimen>'. For example,
%\begintt
%$$\displaystyle{a\over b}\above1pt\displaystyle{c\over d}$$
%\endtt
%will produce a ^{compound fraction} with a heavier ($1\pt$ thick) rule as
%its main bar:
%$${\displaystyle{a\over b}\above 1pt\displaystyle{c\over d}}.$$
%This sort of thing occurs primarily in textbooks on elementary mathematics.
\danger  \TeX\ 还有一个比的 |\over| 和 |\atop| 更灵活命令,
用它可以准确地给出横线的粗细,只要输入`|\above|\<dimen>'即可。%
例如,
\begintt
$$\displaystyle{a\over b}\above1pt\displaystyle{c\over d}$$
\endtt
将得到一个复合分数,其主横线更粗($1\pt$):
$${\displaystyle{a\over b}\above 1pt\displaystyle{c\over d}}.$$
这种东西主要出现在初等数学的教科书中。

%\goodbreak
%Mathematicians often use the sign $\sum$ to stand for ``^{summation}''
%and the sign $\int$ to stand for ``^{integration}.'' If you're a typist but not
%a mathematician, all you need to remember is that ^|\sum| stands for
%$\sum$ and ^|\int| for $\int$; these abbreviations appear in Appendix~F
%together with all the other symbols, in case you forget. Symbols like
%$\sum$ and $\int$ (and a few others like $\bigcup$ and $\prod$ and $\oint$
%and~$\bigotimes$, all listed in Appendix~F) are called {\sl ^{large operators}},
%^^{collective signs, see large operators} ^^{sigma signs, see sum}
%and you type them just as you type ordinary symbols or letters. The
%difference is that \TeX\ will choose a {\sl larger\/} large operator in
%display style than it will in text style. For example,
%$$\halign{\indent#\hfil\qquad yields\qquad&$#\hfil$\qquad&#\hfil\cr
%|$\sum x_n$|&\sum x_n&($T$ style)\cr
%\noalign{\vskip3pt}
%|$$\sum x_n$$|&\displaystyle\sum x_n&($D$ style).\cr}$$
\goodbreak
\1数学家通常用符号 $\sum$ 来表示``求和'',
用符号 $\int$ 表示``积分''。%
如果你只是一个排版者而不是数学家,就只需要记住 |\sum| 表示 $\sum$ 和%
~|\int| 表示 $\int$; 如果你忘记了,这些简写与其它所有符号都在附录 F 中。%
象 $\sum$ 和 $\int$ 这样的符号(以及列在附录 F 中的象 $\bigcup$, $\prod$,
$\oint$ 和 $\bigotimes$ 这样的其它符号)称为{\KT{10}巨算符},
在输入时象输入普通符号或字母一样。%
差别在于, \TeX\ 在陈列样式中选择的巨算符比文本样式中要{\KT{10}更大}。%
例如,
$$\halign{\indent#\hfil\qquad 得到的是\qquad&$#\hfil$\qquad&#\hfil\cr
|$\sum x_n$|&\sum x_n&(样式 $T$)\cr
\noalign{\vskip3pt}
|$$\sum x_n$$|&\displaystyle\sum x_n&(样式 $D$)。\cr}$$

%A displayed |\sum| usually occurs with ``^{limits},'' i.e., with
%subformulas that are to appear above and below it. You type limits just
%as if they were superscripts and subscripts; for example, if you want
%$$\sum_{n=1}^m$$
%you type either `|$$\sum_{n=1}^m$$|' or `|$$\sum^m_{n=1}$$|'.  According
%to the normal conventions of mathematical typesetting, \TeX\ will change
%this to `$\sum_{n=1}^m$' (i.e., without limits) if it occurs in text
%style rather than in display style.
陈列公式的 |\sum| 通常伴有``上下限'', 即在它上面和下面有子公式。%
输入上下限就象输入上下标一样;
例如,如果要得到
$$\sum_{n=1}^m$$
可以输入`|$$\sum_{n=1}^m$$|'或`|$$\sum^m_{n=1}$$|'。%
按照数学排版的正常约定,
如果出现在文本样式而不是陈列样式中, \TeX\ 就把它变成`$\sum_{n=1}^m$'%
(即没有上下限了)。

%Integrations are slightly different from summations, in that the superscripts
%and subscripts are not set as limits even in display style:
%$$\halign{\indent\hbox to2.3in{#\hfil}\hbox to.6in{yields\hfil}&
%  $#\hfil$\qquad&#\hfil\cr
%|$\int_{-\infty}^{+\infty}$|&\int_{-\infty}^{+\infty}&($T$ style)\cr
%\noalign{\vskip3pt}
%|$$\int_{-\infty}^{+\infty}$$|&\displaystyle\int_{-\infty}^{+\infty}&
%  ($D$ style).\cr}$$
积分与求和略有不同,即使在陈列样式中,上下标也不变成上下限:
$$\halign{\indent\hbox to2.3in{#\hfil}\hbox to.6in{得到的是\hfil}&
  $#\hfil$\qquad&#\hfil\cr
|$\int_{-\infty}^{+\infty}$|&\int_{-\infty}^{+\infty}&(样式 $T$)\cr
\noalign{\vskip3pt}
|$$\int_{-\infty}^{+\infty}$$|&\displaystyle\int_{-\infty}^{+\infty}&
  (样式 $D$)。\cr}$$

%\danger Some printers prefer to set limits above and below $\int$ signs;
%this takes more space on the page, but it
%gives a better appearance if the subformulas are complex, because it
%keeps them out of the way of the rest of the formula. Similarly, limits
%are occasionally desirable in text style or script style; but some
%printers prefer not to set limits on displayed $\sum$ signs. You can change
%\TeX's convention by simply typing `^|\limits|' or `^|\nolimits|' immediately
%after the large operator.
%For example,
%$$\halign{\indent\hbox to2.3in{#\hfil}\hbox to.6in{yields\hfil}&
%  $\displaystyle{#}$\hfil\cr
%|$$\int\limits_0^{\pi\over2}$$|&\int\limits_0^{\pi\over2}\cr
%\noalign{\vskip 4pt}
%|$$\sum\nolimits_{n=1}^m$$|&\sum\nolimits_{n=1}^m\cr}$$
\danger 有些排版者希望在 $\int$ 上有上下限;
这要占用更多的页面,但是如果子公式很复杂时,这样做效果很好,
因为它把上下限从公式的其它内容区分开了。%
类似地,有时候希望在文本样式或标号样式中使用上下限;
但是某些用户不要在陈列公式的 $\sum$ 上出现上下限。%
直接在巨算符后面输入`|\limits|'或`|\nolimits|', 就可以改变 \TeX\ 的约定。
例如:
$$\halign{\indent\hbox to2.3in{#\hfil}\hbox to.6in{得到的是\hfil}&
  $\displaystyle{#}$\hfil\cr
|$$\int\limits_0^{\pi\over2}$$|&\int\limits_0^{\pi\over2}\cr
\noalign{\vskip 4pt}
|$$\sum\nolimits_{n=1}^m$$|&\sum\nolimits_{n=1}^m\cr}$$

%\ddanger If you say `|\nolimits\limits|' (presumably because some macro
%like |\int| specifies |\nolimits|, but you do want them), the last word
%takes precedence.  There's also a command `^|\displaylimits|' that can be
%used to restore \TeX's normal conventions; i.e., the limits will be
%displayed only in styles $D$ and $D'$.
\ddanger 如果输入的是`|\nolimits\limits|'(大概是象 |\int| 这样的某些宏已经%
给出了 |\nolimits|, 但是你又想要上下限),
那么最后一个优先。%
还有一个命令`|\displaylimits|', 用它来恢复 \TeX\ 的正常约定;
即上下限只在样式 $D$ 和 $D'$ 中出现。

%\danger Sometimes you need to put two or more rows of limits under a large
%operator; you can do this with `^|\atop|'. For example, if you want
%the displayed formula
%$$\sum_{\scriptstyle0\le i\le m\atop\scriptstyle0<j<n}P(i,j)$$
%the correct way to type it is
%\begintt
%$$\sum_{\scriptstyle0\le i\le m\atop\scriptstyle0<j<n}P(i,j)$$
%\endtt
%(perhaps with a few more spaces to make it look nicer in the manuscript
%file). The instruction `^|\scriptstyle|' was necessary here,
%twice---otherwise the lines `$0\le i\le m$' and `$0<j<n$' would have been in
%scriptscript size, which is too small. This is another instance of a rare
%case where \TeX's automatic style rules need to be overruled.
\danger \1有时候可能要在巨算符下面放两行或多行极限;
可以用`|\atop|'来实现。%
例如,如果要得到陈列公式
$$\sum_{\scriptstyle0\le i\le m\atop\scriptstyle0<j<n}P(i,j)$$
应该输入
\begintt
$$\sum_{\scriptstyle0\le i\le m\atop\scriptstyle0<j<n}P(i,j)$$
\endtt
(在文稿中可以用几个间距把它调整得更好看)。
命令`|\scriptstyle|'必须在这里出现两次——否则`$0\le i\le m$'和%
`$0<j<n$'就使用小标号尺寸,那太小了。%
这是另一个 \TeX\ 的自动规则应该改变的少见的例子。

%\exercise How would you type the displayed formula $\displaystyle
%\sum_{i=1}^p\sum_{j=1}^q\sum_{k=1}^ra_{ij}b_{jk}c_{ki}$\enspace?
%\answer |$$\sum_{i=1}^p\sum_{j=1}^q\sum_{k=1}^ra_{ij}b_{jk}c_{ki}$$|.
\exercise 怎样输入陈列公式
~$\displaystyle
\sum_{i=1}^p\sum_{j=1}^q\sum_{k=1}^ra_{ij}b_{jk}c_{ki}$\enspace ?
\answer |$$\sum_{i=1}^p\sum_{j=1}^q\sum_{k=1}^ra_{ij}b_{jk}c_{ki}$$|。

%\dangerexercise And how would you handle $\displaystyle
%\sum_{{\scriptstyle1\le i\le p\atop\scriptstyle1\le j\le q}
%    \atop\scriptstyle1\le k\le r}a_{ij}b_{jk}c_{ki}$\enspace?
%\answer |$$\sum_{{\scriptstyle 1\le i\le p \atop \scriptstyle 1\le j\le q}
%    \atop \scriptstyle 1\le k\le r} a_{ij} b_{jk} c_{ki}$$|.
\dangerexercise 怎样得到 $\displaystyle
\sum_{{\scriptstyle1\le i\le p\atop\scriptstyle1\le j\le q}
    \atop\scriptstyle1\le k\le r}a_{ij}b_{jk}c_{ki}$\enspace ?
\answer |$$\sum_{{\scriptstyle 1\le i\le p \atop \scriptstyle 1\le j\le q}
    \atop \scriptstyle 1\le k\le r} a_{ij} b_{jk} c_{ki}$$|。

%Since mathematical formulas can get horribly large, \TeX\ has to have some
%way to make ever-larger symbols. For example, if you type
%\begintt
%$$\sqrt{1+\sqrt{1+\sqrt{1+
%            \sqrt{1+\sqrt{1+\sqrt{1+\sqrt{1+x}}}}}}}$$
%\endtt
%the result shows a variety of available ^{square-root signs}:
%\begindisplay
%$\displaystyle\sqrt{1+\sqrt{1+\sqrt{1+
%            \sqrt{1+\sqrt{1+\sqrt{1+\sqrt{1+x}}}}}}}$
%\enddisplay
%The three largest signs here are all essentially the same, except for a
%vertical segment `\vbox{\hbox{\tenex\char'165}\vss}' that gets repeated as
%often as necessary to reach the desired size; but the smaller signs are
%distinct characters found in \TeX's math fonts.
因为数学公式可以大得惊人,所以 \TeX\ 必须可以生成不断增大的符号。%
例如,如果输入
\begintt
$$\sqrt{1+\sqrt{1+\sqrt{1+
            \sqrt{1+\sqrt{1+\sqrt{1+\sqrt{1+x}}}}}}}$$
\endtt
就在结果中出现了各种用到的根号:
\begindisplay
$\displaystyle\sqrt{1+\sqrt{1+\sqrt{1+
            \sqrt{1+\sqrt{1+\sqrt{1+\sqrt{1+x}}}}}}}$
\enddisplay
在这里,最大的三个符号基本上是一样的,除了垂直线段%
`\vbox{\hbox{\tenex\char'165}\vss}'必要地重复到所要求的尺寸外;
但是更小的符号是 \TeX\ 的数学字体中的不同字符。

%A similar thing happens with parentheses and other so-called
%``^{delimiter}'' symbols.  For example, here are some of the different sizes of
%^^{fences, see delimiters}
%^{parentheses} and ^{braces} that plain \TeX\ might use in formulas:
%\begindisplay
%$\displaystyle
%\left(\vbox to 27pt{}\left(\vbox to 24pt{}\left(\vbox to 21pt{}
%\Biggl(\biggl(\Bigl(\bigl(({\scriptstyle({\scriptscriptstyle(\hskip3pt
%)})})\bigr)\Bigr)\biggr)\Biggr)\right)\right)\right)
%\left\{\vbox to 27pt{}\left\{\vbox to 24pt{}\left\{\vbox to 21pt{}
%\Biggl\{\biggl\{\Bigl\{\bigl\{\{{\scriptstyle\{{\scriptscriptstyle\{\hskip3pt
%\}}\}}\}\bigr\}\Bigr\}\biggr\}\Biggr\}\right\}\right\}\right\}$
%\enddisplay
%The three largest pairs in each case are made with repeatable extensions,
%so they can become as large as necessary.
%^^{pieces of symbols}
类似情况还出现在括号和其它所谓是``分界符''上。%
例如,下面是 plain \TeX\ 在公式中使用的各种尺寸的圆括号和大括号:
\begindisplay
$\displaystyle
\left(\vbox to 27pt{}\left(\vbox to 24pt{}\left(\vbox to 21pt{}
\Biggl(\biggl(\Bigl(\bigl(({\scriptstyle({\scriptscriptstyle(\hskip3pt
)})})\bigr)\Bigr)\biggr)\Biggr)\right)\right)\right)
\left\{\vbox to 27pt{}\left\{\vbox to 24pt{}\left\{\vbox to 21pt{}
\Biggl\{\biggl\{\Bigl\{\bigl\{\{{\scriptstyle\{{\scriptscriptstyle\{\hskip3pt
\}}\}}\}\bigr\}\Bigr\}\biggr\}\Biggr\}\right\}\right\}\right\}$
\enddisplay
在每种情况下,最大的三对都是通过重复扩展而得到的,
所以它们可以变得如所需要的那样大。

%Delimiters are important to mathematicians, because they provide good
%visual clues to the underlying structure of complex expressions; they delimit
%the boundaries of individual subformulas. Here is a list of the 22~basic
%delimiters provided by plain \TeX:
%\begindisplay
%\it Input&\it Delimiter\cr
%\noalign{\vskip2pt}
%|(|&left parenthesis: $($\cr
%|)|&right parenthesis: $)$\cr
%|[| or ^|\lbrack|&left bracket: $[$\cr
%|]| or ^|\rbrack|&right bracket: $]$\cr
%|\{| or ^|\lbrace|&left curly brace: $\{$\cr
%|\}| or ^|\rbrace|&right curly brace: $\}$\cr
%^|\lfloor|&left floor bracket: $\lfloor$\cr
%^|\rfloor|&right floor bracket: $\rfloor$\cr
%^|\lceil|&left ceiling bracket: $\lceil$\cr
%^|\rceil|&right ceiling bracket: $\rceil$\cr
%^|\langle|&left angle bracket: $\langle$\cr
%^|\rangle|&right angle bracket: $\rangle$\cr
%|/|&slash: $/$\cr
%^|\backslash|&reverse slash: $\backslash$\cr
%\| or ^|\vert|&vertical bar: $\vert$\cr
%|\|\| or ^|\Vert|&double vertical bar: $\Vert$\cr
%^|\uparrow|&upward arrow: $\uparrow$\cr
%^|\Uparrow|&double upward arrow: $\Uparrow$\cr
%^|\downarrow|&downward arrow: $\downarrow$\cr
%^|\Downarrow|&double downward arrow: $\Downarrow$\cr
%^|\updownarrow|&up-and-down arrow: $\updownarrow$\cr
%^|\Updownarrow|&double up-and-down arrow: $\Updownarrow$\cr
%\enddisplay
%^^{bent bars, see langle, rangle} ^^{curly braces, see lbrace, rbrace}
%^^{leftbracket}^^{rightbracket}^^{leftbrace}^^{rightbrace}^^{/}
%In some cases, there are two ways to get the same delimiter; for example,
%you can specify a left bracket by typing either `|[|' or `|\lbrack|'. The
%latter alternative has been provided because the symbol `|[|' is not
%readily available on all computer keyboards. Remember, however,
%that you should never try to specify a left brace or right brace simply by
%typing `|{|' or `|}|'; the |{| and |}| symbols are reserved for grouping.
%The right way is to type `|\{|' or `|\}|' or `|\lbrace|' or `|\rbrace|'.
\1对数学家而言,分界符很重要,因为它们从外观上把复杂的公式的内在结构给理顺了;
它们把各个不同的子公式分开。%
下面给出 plain \TeX\ 的 22 个基本分界符:
\begindisplay
{\KT{10}输入}&{\hbox{\KT{10}分界符}}\cr
\noalign{\vskip2pt}
|(|&left parenthesis: $($\cr
|)|&right parenthesis: $)$\cr
|[| or |\lbrack|&left bracket: $[$\cr
|]| or |\rbrack|&right bracket: $]$\cr
|\{| or |\lbrace|&left curly brace: $\{$\cr
|\}| or |\rbrace|&right curly brace: $\}$\cr
|\lfloor|&left floor bracket: $\lfloor$\cr
|\rfloor|&right floor bracket: $\rfloor$\cr
|\lceil|&left ceiling bracket: $\lceil$\cr
|\rceil|&right ceiling bracket: $\rceil$\cr
|\langle|&left angle bracket: $\langle$\cr
|\rangle|&right angle bracket: $\rangle$\cr
|/|&slash: $/$\cr
|\backslash|&reverse slash: $\backslash$\cr
\| or |\vert|&vertical bar: $\vert$\cr
|\|\| or |\Vert|&double vertical bar: $\Vert$\cr
|\uparrow|&upward arrow: $\uparrow$\cr
|\Uparrow|&double upward arrow: $\Uparrow$\cr
|\downarrow|&downward arrow: $\downarrow$\cr
|\Downarrow|&double downward arrow: $\Downarrow$\cr
|\updownarrow|&up-and-down arrow: $\updownarrow$\cr
|\Updownarrow|&double up-and-down arrow: $\Updownarrow$\cr
\enddisplay
在某些情况下,可以通过两种方法得到同一个分界符;
例如,可以通过`|[|'或`|\lbrack|'得到左方括号。%
给出后一种方法是因为不是在所有的键盘上,符号`|[|'都那样好用。%
但是要记住,不要直接输入`|{|'或`|}|'来得到左或右大括号;
符号 |{| 和 |}| 已经保留给编组使用了。%
正确的方法是输入`|\{|'或`|\}|'或`|\lbrace|'或`|\rbrace|'。

%In order to get a slightly larger version of any of these symbols, just
%precede them by `^|\bigl|' (for opening delimiters) or `^|\bigr|' (for
%closing ones). This makes it easier to read formulas that contain
%delimiters inside delimiters:
%\beginlongmathdemo
%\it Input&\it Output\cr
%\noalign{\vskip2pt}
%|$\bigl(x-s(x)\bigr)\bigl(y-s(y)\bigr)$|&
%  \bigl(x-s(x)\bigr)\bigl(y-s(y)\bigr)\cr
%|$\bigl[x-s[x]\bigr]\bigl[y-s[y]\bigr]$|&
%  \bigl[x-s[x]\bigr]\bigl[y-s[y]\bigr]\cr
%|$\bigl|\|| |\||x|\||-|\||y|\|| \bigr|\||$|&
%  \bigl\vert\vert x\vert-\vert y\vert\bigr\vert\cr
%|$\bigl\lfloor\sqrt A\bigr\rfloor$|&
%  \bigl\lfloor\sqrt A\bigr\rfloor\cr
%\endmathdemo
%The |\big| delimiters are just enough bigger than ordinary ones so that
%the difference can be perceived, yet small enough to be used in the text
%of a paragraph. Here are all~22 of them, in the ordinary size and in
%the |\big| size:
%\begindisplay
%$(\,)\,[\,]\,\{\,\}\,\lfloor\,\rfloor\,\lceil\,\rceil\,\langle\,\rangle
%  \,/\,\backslash\,\vert\,\Vert\,\uparrow\,\Uparrow\,\downarrow\,\Downarrow
%  \,\updownarrow\,\Updownarrow$\cr
%\noalign{\smallskip}
%$\bigl(\,\bigr)\,\bigl[\,\bigr]\,\bigl\{\,\bigr\}\,\bigl\lfloor
%  \,\bigr\rfloor\,\bigl\lceil\,\bigr\rceil\,\bigl\langle\,\bigr\rangle
%  \,\big/\,\big\backslash\,\big\vert\,\big\Vert\,\bigm\uparrow\,\bigm\Uparrow
%  \,\bigm\downarrow\,\bigm\Downarrow\,\bigm\updownarrow\,\bigm\Updownarrow$\cr
%\enddisplay
%You can also type ^|\Bigl| and ^|\Bigr| to get larger symbols suitable for
%displays:
%\begindisplay
%$\Bigl(\,\Bigr)\,\Bigl[\,\Bigr]\,\Bigl\{\,\Bigr\}\,\Bigl\lfloor
%  \,\Bigr\rfloor\,\Bigl\lceil\,\Bigr\rceil\,\Bigl\langle\,\Bigr\rangle
%  \,\Big/\,\Big\backslash\,\Big\vert\,\Big\Vert\,\Bigm\uparrow\,\Bigm\Uparrow
%  \,\Bigm\downarrow\,\Bigm\Downarrow\,\Bigm\updownarrow\,\Bigm\Updownarrow$
%\enddisplay
%These are 50\% taller than their |\big| counterparts. Displayed formulas
%most often use delimiters that are even taller (twice the size of\/ |\big|);
%such delimiters are constructed by ^|\biggl| and ^|\biggr|, and they
%look like this:
%\begindisplay
%$\biggl(\,\biggr)\,\biggl[\,\biggr]\,\biggl\{\,\biggr\}\,\biggl\lfloor
%  \,\biggr\rfloor\,\biggl\lceil\,\biggr\rceil\,\biggl\langle\,\biggr\rangle
%  \,\bigg/\,\bigg\backslash\,\bigg\vert\,\bigg\Vert\,\biggm\uparrow
%  \,\biggm\Uparrow\,\biggm\downarrow\,\biggm\Downarrow\,\biggm\updownarrow
%  \,\biggm\Updownarrow$
%\enddisplay
%Finally, there are ^|\Biggl| and ^|\Biggr| versions, 2.5 times as tall
%as the |\bigl| and |\bigr| delimiters:
%\begindisplay
%$\Biggl(\,\Biggr)\,\Biggl[\,\Biggr]\,\Biggl\{\,\Biggr\}\,\Biggl\lfloor
%  \,\Biggr\rfloor\,\Biggl\lceil\,\Biggr\rceil\,\Biggl\langle\,\Biggr\rangle
%  \,\Bigg/\,\Bigg\backslash\,\Bigg\vert\,\Bigg\Vert\,\Biggm\uparrow
%  \,\Biggm\Uparrow\,\Biggm\downarrow\,\Biggm\Downarrow\,\Biggm\updownarrow
%  \,\Biggm\Updownarrow$
%\enddisplay
要得到略大的任何这些符号,只需要在它们前面加上`|\bigl|'(对开分界符)或%
`|\bigr|'(对闭分界符)。%
这使得包含多层分界符的公式容易阅读:
\beginlongmathdemo
{\KT{10}输入}&{\hbox{\KT{10}输出}}\cr
\noalign{\vskip2pt}
|$\bigl(x-s(x)\bigr)\bigl(y-s(y)\bigr)$|&
  \bigl(x-s(x)\bigr)\bigl(y-s(y)\bigr)\cr
|$\bigl[x-s[x]\bigr]\bigl[y-s[y]\bigr]$|&
  \bigl[x-s[x]\bigr]\bigl[y-s[y]\bigr]\cr
|$\bigl|\|| |\||x|\||+|\||y|\|| \bigr|\||$|&
  \bigl\vert\vert x\vert+\vert y\vert\bigr\vert\cr
|$\bigl\lfloor\sqrt A\bigr\rfloor$|&
  \bigl\lfloor\sqrt A\bigr\rfloor\cr
\endmathdemo
\1|\big| 分界符只比普通的要大得足以感觉到不同,
但是还是足够小得可在段落的文本中使用。%
这里是它们 22 个的全部,为普通尺寸和 |\big| 尺寸:
\begindisplay
$(\,)\,[\,]\,\{\,\}\,\lfloor\,\rfloor\,\lceil\,\rceil\,\langle\,\rangle
  \,/\,\backslash\,\vert\,\Vert\,\uparrow\,\Uparrow\,\downarrow\,\Downarrow
  \,\updownarrow\,\Updownarrow$\cr
\noalign{\smallskip}
$\bigl(\,\bigr)\,\bigl[\,\bigr]\,\bigl\{\,\bigr\}\,\bigl\lfloor
  \,\bigr\rfloor\,\bigl\lceil\,\bigr\rceil\,\bigl\langle\,\bigr\rangle
  \,\big/\,\big\backslash\,\big\vert\,\big\Vert\,\bigm\uparrow\,\bigm\Uparrow
  \,\bigm\downarrow\,\bigm\Downarrow\,\bigm\updownarrow\,\bigm\Updownarrow$\cr
\enddisplay
还可以通过 |\Bigl| 和 |\Bigr| 来得到陈列公式中的适当大小的符号:
\begindisplay
$\Bigl(\,\Bigr)\,\Bigl[\,\Bigr]\,\Bigl\{\,\Bigr\}\,\Bigl\lfloor
  \,\Bigr\rfloor\,\Bigl\lceil\,\Bigr\rceil\,\Bigl\langle\,\Bigr\rangle
  \,\Big/\,\Big\backslash\,\Big\vert\,\Big\Vert\,\Bigm\uparrow\,\Bigm\Uparrow
  \,\Bigm\downarrow\,\Bigm\Downarrow\,\Bigm\updownarrow\,\Bigm\Updownarrow$
\enddisplay
它们比 |\big| 符号大50\%。%
陈列公式中最经常使用的分界符甚至更高(|\big| 尺寸的两倍);
这样的分界符由 |\biggl| 和 |\biggr| 构造,它们看起来象:
\begindisplay
$\biggl(\,\biggr)\,\biggl[\,\biggr]\,\biggl\{\,\biggr\}\,\biggl\lfloor
  \,\biggr\rfloor\,\biggl\lceil\,\biggr\rceil\,\biggl\langle\,\biggr\rangle
  \,\bigg/\,\bigg\backslash\,\bigg\vert\,\bigg\Vert\,\biggm\uparrow
  \,\biggm\Uparrow\,\biggm\downarrow\,\biggm\Downarrow\,\biggm\updownarrow
  \,\biggm\Updownarrow$
\enddisplay
最后,~|\Biggl| 和 |\Biggr| 的分界符是 |\bigl| 和 |\bigr| 的 2.5 倍:
\begindisplay
$\Biggl(\,\Biggr)\,\Biggl[\,\Biggr]\,\Biggl\{\,\Biggr\}\,\Biggl\lfloor
  \,\Biggr\rfloor\,\Biggl\lceil\,\Biggr\rceil\,\Biggl\langle\,\Biggr\rangle
  \,\Bigg/\,\Bigg\backslash\,\Bigg\vert\,\Bigg\Vert\,\Biggm\uparrow
  \,\Biggm\Uparrow\,\Biggm\downarrow\,\Biggm\Downarrow\,\Biggm\updownarrow
  \,\Biggm\Updownarrow$
\enddisplay

%\medskip
%\exercise Guess how to type the formula $\displaystyle
%\biggl({\partial^2\over\partial x^2}+{\partial^2\over\partial y^2}
%  \biggr)\bigl\vert\varphi(x+iy)\bigr\vert^2=0$, in display style,
%using |\bigg| delimiters for the large parentheses. \ (The symbols $\partial$
%and $\varphi$ that appear here are called ^|\partial| and ^|\varphi|.)
%\answer |$\displaystyle\biggl({\partial^2\over\partial x^2}+|\hfil\break
%|{\partial^2\over\partial y^2}\biggr)\bigl|\||\varphi(x+iy)\bigr|\||^2=0$|.
\medskip
\exercise 看看怎样用陈列样式输入公式 $\displaystyle
\biggl({\partial^2\over\partial x^2}+{\partial^2\over\partial y^2}
  \biggr)\bigl\vert\varphi(x+iy)\bigr\vert^2=0$,
使用 |\bigg| 分界符来得到大的括号。%
(符号 $\partial$ 和 $\varphi$ 分别叫做 |\partial| 和 |\varphi|。)
\answer |$\displaystyle\biggl({\partial^2\over\partial x^2}+|\hfil\break
|{\partial^2\over\partial y^2}\biggr)\bigl|\||\varphi(x+iy)\bigr|\||^2=0$|。

%\dangerexercise In practice, |\big| and |\bigg| delimiters are used much
%more often than |\Big| and |\Bigg| ones. Why do you think this is true?
%\answer Formulas that are more than one line tall are usually two lines tall,
%not 1$1\over2$ or 2$1\over2$ lines tall.
\dangerexercise 在实际使用中,|\big| 和 |\bigg| 分界符比 |\Big| 和 |\Bigg|
分界符更常用。想想为什么?
\answer 多于一行的公式通常为两行高,而非 1$1\over2$ 或 2$1\over2$ 行高。

%\danger A |\bigl| or |\Bigl| or |\biggl| or |\Biggl| delimiter is an
%^{opening}, like a left parenthesis;
%a |\bigr| or |\Bigr| or |\biggr| or |\Biggr| delimiter is a
%^{closing}, like a right parenthesis. Plain \TeX\ also provides
%^|\bigm| and ^|\Bigm| and ^|\biggm| and ^|\Biggm| delimiters, for use
%in the middle of formulas; such a delimiter plays the r\^ole of a ^{relation},
%like an equals sign, so \TeX\ puts a bit of space on either side of it.
%\beginlongmathdemo
%|$\bigl(x\in A(n)\bigm|\||x\in B(n)\bigr)$|&
%  \tenmath\bigl(x\in A(n)\bigm\vert x\in B(n)\bigr)\cr
%\noalign{\vskip2pt}
%|$\bigcup_n X_n\bigm\|\||\bigcap_n Y_n$|&
%  \tenmath\bigcup_n X_n\bigm\Vert\bigcap_n Y_n\cr
%\endmathdemo
%^^|\bigcup|^^|\bigcap|^^|\verticalline|^^|\in|
%You can also say just ^|\big| or ^|\Big| or ^|\bigg| or ^|\Bigg|; this produces
%a delimiter that acts as an ordinary variable. It is used primarily with
%slashes and backslashes, as in the following example.
%\beginlongmathdemo
%\noalign{\vskip-2pt}
%|$${a+1\over b}\bigg/{c+1\over d}$$|&
%  \tenmath\displaystyle{a+1\over b}\bigg/{c+1\over d}\cr
%\endmathdemo
\danger |\bigl|, |\Bigl|, |\biggl| 或 |\Biggl| 分界符是开符号,象左圆括号一样;
|\bigr|, |\Bigr|, |\biggr| 或 |\Biggr| 分界符是闭符号,象右圆括号一样。%
Plain \TeX\ 还给出了|\bigm|, |\Bigm|, |\biggm| 和 |\Biggm| 分界符,
它们用在公式中央;这样的分界符起着表示关系的作用,就象等号一样,
所以 \TeX\ 在它两边都添加一点间距。
\beginlongmathdemo
|$\bigl(x\in A(n)\bigm|\||x\in B(n)\bigr)$|&
  \tenmath\bigl(x\in A(n)\bigm\vert x\in B(n)\bigr)\cr
\noalign{\vskip2pt}
|$\bigcup_n X_n\bigm\|\||\bigcap_n Y_n$|&
  \tenmath\bigcup_n X_n\bigm\Vert\bigcap_n Y_n\cr
\endmathdemo
也可以只用 |\big|, |\Big|, |\bigg| 或 |\Bigg|;
它得到的分界符就一个普通变量一样。%
它主要用于斜线和反斜线,就象下面的例子一样。
\beginlongmathdemo
\noalign{\vskip-2pt}
|$${a+1\over b}\bigg/{c+1\over d}$$|&
  \tenmath\displaystyle{a+1\over b}\bigg/{c+1\over d}\cr
\endmathdemo

%\dangerexercise What's the professional way to type
%$\tenmath\bigl(x+f(x)\bigr)\big/\bigl(x-f(x)\bigr)$? \ (Look closely.)
%\answer |$\bigl(x+f(x)\bigr) \big/ \bigl(x-f(x)\bigr)$|. \ Notice especially
%the `|\big/|'; an ordinary ^{slash} would look too small between the
%|\big| parentheses.
\dangerexercise 用专业的方法输入
$\tenmath\bigl(x+f(x)\bigr)\big/\bigl(x-f(x)\bigr)$?(看仔细点。)
\answer |$\bigl(x+f(x)\bigr) \big/ \bigl(x-f(x)\bigr)$|。
特别注意 `|\big/|';普通的^{斜线}放在 |\big| 括号之间看起来太小。

%\TeX\ has a built-in mechanism that figures out how tall a pair of delimiters
%needs to be, in order to enclose a given subformula; so you can use this
%method, instead of deciding whether a delimiter should be |\big| or
%|\bigg| or whatever. All you do is say
%\begindisplay
%^|\left|\<delim$_1$>\<subformula>^|\right|\<delim$_2$>
%\enddisplay
%and \TeX\ will typeset the subformula, putting the specified delimiters at
%the left and the right. The size of the delimiters will be just big enough
%to cover the subformula. For example, in the display
%\beginlongdisplaymathdemo
%|$$1+\left(1\over1-x^2\right)^3$$|&1+\left(1\over1-x^2\right)^3\cr
%\endmathdemo
%\TeX\ has chosen |\biggl(| and |\biggr)|, because smaller delimiters
%would be too small for this particular fraction. A simple formula like
%`|$\left(x\right)$|' yields just `$\left(x\right)$'; thus, |\left| and
%|\right| sometimes choose delimiters that are smaller than |\bigl| and |\bigr|.
\1\TeX\ 有一个内置的机制,能确定要封装住给定公式所需要的这对分界符的高度;
因此,你可以使用这种方法,而不在 |\big|, |\bigg| 或其它中筛选。%
所要做的只是
\begindisplay
|\left|\<delim$_1$>\<subformula>|\right|\<delim$_2$>
\enddisplay
 \TeX\ 将排版此子公式,把规定的分界符放在左右两边。%
分界符的大小正好可以框住子公式。%
例如,在陈列公式
\beginlongdisplaymathdemo
|$$1+\left(1\over1-x^2\right)^3$$|&1+\left(1\over1-x^2\right)^3\cr
\endmathdemo
中, \TeX\ 选择了 |\biggl(| 和 |\biggr)|, 因为再小的话就框不住这个特殊的分数了。%
象`|$\left(x|\allowbreak|\right)$|'得到的就是`$\left(x\right)$'这样的简单公式,
因此,|\left| 和 |\right| 有时候选择的分界符比 |\bigl| 和 |\bigr| 更小。

%Whenever you use |\left| and |\right| they must pair up with each other,
%just as braces do in groups. You can't have |\left| in one formula
%and |\right| in another, nor are you allowed to type things like
%`|\left(...{...\right)...}|' or
%`|\left(...\begingroup...\right)...\endgroup|'.
%This restriction makes sense, because \TeX\ needs to typeset the
%subformula that appears between |\left| and |\right| before it can decide
%how big to make the delimiters.  But it is worth explicit mention here,
%because you do {\sl not\/} have to match ^{parentheses} and ^{brackets}, etc.,
%^^{crotchets, see brackets}
%when you are not using |\left| and |\right|: \TeX\ will not complain if
%you input a formula like `|$[0,1)$|' or even `|$)($|' or just `|$)$|'.\
%(And it's a good thing \TeX\ doesn't, for such unbalanced formulas occur
%surprisingly often in mathematics papers.) \ Even when you do use |\left|
%and |\right|, \TeX\ doesn't look closely at the particular delimiters that
%you happen to choose; thus, you can type strange things like `|\left)|'
%and/or `|\right(|' if you know what you're doing. Or even if you don't.
只要用到 |\left| 和 |\right|,它们就必须成对的出现,就像编组中的花括号一样。
你不可以在一个公式中使用 |\left|,却在另一个公式中使用 |\right|;
也不可以像 `|\left(...{...\right)...}|' 或者
`|\left(...\begingroup...|\allowbreak|\right)...\endgroup|' 这样输入。
这个限制的意义在于,在 \TeX\ 决定使用多大的分界符前,
需要把 |\left| 和 |\right| 之间的子公式排版完毕。
但是这里值得提醒的是,因为当你不使用 |\left| 和 |\right| 时,
{\sl 不必}配对圆括号和方括号等等这些:
如果输入像 `|$[0,1)$|'、`|$)($|' 或者 `|$)$|' 这样的公式,
\TeX\ 不会认为是错误的。(\TeX\ 这样做是对的,
因为这样不配对的公式太经常出现在数学文章中了。)%
甚至当使用 |\left| 和 |\right| 时,\TeX\ 也不要求配对的是以前出现的那个分界符;
因此,只要你知道在做什么(甚至不知道也没关系),就可以输入像 `|\left)|'
和/或 `|\right(|' 这样奇怪的配对。

%The |\over| operation in the example displayed above does not involve the
%`|1+|' at the beginning of the formula; this happens because |\left| and
%|\right| have the function of ^{grouping}, in addition to their function
%of delimiter-making.  Any definitions that you happen to make between
%|\left| and |\right| will be local, as if braces had appeared around the
%enclosed subformula.
在上面的例子中,|\over| 没有作用在公式开头的`|1+|'上;
这是因为 |\left| 和 |\right| 有编组的功能,而不仅仅是选择分界符大小。%
出现在 |\left| 和 |\right| 之间的任何定义都是局部的,就象大括号出现在%
要封装的子公式外面一样。

%\exercise Use |\left| and |\right| to typeset the following display
%(with ^|\phi| for $\phi$):
%$$\pi(n)=\sum_{k=2}^n\left\lfloor\phi(k)\over k-1\right\rfloor.$$
%\answer |$$\pi(n)=\sum_{k=2}^n\left\lfloor\phi(k)\over k-1\right\rfloor.$$|
\exercise 利用 |\left| 和 |\right| 输入下列陈列公式(用 |\phi| 得到 $\phi$):
$$\pi(n)=\sum_{k=2}^n\left\lfloor\phi(k)\over k-1\right\rfloor.$$
\answer |$$\pi(n)=\sum_{k=2}^n\left\lfloor\phi(k)\over k-1\right\rfloor.$$|

%At this point you are probably wondering why you should bother learning about
%|\bigl| and |\bigr| and their relatives, when |\left| and |\right| are there
%to calculate sizes for you automatically. Well, it's true that |\left|
%and |\right| are quite handy, but there are at least three situations in which
%you will want to use your own wisdom when selecting the proper delimiter size:
%\ (1)~Sometimes |\left| and |\right| choose a smaller delimiter than you want.
%For example, we used |\bigl| and |\bigr| to produce $\bigl\vert\vert x\vert-
%\vert y\vert\bigr\vert$ in one of the previous illustrations; |\left| and
%|\right| don't make things any bigger than necessary, so
%`|$\left|\||\left|\||x\right|\||-\left|\||y\right|\||\right|\||$|'
%yields only
%`$\left\vert \left\vert x\right\vert -\left\vert y\right\vert \right\vert$'.
%\ (2)~Sometimes |\left| and |\right| choose a larger delimiter than you want.
%This happens most frequently when they enclose a large operator in a display;
%for example, compare the following two formulas:
%\beginlongdisplaymathdemo
%\noalign{\vskip 6pt}
%|$$\left( \sum_{k=1}^n A_k \right)$$|&\left( \sum_{k=1}^n A_k \right)\cr
%\noalign{\vskip 3pt}
%|$$\biggl( \sum_{k=1}^n A_k \biggr)$$|&\biggl( \sum_{k=1}^n A_k \biggr)\cr
%\endmathdemo
%The rules of\/ |\left| and |\right| cause them to enclose the ^|\sum| together
%with its ^{limits}, but in special cases like this it looks better to let
%the limits hang out a~bit; |\bigg| delimiters are better here.
%\ (3)~Sometimes you need to break a huge displayed
%formula into two or more separate lines, and you want to make sure that
%its opening and closing delimiters have the same size; but you can't use
%|\left| on the first line and |\right| on the last, since |\left| and
%|\right| must occur in pairs. The solution is to use |\Biggl| (say) on
%the first line and |\Biggr| on the last.
这时,你可能想知道,既然 |\left| 和 |\right| 会自动算出大小,
为什么还要花精力学习 |\bigl| 和 |\bigr| 这些东西呢?
嗯,|\left| 和 |\right| 的确很方便,但是至少在三种情况下你要%
凭自己的才智来得到正确的分界符尺寸:
\1(1). 有时候 |\left| 和 |\right| 选择的比所要的小。%
例如,在前面的一个示例中,我们用 |\bigl| 和 |\bigr| 得到了$\bigl\vert\vert x\vert+
\vert y\vert\bigr\vert$;
|\left| 和 |\right| 不能把分界符变得比所要求的更大,因此%
`|$\left|\||\left|\||x\right|\||+\left|\||y\right|\||\right|\||$|'
得到的只是%
`$\left\vert \left\vert x\right\vert +\left\vert y\right\vert \right\vert$'。%
(2). 有时候 |\left| 和 |\right| 选择的比所要的大。%
当在陈列公式中把一个巨算符封装起来时常常会出现这种情况;
例如,比较下列两个公式:
\beginlongdisplaymathdemo
\noalign{\vskip 6pt}
|$$\left( \sum_{k=1}^n A_k \right)$$|&\left( \sum_{k=1}^n A_k \right)\cr
\noalign{\vskip 3pt}
|$$\biggl( \sum_{k=1}^n A_k \biggr)$$|&\biggl( \sum_{k=1}^n A_k \biggr)\cr
\endmathdemo
|\left| 和 |\right| 的规则使得它们要把 |\sum| 以及上下限都封装起来,
但是在这样的特殊情况下,最好让上下限露出一点;
|\bigg| 分界符更好一些。%
(3). 有时候需要把一个大的陈列公式分在两个或多个行,
并且知道开和闭分界符大小相同;
而你不能在第一行用 |\left| 并且在最后一行用 |\right|,
因为 |\left| 和 |\right| 必须配对出现。%
解决方法就是(比如说)在第一行用 |\Biggl| 并且在最后一行用 |\Biggr|。

%\danger Of course, one of the advantages of\/ |\left| and |\right| is that
%they can make arbitrarily large delimiters---much bigger than |\biggggg|!
%The slashes and angle brackets do have a maximum size, however; if you
%ask for really big versions of those symbols you will get the largest
%ones available.
\danger 当然,|\left| 和 |\right| 的一个优点就是可以得到任意大的分界符%
——比 |\biggggg| 还要大很多!
但是斜线和角括号却有最大尺寸;
如果要使用这些符号的大尺寸,得到的是可用的最大尺寸。

%\exercise Prove that you have mastered delimiters: Coerce \TeX\ into
%producing the formula
%$$\pi(n)=\sum_{m=2}^n\left\lfloor\biggl(\sum_{k=1}^{m-1}\bigl\lfloor
%  (m/k)\big/\lceil m/k\rceil\bigr\rfloor\biggr)^{-1}\right\rfloor.$$
%\answer |$$\pi(n)=\sum_{m=2}^n\left\lfloor\biggl(\sum_{k=1}^{m-1}\bigl|
%\hfil\break
%|\lfloor(m/k)\big/\lceil m/k\rceil\bigr\rfloor\biggr)^{-1}\right\rfloor.$$|
\exercise 看看你掌握了分界符了么:强制 \TeX\ 生成下列公式
$$\pi(n)=\sum_{m=2}^n\left\lfloor\biggl(\sum_{k=1}^{m-1}\bigl\lfloor
  (m/k)\big/\lceil m/k\rceil\bigr\rfloor\biggr)^{-1}\right\rfloor.$$
\answer |$$\pi(n)=\sum_{m=2}^n\left\lfloor\biggl(\sum_{k=1}^{m-1}\bigl|
\hfil\break
|\lfloor(m/k)\big/\lceil m/k\rceil\bigr\rfloor\biggr)^{-1}\right\rfloor.$$|

%\danger If you type `|.|'\ after |\left| or |\right|, instead of
%specifying one of the basic delimiters, you get a so-called ^{null
%delimiter} (which is blank). Why on earth would anybody want that, you may
%ask. Well, you sometimes need to produce formulas that contain only one
%large delimiter. For example, the display
%$$\vert x\vert=\cases{x,&if $x\ge0$\cr
%  -x,&if $x<0$\cr}$$
%has a `$\{$' but no `$\}$'. It can be produced by a construction of the form
%\begindisplay
%|$$|\||x|\||=\left\{ ... \right.$$|
%\enddisplay
%Chapter 18 explains how to fill in the `\hbox{|...|}' to finish this
%construction; let's just notice for now that the `|\right.|'\ makes it
%possible to have an invisible right delimiter to go with the visible
%left brace.
\danger 如果在 |\left| 和 |\right| 后面跟的是`|.|', 而不是所规定的基本分界符,
就得到所谓的空分界符(它是一个空白)。%
可能你想知道要这个有什么用。%
嗯,有时候你要得到一个只包含一个巨分界符的公式。%
例如,陈列公式
$$\vert x\vert=\cases{x,&if $x\ge0$\cr
  -x,&if $x<0$\cr}$$
有`$\{$'而没有`$\}$'。%
可以通过下列方法得到它:
\begindisplay
|$$|\||x|\||=\left\{ ... \right.$$|
\enddisplay
第十八章讨论了`\hbox{|...|}'所代表的内容;
现在我们只需知道`|\right.|'在可见的左大括号后面产生一个不可见的右分界符。

%\ddanger A null delimiter isn't completely void; it is an empty box
%whose width is a \TeX\ parameter called ^|\nulldelimiterspace|.
%We will see later that null delimiters are inserted next to fractions.
%Plain \TeX\ sets |\nulldelimiterspace=1.2pt|.
\ddanger \1空分界符不是完全置空的;
它是一个空盒子,其宽度为 \TeX\ 中一个叫 |\nulldelimiterspace| 的参数。
后面我们将看到,空分界符被插入到分数后面。
Plain \TeX\ 设置 |\nulldelimiterspace|\allowbreak|=1.2pt|。

%You can type `|<|' or `|>|' as convenient abbreviations for ^|\langle| and
%^|\rangle|, when \TeX\ is looking for a delimiter. For example,
%`|\bigl<|' is equivalent to `|\bigl\langle|', and `|\right>|' is
%equivalent to `|\right\rangle|'. Of course `|<|' and `|>|' ordinarily
%produce the ^{less-than} and ^{greater-than} relations `${<}\,{>}$', which
%are quite different from ^{angle brackets} `$\langle\,\rangle$'.
在 \TeX\ 需要分界符时,你可以用 `|<|' 或 `|>|' 来代替 |\langle| 和 |\rangle|,
例如,`|\bigl<|' 等价于 `|\bigl\langle|',`|\right>|' 等价于 `|\right\rangle|'。
当然,`|<|' 和 `|>|' 一般得到的是小于和大于关系符号,
它们与角括号 `$\langle\,\rangle$' 差别很大。

%\danger Plain \TeX\ also makes available a few more delimiters, which were
%not listed in the basic set of~22 because they are sort of special.
%The control sequences ^|\arrowvert|, ^|\Arrowvert|, and ^|\bracevert| produce
%delimiters made from the repeatable parts of the vertical arrows, double
%vertical arrows, and large braces, respectively, without the arrowheads
%or the curly parts of the braces. They produce results similar to
%^|\vert| or ^|\Vert|, but they are surrounded by more white space and
%they have a different weight. You can also use ^|\lgroup| and ^|\rgroup|,
%which are constructed from braces without the middle parts; and
%^|\lmoustache| and ^|\rmoustache|, ^^{moustaches}
%which give you the top and bottom halves of large braces. For example,
%here are the |\Big| and |\bigg| versions of\/ |\vert|, |\Vert|, and
%these seven special delimiters:
%$$\halign{\indent$#\hfil$\cr
%\ldots\Big\vert\ldots\Big\Vert
%\ldots\Big\arrowvert\ldots\Big\Arrowvert\ldots\Big\bracevert
%\ldots\Big\lgroup\ldots\Big\rgroup\ldots\Big\lmoustache\ldots\Big\rmoustache
%\ldots\,;\cr
%\noalign{\smallskip}
%\ldots\bigg\vert\ldots\bigg\Vert
%\ldots\bigg\arrowvert\ldots\bigg\Arrowvert\ldots\bigg\bracevert
%\ldots\bigg\lgroup\ldots\bigg\rgroup\ldots\bigg\lmoustache\ldots\bigg\rmoustache
%\ldots\,.\cr}$$
%Notice that |\lgroup| and |\rgroup| are rather like bold parentheses, with
%sharper bends at the corners; this makes them attractive for certain large
%displays. But you cannot use them exactly like parentheses, because
%they are available only in large sizes (|\Big|~or~more).
\danger Plain \TeX\ 还有几个分界符,它们不在 22 个的基本集合中,
因为它们是特殊类型的。%
控制系列 |\arrowvert|, |\Arrowvert| 和 |\bracevert| 是通过分别重复垂直箭头,
垂直双箭头和巨括号的垂直部分而得到的。%
它们生成的结果类似于 |\vert| 或 |\Vert|, 但是它们两边要添加间距,
并且其厚度也不同。%
你也可以用 |\lgroup| 和 |\rgroup| 来得到没有中间部分的括号;
并且 |\lmoustache| 和 |\rmoustache| 将给出巨括号的顶部和底部字符。%
例如,下面是 |\vert|, |\Vert| 以及七个特殊分界符的的 |\Big| 和 |\bigg| 尺寸:
$$\halign{\indent$#\hfil$\cr
\ldots\Big\vert\ldots\Big\Vert
\ldots\Big\arrowvert\ldots\Big\Arrowvert\ldots\Big\bracevert
\ldots\Big\lgroup\ldots\Big\rgroup\ldots\Big\lmoustache\ldots\Big\rmoustache
\ldots\,;\cr
\noalign{\smallskip}
\ldots\bigg\vert\ldots\bigg\Vert
\ldots\bigg\arrowvert\ldots\bigg\Arrowvert\ldots\bigg\bracevert
\ldots\bigg\lgroup\ldots\bigg\rgroup\ldots\bigg\lmoustache\ldots\bigg\rmoustache
\ldots\,.\cr}$$
注意,|\lgroup| 和 |\rgroup| 非常类似与加粗的圆括号,只是角上弯得更厉害;
这在大的陈列公式中使用很好。%
但是不能把它们象圆括号那样使用,
因为它们只能用在大尺寸(|\Big| 或更大)。

%\ddanger Question: What happens if a ^{subscript} or ^{superscript}
%follows a large delimiter? Answer:~That's a good question. After a |\left|
%delimiter, it is the first subscript or superscript of the enclosed
%subformula, so it is effectively preceded by |{}|. After a |\right|
%delimiter, it is a subscript or superscript of the entire |\left...\right|
%subformula. And after a |\bigl| or |\bigr| or |\bigm| or |\big| delimiter,
%it applies only to that particular delimiter. Thus, `|\bigl(_2|' works
%quite differently from `|\left(_2|'.
\ddanger 问题:如果巨分界符后面跟一个上下标时会出现什么情况?
回答:问得好。%
在 |\left| 分界符后面,它就是所封装子公式中的第一个上下标,因此在它前面的是%
~|{}|。在 |\right| 分界符后面,它就是整个 |\left...\right| 子公式的上下标。%
在 |\bigl|, |\bigr|, |\bigm| 或 |\big| 分界符后面,它只是那个分界符的上下标。%
因此,`|\bigl(_2|'与`|\left(_2|'得到的结果不同。

%\danger If you look closely at the examples of math typesetting in this
%chapter, you will notice that large parentheses and brackets are
%symmetric with respect to an invisible horizontal line that runs a little
%bit above the ^{baseline}; when a delimiter gets larger, its height and
%depth both grow by the same amount. This horizontal line is called the
%{\sl^{axis}\/} of the formula; for example, a formula in the text of the
%present paragraph would have an axis at this level: $\hskip 2em\over$. The
%bar line in every fraction is centered on the axis, regardless of the size
%of the numerator or denominator.
\danger 如果你仔细观察本章的数学排版,就会注意到巨圆括号和大括号关于基线%
附近一条看不见的水平线对称;
当分界符变大时,高度和深度增加同样的量。%
这个水平线称为公式的{\KT{10}轴};
例如,当前段落的文本中公式的轴为 $\hskip 2em\over$。%
在每个分数中的横线都在轴上,不管分子或分母的大小。

%\danger Sometimes it is necessary to create a special box that should be
%centered vertically with respect to the axis. \ (For example, the
%`$\vert x\vert=\bigl\{\,\ldots$' example above was done with such a box.) \
%\TeX\ provides a simple way to do this: You just say
%\begindisplay
%|\vcenter{|\<vertical mode material>|}|
%\enddisplay
%and the vertical mode material will be packed into a box just as if
%^|\vcenter| had been ^|\vbox|. Then the box will be raised or lowered until
%its top edge is as far above the axis as the bottom edge is below.
\danger 有时候必须生成一个关于轴垂直居中的特殊的盒子。%
(例如,上面的`$\vert x\vert=\bigl\{\,\ldots$'就用了这样的盒子。)
 \TeX\ 提供了实现它的简单方法:
只要输入
\begindisplay
|\vcenter{|\<vertical mode material>|}|
\enddisplay
\1垂直模式的内容就放在一个盒子中,就象 |\vcenter| 是一个 |\vbox| 一样。%
接着盒子被升降到顶边和底边距轴一样的地方。

%\ddanger The concept of ``axis'' is meaningful for \TeX\ only in math
%formulas, not in ordinary text; therefore \TeX\ allows you to use
%|\vcenter| only in math mode. If you really need to center something
%vertically in horizontal mode, the solution is to say `|$\vcenter{...}$|'.
% \ (Incidentally, the constructions `|\vcenter| |to|\<dimen>'
%and `|\vcenter| |spread|\<dimen>' are legal too, in math mode;
%vertical glue is always set by the rules for |\vbox| in
%Chapter~12. But |\vcenter| by itself is usually sufficient.)
\ddanger ``轴''的概念只对数学公式有意义,对普通文本没意义;
因此, \TeX\ 只允许在数学模式下使用 |\vcenter|。%
如果的确需要在水平模式下垂直居中某些内容,只能用`|$\vcenter{...}$|'。%
(顺便说一下,命令`|\vcenter| |to|\<dimen>'和`|\vcenter| |spread|\<dimen>'%
在数学模式下也是可以的;
垂直粘连总可以通过第十二章的规则来设置。%
但是一般直接用 |\vcenter| 就可以了。)

%\danger Any box can be put into a formula by simply saying ^|\hbox| or
%|\vbox| or ^|\vtop| or ^|\box| or ^|\copy| in the normal way, even when
%you are in math mode.  Furthermore you can use ^|\raise| or ^|\lower|, as
%if you were in horizontal mode, and you can insert vertical rules with
%^|\vrule|.  Such constructions, like |\vcenter|, produce boxes that can be
%used like ordinary symbols in math formulas.
\danger 通过直接按照正常方法输入 |\hbox|, |\vbox|, |\vtop|, |\box| 或 |\copy|~%
就可以把任何盒子放在公式中,即使是在数学模式下。%
还有,可以象在水平模式下那样使用 |\raise| 或 |\lower|,
并且可以用 |\vrule| 插入垂直标尺。%
象 |\vcenter| 这样得到的盒子在数学公式中象普通符号那样使用。

%\ddanger Sometimes you need to make up your own symbols, when you run across
%something unusual that doesn't occur in the fonts. If the new symbol
%occurs only in one place, you can use |\hbox| or |\vcenter| or something
%to insert exactly what you want; but if you are defining a macro for
%general use, you may want to use different constructions in different
%styles. \TeX\ has a special feature called ^|\mathchoice| that comes
%to the rescue in such situations: You write
%\begindisplay
%|\mathchoice{|\<math>|}{|\<math>|}{|\<math>|}{|\<math>|}|
%\enddisplay
%where each \<math> specifies a subformula. \TeX\ will choose the first
%subformula in style $D$ or~$D'$, the second in style $T$ or~$T'$, the
%third in style $S$ or~$S'$, the fourth in style $\SS$ or $\SS'$.
%\ (\TeX\ actually typesets all four subformulas, before it chooses the
%final one, because the actual style is not always known at the time a
%|\mathchoice| is encountered; for example, when you type `|\over|' you often
%change the style of everything that has occurred earlier in the formula.
%Therefore |\mathchoice| is somewhat expensive in terms of time and space,
%and you should use it only when you're willing to pay the price.)
\ddanger 当遇到字体中没有的不常用的符号时,有时候需要制作你自己的符号。%
如果新符号只在一个地方使用,就可以用 |\hbox| 或 |\vcenter| 或其它东西插入%
你想要的;
但是如果要定义一个普遍使用的宏,那么可能要在不同样式中使用不同的命令。%
 \TeX\ 提供了一种这种情况下使用的特殊命令,叫做 |\mathchoice|:
给出
\begindisplay
|\mathchoice{|\<math>|}{|\<math>|}{|\<math>|}{|\<math>|}|
\enddisplay
其中每一个 \<math> 规定了一个子公式。%
在样式 $D$ 或 $D'$ 中 \TeX\ 选择第一个子公式,
在 $T$ 或 $T'$ 中选第二个,在 $S$ 或 $S'$ 中选第三个,在 $SS$ 或 $SS'$ 中选第四个。%
(实际上,在 \TeX\ 选定最后一个之前,它把所有四个子公式都排版了,
因为在读入 |\mathchoice| 时一般还不知道实际要用的样式;
例如,当使用`|\over|'时,你通常改变了公式前半部分出现的所有东西的样式。
因此,|\mathchoice| 要花费很多时间和空间,因此只有在舍得的时候再使用它。)

%\ddangerexercise Guess what output is produced by the following commands:
%\begintt
%\def\puzzle{{\mathchoice{D}{T}{S}{SS}}}
%$$\puzzle{\puzzle\over\puzzle^{\puzzle^\puzzle}}$$
%\endtt
%\answer A displayed formula equivalent to |$${D}{{T}\over{T}^{{S}^{SS}}}$$|.
\ddangerexercise 看看下列命令输出结果是什么:
\begintt
\def\puzzle{{\mathchoice{D}{T}{S}{SS}}}
$$\puzzle{\puzzle\over\puzzle^{\puzzle^\puzzle}}$$
\endtt
\answer 一个等价于 |$${D}{{T}\over{T}^{{S}^{SS}}}$$| 的陈列公式。

%\ddangerexercise Devise a `^|\square|' macro that produces a
%\def\sqr#1#2{{\vcenter{\vbox{\hrule height.#2pt
%      \hbox{\vrule width.#2pt height#1pt \kern#1pt \vrule}
%      \hrule height.#2pt}}}}%
%`$\,\sqr34\,$' for use in math formulas. The box should be symmetrical
%with respect to the axis, and its inside dimensions should be $3\pt$ in
%display and text styles, $2.1\pt$ in script styles, and $1.5\pt$ in
%scriptscript styles. The rules should be $0.4\pt$ thick in
%display and text styles, $0.3\pt$ thick otherwise.
%\answer |\def\sqr#1#2{{\vcenter{\vbox{\hrule height.#2pt|\parbreak
%        |        \hbox{\vrule width.#2pt height#1pt \kern#1pt|\parbreak
%        |           \vrule width.#2pt}|\parbreak
%        |        \hrule height.#2pt}}}}|\parbreak
%        |\def\square{\mathchoice\sqr34\sqr34\sqr{2.1}3\sqr{1.5}3}|
\ddangerexercise 设计一个叫`|\square|'的宏,以得到数学公式中使用的%
\def\sqr#1#2{{\vcenter{\vbox{\hrule height.#2pt
      \hbox{\vrule width.#2pt height#1pt \kern#1pt \vrule}
      \hrule height.#2pt}}}}%
盒子关于轴是对称的,并且其内部尺寸在陈列样式和文本样式中为 $3\pt$,
在标号样式中为 $2.1\pt$, 在小标号样式中为 $1.5\pt$。%
在陈列样式和文本样式中为 $0.4\pt$ 厚,其它为 $0.3\pt$ 厚。
\answer |\def\sqr#1#2{{\vcenter{\vbox{\hrule height.#2pt|\parbreak
        |        \hbox{\vrule width.#2pt height#1pt \kern#1pt|\parbreak
        |           \vrule width.#2pt}|\parbreak
        |        \hrule height.#2pt}}}}|\parbreak
        |\def\square{\mathchoice\sqr34\sqr34\sqr{2.1}3\sqr{1.5}3}|

%\ddanger Plain \TeX\ has a macro called ^|\mathpalette| that is useful
%for |\mathchoice| constructions; `|\mathpalette\a{xyz}|' expands to
%the four-pronged array of choices
%`|\mathchoice|\stretch|{\a|\stretch|\displaystyle|\stretch|{xyz}}|\stretch
%|...|\stretch|{\a|\stretch|\scriptscriptstyle|\stretch|{xyz}}|\stretch'.
%Thus the first argument to |\mathpalette| is a control sequence whose
%first argument is a style selection. Appendix~B contains several examples
%that show how |\mathpalette| can be applied. \ (See in particular the
%definitions of\/ |\phantom|, |\root|, and |\smash|; the ^{congruence sign}
%^|\cong| ($\cong$) is also constructed from $=$ and $\sim$ using
%|\mathpalette|.)
%^^{constructing new math symbols}
%^^{math symbols, construction of}
\ddanger Plain \TeX\ 有一个名为 |\mathpalette| 的宏,
它使得 |\mathchoice| 构造更加容易使用;`|\mathpalette\a{xyz}|' 展开了四个分支
\begintt
\mathchoice{\a\displaystyle{xyz}}...{\a\scriptscriptstyle{xyz}}.
\endtt
因此,|\mathpalette| 的第一个参量是一个控制系列,而这个控制系列的第一个
参量是样式选择。附录 B 包含了几个展示 |\mathpalette| 的应用的例子。%
(特别是 |\phantom|、|\root| 和 |\smash| 的定义;全等符号 |\cong|($\cong$)%
也是通过 |\mathpalette| 从 $=$ 和 $\sim$ 得到的。

%\ddanger At the beginning of this chapter we discussed the commands
%|\over|, |\atop|, |\choose|, and |\above|. These are special cases of
%\TeX's ``^{generalized fraction}'' feature, which includes also the
%three primitives
%\begindisplay
%|\overwithdelims|\<delim$_1$>\<delim$_2$>\cr
%|\atopwithdelims|\<delim$_1$>\<delim$_2$>\cr
%|\abovewithdelims|\<delim$_1$>\<delim$_2$>\<dimen>\cr
%\enddisplay
%The third of these is the most general, as it encompasses all of the other
%generalized fractions:  ^|\overwithdelims| uses a ^{fraction} bar whose
%thickness is the default for the current size, and ^|\atopwithdelims| uses
%an invisible fraction bar whose thickness is zero, while
%^|\abovewithdelims| uses a bar whose thickness is specified explicitly.
%\TeX\ places the immediately preceding subformula (the ^{numerator}) over
%the immediately following subformula (the ^{denominator}), separated by a
%bar line of the desired thickness; then it puts \<delim$_1$> at the left
%and \<delim$_2$> at the right. For example, `^|\choose|' is equivalent to
%`|\atopwithdelims()|'. If you define |\legendre| to be
%`|\overwithdelims()|', you can typeset the ^{Legendre symbol}
%\def\legendre{\overwithdelims()}%
%`$a\legendre b$' by saying `|{a\legendre b}|'. The size of the surrounding
%delimiters depends only on the style, not on the size of the fractions;
%larger delimiters are used in styles $D$ and~$D'$ (see Appendix~G\null). The
%simple commands ^|\over|, ^|\atop|, and ^|\above| are equivalent to the
%corresponding `|withdelims|' commands when the delimiters are null; for
%example, `|\over|' is an abbreviation for `|\overwithdelims..|'.
\ddanger \1在本章开头,我们讨论了命令 |\over|, |\atop|, |\choose| 和 |\above|。%
它们是 \TeX\ 的广义分数特性的特殊情形,
这个广义分数还包括三个原始控制系列:
\begindisplay
|\overwithdelims|\<delim$_1$>\<delim$_2$>\cr
|\atopwithdelims|\<delim$_1$>\<delim$_2$>\cr
|\abovewithdelims|\<delim$_1$>\<delim$_2$>\<dimen>\cr
\enddisplay
其中的第三个更广义,因为它包含了所有其它的广义分数:
|\overwithdelims| 使用了厚度为当前尺寸的分数横线,
|\atopwithdelims| 使用了厚度为零的看不见的分数横线,
而 |\abovewithdelims| 使用的是厚度为任意给定的分数横线。%
 \TeX\ 直接把前面的子公式(分子)放在后面的子公式(分母)上面,
中间是所要求厚度的横线;
接着把 \<delim$_1$> 放在左边,把 \<delim$_2$> 放在右边。%
例如,`|\choose|'等价于`|\atopwithdelims()|'。%
如果要定义 |\legendre| 为`|\overwithdelims()|', 那么可以通过输入%
\def\legendre{\overwithdelims()}%
`|{a\legendre b}|'来排版出 Legendre 符号`$a\legendre b$'。%
两边的分界符大小与样式有关,与分数的大小无关;
在样式 $D$ 和 $D'$ 中用大的分界符(见附录 G)。%
简单命令 |\over|, |\atop| 和 |\above| 等价于空分界符时`|withdelims|'的相应命令;
例如,`|\over|'就等价于`|\overwithdelims..|'。

%\def\euler{\atopwithdelims<>}
%\ddangerexercise Define a control sequence |\euler| so that the
%^{Eulerian number} $n\euler k$ will be produced when you type `|{n\euler k}|'
%in a formula.
%\answer|\def\euler{\atopwithdelims<>}|.
\def\euler{\atopwithdelims<>}
\ddangerexercise 定义一个控制系列 |\euler|,
使得在公式中输入`|{n\euler k}|'时得到 Euler 数 $n\euler k$。
\answer|\def\euler{\atopwithdelims<>}|。

%\ddanger Appendix G explains exactly how \TeX\ computes the desired size
%of delimiters for |\left| and~|\right|. The general idea is that delimiters
%are vertically centered with respect to the ^{axis}; hence, if we want
%to cover a subformula between |\left| and |\right| that extends $y_1$~units
%above the axis and $y_2$~units below, we need to make a delimiter whose
%height plus depth is at least $y$~units, where $y=2\max(y_1,y_2)$.
%It is usually best not to cover the formula completely, however,
%but just to come close; so \TeX\ allows you to specify
%two parameters, the ^|\delimiterfactor|~$f$ (an~integer) and the
%^|\delimitershortfall|~$\delta$ (a~dimension). The minimum delimiter size
%is taken to be at least $y\cdot f/1000$, and at least $y-\delta$. Appendix~B
%sets $f=901$ and $\delta=5\pt$. Thus, if $y=30\pt$, the plain \TeX\ format
%causes the delimiter to be more than $27\pt$ tall; if $y=100\pt$, the
%corresponding delimiter will be at least $95\pt$ tall.
\ddanger 附录 G 讨论了怎样精确算出 |\left| 和 |\right| 分界符所要的尺寸。%
一般思路是,分界符关于轴垂直居中;
因此如果要框住 |\left| 和 |\right| 之间的子公式,并且设轴上面的高度为 $y_1$,
轴下面的高度为 $y_2$, 那么我们要制作的分界符的高度加深度必须至少%
为 $y=2\max(y_1,y_2)$。%
但是一般最好不完全框住公式,而是仅仅基本上框住就可以了;
因此 \TeX\ 允许给出两个参数,|\delimiterfactor|~$f$(一个整数)和%
~|\delimitershortfall|~$\delta$~(一个尺寸)。%
分界符的最小尺寸至少取为 $y\cdot f/1000$, 并且至少为 $y-\delta$。%
附录 B 设置 $f=901$ 和 $\delta=5\pt$。%
因此,如果 $y=30\pt$, 那么 plain \TeX\ 公式就把分界符的高度变成 $27\pt$;
如果 $y=100\pt$, 相应的分界符的高度至少为 $95\pt$。

%\danger So far we have been discussing the rules for typing math formulas,
%but we haven't said much about how \TeX\ actually goes about converting
%its input into lists of boxes and glue. Almost all of the control
%sequences that have been mentioned in Chapters 16 and~17 are ``high level''
%features of the plain \TeX\ format; they are not built into \TeX\ itself.
%Appendix~B defines those control sequences in terms of more primitive
%commands that \TeX\ actually deals with. For example, `|\choose|' is
%an abbreviation for `|\atopwithdelims()|'; Appendix~B not only introduces
%|\choose|, it also tells \TeX\ where to find the delimiters |(| and~|)|
%in various sizes. The plain \TeX\ format defines all of the special
%characters like |\alpha| and~|\mapsto|, all of the special accents like
%|\tilde| and~|\widehat|, all of the large operators like |\sum| and~|\int|,
%and all of the delimiters like |\lfloor| and~|\vert|. Any of these things
%can be redefined, in order to adapt \TeX\ to other mathematical styles
%and/or to other fonts.
\danger 现在我们已经讨论了排版数学公式的规则,
但是关于 \TeX\ 怎样把输入内容变成盒子和粘连列还没有讲述。%
在第十六和十七章提到的几乎所有控制系列都是 plain \TeX\ 格式的``高级''命令;
它们不是 \TeX\ 自己内建的。%
附录 B 用 \TeX\ 实际使用的原始命令定义了这些控制系列。%
例如,`|\choose|'的定义为`|\atopwithdelims()|';
附录 B 不仅定义了 |\choose|, 它还告诉 \TeX\ 怎样得到各种大小的分界符\hbox{~|(| 和 |)|。}%
Plain \TeX\ 格式定义了所有象 |\alpha| 和 |\mapsto| 这样的特殊字符,
所有象 |\tilde| 和 |\widehat| 这样的特殊重音,所有象 |\sum| 和 |\int| 这样%
的巨算符,所有象 |\lfloor| 和 |\vert| 这样的分界符。
这些东西的任一个都可重新定义,使得 \TeX\ 能够使用其他数学样式和/或其他字体。

%\danger The remainder of this chapter discusses the low-level commands
%that \TeX\ actually obeys behind the scenes.  Every paragraph on the next
%few pages is marked with double dangerous bends, so you should skip to
%Chapter~18 unless you are a glutton for \TeX nicalities.
\danger \1本章剩下的内容要讨论 \TeX\ 在幕后实际使用的低级命令。%
在下几页的每段前都有两个``危险''标识,
因此除非特别着迷,你可以跳过第十八章了。

%\ninepoint
%\ddanger All characters that are typeset in math mode belong to one of
%sixteen {\sl^{families} of fonts}, numbered internally from 0 to~15. Each
%of these families consists of three fonts: one for text size, one for
%script size, and one for scriptscript size. The commands ^|\textfont|,
%^|\scriptfont|, and ^|\scriptscriptfont| are used to specify the members
%of each family.  For example, ^{family~0} in the plain \TeX\ format is
%used for roman letters, and Appendix~B contains the instructions
%\begintt
%\textfont0=\tenrm
%\scriptfont0=\sevenrm
%\scriptscriptfont0=\fiverm
%\endtt
%to set up this family: The 10-point roman font (^|\tenrm|) is used for
%normal symbols, 7-point roman (^|\sevenrm|) is used for subscripts, and
%5-point roman (^|\fiverm|) is used for sub-subscripts. Since there are up to
%256~characters per font, and 3~fonts per family, and 16~families, \TeX\ can
%access up to 12,288 characters in any one formula (4096 in~each of
%the three sizes). Imagine that.
\ninepoint
\ddanger 在数学模式下排版的所有字符都属于 16 个{\KT{9}字体族}中的一个,
在内部用从 0 到 15 来编号。%
这些族中的每个都包含三种字体:
一个用于文本尺寸,一个用于标号尺寸,一个用于小标号尺寸。%
命令 |\textfont|, |\scriptfont| 和 |\scriptscriptfont| 对应于每族的成员。%
例如,~plain \TeX\ 格式中的第 0 族被用于罗马体字母,
并且附录 B 用下列指令
\begintt
\textfont0=\tenrm
\scriptfont0=\sevenrm
\scriptscriptfont0=\fiverm
\endtt
来设定此族:
10-point 罗马字体 (|\tenrm|) 用于正常符号,
7-point 罗马字体 (|\sevenrm|) 用于上下标,
5-point 罗马字体 (|\fiverm|) 用于小上下标。%
因为每种字体有 256 个字符,每族有 3 种字体,总共有 16 族,
那么在任一公式中, \TeX\ 都可直接得到 12,228 个字符(每种尺寸为 4096 个)。%
想像一下有多少。

%\ddanger A definition like |\textfont|\<family number>|=|\<font identifier>
%is local to the group that contains it, so you can easily change family
%membership from one set of conventions to another and back again. Furthermore
%you can put any font into any family; for example, the command
%\begintt
%\scriptscriptfont0=\scriptfont0
%\endtt
%makes sub-subscripts in family~0 the same size as the subscripts currently
%are. \TeX\ doesn't check to see if the families are sensibly organized; it
%just follows instructions. \ (However, fonts cannot be used in families
%2 and~3 unless they contain a certain number of special parameters, as we
%shall see later.) \ Incidentally, \TeX\ uses ^|\nullfont|, which contains
%no characters, for each family member that has not been defined.
\ddanger 象 |\textfont|\<family number>|=|\<font identifier> 这样的定义%
对包含它的编组而言是局部的,因此可以很容易在族成员之间从一组设定改为另一组%
并且再改回来。%
还有,可以把任何字体放在任何族中;例如,命令
\begintt
\scriptscriptfont0=\scriptfont0
\endtt
把第 0 族小标号尺寸变得同当前的标号尺寸一样大。%
 \TeX\ 不其检验族是否建立起来了;
它只遵循指令。%
(但是,我们后面将看到,字体不能用在第 2 和 3 族,除非它们包含一定数量的参数。)
顺便说一下,对没有定义的每族成员, \TeX\ 都用不包含字符的 |\nullfont| 代替。

%\ddanger During the time that a math formula is being read,
%\TeX\ remembers each symbol as being ``character position so-and-so in
%family number such-and-such,'' but it does not take note of what fonts
%are actually in the families until reaching the end of the formula.
%Thus, if you have loaded a font called |\Helvetica| that contains Swiss-style
%numerals, and if you say something like
%\begintt
%$\textfont0=\tenrm 9 \textfont0=\Helvetica 9$
%\endtt
%you will get two 9's in font |\Helvetica|, assuming that \TeX\ has been
%set up to take 9's from family~0. The reason is that |\textfont0|
%is~|\Helvetica| at the end of the formula, and that's when it counts. On
%the other hand, if you say
%\begintt
%$\textfont0=\tenrm 9 \hbox{$9\textfont0=\Helvetica$}$
%\endtt
%the first 9 will be from |\tenrm| and the second from |\Helvetica|, because
%the formula in the hbox will be typeset before it is incorporated into
%the surrounding formula.
\ddanger 在数学公式正在读入期间, \TeX\ 把每个符号记作``在某某族中的某某字符位置'',
但是直到公式结尾,它才去注意族中实际是什么字体。%
因此,如果你已经把叫做 |\Helvetica|~(它包含瑞士数字)的字体载入,
并且给出象
\begintt
$\textfont0=\tenrm 9 \textfont0=\Helvetica 9$
\endtt
这样的指令,那么就得到字体 |\Helvetica| 中的两个 9,
如果 \TeX\ 设定得要从第 0 族得到 9 的话。%
原因是在公式结尾处 |\textfont0| 是 |\Helvetica|,
并且那是它正在起作用的时候。%
另一方面,如果使用
\begintt
$\textfont0=\tenrm 9 \hbox{$9\textfont0=\Helvetica$}$
\endtt
那么第一个 9 是 |\tenrm| 而第二个是 |\Helvetica|,
因为 hbox 中的公式在合并到外面公式前就已经排版好了。

%\ddangerexercise If you say `|${\textfont0=\Helvetica 9}$|', what
%font will be used for the~9?
%\answer The |\textfont0| that was current at the beginning of the formula
%will be used, because this redefinition is local to the braces. \
%(It would be a different story if `^|\global||\textfont|' had appeared instead;
%that would have changed the meaning of\/ |\textfont0| at all levels.)
\ddangerexercise 如果输入`|${\textfont0=\Helvetica 9}$|',
那么 9 所用的字体是什么?
\answer 因为花括号内的重定义是局部的,
所用的字体将是在公式开始时的当前 |\textfont0|。%
(如果写成 `^|\global||\textfont|' 结果就会不一样;
这将在各个层级都改变 |\textfont0| 的含义。)

%\ddanger Every ^{math character} is given an identifying code
%number between 0 and~4095, obtained by adding 256~times the family number
%to the position number. This is easily expressed in ^{hexadecimal
%notation}, using one hexadecimal digit for the family and two for the
%character; for example, \hex{24A} stands for character~\hex{4A} in
%family~2. Each character is also assigned to one of eight classes,
%^^{classes of math characters, table} ^^{math codes} ^^{table of ...}
%numbered 0 to~7, as follows:
%$$\halign{\indent#\hfil&\quad#\hfil&\quad#\hfil&
%\hskip4em#\hfil&\quad#\hfil&\quad#\hfil\cr
%\it \kern-2pt Class&\it Meaning&\kern-2pt\it Example&
%\it \kern-2pt Class&\it Meaning&\kern-2pt\it Example\cr
%\noalign{\vskip2pt}
%0&Ordinary&|/|&
%4&Opening&|(|\cr
%1&Large operator&|\sum|&
%5&Closing&|)|\cr
%2&Binary operation&|+|&
%6&Punctuation&|,|\cr
%3&Relation&|=|&
%7&Variable family&|x|\cr
%}$$
%^^{large operator}^^{binary operation}^^{relation}^^{opening}^^{closing}
%^^{punctuation}^^{variable family}
%Classes 0 to 6 tell what ``part of speech'' the character belongs to, in
%math-printing language; class~7 is a special case discussed below. The class
%number is multiplied by 4096 and added to the character number, and this
%is the same as making it the leading digit of a four-digit hexadecimal
%number. For example, Appendix~B defines |\sum| to be the math character
%\hex{1350}, meaning that it is a large operator (class~1) found in position
%\hex{50} of family~3.
\ddanger \1每个数学字符被指定一个识别码数字,在 0 和 4095 之间,
它等于 256 乘以族数再加上位置数。%
这很容易用十六进制表示,一位十六进制数为族数,两位为字符位置;
例如,\hex{24A} 表示第 2 族的 \hex{4A} 字符。%
每个字符还可以指定到 8 类中的一类,
编号从 1 到 7, 如下:
$$\halign{\indent#\hfil&\quad#\hfil&\quad#\hfil&
\hskip2em#\hfil&\quad#\hfil&\quad#\hfil\cr
\it \kern-2pt {\KT{9}类}&\it {\KT{9}意思}&\kern-2pt\it {\KT{9}例子}&
\it \kern-2pt {\KT{9}类}&\it {\KT{9}意思}&\kern-2pt\it {\KT{9}例子}\cr
\noalign{\vskip2pt}
0&Ordinary(普通符号)&|/|&
4&Opening(开符号)&|(|\cr
1&Large operator(巨符号)&|\sum|&
5&Closing(闭符号)&|)|\cr
2&Binary operation(二元运算)&|+|&
6&Punctuation(标点)&|,|\cr
3&Relation(关系符号)&|=|&
7&Variable family(变量族)&|x|\cr
}$$
第 0 到 6 类是数学排版中字符属于哪个``讨论过的部分'';
第 7 类是下面要讨论的特殊情形。%
这些类数乘以 4096 再加到字符代码上,而且这个类数就是就是四位十六进制数%
的第一位数。%
例如,附录 B 把 |\sum| 定义为数学字符 \hex{1350}, 意思是它是巨算符(第 1 类),
在第 3 类的位置 \hex{50} 上。

%\ddangerexercise The ^|\oplus| and ^|\bullet| symbols ($\oplus$ and $\bullet$)
%are binary operations that appear in positions 8 and~15 (decimal)
%of family~2, when the fonts of plain~\TeX\ are being used. Guess
%what their math character codes are. \ (This is too easy.)
%\answer \hex{2208} and \hex{220F}.
\ddangerexercise 符号 |\oplus| 和 |\bullet|($\oplus$ 和 $\bullet$)是二元运算符;
在 plain \TeX\ 的字体中,它们出现在第 2 族位置 8 和 15(十进制数)上。
其数学字符代码是什么?(这太简单了。)
\answer \hex{2208} 和 \hex{220F}。

%\ddanger Class 7 is a special case that allows math symbols to change families.
%It behaves exactly like class~0, except that the specified family is
%replaced by the current value of an integer parameter called ^|\fam|,
%provided that |\fam| is a legal family number (i.e., if it lies between
%0 and~15). \TeX\ automatically sets |\fam=-1| whenever math mode is entered;
%therefore class~7 and class~0 are equivalent unless |\fam| has been
%given a new value. Plain \TeX\ changes |\fam| to~0 when the user
%types `^|\rm|'; this makes it convenient to get roman letters in formulas,
%as we will see in Chapter~18, since letters belong to class~7. \ (The
%control sequence |\rm| is an abbreviation for `|\fam=0 \tenrm|'; thus,
%|\rm| causes
%|\fam| to become zero, and it makes |\tenrm| the ``^{current font}.''
%In horizontal mode, the |\fam| value is irrelevant and the current font
%governs the typesetting of letters; but in math mode, the current font is
%irrelevant and the |\fam| value governs the letters. The current font
%affects math mode only if\/ |\|\] is used ^^{control space} or if
%dimensions are given in ^|ex| or ^|em| units;
%it also has an effect if an |\hbox| appears inside a formula, since
%the contents of an hbox are typeset in horizontal mode.)
\ddanger 第 7 类是一种特殊情形,它允许数学符号改变族数。%
它的性质象第 0 类,但是给出的族数被叫做 |\fam| 的整数参数的当前值所代替,
只要 |\fam| 是一个合理的族数(即处在 0 和 15 之间)。%
只要进入了数学模式, \TeX\ 就自动设置 |\fam=-1|;
因此,如果 |\fam| 没有得到新的值,那么第 7 类与第 0 类就是等价的。%
当用户输入`|\rm|'时,~plain \TeX\ 就把 |\fam| 变为零;
这样就很容易得到 roman 字母,就象我们在第十八章将看到的那样,
因为字母属于第 7 类。%
控制系列 |\rm| 的定义为`|\fam=0 \tenrm|';
因此,|\rm| 使 |\fam| 变成零,并且它把``当前字体''变成 |\tenrm|。%
在水平模式下,|\fam| 的值无关紧要,并且当前字体决定字母的排版;
但是在数学模式下,当前字体是无关紧要的,|\fam| 的值决定字母的排版。%
只有当用到 |\|\] 或要用 |ex| 或 |\em| 的方式给出尺寸时,
才在数学模式下用到当前字体;
如果 hbox 出现在公式中,当前字体也起作用,因为 hbox 的内容要在水平模式下排版。

%\ddanger The interpretation of characters in math mode is defined by a
%table of~256 ``mathcode'' values; these table entries can be changed
%by the ^|\mathcode| command, just as the category codes are changed
%by ^|\catcode| (see Chapter~7). Each mathcode specifies class, family, and
%character position, as described above. For example, Appendix~B contains
%the commands
%\begintt
%\mathcode`<="313C
%\mathcode`*="2203
%\endtt
%which cause \TeX\ to treat the character `|<|' in math mode as a relation
%^^{less than} (class~3) found in position \hex{3C} of family~1, and to treat an
%^{asterisk} `|*|' as a binary operation found in position~3 of family~2.
%The initial value of\/ |\mathcode`b| is \hex{7162}; thus, |b|~is character
%\hex{62} in ^{family~1} (italics), and its family will vary with |\fam|.
%\ (|INITEX| starts out with |\mathcode|$\,x=x$ for all characters~$x$
%that are neither
%letters nor digits. The ten digits have |\mathcode|$\,x=x+\hbox{\hex{7000}}$;
%the 52 letters have |\mathcode|$\,x=x+\hbox{\hex{7100}}$.) \
%\TeX\ looks at the mathcode only when it is typesetting a character whose
%catcode is 11~(letter) or 12~(other), or when it encounters a character that
%is given explicitly as ^|\char|\<number>.
\ddanger 数学模式下字符的含义由 256 个``mathcode''的值这个表来定义;
这些表的单元可以用命令 |\mathcode| 来改变,就象类代码可以用 |\catcode| 改变%
一样(见第七章)。%
每个 mathcode 规定了类,族和字符位置,就象上面讲述的那样。%
例如,附录 B 中含有命令
\begintt
\mathcode`<="313C
\mathcode`*="2203
\endtt
它告诉 \TeX\ , 在数学模式下,把字符`|<|'看作关系符号(第 3 类),
在第 1 族的位置 \hex{3C} 上,
把星号`|*|'看作二元运算,在第 2 族的位置 3 上。%
|\mathcode`b| 的初始值为 \hex{7162};
因此 |b| 是第 1 族(italic)的字符 \hex{62}, 并且其族要随着 |\fam| 变化。%
(\1对不是字母和数字的所有字符,|INITEX| 开始设置为 |\mathcode|$\,x=x$。%
10 个数字为 |\mathcode|$\,x=x+\hbox{\hex{7000}}$;
52 个字母为 |\mathcode|$\,x=x+\hbox{\hex{7100}}$。)
只有当 \TeX\ 排版类代码为 11~(字母)或 12~(其它)的字符时,
或者当读入的字符由 |\char|\<number> 给出时,
它才用到 mathcode。

%\ddanger A |\mathcode| can also have the special value \hex{8000}, which
%causes the character to behave as if it has catcode~13 (active). Appendix~B
%uses this feature to make |'| ^^{apostrophe} expand to |^{|^|\prime||}| in a
%slightly tricky way. The mathcode of |'| does not ^^{active math character}
%interfere with the use of |'| in ^{octal} constants.
\ddanger |\mathcode| 也可以有一个特殊值 \hex{8000},
它使得字符的性质同类代码 13~(活动符)一样。%
附录 B 中利用这个性质略施小计,把 |'| 定义为 |{||\prime||}|。%
|'| 的 mathcode 不影响用八进制常数来使用 |'|。

%\ddanger The mathcode table allows you to refer indirectly to any character in
%any family, with the touch of a single key. You can also specify a math
%character code directly, by typing ^|\mathchar|, which is analogous to
%^|\char|. For example, the command `|\mathchar"1ABC|' specifies a
%character of class~1, family~10 (\hex A), and position \hex{BC}. A~hundred
%or so definitions like
%\begintt
%\def\sum{\mathchar"1350 }
%\endtt
%would therefore suffice to define the special symbols of plain \TeX\null. But
%there is a better way: \TeX\ has a primitive command ^|\mathchardef|,
%which relates to |\mathchar| just as ^|\chardef| does to |\char|.
%Appendix~B has a hundred or so definitions like
%\begintt
%\mathchardef\sum="1350
%\endtt
%to define the special symbols. A |\mathchar| must be between 0 and 32767
%(\hex{7FFF}).
\ddanger mathcode 允许你把单个键间接地指向任何族中任何字符。%
你还可以通过输入 |\mathchar| 直接给出数学字符,它类似于 |\char|。%
例如,命令`|\mathchar"1ABC|'给出了第 1 类,第 10 (\hex A) 族,位置 \hex{BC} 上%
的字符。%
因此象
\begintt
\def\sum{\mathchar"1350 }
\endtt
这样大概 100 个命令足以定义完 plain \TeX\ 中的特殊符号了。%
但是,有一个更好的方法: \TeX\ 有一个原始命令 |\mathchardef|,
它与 |\mathchar| 的关系就象 |\chardef| 与 |\char| 的一样。%
附录 B 给出了象
\begintt
\mathchardef\sum="1350
\endtt
这样的 100 个左右的定义来定义特殊符号。%
|\mathchar| 必须在 0 和 32767 (\hex{7FFF}) 之间。

%\ddanger A character of class~1, i.e., a ^{large operator} like |\sum|, will
%be vertically centered with respect to the axis when it is typeset. Thus,
%the large operators can be used with different sizes of type. This vertical
%adjustment is not made for symbols of the other classes.
\ddanger 当排版第 1 类字符,即象 |\sum| 这样的巨算符时,
它应该关于轴垂直居中。%
因此,巨算符可以使用不同的字体尺寸。%
这个垂直调整对其它类的符号是没有的。

%\ddanger \TeX\ associates classes with subformulas as well as with individual
%characters. Thus, for example, you can treat a complex construction as if
%it were a binary operation or a relation, etc., if you want to. The
%commands ^|\mathord|, ^|\mathop|, ^|\mathbin|, ^|\mathrel|, ^|\mathopen|,
%^|\mathclose|, and ^|\mathpunct| are used for this purpose; each of them
%is followed either by a single character or by a subformula in braces.
%For example, |\mathopen\mathchar"1234| is equivalent to |\mathchar"4234|,
%because |\mathopen| forces class~4 (opening). In the formula
%`|$G\mathbin:H$|', the ^{colon} is~treated as a binary operation.
%And Appendix~B constructs large opening symbols by
%defining ^|\bigl||#1| to be an abbreviation for
%\begintt
%\mathopen{\hbox{$\left#1 ...\right.$}}
%\endtt
%There's also an eighth classification, ^|\mathinner|, which is not
%normally used for individual symbols; fractions and ^|\left||...|^|\right|
%constructions are treated as ``inner'' subformulas, which means that
%they will be surrounded by additional space in certain circumstances.
%All other subformulas are generally treated as ordinary symbols,
%whether they are formed by |\overline| or |\hbox| or |\vcenter| or
%by simply being enclosed in braces. Thus, |\mathord| isn't really
%a necessary part of the \TeX\ language; instead of typing
%`|$1\mathord,234$|' you can get the same effect from `|$1{,}234$|'.
\ddanger  \TeX\ 把类型与子公式以及单个字符联系起来。%
因此,例如,如果需要的话,可以把一个复杂的式子看作二元运算或关系符号等等。%
为此要用到命令 |\mathord|, |\mathop|, |\mathbin|, |\mathrel|, |\mathopen|,
|\mathclose| 和 |\mathpunct|;
它们的每个后面都可以跟单个字符或者放在大括号中的子公式。%
例如,|\mathopen\mathchar"1234| 等价于 |\mathchar"4234|,
因为 |\mathopen| 把类强制变成第 4 类(开符号)。%
在公式`|$G\mathbin:H$|'中,冒号被看作二元运算。%
通过把 |\bigl||#1| 定义为
\begintt
\mathopen{\hbox{$\left#1 ...\right.$}}
\endtt
附录 B 构造了大量开符号。%
还有第 8 类,|\mathinner|, 它一般不用在单个符号上;
分数或 |\left||...||\right| 式子被看作``内部''子公式时,
就意味着在某些时候它们外面要有额外的间距。%
所有其它子公式一般被看作普通符号,不管它们是由 |\overline|,
|\hbox| 或 |\vcenter| 生成的,或者是封装在大括号中的。%
因此,|\mathord| 其实在 \TeX\ 语言中是不必要的;
不用输入`|$1\mathord,234$|', 照样可以用`|$1{,}234$|'得到同样的效果。

%\ddangerexercise Commands like |\mathchardef\alpha="010B| are used in
%Appendix~B to define the lowercase ^{Greek} letters. Suppose that you want
%to extend plain \TeX\ by putting ^{boldface math italic} letters
%in family~9, analogous to the normal math italic letters in family~1.
%\ (Such fonts aren't available in stripped down versions of \TeX, but
%let's assume that they exist.) \ Assume that the control sequence
%|\bmit| has been defined as an abbreviation for `|\fam=9|'; hence
%`|{\bmit b}|' will give a boldface math italic~|b|. What change to the
%definition of\/ |\alpha| will make |{\bmit\alpha}| produce a boldface~alpha?
%\checkequals\bmiexno\exno
%\answer |\mathchardef\alpha="710B|. Incidentally, |{\rm\alpha}| will
%then give a spurious result, because character position \hex{0B} of
%roman fonts does not contain an alpha; you should warn
%your users about what characters they are allowed to type under the
%influence of special conventions like ^|\rm|.
\ddangerexercise \1像 |\mathchardef\alpha="010B| 这样的命令在附录 B
中用来定义小写^{希腊字母}。假定你要扩展 plain \TeX\ ,
把加粗的数学意大利字母放在第 9 族,就像常用的数学意大利字母放在第 1 族一样。%
(这样的字体在最简版 \TeX\ 中不能用,但是我们假定它存在。)%
假设控制系列 |\bmit| 已经定义为 `|\fam=9|';
因此 `|{\bmit b}|' 将给出加粗的数学意大利字母 |b|。
怎样改变定义使得 |{\bmit\alpha}| 得到的是加粗的 alpha?
\checkequals\bmiexno\exno
\answer |\mathchardef\alpha="710B|。顺便说一下,
这样 |{\rm\alpha}| 将给出错误的结果,因为罗马字体的 \hex{0B} 位置不是 alpha 字符;
你应当警告用户在使用 ^|\rm| 时允许使用哪些字符。

%\ddanger ^{Delimiters} are specified in a similar but more complicated
%way. Each character has not only a~|\catcode| and a~|\mathcode| but also
%a~^|\delcode|, which is either negative (for characters that should not
%act as delimiters) or less than \hex{1000000}. In other words,
%nonnegative delcodes consist of six hexadecimal digits. The first three
%digits specify a ``small'' variant of the delimiter, and the last three
%specify a ``large'' variant. For example, the command
%\begintt
%\delcode`x="123456
%\endtt
%means that if the letter |x| is used as a delimiter, its small variant
%is found in position \hex{23} of family~1, and its large variant is found
%in position \hex{56} of family~4. If the small or large variant is
%given as |000|, however (position~0 of ^{family~0}), that variant is ignored.
%\TeX\ looks at the delcode when a character follows ^|\left| or ^|\right|,
%or when a character follows one of the ^|withdelims| commands; a
%negative delcode leads to an error message, but otherwise \TeX\ finds
%a suitable delimiter by first trying the small variant and then
%the large. \ (Appendix~G discusses this process in more detail.) \
%For example, Appendix~B contains the commands
%\begintt
%\delcode`(="028300  \delcode`.=0
%\endtt
%which specify that the small variant of a left parenthesis is found in
%position \hex{28} of family~0, and that the large variant is in position~0
%of family~3; also, a period has no variants, hence `|\left.|'\ will produce
%a ^{null delimiter}. There actually are several different left parenthesis
%symbols in family~3; the smallest is in position~0, and the others are
%linked together by information that comes with the font. All delcodes
%are~$-1$ until they are changed by a |\delcode| command.
\ddanger 分界符由一种类似但更复杂的方法给出。%
每个字符不但有一个 |\catcode| 和一个 |\mathcode|,
还有一个 |\delcode|, 它或者是负值(对不作为分界符使用的字符),
或者小于 \hex{1000000}。%
换句话说,非负的 delcode 由 6 位十六进制数组成。%
前三个规定了分界符的``小''组分,后三个规定了``大''组分。%
例如,命令
\begintt
\delcode`x="123456
\endtt
意思是,如果字母 |x| 被作为分界符使用,它的小组分在第 1 族的位置 \hex{23} 上,
大组分在第 4 族的位置 \hex{56} 上。%
但是如果大组分或小组分为 |000|~(第 0 族的位置 0 上),
此组分将忽略掉。%
当组分跟在 |\left| 或 |\right| 后面,或者在某个 |withdelims| 命令后时,
 \TeX\ 才用到 delcode;
负的 delcode 会产生一个错误信息,
但是其它时候 \TeX\ 通过先找小组分再找大组分而得到一个适当的分界符。%
(附录 G 讨论了这个过程的细节。)
例如,附录 B 中包含命令
\begintt
\delcode`(="028300  \delcode`.=0
\endtt
它表明,左圆括号的小组分在第 0 族的位置 \hex{28} 上,
大组分在第 3 族的位置 0 上;
还有,句点没有什么组分,因此`|\left.|'得到的是一个空分界符。%
实际上,在第 3 族中有几个不同的左括号符号;
最小的在位置 0,
其它的通过字体中的信息链接在一起。%
所有的 delcode 在被命令 |\delcode| 改变之前都是 $-1$。

%\ddangerexercise Appendix~B defines |\delcode`<| so that there is a
%shorthand notation for ^{angle brackets}. Why do you think Appendix~B
%doesn't go further and define |\delcode`{|?
%\answer If\/ |\delcode`{| were set to some nonnegative delimiter code, you
%would get no error message when you wrote something like `|\left{|'.
%This would be bad because strange effects would happen when certain
%subformulas were given as arguments to macros, or when they appeared
%in alignments. But it has an even worse defect, because a user who
%gets away with `|\left{|' is likely to try also `|\bigl{|', which
%fails miserably.
\ddangerexercise 附录 B 定义了 |\delcode`<|,使得^{角括号}有简短的名称。
想想为什么附录 B 不进一步定义出 |\delcode`{|?
\answer 如果给 |\delcode`{| 设定了非负的分界码,
在像 `|\left{|' 这样写时你不会得到任何错误信息。
这个定义是糟糕的,因为当某些子公式被作为宏参量,或者当它们出现在阵列中时,
将会有奇怪的效果。但它还有一个更加糟糕的缺点,
因为顺利使用 `|\left{|' 的用户将可能尝试使用 `|\bigl{|',
这将会失败得很惨。

%\ddanger A delimiter can also be given directly, as `^|\delimiter|\<number>'.
%In this case the number can be as high as \hex{7FFFFFF}, i.e., seven
%hexadecimal digits; the leading digit specifies a class, from 0 to~7,
%as in a |\mathchar|. For example, Appendix~B contains the definition
%\begintt
%\def\langle{\delimiter"426830A }
%\endtt
%and this means that ^|\langle| is an opening (class 4) whose small
%variant is \hex{268} and whose large variant is \hex{30A}. When |\delimiter|
%appears after |\left| or |\right|, the class digit is ignored; but
%when |\delimiter| occurs in other contexts, i.e., when \TeX\ isn't
%looking for a delimiter, the three rightmost digits are dropped and
%the remaining four digits act as a |\mathchar|. For example, the expression
%`|$\langle x$|' is treated as if it were `|$\mathchar"4268 x$|'.
\ddanger 分界符也可以直接给出,就象`|\delimiter|\<number>'这样。%
在这种情况下,数字最大为 \hex{7FFFFFF},
即七位十六进制数;领头的数字给出类,从 0 到 7, 象在 |\mathchar| 中一样。%
例如,附录 B 包含定义
\begintt
\def\langle{\delimiter"426830A }
\endtt
它的意思是,|\langle| 是一个开符号(第 4 类), 其小组分为 \hex{268},
大组分为 \hex{30A}。%
当 |\delimiter| 出现在 |\left| 或 |\right| 后面时,
类的数字忽略掉;
但是当 |\delimiter| 出现在其它情况下时,即,当 \TeX\ 不把它看作分界符时,
三个最右边的数字去掉,剩下的四个作为 |\mathchar| 出现。%
例如,式子`|$\langle x$|'将被看作`|$\mathchar"4268 x$|'。

%\ddangerexercise What goes wrong if you type
%`|\bigl\delimiter"426830A|'\thinspace?
%\answer Since |\bigl| is defined as a macro with one parameter,
%it gets just `|\delimiter|' as the argument. You have to write
%`|\bigl{\delimiter"426830A}|' to make this work. On the other hand,
%|\left| will balk if the following character is a left brace. Therefore
%it's best to have control sequence names for all delimiters.
\ddangerexercise \1输入`|\bigl\delimiter"426830A|'会产生什么错误?
\answer 由于 |\bigl| 被定义为有一个参数的宏,它将以 `|\delimiter|' 为参数。
你得写成 `|\bigl{\delimiter"426830A}|' 以让它正常执行。
另一方面,当随后的字符为左花括号时 |\left| 将出错。
因此,最好给所有分界符都定义一个控制系列名。

%\ddanger Granted that these numeric conventions for |\mathchar| and
%|\delimiter| are not beautiful, they sure do pack a lot of information into
%a small space. That's why \TeX\ uses them for low-level definitions inside
%formats. Two other low-level primitives also deserve to be mentioned:
%^|\radical| and ^|\mathaccent|. Plain \TeX\ makes ^{square root signs}
%and math accents available by giving the commands
%\begintt
%\def\sqrt{\radical"270370 }
%\def\widehat{\mathaccent"362 }
%\endtt
%and several more like them. The idea is that |\radical| is followed by
%a delimiter code and |\mathaccent| is followed by a math character code,
%so that \TeX\ knows the family and character positions for the symbols
%used in radical and accent constructions. Appendix~G gives precise
%information about the positioning of these characters. By changing the
%definitions, \TeX\ could easily be extended so that it would typeset a
%variety of different radical signs and a variety of different accent
%signs, if such symbols were available in the fonts.
%^^{surd signs, see radical}
\ddanger 当然 |\mathchar| 和 |\delimiter| 的这些数字约定不好看,
它们确实太臃肿了。%
这就是 \TeX\ 把它们作为格式中低级定义的原因。%
还有两个低级原始命令要提一下:
~|\radical| 和 |\mathaccent|。%
Plain \TeX\ 通过命令
\begintt
\def\sqrt{\radical"270370 }
\def\widehat{\mathaccent"362 }
\endtt
给出根号符号和数学重音,还有几个类似的命令。%
思路是,|\radical| 后面跟的是分界符代码,|\mathaccent| 后面跟的是%
数学组分代码,使得 \TeX\ 知道了用在根号和重音构造中的族和字符位置。%
附录 G 给出了这些字符位置的准确信息。%
通过改变定义, \TeX\ 很容易得扩展到排版各种不同的根号和各种不同的重音,
只要这样的符号在字体中可用。

%\ddanger Plain \TeX\ uses ^{family~1} for math italic letters, ^{family~2} for
%ordinary math symbols, and ^{family~3} for large symbols. \TeX\ insists that
%^^{math fonts}
%the fonts in families 2 and~3 have special ^|\fontdimen| parameters,
%which govern mathematical spacing according to the rules in Appendix~G\null;
%the ^|cmsy| and ^|cmex| ^{symbol fonts} have these parameters, so
%their assignment to families 2 and~3 is almost mandatory. \ (There is, however,
%a way to modify the parameters of any font, using the ^|\fontdimen| command.) \
%^|INITEX| initializes the mathcodes of all ^{letters} |A| to~|Z| and |a| to~|z|
%so that they are symbols of class~7 and family~1; that's why
%it is natural to use family~1 for math italics.  Similarly, the digits |0|
%to~|9| are class~7 and family~0.  None of the other families
%is treated in any special way by \TeX. Thus, for example, plain \TeX\ puts
%^{text italic} in family~4, slanted roman in family~5, bold roman in family~6,
%and typewriter type in family~7, but any of these numbers could be
%switched around. There is a macro ^|\newfam|, analogous to |\newbox|,
%that will assign symbolic names to families that aren't already used.
\ddanger Plain \TeX\ 把第 1 族用于数学 italic 字母,
第 2 族用于普通数学符号,第 3 族用于巨符号。%
 \TeX\ 的第 2 和 3 族中给出了一个叫 |\fontdimen| 的特殊参数,
由它按照附录 G 的规则来控制数学间距;
符号字体 |cmsy| 和 |cmex| 有这些参数,
因此,差不多把它们都指定到第 2 和 3 族中。%
(但是,用命令 |\fontdimen| 可用修改任何字体的参数。)
|INITEX| 初始化所有字母 |A| 到 |Z| 和 |a| 到 |z| 的 mathcode,
使得它们变成第 7 类第 1 族的符号;
这就是把第 1 族用于数学 italic 的原因。%
类似地,数字 |0| 到 |9| 是第 7 类第 0 族。%
 \TeX\ 不把其它族进行特殊对待了。%
因此,例如,~plain \TeX\ 把文本 italic 字体放在第 4 族,~slanted roman 字体%
放在第 5 族,bold roman 放在第 6 族,
typewriter 字体放在第 7 族,但是这些数字可以转换。%
有一个叫 |\newfam| 的宏,类似于 |\newbox|,
它把符号的名称指向未使用的族。

%\ddanger When \TeX\ is in horizontal mode, it is making a horizontal list;
%in vertical mode, it is making a vertical list. Therefore it should come
%as no great surprise that \TeX\ is making a ^{math list} when it is in
%^{math mode}. The contents of horizontal lists were explained in Chapter~14,
%and the contents of vertical lists were explained in Chapter~15; it's time
%now to describe what math lists are made of. Each item in a math list
%is one of the following types of things:\enddanger
\ddanger 当 \TeX\ 在水平模式时,它制作一个水平列;
在垂直模式下,制作一个垂直列。%
因此,在数学模式下当然得到的是数学列了。%
水平列的讨论见第十四章,
垂直列的讨论见第十五章;
现在讨论的是数学列。%
数学列中的每个项目都是下列某个类型:\enddanger

%\smallskip
%\item\bull an ^{atom} (to be explained momentarily);
\smallskip
\item\bull 原子(马上就讨论);

%\item\bull horizontal material (a rule or discretionary or penalty or
%``whatsit'');
\item\bull 水平内容(标尺,可断点,惩罚或``无名'');

%\item\bull vertical material (from |\mark| or |\insert| or |\vadjust|);
\item\bull 垂直内容(来自 |\mark|, |\insert| 或 |\vadjust|);

%\item\bull a glob of ^{glue} (from |\hskip| or |\mskip| or |\nonscript|);
\item\bull 粘连团(来自 |\hskip|, |\mskip| 或 |\nonscript|);

%\item\bull a ^{kern} (from |\kern| or |\mkern|);
\item\bull 紧排(来自 |\kern| 或 |\mkern|);

%\item\bull a ^{style change} (from |\displaystyle|, |\textstyle|, etc.);
\item\bull 样式改变(来自 |\displaystyle|, |\textstyle|, 等等);

%\item\bull a ^{generalized fraction} (from |\above|, |\over|, etc.);
\item\bull 广义分数(来自 |\above|, |\over|, 等等);

%\item\bull a ^{boundary} (from |\left| or |\right|);
\item\bull 分界符(来自 |\left| 或 |\right|);

%\item\bull a four-way ^{choice} (from ^|\mathchoice|).
\item\bull 四分支选择(来自 |\mathchoice|).

%\ddanger The most important items are called {\sl atoms}, and they have
%three parts: a {\sl^{nucleus}}, a {\sl^{superscript}}, and a {\sl^{subscript}}.
%For example, if you type
%\begintt
%(x_i+y)^{\overline{n+1}}
%\endtt
%in math mode, you get a math list consisting of five atoms:
%$($, $x_i$, $+$, $y$, and~$)^{\overline{n+1}}$. The nuclei of these atoms
%are $($, $x$, $+$, $y$, and~$)$; the subscripts are empty except for the
%second atom, which has subscript~$i$; the superscripts are empty except for the
%last atom, whose superscript is~$\overline{n+1}$. This superscript is
%itself a math list consisting of one atom, whose nucleus is~$n+1$; and that
%nucleus is a math list consisting of three atoms.
\ddanger \1最重要的项目称为{\KT{9}原子}, 有三个部分:
{\KT{9}核,上标,下标}。%
例如,在数学模式下,如果输入
\begintt
     (x_i+y)^{\overline{n+1}}
\endtt
就得到由五个原子组成的数学列:
$($, $x_i$, $+$, $y$ 和 $)^{\overline{n+1}}$。%
这些原子的核为 $($, $x$, $+$, $y$ 和 $)$;
除了第二项下标为 $i$ 外其它下标都是空的;
除了最后一项的上标为 $\overline{n+1}$ 其它上标都是空的。%
这个上标自己也是由一个原子组成的数学列,其核为 $n+1$;
这个核是由三个原子组成的数学列。

%\ddanger There are thirteen kinds of atoms, each of which might act
%differently in a formula; for example, `$($' is an Open atom because
%^^{atomic types, table}
%it comes from an opening. Here is a complete list of the different kinds:
%$$\halign{\indent#\hfil&\enskip#\hfil\cr
%Ord&is an ordinary atom like `$x$'\thinspace;\cr
%Op&is a large operator atom like `$\sum$'\thinspace;\cr
%Bin&is a binary operation atom like `$+$'\thinspace;\cr
%Rel&is a relation atom like `$=$'\thinspace;\cr
%Open&is an opening atom like `$($'\thinspace;\cr
%Close&is a closing atom like `$)$'\thinspace;\cr
%Punct&is a punctuation atom like `$,$'\thinspace;\cr
%Inner&is an inner atom like `$1\over2$'\thinspace;\cr
%Over&is an overline atom like `$\overline x$'\thinspace;\cr
%Under&is an underline atom like `$\underline x$'\thinspace;\cr
%Acc&is an accented atom like `$\hat x$'\thinspace;\cr
%Rad&is a radical atom like `$\sqrt2$'\thinspace;\cr
%Vcent&is a vbox to be centered, produced by |\vcenter|.\cr
%}$$
\ddanger 有十三种原子,每种在公式中的表现都不同;
例如,`$($'是一个开原子,因为它来自开符号。%
下面是不同种类的完整列表:
$$\halign{\indent#\hfil&\enskip#\hfil\cr
Ord&普通原子,如`$x$'\thinspace;\cr
Op&巨原子,如`$\sum$'\thinspace;\cr
Bin&二元运算原子,如`$+$'\thinspace;\cr
Rel&关系原子,如`$=$'\thinspace;\cr
Open&开原子,如`$($'\thinspace;\cr
Close&闭原子,如`$)$'\thinspace;\cr
Punct&标点原子,如`$,$'\thinspace;\cr
Inner&内部原子,如`$1\over2$'\thinspace;\cr
Over&上划线原子,如`$\overline x$'\thinspace;\cr
Under&下划线原子,如`$\underline x$'\thinspace;\cr
Acc&重音原子,如`$\hat x$'\thinspace;\cr
Rad&根号原子,如`$\sqrt2$'\thinspace;\cr
Vcent&|\vcenter| 生成的垂直居中 vbox。\cr
}$$

%\ddanger An atom's nucleus, superscript, and subscript are called its
%{\sl ^{fields}}, and there are four possibilities for each of these fields.
%A field can be\enddanger
\ddanger 原子的核,上标和下标称为{\KT{9}字段},
并且每个字段有四种可能;一个字段可以是\enddanger

%\smallskip
%\item\bull empty;
\smallskip
\hskip 20pt\item\bull 空的;

%\item\bull a math symbol (specified by family and position number);
\item\bull 数学符号(由族和位置给出);

%\item\bull a box; or
\item\bull 盒子;或者

%\item\bull a math list.
\item\bull 数学列。

%\smallskip\noindent
%For example, the Close atom $)^{\overline{n+1}}$ considered above has an
%empty subscript field; its nucleus is the symbol `$)$', which is
%character~\hex{28} of family~0 if the conventions of plain \TeX\ are
%in force; and its superscript field is the math list $\overline{n+1}$.
%The latter math list consists of an Over atom whose nucleus
%is the math list $n+1$; and that math list, in turn, consists of
%three atoms of types Ord, Bin, Ord.
\smallskip\noindent
例如,上面讨论的闭原子 $)^{\overline{n+1}}$ 的上标字段是空的;
核为符号`$)$', 如果采用 plain \TeX\ 的约定,那么它是第 0 族的字符 \hex{28};
其上标字段为数学列 $\overline{n+1}$。%
后一个数学列由一个上划线原子组成,其核为数学列 $n+1$;
而这个数学列由三个原子组成,分别是普通原子,二元运算原子,普通原子。

%\ddanger You can see \TeX's view of a math list by typing ^|\showlists|
%in math mode. ^^{internal list format}
%For example, after `|$(x_i+y)^{\overline{n+1}}\showlists|' your log
%file gets the following curious data:
%\begindisplay
%|\mathopen|\cr
%|.\fam0 (|\cr
%|\mathord|\cr
%|.\fam1 x|\cr
%|_\fam1 i|\cr
%\noalign{\penalty-500}
%|\mathbin|\cr
%|.\fam0 +|\cr
%\noalign{\penalty-500}
%|\mathord|\cr
%|.\fam1 y|\cr
%\noalign{\penalty-500}
%|\mathclose|\cr
%|.\fam0 )|\cr
%|^\overline|\cr
%|^.\mathord|\cr
%|^..\fam1 n|\cr
%|^.\mathbin|\cr
%|^..\fam0 +|\cr
%|^.\mathord|\cr
%|^..\fam0 1|\cr
%\enddisplay
%In our previous experiences with |\showlists| we observed that there can
%be boxes within boxes, and that each line in the log file is
%prefixed by dots to indicate its position in the hierarchy. Math lists
%have a slightly more complex structure; therefore a dot is used to denote
%the nucleus of an atom, a~`|^|' is used for the superscript field, and
%a~`|_|' is used for the subscript field. Empty fields are not shown. Thus,
%for example, the Ord atom~$x_i$ is represented here by three lines
%`|\mathord|', `|.\fam1 x|', and `|_\fam1 i|'.
\ddanger 可以通过在数学模式下输入 |\showlists| 看看 \TeX\ 是怎样处理的。%
例如,编译 `|$(x_i+y)^{\overline{n+1}}\showlists$|' 后,
在日志文件中得到下列古怪的信息:
\begindisplay
|\mathopen|\cr
|.\fam0 (|\cr
|\mathord|\cr
|.\fam1 x|\cr
|_\fam1 i|\cr
\noalign{\penalty-500}
|\mathbin|\cr
|.\fam0 +|\cr
\noalign{\penalty-500}
|\mathord|\cr
|.\fam1 y|\cr
\noalign{\penalty-500}
|\mathclose|\cr
|.\fam0 )|\cr
|^\overline|\cr
|^.\mathord|\cr
|^..\fam1 n|\cr
|^.\mathbin|\cr
|^..\fam0 +|\cr
|^.\mathord|\cr
|^..\fam0 1|\cr
\enddisplay
\1依照我们前面关于 |\showlists| 的经验可以看出,
有盒子套盒子,而且 log 文件中每行前面的点表明它所处的层次。%
数学列的结构略微复杂一些;
因此,一个点表示原子的核,一个`|^|'用来表示上标字段,
一个`|_|'用来表示下标字段。%
空字段没有显示。%
因此,例如,普通原子 $x_i$ 在这里用三行来表示:
`|\mathord|', `|.\fam1 x|'和`|_\fam1 i|'。

%\ddanger Certain kinds of atoms carry additional information besides their
%nucleus, subscript, and superscript fields: An Op atom will be marked
%`^|\limits|' or `^|\nolimits|' if the normal ^|\displaylimits|
%convention has been overridden; a Rad atom contains
%a delimiter field to specify what radical sign is to be used; and an Acc atom
%contains the family and character codes of the accent symbol.
\ddanger 某些种类的原子除了核,上标和下标字段外还有额外的东西:
在正常的 |\displaylimits| 被覆盖时,巨算符原子要标记上`|\limits|' 或 `|\nolimits|';
根号原子包含根号要用到的分界符字段;
重音原子包含重音符号的族和字符代码。

%\ddanger When you say ^|\hbox||{...}| in math mode, an Ord atom is placed
%on the current math list, with the hbox as its nucleus.  Similarly,
%^|\vcenter||{...}| produces a Vcent atom whose nucleus is a box. But in
%most cases the nucleus of an atom will be either a symbol or a math list.
%You can experiment with |\showlists| to discover how other things like
%fractions and mathchoices are represented internally.
\ddanger 当在数学模式下输入 |\hbox||{...}| 时,普通原子就放在了当前数学列上,
其中 hbox 为其核。%
类似地,|\vcenter||{...}| 得到的 Vcent 原子的核是一个盒子。%
但是在大多数情况下,原子的核是一个符号或一个数学列。%
可以用 |\showlists| 试试象分数和 mathchoice 在内部表示为什么东西了。

%\ddanger Chapter~26 contains complete details of how math lists are
%constructed.  As soon as math mode ends (i.e., when the closing `|$|'
%occurs), \TeX\ dismantles the current math list and converts it into a
%horizontal list.  The rules for this conversion are spelled out in
%Appendix~G\null. You can see ``before and after'' representations of such math
%typesetting by ending a formula with `|\showlists$\showlists|'; the first
%|\showlists| will display the math list, and the second will show the
%(possibly complex) horizontal list that is manufactured from it.
\ddanger 第二十六章包含了构造数学列的完整细节。%
一旦数学列结束了(即闭符号`|$|'出现了),  \TeX\ 将卸开当前数学列,
并且把它转换到水平列。%
这个转换的规则在附录 G 给出。%
可以对比这样的数学排版的``前后''表示:在公式结尾输入`|\showlists$\showlists|';
第一个 |\showlists| 显示的是数学列,第二个显示的是从它加工出的水平列(可能很复杂)。

\endchapter

The learning time is short. A few minutes gives the general flavor, and
typing a page or two of a paper generally uncovers most of the misconceptions.
\author ^{KERNIGHAN} and ^{CHERRY}, {\sl A System for %
  Typesetting Mathematics\/} (1975)
  % in {\sl Communications of the ACM\/} p152

\bigskip

Within a few hours (a few days at most)
a typist with no math or typesetting experience
can be taught to input even the most complex equations.
\author PETER J. ^{BOEHM}, {\sl Software and Hardware Considerations %
  for a\break Technical Typesetting System\/} (1976)
%  in {\sl IEEE Transactions on Professional Communication\/} PC-19, pp15--19

\vfill\eject\byebye
