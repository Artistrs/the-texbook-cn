% -*- coding: utf-8 -*-

\input macros

%\beginchapter Chapter 20. Definitions\\(also called Macros)
\beginchapter Chapter 20. 宏定义

\origpageno=199

%You can often save time typing math formulas by letting ^{control sequences}
%stand for constructions that occur frequently in a particular manuscript.
%For example, if some document uses the vector `$(x_1,\ldots,x_n)$' a lot,
%you can type
%\begintt
%\def\xvec{(x_1,\ldots,x_n)}
%\endtt
%and |\xvec| will henceforth be an abbreviation for `|(x_1,\ldots,x_n)|'.
%Complex displays \rlap{like}
%$$\def\xvec{(x_1,\ldots,x_n)}
%\sum_{\xvec\ne(0,\ldots,0)} \bigl(f\xvec+g\xvec\bigr)$$
%can then be typed simply as
%\begintt
%$$\sum_{\xvec\ne(0,\ldots,0)} \bigl(f\xvec+g\xvec\bigr)$$
%\endtt
%instead of in a tedious long form. By ^{defining a control sequence}
%like |\xvec|, you not only cut down on the number of keystrokes that you need
%to make, you also reduce your chances of introducing typographical errors
%and inconsistencies. ^^{abbreviations, see macros}
\1在 \TeX\ 文稿中,把经常用到的命令和符号的组合用控制系列来表示,
在输入时就可以节约时间了。%
例如,如果在文档中用到很多`$(x_1,\ldots,x_n)$',
那么就可以输入
\begintt
\def\xvec{(x_1,\ldots,x_n)}
\endtt
这样就把 |\xvec| 定义为`|(x_1,\ldots,x_n)|'了。%
因此,象
$$\def\xvec{(x_1,\ldots,x_n)}
\sum_{\xvec\ne(0,\ldots,0)} \bigl(f\xvec+g\xvec\bigr)$$
这样复杂的陈列公式就可以简单地输入为
\begintt
$$\sum_{\xvec\ne(0,\ldots,0)} \bigl(f\xvec+g\xvec\bigr)$$
\endtt
而不需要用冗长的输入。%
通过定义象 |\xvec| 这样的控制系列,不但可以减轻输入的工作量,
还可以在输入时减少出现错误和矛盾的机会。

%Of course, you usually won't be making a definition just to speed up the
%typing of one isolated formula; that doesn't gain anything, because time goes by
%when you're deciding whether or not to make a definition, and when you're
%typing the definition itself. The real payoff comes when some cluster of
%symbols is used dozens of times throughout a manuscript. A wise typist will
%look through a document before typing anything, thereby getting a feeling
%for what sorts of problems will arise and what sorts of definitions will
%be helpful. For example, Chapter~16 recommends that the control sequence
%|\Ahat| be defined at the beginning of any manuscript that makes frequent
%use of the symbol~$\hat A$.
当然,通常不要单单为快速输入一个公式而给出定义;
这样没什么好处,因为在你决定是否给出定义时,以及输入定义本身时就花掉时间了。%
只有当整个文稿中用到某些符号组合的次数超过一打时,才真正能得到好处。%
在排版前,聪明的人会浏览一下整个文档,对要出现的问题和要用到的定义有一个印象。%
例如,第十六章建议,对频繁使用符号 $\hat A$ 的文稿,在其开头给出控制系列 |\Ahat| 的定义。

%Abbreviations like |\xvec| turn out to be useful in many applications of
%computers, and they have come to be known as {\sl^{macros}\/} because they
%are so powerful; one little macro can represent an enormous amount of
%material, so it has a sort of macroscopic effect. System programs like
%\TeX\ that are designed to deal with macro definitions are said to {\sl
%expand\/} the user's macros; for example, |\xvec| expands into
%|(x_1,\ldots,x_n)|, and ^|\ldots| in turn is a macro that expands into
%|\mathinner{\ldotp\ldotp\ldotp}|.  Thus, |\xvec| is actually an
%abbreviation for `|(x_1,\mathinner{\ldotp\ldotp\ldotp},x_n)|'.  \ (The
%expansion stops here, because ^|\mathinner| is a primitive control
%sequence of \TeX, and because |\ldotp| has been defined with
%^|\mathchardef|; thus |\mathinner| and |\ldotp| are not macros.)
像 |\xvec| 这样的缩写在很多计算机应用中都很有用,
并且因为其威力巨大而被称为{\KT{10}宏};
一个很小的宏可以表示很多的内容,因此它有一种宏观效果。
像 \TeX\ 这样的系统程序,它处理用户定义的方法被称为{\KT{10}宏展开};
例如 |\xvec| 展开为 |(x_1,\ldots,x_n)|,而 |\ldots| 实际也是一个宏,
它接着展开为 |\mathinner{\ldotp\ldotp|\allowbreak |\ldotp}|。因此,
|\xvec| 实际上是 `|(x_1,\mathinner{\ldotp\ldotp|\allowbreak|\ldotp},x_n)|'
的缩写。(展开到此为止,因为 |\mathinner| 是 \TeX\ 的原始控制系列,
而且 |\ldotp| 已经由 |\mathchardef| 定义好了;
因此,|\mathinner| 和 |\ldotp| 不是宏。)

%\TeX\ users generally build up their own personal ^{library of macros} for
%things that they want to do in different documents. For example, it is common
%to have a file called |macros.tex| that contains definitions of your favorite
%special control sequences, perhaps together with commands that load your
%favorite special fonts, etc. If you begin a document with the command
%\begintt
%\input macros
%\endtt
%then \TeX\ will read all those definitions, ^^|\input|
%saving you all the trouble of retyping them. Of course, \TeX's memory is
%limited, and it takes time to read a file, so you shouldn't put thousands
%of definitions into |macros.tex|. A large collection of macro definitions
%(e.g., the set of definitions in Appendix~B) is called a {\sl^{format}\/}
%(e.g., ``plain \TeX\ format''); \TeX\ has a special way to input
%a format at high speed, assuming that the format doesn't change very often.
 \TeX\ 用户一般要建立自己个人的宏库,以便在不同的文档中使用。%
例如,一般建立一个叫 |macros.tex| 的文件,其中有你喜爱的专用控制系列的定义,
可能还有载入所喜欢的特殊字体的命令,等等。%
如果在文稿开头输入命令:
\begintt
\input macros
\endtt
那么 \TeX\ 将读入所有这些定义,省得你再次输入它们了。%
当然, \TeX\ 的内存有限,读入文件要花时间,
因此不要在 |macros.tex| 中放上千个定义。%
\1一个大的宏定义集(比如,附录 B 中的定义集)称为一个{\KT{10}格式}(比如,%
``plain \TeX\ 格式'');
如果格式不经常改变,那么 \TeX\ 能用一种特殊的方式快速输入一个格式。

%The |\xvec| and |\Ahat| examples apply to math formulas, but you can make
%good use of macro definitions even when you aren't doing any math at all.
%For example, if you are using \TeX\ for ^{business correspondence},
%you can have a |\yours| macro that stands for `Sincerely yours, A.~U.
%^{Thor}'. If you often write ^{form letters} you can have macros that
%generate entire sentences or paragraphs or groups of paragraphs.
%The ^{Internal Revenue Service} could, for example, make use of the
%following two macros:
%\begintt
%\def\badcheck{A penalty has been added because your
%  check to us was not honored by your bank.\par}
%\def\cheater{A penalty of 50\% of the underpaid tax
%  has been added for fraud.\par}
%\endtt
%Simple macro definitions, like these, start with `|\def|'; then comes the
%control sequence name, e.g., `|\badcheck|'; and then comes the replacement
%text enclosed in `|{|' and~`|}|'. The ^{braces} do not represent
%^{grouping} in this case; they simply show the extent of the replacement
%text in the definition.  You could, of course, define a macro that
%includes actual braces in its replacement text, as long as those braces
%match each other properly. For example, `|\def\xbold{{\bf x}}|'
%makes |\xbold| an abbreviation for `|{\bf x}|'.
例子 |\xvec| 和 |\Ahat| 是应用于数学公式的,但是即使与数学无关时,
宏定义也非常有用。%
例如,如果要用 \TeX\ 进行商业通信,
就可以用宏 |\yours| 了表示`Sincerely yours, A.~U. ^{Thor}'。%
如果经常用到格式信函,就可以用宏生成整个句子或段落,或者是一组段落。%
例如,税务机关可以用下面两个宏:
\begintt
\def\badcheck{A penalty has been added because your
  check to us was not honored by your bank.\par}
\def\cheater{A penalty of 50\% of the underpaid tax
  has been added for fraud.\par}
\endtt
象这样简单的宏定义以`|\def|'开始;
接着是控制系列的名字,比如`|\badcheck|';
再接着是要替换的文本,它们要封装在`|{|'和`|}|'中。%
在这种情况下,大括号不表示编组;
它们直接表明定义中要替换的文本的范围。%
当然,在替换文本中包含大括号的宏也是可以定义的,
只要这些大括号互相正确匹配即可。%
例如,`|\def\xbold{{\bf x}}|'把 |\xbold| 定义为`|{\bf x}|'。

%\exercise
%Write a |\punishment| macro that prints 100 lines containing the message
%`I must not talk in class.' \ [{\sl Hint:\/} First write a macro |\mustnt|
%that prints the message once; then write a macro |\five| that prints it
%five times.] \checkequals\punishexno\exno
%\answer |\def\mustnt{I must not talk in class.\par}|\parbreak
%        |\def\five{\mustnt\mustnt\mustnt\mustnt\mustnt}|\parbreak
%        |\def\twenty{\five\five\five\five}|\parbreak
%        |\def\punishment{\twenty\twenty\twenty\twenty\twenty}|\par
%\smallskip\noindent Solutions to more complicated problems of this type are
%discussed later.
\exercise
定义 |\punishment| 宏,它要将 `I must not talk in class.' 这句话打印 100 行。%
[{\KT{10}提示}:首先定义打印一次的宏 |\mustnt|,再定义把 |\mustnt| 输出五次的宏。]
 \checkequals\punishexno\exno
\answer |\def\mustnt{I must not talk in class.\par}|\parbreak
        |\def\five{\mustnt\mustnt\mustnt\mustnt\mustnt}|\parbreak
        |\def\twenty{\five\five\five\five}|\parbreak
        |\def\punishment{\twenty\twenty\twenty\twenty\twenty}|\par
\smallskip\noindent 稍后我们将讨论如何解决这种类型的更复杂的问题。

%\dangerexercise What is the expansion of\/ |\puzzle|, given the following
%definitions?
%\begintt
%\def\a{\b}
%\def\b{A\def\a{B\def\a{C\def\a{\b}}}}
%\def\puzzle{\a\a\a\a\a}
%\endtt
%\answer |ABCAB|. \ (The first |\a| expands into |A\def\a{B...}|; this redefines
%|\a|, so the second |\a| expands into |B...|, etc.) \ At least, that's what
%happens if\/ |\puzzle| is encountered when \TeX\ is building a list. But if
%|\puzzle| is expanded in an ^|\edef| or ^|\message| or something like that,
%we will see later that the interior |\def| commands are not performed
%while the expansion is taking place, and the control sequences following
%|\def| are expanded; so the result is an infinite string
%\begintt
%A\def A\def A\def A\def A\def A\def A\def A\def A...
%\endtt
%which causes \TeX\ to abort because the program's input stack is finite.
%This example points out that a control sequence (e.g., |\b|) need not be
%defined when it appears in the replacement text of a definition. The example
%also shows that \TeX\ doesn't expand a macro until it needs to.
\dangerexercise 对于下面给出的宏定义,|\puzzle| 宏展开后是什么?
\begintt
\def\a{\b}
\def\b{A\def\a{B\def\a{C\def\a{\b}}}}
\def\puzzle{\a\a\a\a\a}
\endtt
\answer |ABCAB|。(第一个 |\a| 展开为 |A\def\a{B...}|;它重定义了 |\a|,
因此第二个 |\a| 展开为 |B...|,依次类推。)%
至少,若 \TeX\ 是在构建列表时遇到 |\puzzle|,这就是所得到的结果。
但如果 |\puzzle| 是在类似 ^|\edef| 或 ^|\message| 的地方展开,
我们稍后将看到,当执行宏展开时,内部的 |\def| 命令将不会被执行,
而 |\def| 之后的控制系列却被展开;因此结果就是一个无穷的字符串:
\begintt
A\def A\def A\def A\def A\def A\def A\def A\def A...
\endtt
\TeX\ 的输入栈是有限的,因此这导致它中止执行。
此例子指出,控制系列(比如 |\b|)在宏定义的替换文本出现时可以还未定义。
此例子还显示 \TeX\ 仅在需要用到时才展开一个宏。

%\danger As soon as you get the hang of simple macros like those illustrated
%above, you will probably begin to think, ``Boy, wouldn't it be nice if I could
%have a macro in which some of the text in the expansion is changeable? I'd like
%to be able to stick different things into the middle of that text.'' Well,
%\TeX\ has good news for you: Control sequences can be defined in terms of
%{\sl^{parameters}}, and you can supply {\sl^{arguments}\/} that will be
%substituted for the parameters.
\danger 只要你略微思考一下上面例子中的简单宏,可能就会想:
``小子,如果宏展开后某些文本可以改变不是更好?
最好能把不同的东西放在那些文本中间。''
好, \TeX\ 带来了好消息:
控制系列可以用{\KT{10}参数}的方式来定义,
并且你可以给出要代替参数的{\KT{10}参量}。

%\danger For example, let's consider |\xvec| again. Suppose that you not only
%refer to `$(x_1,\ldots,x_n)$', but you also make frequent use of
%`$(y_1,\ldots,y_n)$' and other similar things. Then you might want to type
%\begintt
%\def\row#1{(#1_1,\ldots,#1_n)}
%\endtt
%\def\row#1{(#1_1,\ldots,#1_n)}%
%after which |\row x| will produce `$\row x$' and |\row y| will produce
%`$\row y$'. The symbol |#1| ^^{sharpsign} stands for the first parameter to
%the macro, and when you say `|\row|~|x|' the |x| is a so-called argument that
%will be inserted in place of the |#1|'s in the replacement text. In this
%case the argument consists of a single letter, |x|. You can also say
%|\row\alpha|, in which case the argument will be the control sequence
%^|\alpha|, and the result will be `$\row\alpha$'. If you want the argument
%to contain more than one symbol or control sequence, you can simply enclose
%it in ^{braces}; for example, |\row{x'}| yields $\row{x'}$. The argument
%in this case is |x'| (without the braces). Incidentally, if you say
%|\row{{x'}}|, you get $\row{{x'}}$; the reason is that only one pair of
%braces is stripped off when the argument is collected, and
%$({x'}_1,\ldots,{x'}_n)$ is what you get from
%|({x'}_1,\ldots,{x'}_n)| in math mode, according to the rules
%of Chapter~16. ^^{apostrophe}
\danger 例如,再来看看 |\xvec|。%
假定你不只需要`$(x_1,\ldots,x_n)$', 还需要频繁用到`$(y_1,\ldots,y_n)$'%
和其它类似的公式。%
那么你可能要输入
\begintt
\def\row#1{(#1_1,\ldots,#1_n)}
\endtt
\def\row#1{(#1_1,\ldots,#1_n)}%
\1此后,|\row x| 得到的就是`$\row x$', |\row y| 得到的是`$\row y$'。%
符号 |#1| 表示宏的第一个参数,并且当你输入`|\row|~|x|'时,|x| 就是所谓的参量,
它要代替的是替换文本中 |#1|。%
在这种情况下,参量由单个字母 |x| 组成。%
你还可以使用 |\row\alpha|, 这时参量是一个控制系列,所得到的结果是`$\row\alpha$'。%
如果要让参量包含不止一个符号或控制系列,
那么可以把它们封装在大括号中;
例如,|\row{x'}| 得到的是 $\row{x'}$。%
在这种情况下,参量是 |x'|~(没有大括号)。%
顺便说一句,如果输入 |\row{{x'}}|, 那么得到的是 $\row{{x'}}$;
原因是当取参量时,只剥去一对大括号,
并且按照第十六章的规则,$({x'}_1,\ldots,{x'}_n)$ 是从 |({x'}_1,\ldots,{x'}_n)| 的%
数学模式得到的。%

%\dangerexercise Continuing this example, what is the result of
%|$\row{\bf x}$|?
%\answer \def\row#1{(#1_1,\ldots,#1_n)}$\row{\bf x}$. Note that the
%subscripts are bold here, because the expansion |(\bf x_1,\ldots,\bf x_n)|
%doesn't ``turn off'' ^|\bf|. To prevent this, one should write
%|\row{{\bf x}}|; or (better), |\row\xbold|, in conjunction with
%|\def\xbold{{\bf x}}|.
\dangerexercise 继续讨论这个例子,|$\row{\bf x}$| 得到的是什么?
\answer \def\row#1{(#1_1,\ldots,#1_n)}$\row{\bf x}$。
注意因为宏展开 |(\bf x_1,\ldots,\bf x_n)| 并未 ``关闭'' ^|\bf|,这里的下标也被加粗了。
要避免这个结果,你应该写为 |\row{{\bf x}}|;或者写为 |\row\xbold|,
并补充 |\def\xbold{{\bf x}}|。

%\danger The notation `|#1|' suggests that there might be an opportunity to have
%more than one parameter, and indeed there is. You can write, for example,
%\begintt
%\def\row#1#2{(#1_1,\ldots,#1_#2)}
%\endtt
%\def\row#1#2{(#1_1,\ldots,#1_#2)}%
%after which `|\row xn|' would be the proper protocol for `$\row xn$'. There can
%be as many as nine parameters, |#1| to~|#9|, and when you use them you must
%number them~in order. For example, you can't use |#5| in a definition unless
%the previous parameter in~that definition was called |#4|. \ (This restriction
%applies only to the initial statement of parameters, before the replacement
%text starts; the stated parameters can be used any number of times, in any
%order, in the replacement text itself.)
\danger 符号`|#1|'意味着可以有不止一个参数,的确如此。%
例如,可以输入
\begintt
\def\row#1#2{(#1_1,\ldots,#1_#2)}
\endtt
\def\row#1#2{(#1_1,\ldots,#1_#2)}%
此后,`|\row xn|'就能得到`$\row xn$'了。%
至多可以有九个参数,|#1| 到 |#9|, 并且在使用时必须按顺序来。%
例如,如果定义中没有叫做 |#4| 的前一个参数,就不能用 |#5|。%
(这个限制只适用于替换文本开始前参数的初始声明;
在替换文本自己中,所声明的参数可以任意次序地使用任意次。)

%\danger A control sequence has only one definition at a
%time, so the second definition of\/ |\row| would supersede the first one if
%both had appeared in the same document. Whenever \TeX\ encounters a macro
%that it wants to expand, it uses the most recent definition. However,
%definitions are ^{local} to the group that contains them; old definitions
%will be restored in the usual way when a ^{group} ends.
\danger 一个控制系列同时只能有一个定义,因此,
如果在同一文稿中出现两个 |\row| 的定义,那么第二个就替掉第一个。%
只要 \TeX\ 遇见要展开的宏,它就使用最近的定义。%
但是,对于包含定义的组而言,定义是局部的;
当组结束时,旧定义将按照通常的方法恢复。

%\danger Caution: When you define a macro with simple parameters, as in
%these examples, you must be careful not to put blank spaces before the
%`|{|' that begins the replacement text. For example, `|\def\row #1 #2 {...}|'
%will not give the same result as `|\def\row#1#2{...}|', because the spaces
%after |#1| and~|#2| tell \TeX\ to look for arguments that are followed by
%spaces. \ (Arguments can be ``delimited'' in a fairly general way, as
%explained below.) \ But the space after |\row| is optional, as usual,
%because \TeX\ always disregards spaces after control words. After
%you have said `|\def\row#1#2{...}|', you are allowed to put spaces
%between the arguments (e.g., `|\row x n|'), because \TeX\ doesn't
%use single spaces as undelimited arguments.
\danger 注意:当象这些例子中用简单参数定义宏时,必须注意不要在替换文本%
前面的`|{|'前留下空格。%
例如,`|\def\row #1 #2 {...}|'与`|\def\row#1#2{...}|'得到的结果是不同的,
因为 |#1| 和 |#2| 后的空格就要求 \TeX\ 去寻找后面带空格的参量。%
(就象下面讨论的那样,参量可以用相当一般的方法来``分界''。)
但是 |\row| 的空格象通常那样是任意的,因为 \TeX\ 总是忽略掉控制词后的空格。%
在给出`|\def\row#1#2{...}|'后,就允许在参量之间放置空格(比如`|\row x n|'),
因为 \TeX\ 不把单个空格当作未分界的参量。

%\danger The following exercise is particularly recommended for people who
%want to learn to write \TeX\ macros. Even if you have gotten into the
%dangerous habit of skimming other exercises, you should try your hand
%at this one.
\danger 下面的练习特别推荐给要掌握 \TeX\ 宏的人。%
即使你略过其它的练习,也要亲手做下面这个。

%\dangerexercise Extending exercise 20.\punishexno, write a ``generalized
%punishment'' macro that has two parameters, so that |\punishment{run}{the
%halls}| will produce 100 paragraphs that say `I~must not run in the halls.'
%\answer The catch is that the parameters have to percolate down to the
%|\mustnt| macro, if you extend the previous answer:
%\begintt
%\def\mustnt#1#2{I must not #1 in #2.\par}
%\def\five#1#2{\mustnt{#1}{#2}...\mustnt{#1}{#2}}
%\def\twenty#1#2{\five{#1}{#2}...\five{#1}{#2}}
%\def\punishment#1#2{\twenty{#1}{#2}...\twenty{#1}{#2}}
%\endtt
%When you pass parameters from one macro to another in this way, you need to
%enclose them in braces as shown. But actually this particular solution
%punishes \TeX\ much more than it needs to, because it takes a lot of
%time to copy the parameters and read them again and again. There's a
%much more efficient way to do the job, by defining control sequences:
%\begintt
%\def\mustnt{I must not \doit\ in \thatplace.\par}
%\def\punishment#1#2{\def\doit{#1}\def\thatplace{#2}%
%  \twenty\twenty\twenty\twenty\twenty}
%\endtt
%and by defining |\five| and |\twenty| without parameters as before.
%You can also delve more deeply into \TeX nicalities, constructing solutions
%that are more efficient yet; \TeX\ works even faster when macros
%communicate with each other via ^{boxes}.
%^^{efficient macros} ^^{communication between macros}
%For example,
%\begintt
%\def\mustnt{\copy0 }
%\def\punishment#1#2{\setbox0=
%  \vbox{\strut I must not #1 in #2.\strut}%
%  \twenty\twenty\twenty\twenty\twenty}
%\endtt
%sets 100 identical paragraphs at high speed, because \TeX\ has to
%process the paragraph and break it into lines only once. It's much faster
%to ^{copy a box} than to build it up from scratch. \ (The ^{struts} in
%this example keep the interbaseline distances correct between boxed
%paragraphs, as explained in Chapter~12. Two struts are used, for if the
%message takes more than one line there will be a strut at both top
%and bottom. If it were known that each sentence will occupy only a single
%line, no struts would be needed, because interline glue is added as
%usual when a box created by |\copy| is appended to the current vertical list.)
\dangerexercise 扩展练习 20.\punishexno ,
定义一个适用更广的``处罚''宏,它有两个参数,且使用
|\punishment{run}{the halls}| 得到的是 100 句 `I~must not run in the halls.'。
\answer 陷阱在于,如果你在之前答案的基础上改进,
参数必须逐级传入到 |\mustnt| 宏的定义中:
\begintt
\def\mustnt#1#2{I must not #1 in #2.\par}
\def\five#1#2{\mustnt{#1}{#2}...\mustnt{#1}{#2}}
\def\twenty#1#2{\five{#1}{#2}...\five{#1}{#2}}
\def\punishment#1#2{\twenty{#1}{#2}...\twenty{#1}{#2}}
\endtt
当你用这种方式将参数从一个宏传递到另一个宏,你需要如上所示将参数包围在花括号中。
但实际上此种解法太过为难 \TeX 了,因为它将让 \TeX\ 费时地一次次复制并读取参数。
还有一种更有效率的解决方法,即定义下面的控制系列:
\begintt
\def\mustnt{I must not \doit\ in \thatplace.\par}
\def\punishment#1#2{\def\doit{#1}\def\thatplace{#2}%
  \twenty\twenty\twenty\twenty\twenty}
\endtt
而且像之前那样将 |\five| 和 |\twenty| 定义为无参数的宏。
通过深入探究 \TeX\ 技术,你还可以构造更加有效率的解法;
当宏与宏之间利用^{盒子}传递内容时,\TeX\ 的运行速度快多了。
^^{efficient macros} ^^{communication between macros}
例如,
\begintt
\def\mustnt{\copy0 }
\def\punishment#1#2{\setbox0=
  \vbox{\strut I must not #1 in #2.\strut}%
  \twenty\twenty\twenty\twenty\twenty}
\endtt
将以高速设定 100 个相同的段落,因为 \TeX\ 只需处理段落并分段为行一次。
复制一个盒子比从零开始构建它速度快多了。(如同第 12 章中所解释的,
这个例子中的^{支架}用于让盒子中的段落保持正确的基线间距离。
我们需要两个支架,因为如果消息文本需占用超过一行,这将让顶部和底部都有支架。
如果已知道每个句子都只占用一行,则无需使用支架,
因为当用 |\copy| 制造的盒子被添加到当前垂直列时,行间粘连本来将会被加上。)

%\ninepoint % the rest of this chapter is all dangerous
%\ddanger \TeX\ also allows you to define macros whose parameters are delimited
%in quite a general way; you needn't always enclose arguments in braces.
%For example,
%\begintt
%\def\cs #1. #2\par{...}
%\endtt
%defines a control sequence |\cs| with two parameters, and its two arguments
%will be determined as follows: |#1| will consist of all tokens between
%|\cs| and the next subsequent appearance of `|.|\]' (period and space);
%|#2| will consist of all tokens between that `|.|\]' and the next
%|\par| token. \ (The ^|\par| might be given explicitly, or it might be
%generated by a blank line as explained in Chapter~8.) \ For example, when
%\TeX\ expands
%\begintt
%\cs You owe \$5.00. Pay it.\par
%\endtt
%the first argument is `|You owe \$5.00|' and the second is `|Pay it.|'.
%The period in `|\$5.00|' doesn't stop |#1|, in this example,
%because \TeX\ keeps going until finding a period that is followed immediately
%by a space.
\ninepoint % the rest of this chapter is all dangerous
\ddanger \1\TeX\ 还允许定义这样的宏,其参数用相当普遍的方法来分界;
你不需要总是把参量封装在大括号中。%
例如,
\begintt
\def\cs #1. #2\par{...}
\endtt
定义了一个控制系列 |\cs|, 它有两个参数,并且这两个参数如下确定出:
|#1| 由 |\cs| 和下一个随后出现的`|.|\]'\allowbreak(句点和空格)之间的所有内容组成;
|#2| 由这个`|.|\]'和其后出现的 |\par| 之间的所有内容组成。%
(|\par| 可以明确给出,也可以由第十八章讨论的空行生成。)
例如,当 \TeX\ 展开
\begintt
\cs You owe \$5.00. Pay it.\par
\endtt
时,第一个参量是`|You owe \$5.00|', 而第二个是`|Pay it.|'。%
本例中,`|\$5.00|'中的句点不能结束 |#1|,
因为 \TeX\ 要找的是后面紧跟着空格的句点。

%\ddanger Furthermore, an argument will not stop when its delimiter is enclosed
%in braces, because that would produce unbalanced braces. For example, in
%\begintt
%\def\cs #1.#2\par{...}
%\endtt
%the first argument is now delimited by a single period, so |#1| would be
%`|You owe \$5|' and the |#2| would be `|00. Pay it.|' if\/ |\cs| were
%invoked as above. But
%\begintt
%\cs You owe {\$5.00}. Pay it.\par
%\endtt
%satisfactorily hides the first period, making it part of argument |#1|,
%which becomes \hbox{`|You owe {\$5.00}|'}.
\ddanger 还有,如果参量的分界符被封装在大括号时,这个参量不在此结束,
因为这样就得到了不匹配的大括号。例如,在
\begintt
\def\cs #1.#2\par{...}
\endtt
中,现在第一个参量到一个单句点处结束,因此,如果像上面那样调用 |\cs|,
|#1| 为 `|You owe \$5|',而 |#2| 为 `|00. Pay it.|'。但是,
\begintt
\cs You owe {\$5.00}. Pay it.\par
\endtt
圆满地藏起第一个句点,使得它不能结束参量 |#1|,
此时 |#1| 就是 \hbox{`|You owe {\$5.00}|'}。

%\ddanger If you are designing a format for mathematical papers, you will
%probably want to include a macro for the statement of ^{theorems},
%definitions, lemmas, corollaries, and such things. For example, you might
%want to typeset a statement like\enddanger
\ddanger 如果你要设计一个数学论文的格式,那么可能要有陈述定理,定义,引理,
推论等东西的宏。%
例如,可能要排版象下面这样的陈述:
\enddanger

%\proclaim Theorem 1. \TeX\ has a powerful macro capability.\par
%\noindent from the input
%\begintt
%\proclaim Theorem 1. \TeX\ has a powerful macro capability.\par
%\endtt
%In fact, plain \TeX\ includes a ^|\proclaim| macro that does just that;
%its definition is
%\begintt
%\def\proclaim #1. #2\par{\medbreak
%  \noindent{\bf#1.\enspace}{\sl#2}\par\medbreak}
%\endtt
%^^{enunciations, see proclaim} ^^{enspace}
%so the arguments are delimited exactly as in our first |\cs| example. The
%replacement text here uses |\medbreak| to separate the proclaimed
%paragraph from what precedes and follows; the title of the proclamation is
%set in bold face type, while the text itself is set slanted.  \ (The
%actual definition of\/ |\proclaim| in Appendix~B is not quite the same as
%this; the final |\medbreak| has been modified so that a break between
%pages will be discouraged immediately following the statement of a
%theorem. Hence a short theorem will tend to appear at the top of a page
%rather than at the bottom.)
\proclaim Theorem 1. \TeX\ has a powerful macro capability.\par
\noindent 它用下列方法得到:
\begintt
\proclaim Theorem 1. \TeX\ has a powerful macro capability.\par
\endtt
实际上,~plain \TeX\ 中有一个象这样的宏 |\proclaim|;
其定义为
\begintt
\def\proclaim #1. #2\par{\medbreak
  \noindent{\bf#1.\enspace}{\sl#2}\par\medbreak}
\endtt
这样,参量就象我们 |\cs| 的第一个例子那样分界。%
这里的替换文本用 |\medbreak| 把要 proclaim 的段落与其前后内容分开;
proclaim 的题目设置为 bold 字体,而文本设置为 slanted。%
(附录 B 中 |\proclaim| 的实际定义与它不完全一样;
为了使分页不出现在定理陈述紧后面,而修改了最后的 |\medbreak|。%
因此,短小的定理更倾向于放在页面顶部而不是底部。)

%\ddanger By making changes to the |\proclaim| macro, you can change the
%format of all the proclamations in your paper, without changing the text
%of the paper itself. For example, you could produce something like\enddanger
\ddanger 通过改变宏 |\proclaim|, 你可以改变论文中所有 proclaim 的格式,
而不需要改动论文自己的文本。%
例如,可以得到象下面这样的东西:
\enddanger

%\medbreak
%\font\tencsc=cmcsc10
%\noindent {\tencsc Theorem 1:}\enspace
%{\it\TeX\ has a powerful macro capability.}
%\goodbreak\medbreak\noindent
%by making simple alterations to the replacement text of\/ |\proclaim|,
%assuming that you have a ``^{caps and small caps}'' font. \TeX\ is
%intended to support ^{higher-level languages for composition} in which all of
%the control sequences that a user actually types are macros rather than
%\TeX\ primitives. The ideal is to be able to describe important classes of
%documents in terms of their components, without mentioning actual fonts or
%point sizes or details of spacing; a single ^{style-independent document}
%^^{format-independent document}^^{generic coding}
%can then be set in many different styles.
\medbreak
\font\tencsc=cmcsc10
\noindent {\tencsc Theorem 1:}\enspace
{\it\TeX\ has a powerful macro capability.}
\goodbreak\medbreak\noindent
\1这只需要直接修改 |\proclaim| 的替换文本即可,我们假定你有所用到的字体。%
 \TeX\ 希望支持复合的高级语言,其中用户使用的所有的控制系列都是宏,
而不是 \TeX\ 原始命令。%
思路是要能用几个要素来描述出重要类型的文档,
而不涉及到实际的字体,字体尺寸或详细的间距;
因此,可以用很多不同的字体来设置一个与字体无关的文档。

%\ddanger Now that we have seen a number of examples, let's look at the
%precise rules that govern \TeX\ macros. Definitions have the general form
%\begindisplay
%|\def|\<control sequence>\<parameter text>|{|\<replacement text>|}|
%\enddisplay
%where the \<parameter text> contains no ^{braces}, and where all occurrences
%of |{| and |}| in the \<replacement text> are properly nested. Furthermore
%the |#| symbol has a special significance: In the \<parameter text>, the
%first appearance of |#| must be followed by~|1|, the next by~|2|, and
%so on; up to nine |#|'s are allowed. In the \<replacement text> each~|#|
%must be followed by a digit that appeared after~|#| in the \<parameter text>,
%or else the~|#| should be followed by another~|#|. The latter case stands
%for a single~|#| token when the macro is expanded; the former case stands
%for insertion of the corresponding argument.
\ddanger 现在我们已经看到一些例子,接下来我们讨论 \TeX\ 宏的准确规则。%
定义的一般形式为
\begindisplay
|\def|\<control sequence>\<parameter text>|{|\<replacement text>|}|
\enddisplay
其中 \<parameter text> 不包含大括号,
并且在 \<replacement text> 中所出现的 |{| 和 |}| 要正确嵌套。%
还有,符号 |#| 有特殊含义:
在 \<parameter text> 中,第一个出现的 |#| 后面必须跟 |1|,
下一个要跟 |2|, 等等;
只允许九个 |#|。%
在 \<replacement text> 中,每个 |#| 后面必须跟一个在 \<parameter text> 中%
的 |#| 后面出现过的数字,或者一个 |#| 后面跟一个 |#|。%
表示当宏展开时,后一种情况表示一个单个 |#|;
前一种情况表示插入相应的参量。

%\ddanger For example, let's consider a ``random'' definition that doesn't
%do anything useful except that it does exhibit \TeX's rules. The definition
%\begintt
%\def\cs AB#1#2C$#3\$ {#3{ab#1}#1 c##\x #2}
%\endtt
%says that the control sequence |\cs| is to have a parameter text consisting of
%nine tokens
%\begindisplay
%|A|$_{11}$, \ |B|$_{11}$, \ |#1|, \ |#2|, \ |C|$_{11}$, \
%  |$|$_3$, \ |#3|, \ \cstok{\char`$}, \ \]$_{10}$
%\enddisplay
%(assuming the ^{category codes} of plain \TeX), and a replacement text
%of twelve tokens
%\begindisplay
%|#3|, \ |{|$_1$, \ |a|$_{11}$, \ |b|$_{11}$, \ |#1|, \
%  |}|$_2$, \ |#1|, \ \]$_{10}$, \ |c|$_{11}$, \ |#|$_6$, \ \cstok{x}, \ |#2|.
%\enddisplay
%Henceforth when \TeX\ reads the control sequence |\cs| it will expect that
%the next two tokens will be |A|$_{11}$ and |B|$_{11}$ (otherwise you will
%get the error message `|Use| |of| |\cs| |doesn't| |match| |its|
%|definition|'); then comes argument~|#1|, followed by argument~|#2|,
%then~|C|$_{11}$, then~|$|$_3$, then argument~|#3|, then |\$|, and finally
%a space token. It is customary to use the word ``argument'' to mean the
%string of tokens that gets substituted for a parameter; parameters appear
%in a definition, and arguments appear when that definition is used. \ (For
%the purposes of these rules, we are extending Chapter~7's definition of
%^{token}: In addition to control sequences and (character code, category
%code) pairs, \TeX\ also recognizes ``^{parameter tokens},'' denoted here
%by |#1|~to~|#9|. Parameter tokens can appear only in token lists for macros.)
\ddanger 例如,我们讨论一个``随便的''定义,
它只是为了展示 \TeX\ 的规则。%
定义
\begintt
\def\cs AB#1#2C$#3\$ {#3{ab#1}#1 c##\x #2}
\endtt
表示,控制系列 |\cs| 有一个参数文本,它由 9 个记号组成(假定采用 plain \TeX\ 的类代码):
\begindisplay
|A|$_{11}$, \ |B|$_{11}$, \ |#1|, \ |#2|, \ |C|$_{11}$, \
  |$|$_3$, \ |#3|, \ \cstok{\char`$}, \ \]$_{10}$
\enddisplay
有一个替换文本,由 12 个记号组成:
\begindisplay
|#3|, \ |{|$_1$, \ |a|$_{11}$, \ |b|$_{11}$, \ |#1|, \
  |}|$_2$, \ |#1|, \ \]$_{10}$, \ |c|$_{11}$, \ |#|$_6$, \ \cstok{x}, \ |#2|.
\enddisplay
因此,当 \TeX\ 读入控制系列 |\cs| 后,要读入的下两个记号应该是 |A|$_{11}$ 和%
~|B|$_{11}$~(否则就会得到错误信息`|Use| |of| |\cs| |doesn't| |match| |its|
|definition|');
接下来是参量 |#1|, 参量 |#2|, ~|C|$_{11}$, ~|$|$_3$, 再接着是参量 |#3|,
|\$|, 最后是空格记号。%
习惯上用``参量''这个词表示要替换参数的记号串;
参数出现在定义中,而当使用定义时,出现的是参量。%
(为此,我们要扩展第七章记号的定义:除了控制系列和(字符代码,类代码)对外,
 \TeX\ 还有``参数记号'', 在这里它用 |#1| 到 |#9| 来 表示。%
参数记号只出现在宏的记号列中。

%\ddanger How does \TeX\ determine where an argument stops, you ask.
%Answer: There are two cases. A {\sl^{delimited parameter}\/} is followed
%in the \<parameter text> by one or more non-parameter tokens, before
%reaching the end of the parameter text or the next parameter token;
%in this case the corresponding argument is the shortest (possibly empty)
%sequence of tokens with properly nested |{...}| groups that is followed in
%the input by this particular list of non-parameter tokens. \ (Category
%codes and character codes must both match, and control sequence names
%must be the same.) \ An {\sl^{undelimited parameter}\/} is followed immediately
%in the \<parameter text> by a parameter token, or it occurs at the very end
%of the parameter text; in this case the corresponding argument is the
%next nonblank token, unless that token is `|{|', when the argument will
%be the entire |{...}| group that follows. In both cases, if the argument
%found in this way has the form `|{|\<nested tokens>|}|', where \<nested
%tokens> stands for any sequence of tokens that is properly nested with
%respect to braces, the outermost braces enclosing the argument are removed
%and the \<nested tokens> will remain. For example, let's continue with
%|\cs| as defined above and suppose that the subsequent text contains
%\begintt
%\cs AB {\Look}C${And\$ }{look}\$ 5.
%\endtt
%Argument |#1| will be the token \cstok{Look}, since |#1| is an undelimited
%parameter (it is followed immediately by~|#2| in the definition);
%in this case \TeX\ ignores the blank space after |B|, and strips the
%braces off~of |{\Look}|. Argument~|#2| will be empty, since |C$| follows
%immediately. And argument~|#3| will be the thirteen tokens corresponding
%to the text |{And\$|\]|}{look}|, because |#3| is to be followed by `|\$|\]',
%and because the first occurrence of `|\$|\]' is within braces. Even though
%argument~|#3| begins with a left brace and ends with a right brace, the
%braces are not removed, since that would leave the unnested tokens `|And\$
%}{look|'.  The net effect then, after substituting arguments for
%parameters in the replacement text, will be that \TeX\ will next read the
%token list
%\begintt
%{And\$ }{look}{ab\Look}\Look|]c#\x5.
%\endtt
%The space \] here will be part of the resulting token list, even though
%it follows the control word |\Look|, because ^{spaces} are removed
%after ^{control word} tokens only when \TeX\ first converts input lines to
%token lists as described in Chapter~8.
\ddanger 问: \TeX\ 怎样才能知道参量在哪里结束?
答:分两种情况。%
在 \<parameter text> 中,{\KT{9}分界的参数}要跟一个或多个非参数记号,
才能到达参数文本的结尾或下一个参数记号;
在这种情况下,相应的参量是长度最短(可能是空)的记号序列,并且组的嵌套要正确,
后面跟的是在输入中的非参数记号列。%
(类代码和字符代码都要匹配,而且控制系列命名必须相同。)
在 \<parameter text> 中,{\KT{9}未分界的参数}后面要紧接一个参数记号,
或者要出现在参数文本的最结尾;
\1在这种情况下,相应的参量就是下一个非空的记号,如果此记号是`|{|', 这时%
参量是所跟的整个组 |{...}|。%
在所有两种情况下,如果用这两种方法找到的的参量的形式为`|{|\<nested tokens>|}|',
其中 \<nested tokens>~ 表示关于大括号正确嵌套的任意记号列,
那么就去掉封装记号的最外层的盒子,只留下 \<nested tokens>。%
例如,我们继续讨论上面的例子 |\cs|, 假定接下来的文本包含
\begintt
\cs AB {\Look}C${And\$ }{look}\$ 5.
\endtt
参量 |#1| 将是记号 \cstok{Look}, 因为 |#1| 是未分界的参数(在定义中,
它后面紧跟的是 |#2|);
在这种情况下, \TeX\ 将忽略掉 |B| 后面的空格,并且剥去 |{\Look}| 外面的大括号。%
参量 |#2| 将是空的,因为 |C$| 紧跟着。%
而参量 |#3| 将是对应于文本 |{And\$|\]|}{look}| 的 13 个记号,
因为 |#3| 后面跟着`|\$|\]',
并且因为第一个出现的`|\$|\]'在大括号里面。%
即使参量 |#3| 以左大括号开头,以右大括号结尾,这些大括号也不会被剥去,
因为那样得到的是未嵌套的记号`|And\$ }{look|'。%
因此,总的结果是,在替换文本中用参量替换参数后, \TeX\ 下一个要读入的记号列为
\begintt
{And\$ }{look}{ab\Look}\Look|]c#\x5.
\endtt
这里的空格 \] 是得到的记号列的一部分,即使它跟在控制词 |\Look|, 因为只有当%
 \TeX\ 首先把输入行转换为记号列时控制词后面的空格才去掉,就象第八章讨论的那样。%

%\ddangerexercise The example definition of\/ |\cs| includes a |##| in
%its replacement text, but the way |##| is actually used in that example
%is rather pointless. Give an example of a definition where |##| serves
%a useful purpose. ^^{sharp sharp}
%\answer The |##| feature is indispensable when the replacement text of
%a definition contains other definitions. For example, consider
%\begintt
%\def\a#1{\def\b##1{##1#1}}
%\endtt
%after which `|\a!|'\ will expand to `|\def\b#1{#1!}|'. We will see later
%that |##| is also important for alignments; see, for example, the definition
%of\/ |\matrix| in Appendix~B.
\ddangerexercise 在定义 |\cs| 的例子中,替换文本里面包含一个 |##|,
但是在此例子中这样使用 |##| 没什么意义。给出一个让 |##| 有意义的例子。
\answer 如果宏定义的替换文本中还包含另一个定义,|##| 的这个功能就是不可或缺的。
比如在下面宏定义
\begintt
\def\a#1{\def\b##1{##1#1}}
\endtt
之后,`|\a!|' 将展开为 `|\def\b#1{#1!}|'。我们稍后将看到 |##|
对于对齐阵列也是很重要的;比如看附录 B 中 |\matrix| 的定义。

%\ddanger A special extension is allowed to these rules: If the very
%last character of the \<parameter text> is~|#|, so that this~|#| is
%immediately followed by~|{|, \TeX\ will behave as if the~|{| had been
%inserted at the right end of both the parameter text and the replacement text.
%For example, if you say `|\def\a#1#{\hbox to #1}|', the
%subsequent text `|\a3pt{x}|' will expand to `|\hbox to 3pt{x}|', because
%the argument of\/ |\a| is delimited by a left brace.
%^^{dimensions as arguments}
\ddanger 这些规则允许一个特殊的扩展:
如果 \<parameter text> 的最后一个字符是 |#|, 使得这个 |#| 后面紧跟着 |{|,
那么看起来就好像 \TeX\ 把 |{| 插入到参数文本的右结尾和替换文本的右结尾处。%
例如,如果给出`|\def\a#1#{\hbox to #1}|', 那么在后面文本中的`|\a3pt{x}|'将%
展开为`|\hbox to 3pt{x}|', 因为 |\a| 的参量以左大括号为定界符。

%\ddanger Tokens that precede the first parameter token in the \<parameter
%text> of a definition are required to follow the control sequence; in
%effect, they become part of the control sequence name. For example, the author
%might have said
%\begintt
%\def\TeX/{...}
%\endtt
%instead of defining ^|\TeX| without the slash. Then it would be necessary
%to type |\TeX/| each time the \TeX\ logo is desired, but the new definition
%would have the advantage that spaces are {\sl not\/} ignored after
%`|\TeX/|'. You can use this idea to define macros that are intended to
%be used in sentences, so that users don't have to worry about the possible
%disappearance of ^{spaces}.
\ddanger 在定义的 \<parameter text>  中,在第一个参数记号前面的记号要跟在控制系列%
后面;实际上,它们变成了控制系列名字的一部分。%
例如,作者可能给出过
\begintt
\def\TeX/{...}
\endtt
而没有使用无斜线的 |\TeX| 的定义。%
这样,每次要得到 \TeX\ 标志时必须输入 |\TeX/|,
但是新定义的优点是,`|\TeX/|'的空格{\KT{9}不}被忽略掉。%
可以利用这种思路来定义在句子中使用的宏,使得用户不必担心空格可能会被吃掉。

%\ddangerexercise Define a control sequence |\a| such that |\a{...}| expands
%to |\b{...}|, and such that \TeX\ gives an error message if\/ |\a| is not
%immediately followed by a left brace.
%\answer |\def\a#{\b}|.
\ddangerexercise 定义一个控制系列 |\a| 使得 |\a{...}| 展开为 |\b{...}|,
并且使得如果 |\a| 后面没有紧跟一个左花括号 \TeX\ 就给出错误信息。
\answer |\def\a#{\b}|。

%\ddanger Complicated macros have a habit of behaving differently from what
%you expect, when you first define them, even though \TeX's rules are
%not especially complicated. If you have trouble understanding why some
%|\def| doesn't work the way you think it should, help is available:
%You can set |\tracingmacros=1|, whereupon \TeX\ will write something in
%your log file whenever it expands a macro, and whenever it has read a
%macro argument. For example, if\/ ^|\tracingmacros| is positive when
%\TeX\ processes the |\cs| example above, it will put the following
%four lines into the log: ^^{debugging macros}
%\begintt
%\cs AB#1#2C$#3\$ ->#3{ab#1}#1 c##\x #2
%#1<-\Look
%#2<-
%#3<-{And\$ }{look}
%\endtt
\ddanger \1当你第一次定义复杂的宏时,结果常常与你所预想的不同,即使 \TeX\ 的规则%
不太复杂。%
如果不知道为什么 |\def| 不听你的话,可以使用帮助:
你可以设置 |\tracingmacros=1|, 这样 \TeX\ 就在它展开宏和读入宏的参量时在你的%
~log 文件中写入一些东西。%
例如,如果在 \TeX\ 处理上面的宏 |\cs| 时 |\tracingmacros| 是正值,
它就把下列四行放到 log 文件中:
\begintt
\cs AB#1#2C$#3\$ ->#3{ab#1}#1 c##\x #2
#1<-\Look
#2<-
#3<-{And\$ }{look}
\endtt

%\ddanger In all of the rules stated above, `|{|' and `|}|' and `|#|' stand
%for any characters whose ^{category codes} are respectively 1, 2, and 6 in
%the token list when \TeX\ reads the macro definition; there's nothing
%sacred about the particular symbols that plain \TeX\ uses to denote
%grouping and parameters.  You can even make use of several different
%characters with these category codes, all at the same time.
\ddanger 在上述所有规则中,`|{|', `|}|'和`|#|'表示 \TeX\ 读入宏定义时%
记号列中类代码分别为 1, 2 和 6 的任意字符;
~plain \TeX\ 表示编组和参数的这些特殊符号没有什么不能改变的。%
你甚至可以同时使用同类代码的几个不同字符。

%\ddangerexercise Suppose that `|[|', `|]|', and `|!|'\ have the respective
%catcodes 1,~2, and~6, as do `|{|',~`|}|', and~`|#|'. See if you can guess
%what the following definition means:
%\begintt
%\def\!!1#2![{!#]#!!2}
%\endtt
%What token list will result when `|\! x{[y]][z}|' is expanded?
%\answer Let's go slowly on this one, so that the answer will give enough
%background to answer all similar questions. The \<parameter text> of the
%definition consists of the three tokens |#1|, |#2|, |[|$_1$; the
%\<replacement text> consists of the six tokens |{|$_1$, |#|$_6$, |]|$_2$,
%|!|$_6$, |#2|, |[|$_1$. \ (When two tokens of category~6 occur in the
%replacement text, the character code of the second one survives; the
%character code of a category-6 character is otherwise irrelevant. Thus,
%`|\def\!#1!2#[{##]!!#2]|' would produce an essentially identical
%definition.) \ When expanding the given token list, argument~|#1| is
%|x|$_{11}$, since it is undelimited. Argument~|#2| is delimited by~|[|$_1$,
%which is different from~|{|$_1$, so it is set provisionally to |{[y]]|;
%but the outer ``braces'' are stripped off, so |#2|~reduces to the
%three tokens |[|$_1$, |y|$_{11}$,~|]|$_2$. The result of the expansion
%is therefore
%\begindisplay
%|{|$_1$ |#|$_6$ |]|$_2$ |!|$_6$ |[|$_1$ |y|$_{11}$ |]|$_2$
%  |[|$_1$ |z|$_{11}$ |}|$_2$.
%\enddisplay
%Incidentally, if you display this with ^|\tracingmacros||=1|, \TeX\ says
%\begintt
%\!!1#2[->{##]!!#2[
%#1<-x
%#2<-[y]
%\endtt
%Category codes are not shown, but a character of category~6 always
%appears twice in succession. A parameter token in the replacement text
%uses the character code of the final parameter in the parameter text.
%^^{token lists, as displayed by TeX}
\ddangerexercise 假定 `|[|'、`|]|' 和 `|!|' 的类别码分别为 1、2 和 6,
就像 `|{|'、`|}|' 和 `|#|' 一样。看看下面定义的意思是什么:
\begintt
\def\!!1#2![{!#]#!!2}
\endtt
当展开`|\! x{[y]][z}|'时得到的记号列是什么?
\answer 我们将慢慢解说这个问题,给出足以回答所有类似问题的来龙去脉。
这个定义的 \<parameter text> 由三个记号 |#1|、|#2|、|[|$_1$ 组成;
而 \<replacement text> 由六个记号 |{|$_1$、|#|$_6$、|]|$_2$、|!|$_6$、
|#2|、|[|$_1$ 组成。(当两个类别 6 的记号出现在替换文本时,
只留下第二个的字符码;类别 6 字符的字符码是无关紧要的。
因此,用 `|\def\!#1!2#[{##]!!#2]|' 给出的是本质上完全相同的定义。)%
在展开给定的记号列时,参量 |#1| 是非定界的,因此它是 |x|$_{11}$。
参量 |#2| 被 |[|$_1$ 定界(此时 |[|$_1$ 和 |{|$_1$ 不同),因此它被临时设定为 |{[y]]|;
但外层的``花括号''将被剥离,因此 |#2| 减少为三个记号 |[|$_1$、|y|$_{11}$、|]|$_2$。
从而宏展开的结果为
\begindisplay
|{|$_1$ |#|$_6$ |]|$_2$ |!|$_6$ |[|$_1$ |y|$_{11}$ |]|$_2$
  |[|$_1$ |z|$_{11}$ |}|$_2$.
\enddisplay
顺便说一下,如果你用 ^|\tracingmacros||=1| 显示这些,\TeX\ 将给出
\begintt
\!!1#2[->{##]!!#2[
#1<-x
#2<-[y]
\endtt
类别码不会显示出来,但类别 6 字符总是连续出现两次。
替换文本中的参数记号使用参数文本最后一个参数的字符码。
^^{token lists, as displayed by TeX}

%\ddanger In practice, we all make mistakes. And one of the most common
%typographic errors is to forget a~`|}|', or to insert an extra~`|{|',
%somewhere in an argument to a macro. If \TeX\ were to follow the rules
%blindly in such a case, it would have to keep absorbing more and
%more tokens in hopes of finding the end of the argument. But a mistyped
%argument is unending, like so many arguments in real life (sigh); so
%\TeX\ would have to go on until the end of the file, or (more likely)
%until tokens completely fill the computer's memory. In either case, a
%single typographical error would have ruined the run, and the user would
%be forced to start over. Therefore \TeX\ has another rule, intended
%to confine such errors to the paragraph in which they occur: {\sl The
%token `\thinspace^|\par|' is not allowed to occur as part of an argument},
%unless you explicitly tell \TeX\ that |\par| is OK. Whenever \TeX\ is
%about to include |\par| as part of an argument, it will abort the current
%macro expansion and report that a ``^{runaway} argument'' has been found.
\ddanger 在实践中,我们总是要犯错误的。%
而且最常见的排版错误就是在宏的参量中落掉一个`|}|'或者多插入一个`|{|'。%
如果在这种情况下 \TeX\ 盲目地遵循规则,就要因为要找到参量的结尾而不得不读入越来越%
多的记号。%
但是象实际使用中如此多参量,只要参量中有一个这样的笔误就无法结束(唉);
因此, \TeX\ 不得不一直读到文件结束,或者(更可能的是)直到记号把计算机内存都塞满为止。%
不论哪种情况,单个的排版失误将导致无法运行,并且用户被迫重新开始。%
因此 \TeX\ 还有另一个规则,用它把这样的错误限制在它所在的段落中:
{\KT{9}记号`\thinspace^|\par|'不允许作为参量的一部分而出现},
除非你明确告诉 \TeX\ 可以使用 |\par|。%
只要 \TeX\ 把 |\par| 作为参量的一部分而包括进来,它就终止当前宏的展开,
并且给出信息说,发现一个``失控的参量''。

%\ddanger If you actually want a control sequence to allow arguments with
%|\par| tokens, you can define it to be a ``long'' macro by saying
%`^|\long|' just before `|\def|'. For example, the |\bold| macro
%defined by
%\begintt
%\long\def\bold#1{{\bf#1}}
%\endtt
%is capable of setting several paragraphs in boldface type. \ (However,
%such a macro is not an especially good way to typeset bold text.
%It would be better to say, e.g.,
%\begintt
%\def\beginbold{\begingroup\bf}
%\def\endbold{\endgroup}
%\endtt
%because this doesn't fill \TeX's memory with a long argument.)
\ddanger 如果你真要定义一个控制系列,其参量允许出现记号 |\par|,
那么在`|\def|'紧前面加上`^|\long|'把它定义为一个``长''宏。%
例如,宏 |\bold| 的定义为
\begintt
\long\def\bold#1{{\bf#1}}
\endtt
它可以把几个段落都设置为 bold 字体。%
(但是,这样的宏并不是排版 bold 文本的特别好的方法。%
更好的是,比如
\begintt
\def\beginbold{\begingroup\bf}
\def\endbold{\endgroup}
\endtt
因为对长的参量,它不会把 \TeX\ 的内存塞满。)

%\ddanger The |\par|-forbidding mechanism doesn't catch all conceivable
%missing-brace errors, however; you might forget the |}| at the end of a
%|\def|, and the same problem would arise. In this case it's harder to
%confine the error, because |\par| is a useful thing in replacement
%texts; we wouldn't want to forbid |\par| there, so \TeX\ has another
%mechanism:  When a macro definition is preceded by `^|\outer|', the
%corresponding control sequence will not be allowed to appear in any place
%where tokens are being absorbed at high speed. ^^{forbidden control sequence}
%An |\outer| macro cannot appear in an argument (not even when |\par| is
%allowed), nor can it appear in the parameter text or the replacement text
%of a definition, nor in the ^{preamble} to an alignment, nor in ^{conditional
%text} that is being skipped over.  If an |\outer| macro does show up in
%such places, \TeX\ stops what it is doing and reports either a
%``runaway'' situation or an ``^{incomplete}'' conditional. The ^{end of an
%input file} or alignment template ^^{endtemplate}
%is also considered to be |\outer| in this sense; for example, a
%file shouldn't end in the middle of a definition. If you are designing a
%format for others to use, you can help them detect errors before too much
%harm is done, by using |\outer| with all control sequences that should
%appear only at ``quiet times'' within a document. For example, Appendix~B
%defines ^|\proclaim| to be |\outer|, since a user shouldn't be stating a
%theorem as part of a definition or argument or preamble.
\ddanger \1但是,禁用 |\par| 这个方法不能预防所有可能出现的落掉大括号的错误;
如果你在 |\def| 的结尾处落掉了 |}|,也要出同样的问题。%
在这种情况下,更难限制住错误,因为 |\par| 在替换文本中是一个有用的东西;
在这里我们不想禁用 |\par|, 因此, \TeX\ 提供了另一种方法:
当宏定义前面有`^|\outer|'时,相应的控制系列不允许出现在记号被高速读入的任何地方。%
|\outer| 宏不允许出现在参量中(即使允许出现 |\par| 它也不能出现),
也不允许出现在定义的参数文本或替换文本中,以及对齐阵列的导言中和要被跳过的条件文本中。%
如果 |\outer| 宏确实在这些地方出现了,那么 \TeX\ 就停下来,
并且给出``失控地点''或者一个``不完整''的条件。%
在此意义上,输入文件的结束也被看作是 |\outer| 的;
例如,文件不能在定义中间结束。%
如果你为他人设计了一个格式,通过在每个在文档中只``默默运行''的控制系列前面加上 |\outer|,
就可以检验出可能出现的错误。%
例如,附录 B 把 |\proclaim| 定义为 |\outer| 型,
因为用户不应该把定理陈述作为定义、参量或导言的一部分。%

%\ddanger We have now seen that |\def| can be preceded by |\long|
%or~|\outer|, and it can also be preceded by ^|\global| if the definition
%is supposed to transcend its group. These three prefixes can be applied
%to |\def| in any order, and they can even appear more than once.
%\TeX\ also has a ^|\gdef| primitive that is equivalent to |\global\def|.
%Thus, for example,
%\begintt
%\long\outer\global\long\def
%\endtt
%means the same thing as `|\outer\long\gdef|'.
\ddanger 我们现在已经知道,~|\def| 前面可以加 |\long| 或 |\outer|,
并且如果定义要延伸到组外还可以在前面加 |\global|。%
这三个前缀可以按任意次序放在 |\def| 前面,并且可以不止出现一次。%
 \TeX\ 还有一个原始命令 |\gdef|, 它等价于 |\global\def|。%
因此,例如,
\begintt
\long\outer\global\long\def
\endtt
与`|\outer\long\gdef|'的意思是一样的。

%\ddanger So far in this manual we have encountered several ways to
%assign a meaning to a control sequence. For example,
%$$\halign{\indent#\hfil\quad&#\hfil\cr
%|\font\cs=|\<external font name>&makes |\cs| a font identifier;\cr
%|\chardef\cs=|\<number>&makes |\cs| a character code;\cr
%|\countdef\cs=|\<number>&makes |\cs| a |\count| register;\cr
%|\def\cs...{...}|&makes |\cs| a macro.\cr
%\noalign{\medskip \hbox{It's time now to reveal another important command
%  of this type:} \medskip}
%|\let\cs=|\<token>&gives |\cs| the token's current meaning.\cr}$$
%^^|\let|
%If the \<token> is another control sequence, |\cs| will acquire the
%same significance as that control sequence. For example, if you say
%`|\let\a=\def|', you could then say `|\a\b...{...}|' to define a macro~|\b|,
%because |\a| would behave like \TeX's primitive |\def| command.
%If you say
%\begintt
%\let\a=\b \let\b=\c \let\c=\a
%\endtt
%you have interchanged the former meanings of\/ |\b| and |\c|. And if you say
%\begintt
%\outer\def\a#1.{#1:}
%\let\b=\a
%\endtt
%the effect is exactly the same as `|\outer\def\b#1.{#1:} \let\a=\b|'.
\ddanger 到现在为止,在本手册我们已经见过几种定义控制系列的方法。%
例如,
$$\halign{\indent#\hfil\quad&#\hfil\cr
|\font\cs=|\<external font name>&把 |\cs| 变成字体标识符;\cr
|\chardef\cs=|\<number>&把 |\cs| 变成字符代码;\cr
|\countdef\cs=|\<number>&把 |\cs| 变成 |\count| 寄存器;\cr
|\def\cs...{...}|&把 |\cs| 变成宏。\cr
\noalign{\medskip \hbox{现在要给出此类中的另一个重要的命令:
} \medskip}
|\let\cs=|\<token>&把记号当前的含义赋予给 |\cs|。\cr}$$
如果 \<token> 是另一个控制系列,|\cs| 就得到与那个控制系列同样的意义。%
例如,如果给出`|\let\a=\def|', 那么用`|\a\b...{...}|'就可以定义宏 |\b|,
因为 |\a| 的性质象 \TeX\ 的原始命令 |\def| 一样。%
如果我们给出
\begintt
\let\a=\b \let\b=\c \let\c=\a
\endtt
就把 |\b| 和 |\c| 中的原来的意思互换了。%
并且如果给出
\begintt
\outer\def\a#1.{#1:}
\let\b=\a
\endtt
其结果与`|\outer\def\b#1.{#1:} \let\a=\b|'完全一样。

%\ddanger If the \<token> in a |\let| is a single character---i.e.,
%if it is a (character code, category code) pair---then
%the control sequence will behave to a certain extent like that
%character; but there are some differences. For example, after
%`|\let\zero=0|' you can't use |\zero| in a numerical constant, because
%\TeX\ requires the tokens in a numerical constant to be digits, after
%macro expansion; |\zero| is not a macro, so it doesn't expand. However,
%such uses of\/ |\let| have their value, as we will see later.
\ddanger \1如果 |\let| 中的 \<token> 是单个字符——即如果它是(字符代码,类代码)%
对——那么在一定程度上控制系列就象此字符一样;
但是有一些不同。%
例如,给出`|\let\zero=0|'后,不能在数字常数中使用 |\zero|,
因为 \TeX\ 要求宏展开后数字常数中的记号是数字;
|\zero| 不是一个宏,因此它不能展开。%
但是,这样的 |\let| 也有其用处,我们将在后面讨论。

%\ddangerexercise Is there a significant difference between `|\let\a=\b|'
%and `|\def\a{\b}|'?
%\answer Yes indeed. In the first case, |\a| receives the meaning of\/~|\b|
%that is current at the time of the |\let|. In the second case, |\a|~becomes
%a~macro that will expand into the token~|\b| whenever |\a|~is used,
%so it has the meaning of\/~|\b| that is current at the time of use.
%You need |\let|, if you want to interchange the meanings of\/ |\a| and~|\b|.
\ddangerexercise `|\let\a=\b|' 和 `|\def\a{\b}|' 之间有无明显的区别?
\answer 确实有区别。在第一种情形中,|\a| 得到运行 |\let| 时 |\b| 的当前含义。
在第二种情形中,|\a| 成为一个每次使用时展开为 |\b| 的宏,
因此它得到在使用时 |\b| 的当前含义。
如果你想交换 |\a| 和 |\b| 的含义,你需要使用 |\let|。

%\ddangerexercise Experiment with \TeX\ to discover the answers to the
%following questions: (a)~If the control sequence ^|\par| has been redefined
%(e.g., `|\def\par{\endgroup\par}|'\thinspace), is |\par| still forbidden
%to appear in an argument? \ (b)~If you say |\let\xpar=\par|, is |\xpar|
%also forbidden in an argument?
%\answer (a) Yes. \ (b) No; any other control sequence can appear
%(except those declared as |\outer| macros).
\ddangerexercise 实际运行 \TeX\ 试着得到下面问题的答案:
(a) 如果控制系列 |\par| 被重新定义(比如 `|\def\par{\endgroup\par}|'),
那么 |\par| 在参量中仍然被禁用吗?
(b) 如果给出了 |\let\xpar=\par|,那么 |\xpar| 在参量中也被禁用吗?
\answer (a) 是的。(b) 不是;其他任何控制系列都能出现%
(哪些已被声明为 |\outer| 的宏除外)。

%\ddanger \TeX\ also allows the construction
%`^|\futurelet||\cs|\<token$_1$>\<token$_2$>', which has the effect of
%`|\let\cs = |\<token$_2$>\<token$_1$>\<token$_2$>'. The idea is that you can
%say, for example, `|\futurelet\a\b|' at the end of the replacement text of
%a macro; \TeX~will set |\a| to the token that follows the macro, after
%which |\b| will be expanded. The control sequence~|\b| can continue the
%processing, and it can examine |\a| to see what's coming up next.
%^^{looking ahead}
\ddanger \TeX\ 也允许 ^|\futurelet||\cs|\<token$_1$>\<token$_2$>' 这种结构,
它的结果与 `|\let\cs = |\<token$_2$>\<token$_1$>\<token$_2$>' 的一样。
其想法是,比如,你可以在宏的替换文本结尾处使用 `|\futurelet\a\b|';
\TeX\ 将把 |\a| 设定为紧随在这个宏后面的记号,然后再展开 |\b|。
控制系列 |\b| 将继续宏的处理,并且它可以从 |\a| 得知后面出现的记号是什么。
^^{looking ahead}

%\danger The next thing a person wants, after getting used to macros with
%parameters, is the ability to write macros that change their behavior
%depending on current conditions. \TeX\ provides a variety of primitive
%commands for this purpose. The general form of such ``^{conditional text}'' is
%\begindisplay
%|\if|\<condition>\<true text>|\else|\<false text>|\fi|
%\enddisplay
%where the \<true text> is skipped unless the \<condition> is true, and
%the \<false text> is skipped unless the \<condition> is false.
%If the \<false text> is empty, you can omit the~^|\else|. The
%`|\if|\<condition>' part of this construction begins with a control
%sequence whose first two letters are `|if|'; for example,
%\begintt
%\ifodd\count0 \rightpage \else\leftpage \fi
%\endtt
%^^|\ifodd|
%specifies a condition that is true when \TeX's integer register ^|\count||0|
%is odd. Since \TeX\ generally keeps the current ^{page number} in
%|\count0|, the macro |\rightpage| will be expanded in this example if
%the page number is odd, while |\leftpage| will be expanded if the
%page number is even. Conditional commands always end with a final `^|\fi|'.
\danger 在学习了含参数的宏之后,要做的下一件事情就是编写宏,
在当前条件下,用它来改变排版的方式。%
为此, \TeX\ 提供了各种原始命令。%
其中``条件文本''的一般形式为
\begindisplay
|\if|\<condition>\<true text>|\else|\<false text>|\fi|
\enddisplay
其中,如果 \<condition> 为假,那么跳过 \<true text>;
如果 \<condition> 为真,那么跳过 \<false text>。%
如果 \<false text> 为空,就可以省略掉 |\else|。%
这个结构`|\if|\<condition>'的部分以一个控制系列开头,其前两个字母是`|if|';
例如,
\begintt
\ifodd\count0 \rightpage \else\leftpage \fi
\endtt
规定了一个条件,当 \TeX\ 的整数寄存器 |\count||0| 是奇数时为真。%
因为 \TeX\ 一般把当前页码放在 |\count0|, 所以在本例中,如果页码是奇数,
将展开宏 |\rightpage|, 而如果页码是偶数就展开 |\leftpage|。%
条件命令总以最后的`^|\fi|'来结束。

%\danger Conditionals are primarily intended for experienced \TeX\ users,
%who want to define high-level macros; therefore
%the remaining paragraphs in this chapter are headed by ``double
%dangerous bends.'' Do not feel guilty about skipping right to Chapter~21;
%in other words, imagine that the manual says `|\ifexperienced|' right
%here, and that there is a matching `|\fi|' at the end of the present chapter.
\danger 条件文本主要是为有经验的 \TeX\ 用户编写高级宏而准备的;
因此,本章剩下的段落前面都有双``危险''标志。%
从这里跳到第二十一章不用内疚;
换句话说,假想本手册在这里使用了`|\ifexperienced|',
而所匹配的`|\fi|'就出现在本章的结尾。

%\ddanger Before we discuss \TeX's repertoire of\/ |\if...|\ commands, let's
%look at another example, so that the general ideas will be clear. Suppose
%that the |\count| register |\balance| holds an amount that somebody
%has paid in excess of his or her income tax; this amount is given in
%pennies, and it might be positive, negative, or zero. Our immediate goal
%will be to write a \TeX\ macro that generates a suitable statement for the
%^{Internal Revenue Service} to include as part of a letter to that person,
%based on the amount of the balance. The statement will be quite different
%for positive balances than for negative ones, so we can exploit \TeX's ability
%to act conditionally:
%\begintt
%\def\statement{\ifnum\balance=0 \fullypaid
%  \else\ifnum\balance>0 \overpaid
%       \else\underpaid
%       \fi
%  \fi}
%\endtt
%Here ^|\ifnum| is a conditional command that compares two numbers; the
%|\statement| macro reduces to |\fullypaid| if the balance is zero, and so on.
\ddanger 在我们讨论 \TeX\ 的 |\if...| 命令的指令集前,
来看另一个例子,这样一般思路就清楚了。%
假定 |\count| 寄存器 |\balance| 存放的是某人付所得税后的余额;
这个量用美分来表示,并且它可以是正的,负的或者是零。%
\1我们的简单目的就是编写一个宏,它要为税务机关按照 balance 的值生成一份报告,
此报告要作为通知书的一部分寄给此人。%
正 balance 的报告与负的差别很大,因此我们可以用 \TeX\ 的条件文本来做:
\begintt
\def\statement{\ifnum\balance=0 \fullypaid
  \else\ifnum\balance>0 \overpaid
       \else\underpaid
       \fi
  \fi}
\endtt
这里,|\ifnum| 是一个比较两个数的条件命令;
如果 balance 为零,宏 |\statement| 就只剩下 |\fullypaid|, 等等。

%\ddanger It is vastly important to notice the spaces after the |0|'s in
%this construction. If the example had said
%\begintt
%...=0\fullypaid...
%\endtt
%then \TeX\ would have begun to expand `|\fullypaid|' before it knew the
%value of the constant |0|, because |\fullypaid| might start with a~|1| or
%something that would change the number. \ (After all, `|01|' is a perfectly
%acceptable \<number>, in \TeX's eyes.) \ In this particular case the
%program would still have worked, because we will see in a moment that
%|\fullypaid| begins with the letter~|Y|; thus, the only problem caused by
%the missing space would be that \TeX\ would go slower, since it would have
%to skip over the whole expansion of\/ |\fullypaid| instead of just skipping
%|\fullypaid| as a single, unexpanded token. But in other situations a
%missing space like this might cause \TeX\ to expand macros when you don't
%want any expansion, and such anomalies can cause subtle and confusing errors.
%For best results, {\sl always put a blank space after a numeric constant\/};
%this blank space tells \TeX\ that the constant is complete, and
%such a space will never ``get through'' to the output.  In fact, when you
%don't have a blank ^{space after a constant}, \TeX\ actually has to do
%more work, because each constant continues until a non-digit has been
%read; if this non-digit is not a space, \TeX\ takes the token you did have
%and backs it up, ready to be read again.  \ (On the other hand, the author
%often omits the space when a constant is immediately followed by some
%other character, because extra spaces do look funny in the file;
%aesthetics are more important than~efficiency.)
\ddanger 对这个结构中 |0| 后面的空格要特别注意。%
如果例子中的给出的是
\begintt
...=0\fullypaid...
\endtt
那么 \TeX\ 在得到常数 |0| 的值之前要展开`|\fullypaid|',
因为 |\fullypaid| 可能以 |1| 或其它东西开头而改变这个数字。%
(毕竟,在 \TeX\ 看来,`|01|'是完全可以接受的 \<number>。)
在这种特殊情况下,程序照样工作,因为待会我们就可以看到,|\fullypaid| 的开头是%
字母 |Y|; 因此,落掉空格后唯一引起的问题就是 \TeX\ 处理变慢,
因为它要跳过的是整个展开的 |\fullypaid|, 而不仅仅是一个未展开的单个记号 |\fullypaid|。%
但是在其它情形,象这样落掉空格可能使得 \TeX\ 在你不希望展开宏时把它展开,
并且这样的反常会得到敏感而混乱的错误。%
要得到最好的结果,{\KT{9}总是要在数值常数后面放一个空格};
这个空格就告诉 \TeX\ 这个常数是完整的,而这样的空格从来不会出现在输出中。%
实际上,当不在常数后面放空格时, \TeX\ 实际上要做更多的事,
因为每个常数都要持续到读入一个非数字字符为止;
如果这个非数字字符不是空格, \TeX\ 就把你的确有的记号取出来并且备份,以便下次再读。%
(另一方面,当某些其它字符紧跟常数时,作者常常忽略掉空格,
因为额外的空格在文件中挺难看的;
美感比效率更重要。)

%\ddangerexercise Continuing the IRS example, assume that |\fullypaid|
%and |\underpaid| are defined as follows:
%\begintt
%\def\fullypaid{Your taxes are fully paid---thank you.}
%\def\underpaid{{\count0=-\balance
%  \ifnum\count0<100
%    You owe \dollaramount, but you need not pay it, because
%    our policy is to disregard amounts less than \$1.00.
%  \else Please remit \dollaramount\ within ten days,
%    or additional interest charges will be due.\fi}}
%\endtt
%Write a macro |\overpaid| to go with these, assuming that |\dollaramount|
%is a macro that generates the contents of\/ |\count0| in dollars and cents.
%Your macro should say that a check will be mailed under separate cover,
%unless the amount is less than \$1.00, in which case the person must
%specifically request a check.
%\answer |\def\overpaid{{\count0=\balance|\parbreak
%        |  You have overpaid your tax by \dollaramount.|\parbreak
%        |  \ifnum\count0<100 It is our policy to refund|\parbreak
%        |    such a small amount only if you ask for it.|\parbreak
%        |  \else A check for this amount is being mailed|\parbreak
%        |    under separate cover.\fi}}|
\ddangerexercise 继续看看税务机关的例子,
假定 |\fullypaid| 和 |\underpaid| 的定义如下:
\begintt
\def\fullypaid{Your taxes are fully paid---thank you.}
\def\underpaid{{\count0=-\balance
  \ifnum\count0<100
    You owe \dollaramount, but you need not pay it, because
    our policy is to disregard amounts less than \$1.00.
  \else Please remit \dollaramount\ within ten days,
    or additional interest charges will be due.\fi}}
\endtt
按此编写宏 |\overpaid|,假定 |\dollaramount| 是一个宏,
它按美元和美分输出 |\count0| 的内容。你的宏应该给出的是:
 a check will be mailed under separate cover,
unless the amount is less than \$1.00, in which case the person must
specifically request a check.
\answer |\def\overpaid{{\count0=\balance|\parbreak
        |  You have overpaid your tax by \dollaramount.|\parbreak
        |  \ifnum\count0<100 It is our policy to refund|\parbreak
        |    such a small amount only if you ask for it.|\parbreak
        |  \else A check for this amount is being mailed|\parbreak
        |    under separate cover.\fi}}|

%\ddangerexercise Write a |\dollaramount| macro, to complete the Internal
%Revenue |\statement|.
%\answer The tricky part is to get the zero in an amount like `|$2.01|'.
%\begintt
%\def\dollaramount{\count2=\count0 \divide\count2 by100
%  \$\number\count2.%
%  \multiply\count2 by-100 \advance\count2 by\count0
%  \ifnum \count2<10 0\fi
%  \number\count2 }
%\endtt
\ddangerexercise \1编写宏 |\dollaramount|,以补全税务机关的 |\statement| 宏。
\answer 有技巧的部分在于从类似 `|$2.01|' 的数目中得到零。
\begintt
\def\dollaramount{\count2=\count0 \divide\count2 by100
  \$\number\count2.%
  \multiply\count2 by-100 \advance\count2 by\count0
  \ifnum \count2<10 0\fi
  \number\count2 }
\endtt

%\ddanger Now let's make a complete survey of \TeX's conditional commands.
%Some of them involve features that have not yet been introduced in this
%manual.\enddanger
\ddanger 现在,我们完整地总结一下 \TeX\ 的条件命令。%
其中有一些本手册还未讨论的东西。
\enddanger

%\nobreak\medskip
%\item\bull^|\ifnum|\<number$_1$>\<relation>\<number$_2$>\quad
%  (compare two integers)
%\nobreak\smallskip\noindent
%The ^\<relation> must be either `|<|$_{12}$' or `|=|$_{12}$' or `|>|$_{12}$'.
%The two integer numbers are compared to each other in the usual way, and
%the result is true or false accordingly.
\nobreak\medskip
\item\bull^|\ifnum|\<number$_1$>\<relation>\<number$_2$>\quad
  (比较两个整数)
\nobreak\smallskip\noindent
\<relation> 编写必须是`|<|$_{12}$', `|=|$_{12}$'或`|>|$_{12}$'。%
两个整数按通常的方法进行比较,因此所得结果为真或假。

%\medbreak
%\item\bull^|\ifdim|\<dimen$_1$>\<relation>\<dimen$_2$>\quad
%  (compare two dimensions)
%\nobreak\smallskip\noindent
%This is like |\ifnum|, but it compares two \<dimen> values. For example,
%to test whether the value of\/ |\hsize| exceeds $100\pt$, you can say
%`|\ifdim\hsize>100pt|'.
\medbreak
\item\bull^|\ifdim|\<dimen$_1$>\<relation>\<dimen$_2$>\quad
  (比较两个尺寸)
\nobreak\smallskip\noindent
它与 |\ifnum| 类似,但是比较的是两个 \<dimen> 的值。%
例如,要检验 |\hsize| 的值是否超过 $100\pt$, 可以用`|\ifdim\hsize>100pt|'。

%\medbreak
%\item\bull^|\ifodd|\<number>\quad(test for odd integer)
%\nobreak\smallskip\noindent
%The condition is true if the \<number> is odd, false if it is even.
\medbreak
\item\bull^|\ifodd|\<number>\quad(奇数测试)
\nobreak\smallskip\noindent
如果 \<number> 为奇数则真,为偶数则假

%\medbreak
%\item\bull^|\ifvmode|\quad(test for vertical mode)
%\nobreak\smallskip\noindent
%True if \TeX\ is in vertical mode or internal vertical mode (see Chapter~13).
\medbreak
\item\bull^|\ifvmode|\quad(垂直模式测试)
\nobreak\smallskip\noindent
如果 \TeX\ 处在垂直模式或者内部垂直模式则真(见第十三章)。

%\medbreak
%\item\bull^|\ifhmode|\quad(test for horizontal mode)
%\nobreak\smallskip\noindent
%True if \TeX\ is in horizontal mode or restricted horizontal mode
%(see Chapter~13).
\medbreak
\item\bull^|\ifhmode|\quad(水平模式测试)
\nobreak\smallskip\noindent
如果 \TeX\ 处在水平模式或受限水平模式则真(见第十三章)。

%\medbreak
%\item\bull^|\ifmmode|\quad(test for math mode)
%\nobreak\smallskip\noindent
%True if \TeX\ is in math mode or display math mode (see Chapter~13).
\medbreak
\item\bull^|\ifmmode|\quad(数学模式测试)
\nobreak\smallskip\noindent
如果 \TeX\ 处在数学模式或陈列数学模式则真(见第十三章)。

%\medbreak
%\item\bull^|\ifinner|\quad(test for an internal mode)
%\nobreak\smallskip\noindent
%True if \TeX\ is in internal vertical mode, or restricted
%horizontal mode, or (nondisplay) math mode (see Chapter~13).
\medbreak
\item\bull^|\ifinner|\quad(内部模式测试)
\nobreak\smallskip\noindent
如果 \TeX\ 处在内部垂直模式或受限水平模式或(非列表)数学模式则真(见第十三章)。

%\medbreak
%\item\bull^|\if|\<token$_1$>\<token$_2$>\quad(test if character codes agree)
%\nobreak\smallskip\noindent
%\TeX\ will expand macros following |\if| until two unexpandable tokens are
%found. If either token is a control sequence, \TeX\ considers it to have
%character code~256 and category code~16, unless the current equivalent of
%that control sequence has been |\let| equal to a non-active character
%token.  In this way, each token specifies a (character~code,
%\hbox{category}~code) pair.  The condition is true if the character codes
%are equal, independent of the category codes. For example, after
%|\def\a{*}| and |\let\b=*| and |\def\c{/}|, the tests `|\if*\a|' and
%`|\if\a\b|' will be true, but `|\if\a\c|' will be false. Also
%`|\if\a\par|' will be false, but `|\if\par\let|' will be true. % Beresford=true
\medbreak
\item\bull^|\if|\<token$_1$>\<token$_2$>\quad(测试字符代码是否相同)
\nobreak\smallskip\noindent
 \TeX\ 将把 |\if| 后面的宏展开为两个不能再展开的记号。%
如果其中一个记号是控制系列,那么 \TeX\ 就把它看作字符代码为 256 和类代码为 16,
除非此控制系列的当前内容被 |\let| 为等于非活动字符记号。%
用这种方法,每个记号给出一个(字符代码,类代码)对。%
如果字符代码相等,条件成立,而与类代码无关。%
例如,在给出 |\def\a{*}|, |\let\b=*| 和 |\def\c{/}| 后,`|\if*\a|'%
和`|\if\a\b|'为真,但是`|\if\a\c|'为假。%
还有,`|\if\a\par|'为假,但是`|\if\par\let|'为真。

%\medbreak
%\item\bull^|\ifcat|\<token$_1$>\<token$_2$>\quad(test if category codes agree)
%\nobreak\smallskip\noindent
%This is just like |\if|, but it tests the ^{category codes}, not the character
%codes. ^{Active characters} have category~13, but you have to say
%`^|\noexpand|\<active character>' in order to suppress expansion when you
%are looking at such characters with |\if| or |\ifcat|. For example, after
%\begintt
%\catcode`[=13 \catcode`]=13 \def[{*}
%\endtt
%the tests `|\ifcat\noexpand[\noexpand]|' and `|\ifcat[*|' will be true,
%but the test `|\ifcat\noexpand[*|' will be false.
\medbreak
\item\bull^|\ifcat|\<token$_1$>\<token$_2$>\quad (测试类代码是否相同)
\nobreak\smallskip\noindent
它就像 |\if| 那样,但是测试的是类别码,而不是字符码。
^{活动字符}的类别码是 13,但在用 |\if| 或 |\ifcat| 测试这类字符时,
你必须用 `^|\noexpand|\<active character>' 取消展开。例如,在
\begintt
\catcode`[=13 \catcode`]=13 \def[{*}
\endtt
之后,测试 `|\ifcat\noexpand[\noexpand]|' 和 `|\ifcat[*|' 的结果将为真,
而测试 `|\ifcat\noexpand[*|' 的结果为假。

%\medbreak
%\item\bull^|\ifx|\<token$_1$>\<token$_2$>\quad(test if tokens agree)
%\nobreak\smallskip\noindent
%In this case, \TeX\ does {\sl not\/} expand control sequences when it
%looks at the two tokens. The condition is true if (a)~the two tokens are
%not macros, and they both represent the same (character code, category
%code) pair or the same \TeX\ primitive or the same ^|\font| or ^|\chardef|
%or ^|\countdef|, etc.; or if (b)~the two tokens are macros, and they both
%have the same status with respect to ^|\long| and ^|\outer|, and they both
%have the same parameters and ``top level'' expansion. For example, after
%`|\def\a{\c}| |\def\b{\d}| |\def\c{\e}| |\def\d{\e}| |\def\e{A}|', an
%|\ifx| test will find |\c| and |\d| equal, but not |\a| and~|\b|, nor |\d|
%and~|\e|, nor any other combinations of\/ |\a|, |\b|, |\c|, |\d|, |\e|.
\medbreak
\item\bull^|\ifx|\<token$_1$>\<token$_2$>\quad\1(测试记号是否相同)
\nobreak\smallskip\noindent
在这种情况下,当 \TeX\ 遇见这两个记号时,{\KT{9}不}展开控制系列。%
如果(a) 两个记号不是宏,并且它们都标识同一(字符代码,类代码)对或同一 \TeX\ 原始%
命令,同一 |\font| 或 |\chardef| 或 |\countdef| 等等;
(b) 两个记号是宏,并且它们对 |\long| 和 |\outer| 都处在相同的状态,
以及它们有同样的测试和``顶级''展开,那么条件为真。%
例如,设置%
`|\def\a{\c}| |\def\b{\d}| |\def\c{\e}| |\def\d{\e}| |\def\e{A}|'后,
在 |\ifx| 测试中,|\c| 和 |\d| 相等,但是 |\a| 和 |\b|, |\d| 和 |\e| 不相等,
|\a|, |\b|, |\c|, |\d|, |\e| 的其它组合也不相等。

%\medbreak
%\item\bull^|\ifvoid|\<number>, ^|\ifhbox|\<number>, ^|\ifvbox|\<number>\quad
%(test a box register)
%\nobreak\smallskip\noindent
%The \<number> should be between 0 and 255. The condition is true if that
%|\box| is void or contains an hbox or a vbox, respectively (see Chapter~15).
\medbreak
\item\bull^|\ifvoid|\<number>, ^|\ifhbox|\<number>, ^|\ifvbox|\<number>\quad
(测试一个盒子寄存器)
\nobreak\smallskip\noindent
\<number> 应该在 0 和 255 之间。%
如果 |\box| 是置空的,或者包含一个 hbox 或 vbox 时条件为真(见第十五章)。

%\medbreak
%\item\bull^|\ifeof|\<number>\quad(test for end of file)
%\nobreak\smallskip\noindent
%The \<number> should be between 0 and 15. The condition is true unless the
%corresponding input stream is open and not fully read. \ (See the command
%^|\openin| below.)
\medbreak
\item\bull^|\ifeof|\<number>\quad(测试文件是否结束)
\nobreak\smallskip\noindent
\<number> 一个在 0 和 15 之间。%
条件为真,除非相应的输入流是开的并且没有读完。(见下面的命令 |\openin|。)

%\medbreak
%\item\bull^|\iftrue|, ^|\iffalse|\quad(always true or always false)
%\nobreak\smallskip\noindent
%These conditions have a predetermined outcome. But they turn out to be
%useful in spite of this, as explained below.
\medbreak
\item\bull^|\iftrue|, ^|\iffalse|\quad(永远为真或为假)
\nobreak\smallskip\noindent
这些条件有预先确定的结果。%
但是它们却非常有用,见下面的讨论。

%\medbreak
%Finally, there's one more conditional construction, which is somewhat different
%from the rest because it is capable of making a many-way branch:
%\begindisplay
%\llap{\bull\enspace}^|\ifcase|\<number>\<text for case 0>|\or|\<text
%  for case 1>|\or|$\;\cdots$\cr
%|         \or|\<text for case $n$>|\else|\<text for all other cases>|\fi|\cr
%\enddisplay
%Here there are $n+1$ cases separated by $n$ ^|\or|'s, where $n$ can be any
%nonnegative number. The \<number> selects the text that will be used. Once
%again the ^|\else| part is optional, if you don't want to specify any text
%for cases when the \<number> is negative or greater than~$n$.
\medbreak
最后,有一个多条件结构,它与其它的不同,因为它有多个分支:
\begindisplay
\llap{\bull\enspace}^|\ifcase|\<number>\<text for case 0>|\or|\<text
  for case 1>|\or|$\;\cdots$\cr
|         \or|\<text for case $n$>|\else|\<text for all other cases>|\fi|\cr
\enddisplay
在这里,~$n+1$ 种情形由 $n$ 个 |\or| 分开,
其中 $n$ 可以是任意非负数。%
\<number> 选择要使用的文本。%
~|\else| 部分还是可选的,如果你不想在 \<number> 为负数或大于 $n$ 的情形下%
给出某些文本的话。

%\ddangerexercise Design a |\category| macro that prints a character's current
%category code symbolically, given a one-character control sequence for
%that character. For example, if the category codes of plain \TeX\ are in
%force, `|\category\\|' should expand to `|escape|', and `|\category\a|'
%should expand to `|letter|'.
%\answer |\def\category#1{\ifcase\catcode`#1|\parbreak
%        |  escape\or begingroup\or endgroup\or math\or|\parbreak
%        |  align\or endline\or parameter\or superscript\or|\parbreak
%        |  subscript\or ignored\or space\or letter\or|\parbreak
%        |  otherchar\or active\or comment\or invalid\fi}|\par
\ddangerexercise 设计一个宏 |\category|,在其后面输入单个字符后,
它输出此字符的当前类别码的名称。例如,假设使用 plain \TeX\ 的类代码,
则 `|\category\\|' 将展开为 `|escape|',`|\category\a|' 将展开为 `|letter|'。
\answer |\def\category#1{\ifcase\catcode`#1|\parbreak
        |  escape\or begingroup\or endgroup\or math\or|\parbreak
        |  align\or endline\or parameter\or superscript\or|\parbreak
        |  subscript\or ignored\or space\or letter\or|\parbreak
        |  otherchar\or active\or comment\or invalid\fi}|\par

%\ddangerexercise Test yourself on the following questions to see if you
%understand certain borderline situations: After the definitions
%`|\def\a{}| |\def\b{**}| |\def\c{True}|',
%which of the following are true?
%(a)~`|\if\a\b|';
%(b)~`|\ifcat\a\b|';
%(c)~`|\ifx\a\b|';
%(d)~`|\if\c|';
%(e)~`|\ifcat\c|';
%(f)~`|\ifx\ifx\ifx|'.
%(g)~`|\if\ifx\a\b\c\else\if\a\b\c\fi\fi|'.
%\answer (a,b)~True. (c,d)~False. (e,f)~True. In case~(e), the \<true text>
%starts with `|ue|'.  (g)~The |\ifx| is false and the inner |\if| is true;
%so the outer |\if| becomes `|\if True...|', which is false. \
%(Interestingly, \TeX\ knows that the outer |\if| is false even before it
%has looked at the |\fi|'s that close the |\ifx| and the inner |\if|.)
\ddangerexercise 用下列这些问题测验一下自己,看看你是否掌握了这些模糊的情形:
在定义 `|\def\a{}| |\def\b{**}| |\def\c{True}|' 之后,下面哪些是真的?
(a)~`|\if\a\b|';
(b)~`|\ifcat\a\b|';
(c)~`|\ifx\a\b|';
(d)~`|\if\c|';
(e)~`|\ifcat\c|';
(f)~`|\ifx\ifx\ifx|'。
(g)~`|\if\ifx\a\b\c\else\if\a\b\c\fi\fi|'。
\answer (a,b) 真。(c,d) 假。(e,f) 真。在情形 (e) 中,\<true text> 以 `|ue|' 开始。
(g)~|\ifx| 为假,而内部的 |\if| 为真;因此外部的 |\if| 成为`|\if True...|',它是假的。%
(有趣的是, \TeX\ 还没看到结束|\ifx| 和内部 |\if| 的两个 |\fi|,
就已经知道外部 |\if| 是假的。)

%\ddanger Notice that all of the control sequences for conditionals
%begin with |\if...|, and they all have a matching~|\fi|. This convention---that
%|\if...|\ pairs up with |\fi|---makes it easier to see the nesting
%of conditionals within your program. The ^{nesting} of\/ |\if...\fi| is
%independent of the nesting of |{...}|; thus, you can begin or end a ^{group}
%in the middle of a conditional, and you can begin or end a conditional
%in the middle of a group. Extensive experience with macros has shown that
%such independence is important in applications; but it can also lead to
%confusion if you aren't careful.
\ddanger 注意,所有的条件控制系列都以 |\if...| 开头,并且它们都要匹配 |\fi|。%
这种 |\if...| 与 |\fi| 配对的约定使程序中条件控制系列的嵌套更好分清。%
|\if...\fi| 的嵌套与 |{...}| 的嵌套无关;
因此,可以在条件控制系列中间开始或结束一个组,
也可以在组中间开始或结束一个条件控制系列。%
编写宏的大量经验表明,这种无关性在应用中很重要;
但是如果不仔细也会出现问题。

%\ddanger It's sometimes desirable to pass information from one macro to
%another, and there are several ways to do this: ^^{communication between
%macros} by passing it as an argument, by putting it into a register, or by
%defining a control sequence that contains the information. For example,
%the macros |\hphantom|, |\vphantom|, and ^|\phantom| in Appendix~B are
%quite similar, so the author ^^{Knuth} wanted to do most of the work in
%another macro |\phant| that would be common to all three. Somehow |\phant|
%was to be told what kind of phantom was desired. The first approach was to
%define control sequences |\hph| and |\vph| something like this:
%\begintt
%\def\hphantom{\ph YN} \def\vphantom{\ph NY} \def\phantom{\ph YY}
%\def\ph#1#2{\def\hph{#1}\def\vph{#2}\phant}
%\endtt
%after which |\phant| could test `|\if Y\hph|' and `|\if Y\vph|'. This
%worked, but there were various ways to make it more efficient; for example,
%`|\def\hph{#1}|' could be replaced by `|\let\hph=#1|', avoiding macro expansion.
%An even better idea then suggested itself:
%\begintt
%\def\yes{\if00} \def\no{\if01}
%\def\hphantom{\ph\yes\no}...\def\phantom{\ph\yes\yes}
%\def\ph#1#2{\let\ifhph=#1\let\ifvph=#2\phant}
%\endtt
%after which |\phant| could test `|\ifhph|' and `|\ifvph|'. \ (This
%construction was tried before |\iftrue| and |\iffalse| were part of the
%\TeX\ language.) \ The idea worked fine, so the author started to use
%|\yes| and |\no| in a variety of other situations. But then one day a
%complex conditional failed, because it contained an |\ifhph|-like test
%inside another conditional:
%\begintt
%\if...  \ifhph...\fi  ... \else  ...  \fi
%\endtt
%Do you see the problem that developed? When the \<true text> of the
%outermost conditional was executed, everything worked fine, because
%|\ifhph| was either |\yes| or |\no| and it expanded into either |\if00| or
%|\if01|.  But when the \<true text> was skipped, the |\ifhph| was not
%expanded, so the first |\fi| was mistakenly paired with the first |\if|;
%everything soon went haywire. That's when ^|\iftrue| and ^|\iffalse|
%were put into the language, in place of\/ |\yes| and |\no|; now
%|\ifhph| is either |\iftrue| or |\iffalse|, so \TeX\ will match it
%properly with a closing~|\fi|, whether or not it is being skipped over.
\ddanger \1有时候要把信息从一个宏传到另一个,而实现它有几种方法:
把它作为一个参量来传递,把它放在一个寄存器中,或者定义一个包含此信息的控制系列。%
例如,附录 B 中的宏 |\hphantom|, |\vphantom| 和 |\phantom| 非常相似,
因此作者希望把它们三个的所有部分放在另一个宏 |\phant| 中。%
用某种方法来告诉 |\phant| 所要的是哪类 phantom。%
第一种方法是定义象下面这样的控制系列 |\hph| 和 |\vph|:
\begintt
\def\hphantom{\ph YN} \def\vphantom{\ph NY} \def\phantom{\ph YY}
\def\ph#1#2{\def\hph{#1}\def\vph{#2}\phant}
\endtt
之后 |\phant| 可测试`|\if Y\hph|'和`|\if Y\vph|'。%
这可以用,但是有几个更有效的方法;
例如,`|\def\hph{#1}|'可以用`|\let\hph=#1|'来代替,以避免展开宏。%
因此,一个更好的方法是:
\begintt
\def\yes{\if00} \def\no{\if01}
\def\hphantom{\ph\yes\no}...\def\phantom{\ph\yes\yes}
\def\ph#1#2{\let\ifhph=#1\let\ifvph=#2\phant}
\endtt
之后 |\phant| 可测试`|\ifhph|'和`|\ifvph|'。%
(这种结构出现在 \TeX\ 语言中有 |\iftrue| 和 |\iffalse| 之前。)
想法很好,因此作者就开始把 |\yes| 和 |\no| 用在其它情形中。%
但是接着有一天,一个复杂的条件控制系列失败了,因为它把象 |\ifhph| 这样的测试%
放在另一个条件文本中了:
\begintt
\if...  \ifhph...\fi  ... \else  ...  \fi
\endtt
能看出问题吗?
当执行最外层条件的 \<true text> 时,所有的都很顺利,
因为 |\ifhph| 是 |\yes| 或 |\no|, 并且它展开为 |\if00| 或 |\if01|。%
但是当跳过 \<true text> 时,|\ifhph| 没有被展开,因此第一个 |\fi| 错误地%
匹配到第一个 |\if| 上;
很快错误就都出来了。%
这时 |\iftrue| 和 |\iffalse| 就被加进 \TeX\ 语言中,来代替 |\yes| 和 |\no|;
现在,|\ifhph| 是 |\iftrue| 或 |\iffalse|,
因此不管它是否被跳过, \TeX\ 都可正确匹配上 |\fi|。

%\ddanger To facilitate |\if...|~constructions, plain \TeX\ has a
%^|\newif| macro, such that after you say `|\newif\ifabc|' three
%control sequences will be defined: |\ifabc| (for testing the switch),
%|\abctrue| (for making the switch true), and |\abcfalse| (for making
%it false). The |\phantom| problem is now solved in Appendix~B by writing
%\begintt
%\newif\ifhph \newif\ifvph
%\def\hphantom{\hphtrue\vphfalse\phant}
%\endtt
%and with similar definitions of\/ |\vphantom| and |\phantom|. There is no
%longer any need for a |\ph| macro; again |\phant| tests |\ifhph| and
%|\ifvph|. Appendix~E contains other examples of conditionals created
%by~|\newif|. New conditionals are initially false.
\ddanger 为了便于构造 |\if...|, ~plain \TeX\ 提供了一个叫 |\newif| 的宏,
这样在给出`|\newif\ifabc|'后,就定义了三个控制系列:
|\ifabc|(测试真假), |\abctrue|(测试为真)和 |\abcfalse|~(测试为假)。%
现在,附录 B 的 |\phantom| 问题就可以如下解决:
\begintt
\newif\ifhph \newif\ifvph
\def\hphantom{\hphtrue\vphfalse\phant}
\endtt
并且有 |\vphantom| 和 |\phantom| 的类似定义。%
不再需要宏 |\ph| 了;
还是 |\phant| 来测试 |\ifhph| 和 |\ifvph|。%
附录 E 中有由 |\newif| 生成的其它条件文本的例子。%
新的条件文本开始都设定为假。

%\ddanger Caution: Don't say anything like `|\let\ifabc=\iftrue|' in
%conditional text. If \TeX\ skips over this command, it will think that
%both |\ifabc| and |\iftrue| require a matching |\fi|, since the |\let| is
%not being executed! Keep such commands buried inside macros, so that \TeX\
%will see the `|\if...|'\ only when it is not skipping over the text that it
%is reading.
\ddanger 注意:不要在条件文本中给出象`|\let\ifabc=\iftrue|'这样的东西。%
如果 \TeX\ 跳过这些命令,就会认为 |\ifabc| 和 |\iftrue| 都要匹配一个 |\fi|,
因为 |\let| 没有被执行!
把这样的命令包在宏中,这样 \TeX\ 只有在不跳过要读入的文本时才能遇见`|\if...|'。

%\ddanger \TeX\ has 256 ``^{token list registers}'' called ^|\toks||0|
%through |\toks255|, so that token lists can easily be shuffled around
%without passing them through \TeX's reading apparatus.
%There's also a ^|\toksdef| instruction so that, e.g.,
%\begintt
%\toksdef\catch=22
%\endtt
%makes |\catch| equivalent to |\toks22|. Plain \TeX\ provides a ^|\newtoks|
%macro that allocates a new token list register; it is analogous to
%|\newcount|. Token list registers behave like the ^{token list parameters}
%|\everypar|, |\everyhbox|, |\output|, |\errhelp|, etc. To assign
%a new value to a token list parameter or register, you say either
%\begindisplay
%\<token variable>|={|\<replacement text>|}|\cr
%\llap{or }\<token variable>|=|\<token variable>\cr
%\enddisplay
%where ^\<token variable> means either a token list parameter or a
%control sequence defined by |\toksdef| or |\newtoks|, or an
%explicit register designation `|\toks|\<number>'.
\ddanger \1\TeX\ 有 256 个``记号列寄存器''叫做 |\toks||0| 到 |\toks255|,
因此记号列可以在不经过 \TeX\ 读入器时很容易地传来传去。%
还有一个 |\toksdef| 指令,使得,比如,
\begintt
\toksdef\catch=22
\endtt
把 |\catch| 与 |\toks22| 等价起来。%
Plain \TeX\ 提供了一个宏 |\newtoks|, 由它来分配新的记号列寄存器;
它类似于 |\newcount|。%
记号列寄存器的性质就象记号列参数 |\everypar|, |\everyhbox|, |\output|,
|\errhelp| 等等。%
为了给记号列参数或寄存器指定新值,可以使用
\begindisplay
\<token variable>|={|\<replacement text>|}|\cr
\llap{或者 }\<token variable>|=|\<token variable>\cr
\enddisplay
其中 \<token variable> 表示一个记号列参数,或者是由 |\toksdef| 或 |\newtoks|~%
定义的一个控制系列,
或者是一个明确的寄存器名字`|\toks|\<number>'。

%\ddanger Everyone who makes extensive use of a powerful macro facility
%encounters situations when the macros do surprising things. We have
%already mentioned the possibility of setting |\tracingmacros=1|, in order
%to see when \TeX\ expands macros and what arguments it finds. There's also
%another helpful way to watch what \TeX\ is doing: If you set
%^|\tracingcommands||=1|, \TeX\ will show every command that it executes,
%as we saw in Chapter~13.  Furthermore, if you set
%|\tracingcommands=2|, \TeX\ will show all conditional commands and
%their outcomes, as well as the unconditional commands that are actually
%performed or expanded. This diagnostic information goes into your log
%file. You can also see it on your terminal, if you say
%^|\tracingonline||=1|. \ (Incidentally, if you make |\tracingcommands|
%greater than~2, you get the same information as when it equals~2.) \
%Similarly, ^|\tracingmacros||=2| will trace |\output|, |\everypar|, etc.
\ddanger 经常使用宏这个便利工具的每个人都会遇到编写的宏出问题的时候。%
我们已经说过,为了看看 \TeX\ 是怎样展开宏和它找到的参量是什么,
我们可以设置 |\tracingmacros=1|。%
还有另一个有用的方法来看看 \TeX\ 在做什么:
如果设置 |\tracingcommands||=1|, 那么 \TeX\ 将显示出它所执行的每个命令,
就象第十三章那样。%
还有,如果设置 |\tracingcommands=2|, 那么 \TeX\ 将显示所有条件命令及其结果,
以及实际执行或展开的非条件命令。%
这些诊断信息出现在 log 文件中。%
如果还设置了 |\tracingonline||=1|, 那么在终端上也可以看到。%
(顺便说一下,如果设置 |\tracingcommands| 大于 2, 那么得到的信息同等于 2 一样。)
类似地,|\tracingmacros||=2| 将跟踪 |\output|, |\everypar|, 等等。

%\ddanger One way to understand the occasional strangeness of macro
%operation is to use the tracing features just described, so that you can
%watch what \TeX\ does in slow motion. Another way is to learn the rules for
%how macros are expanded; we shall now discuss those rules.
\ddanger 要知道宏命令出毛病的一个方法就是用刚才讨论的跟踪方法,
这样就能看到 \TeX\ 每步在做什么。%
另一种就是掌握宏是怎样展开的;现在我们来讨论这个规则。

%\ddanger \TeX's mastication process converts your input to a long token
%list, as explained in Chapter~8; and its digestive processes work
%strictly on this token list. When \TeX\ encounters a control sequence in
%the token list, it looks up the current meaning, and in certain cases
%it will expand that token into a sequence of other tokens before
%continuing to read. The expansion process applies to macros and to
%certain other special primitives like |\number| and |\if| that we shall
%consider momentarily. Sometimes, however, the ^{expansion} is not carried out;
%for example, when \TeX\ is taking care of a |\def|, the \<control sequence>,
%the \<parameter text>, and the \<replacement text> of that |\def| are
%not subject to expansion. Similarly, the two tokens after |\ifx| are
%never expanded. A complete list of occasions when tokens are not expanded
%appears later in this chapter; you can use it for reference in an emergency.
\ddanger  \TeX\ 的咀嚼过程把你的输入变成一个长记号列,就象第八章讨论的那样;
其消化过程严格按照这个记号列进行。%
当 \TeX\ 遇见记号列中的控制系列时,要查找其当前的意思,
并且在某些情况下,在继续读入之前要把此记号展开为一系列其它记号。%
展开过程作用的对象是宏和某些其它特殊的原始命令,象 |\number|~%
和我们刚刚讨论过的 |\if| 这样。%
但是有时候,却不进行展开;
例如,当 \TeX\ 处理一个 |\def|, 此 |\def| 的 \<control sequence>, \<parameter text>~%
和 \<replacement text> 不被展开。%
类似地,~|\ifx| 后面的两个记号也不被展开。%
本章后面要给出一个完整列表,在这些情况下不展开记号;
在实在没办法时,你可以用它作为参考。

%\ddanger Now let's consider the control sequences that are expanded whenever
%expansion has not been inhibited. Such control sequences
%fall into several classes:\enddanger
\ddanger 现在我们来看看不禁止展开时控制系列的展开情况。%
这样的控制系列分几种:\enddanger

%\nobreak\medskip
%\textindent\bull Macros. When a macro is expanded, \TeX\ first determines its
%arguments (if any), as explained earlier in this chapter. Each argument
%is a token list; the tokens are not expanded when they are being
%accepted as arguments. Then \TeX\ replaces the macro and its arguments
%by the replacement text.
\nobreak\medskip
\textindent\bull 宏:\ 当宏被展开时, \TeX\ 首先确定其参量(如果有的话),
就象本章前面讨论的那样。%
每个参量是一个记号列;但记号作为参量而看待时,它们不被展开。%
接着 \TeX\ 用替换文本代替宏及其参量。

%\smallbreak
%\textindent\bull Conditionals. When an |\if...| is expanded, \TeX\ reads
%ahead as far as necessary to determine whether the condition is true or
%false; and if false, it skips ahead (keeping track of\/ |\if...\fi| nesting)
%until finding the |\else|, |\or|, or~|\fi| that ends the skipped text.
%Similarly, when |\else|, |\or|, or~|\fi| is expanded, \TeX\ reads to the
%end of any text that ought to be skipped. The ``expansion'' of a conditional
%is empty. \ (Conditionals always reduce the number of tokens that are seen by
%later stages of the digestive process, while macros usually increase the
%number of tokens.)
\smallbreak
\textindent\bull 条件文本:\1当 |\if...| 被展开时, \TeX\ 读入必要的内容来%
确定条件的真假;如果是假,将跳过前面(保持 |\if...\fi| 的嵌套)直到找到%
要结束所跳过文本的 |\else|, |\or| 或 |\fi|。%
类似地,当 |\else|, |\or| 或 |\fi| 被展开时, \TeX\ 读入要跳过的任何文本的结尾。%
条件文本的``展开''是空的。(条件文本总是减少在后面的消化阶段所遇见的记号量,
而宏一般增加记号的量。)

%\smallbreak
%\textindent\bull ^|\number|\<number>. When \TeX\ expands |\number|, it reads
%the \<number> that follows (expanding tokens as it goes); the final
%expansion consists of the ^{decimal representation} of that number,
%preceded by `|-|' if negative.
\smallbreak
\textindent\bull ^|\number|\<number>: \ 当 \TeX\ 展开 |\number| 时,
它读入所跟的 \<number>~(如果需要就展开记号);
最后的展开由此数的小数表示组成,如果是负的前面要有`|-|'。

%\smallbreak
%\textindent\bull ^|\romannumeral|\<number>. This is like |\number|, but
%the expansion consists of lowercase roman numerals. For example,
%`|\romannumeral 1984|' produces `|mcmlxxxiv|'. The expansion is
%empty if the number is zero or negative.
\smallbreak
\textindent\bull ^|\romannumeral|\<number>: \ 它与 |\number| 类似,
但是展开由小写 roman 数字组成。%
例如,`|\romannumeral 1984|'得到的是`|mcmlxxxiv|'。%
如果数字是零或负,展开为空。

%\smallbreak
%\textindent\bull ^|\string|\<token>. \TeX\ first reads the \<token> without
%expansion. If a control sequence token appears, its |\string| expansion
%consists of the control sequence name (including ^|\escapechar| as an escape
%character, if the control sequence isn't simply an active character).
%Otherwise the \<token> is a character token, and its character code
%is retained as the expanded result.
\smallbreak
\textindent\bull ^|\string|\<token>: \  \TeX\ 首先读入 \<token> 而不展开。%
如果控制系列记号出现,那么它的 |\string| 展开由控制系列的名字组成(如果控制系列%
不单单是活动字符,就把 |\escapechar| 包括进来作为转义符)。%
否则,~\<token> 就是字符记号,并且其字符代码保持为展开后的结果。

%\smallbreak
%\textindent\bull ^|\jobname|. The expansion is the name that \TeX\ has chosen
%for this job. For example, if \TeX\ is putting its output on files
%|paper.dvi| and |paper.log|, ^^|dvi| then |\jobname| expands to `|paper|'.
\smallbreak
\textindent\bull ^|\jobname|: \ 展开为 \TeX\ 为此进程选定的名字。%
例如,如果 \TeX\ 把它的输出放在文件 |paper.dvi| 和 |paper.log| 中,
那么 |\jobname| 就展开为`|paper|'。

%\smallbreak
%\textindent\bull ^|\fontname|\<font>. The expansion is the external file name
%corresponding to the given font; e.g., `|\fontname\tenrm|' might expand
%to `|cmr10|' (five tokens). If the font is not being used at its design size,
%the ``^{at size}'' also appears in the expansion.  A ^\<font> is either an
%identifier defined by ^|\font|; or |\textfont|\<number>,
%|\scriptfont|\<number>, or |\scriptscriptfont|\<number>; or |\font|, which
%denotes the current font.
\smallbreak
\textindent\bull ^|\fontname|\<font>:展开为给定字体对应的外部文件名;
如 `|\fontname\tenrm|' 将展开为 `|cmr10|'(五个记号)。
如果字体所用的不是其设计尺寸,那么``^{at size}''也出现在展开中。
\<font> 是一个由 |\font| 定义的标识符;或是 |\textfont|\<number>,
|\scriptfont|\<number> 或 |\scriptscriptfont|\<number>;
或者是 |\font|,它表示当前字体。

%\smallbreak
%\textindent\bull ^|\meaning|\<token>. \TeX\ expands this to the sequence of
%characters that would be displayed on your terminal by the commands
%`|\let\test=|\<token> |\show\test|'. For example, `|\meaning A|' usually
%expands to `|the letter A|'; `|\meaning\A|' after `|\def\A#1B{\C}|'
%expands to `|macro:#1B->\C |'.
\smallbreak
\textindent\bull ^|\meaning|\<token>:\TeX\ 把它展开为一系列字符,
它们是命令 `|\let\test= |\<token> |\show\test|' 在你的终端上显示的内容。
例如,`|\meaning A|' 展开的一般是 `|the letter A|';
在 `|\def\A#1B{\C}|' 后,`|\meaning\A|' 展开的是 `|macro:#1B->\C |'。

%\smallbreak
%\textindent\bull ^|\csname||...|^|\endcsname|. When \TeX\ expands |\csname|
%it reads to the matching |\endcsname|, expanding tokens as it goes;
%only character tokens should remain after this expansion has taken place.
%Then the ``expansion'' of the entire |\csname...\endcsname| text
%will be a single control sequence token, defined to be like |\relax| if
%its meaning is currently undefined.
\smallbreak
\textindent\bull ^|\csname||...|^|\endcsname|:\TeX\ 在展开 |\csname|
时要一直读取直到找到匹配的 |\endcsname|,在查找过程中展开读到的记号;
在展开时,只有字符记号或保留下来。因此,整个 |\csname...\endcsname| 文本的
``展开''将是单个控制系列记号,如果它当前没有定义,则定义为 |\relax|。

%\smallbreak
%\textindent\bull ^|\expandafter|\<token>. \TeX\ first reads the token that
%comes immediately after |\expandafter|, without expanding it;
%let's call this token~$t$. Then \TeX\ reads the token that comes after~$t$
%(and possibly more tokens, if that token has an argument), replacing it by
%its expansion. Finally \TeX\ puts~$t$ back in front of that expansion.
\smallbreak
\textindent\bull ^|\expandafter|\<token>: \
 \TeX\ 首先读入 |\expandafter| 紧后面的记号,而不展开它;
我们称此记号为 $t$。%
接着, \TeX\ 读入 $t$ 后面的记号(如果此记号有一个参量,那么可能有更多的记号),
把它用其展开代替。%
最后 \TeX\ 把 $t$ 放在此展开的前面。

%\smallbreak
%\textindent\bull ^|\noexpand|\<token>. The expansion is the token itself;
%but that token is interpreted as if its meaning were `|\relax|' if it is a
%control sequence that would ordinarily be expanded by \TeX's expansion rules.
\smallbreak
\textindent\bull ^|\noexpand|\<token>: \
展开为记号自己;但是如果此记号是一个按 \TeX\ 的展开规则一般要被展开的控制系列,
那么其含义与`|\relax|'一样。

%\smallbreak
%\textindent\bull ^|\topmark|, ^|\firstmark|, ^|\botmark|, ^|\splitfirstmark|,
%and ^|\splitbotmark|. \kern-1.7pt % This saves an overfull box (March 27, 1983)
%The expansion is the token list in the corresponding ``^{mark}'' register
%(see Chapter~23).
\smallbreak
\textindent\bull ^|\topmark|, ^|\firstmark|, ^|\botmark|, ^|\splitfirstmark|,
和 |\splitbotmark|: \
展开为相应``标记''寄存器中的记号列(见第二十三章)。

%\smallbreak
%\textindent\bull ^|\input|\<file name>. The expansion is null; but \TeX\
%prepares to read from the specified file before looking at any more
%tokens from its current source.
\smallbreak
\textindent\bull ^|\input|\<file name>: \1
展开为空;但是 \TeX\ 做好准备从给定文件的读入内容,
之后再在当前文件中继续读入其它记号。

%\smallbreak
%\textindent\bull ^|\endinput|. The expansion is null. The next time \TeX\ gets
%to the end of an |\input| line, it will stop reading from the file containing
%that line.
\smallbreak
\textindent\bull ^|\endinput|: \
展开为空。%
 \TeX\ 到达了 |\input| 行的结尾,将停止从包含此行的文件中读入。

%\smallbreak
%\textindent\bull ^|\the|\<internal quantity>. The expansion is a list of
%tokens representing the current value of one of \TeX's variables, as
%explained below. For example, `|\the\skip5|' might expand into
%`|5.0pt plus 2.0fil|' (17~tokens).
\smallbreak
\textindent\bull ^|\the|\<internal quantity>: \
展开为一列记号,它表示某个 \TeX\ 变量的当前值,讨论见下面。%
例如,`|\the\skip5|'将展开为`|5.0pt plus 2.0fil|'(17 个记号)。

%\ddanger The powerful |\the| operation has many subcases, so we shall discuss
%them one at a time. A variety of internal numeric quantities
%can be brought up front:\enddanger
\ddanger |\the| 这个有用的命令有很多子情形,因此我们要同时讨论它们。%
各种内部数字量都可以被提出来使用:
\enddanger

%\nobreak\medskip
%\textindent\bull |\the|\<parameter>, where \<parameter> is the name of
%one of \TeX's ^{integer parameters} (e.g., |\the\widowpenalty|),
%^{dimension parameters} (e.g., |\the\parindent|), ^{glue parameters}
%(e.g., |\the\leftskip|), or ^{muglue parameters} (e.g., |\the\thinmuskip|).
\nobreak\medskip
\textindent\bull |\the|\<parameter>,其中 \<parameter> 是 \TeX\ 的某一个
整数参数(比如说\break |\the\widowpenalty|)、尺寸参数(比如说 |\the\parindent|)、
粘连参数(比如说\break |\the\leftskip|)或 muglue 参数(比如说 |\the\thinmuskip|)的名字。

%\smallbreak
%\textindent\bull |\the|\<register>, where \<register> is the name of
%one of \TeX's integer ^{registers} (e.g., |\the\count|\stretch|0|),
%dimension registers (e.g., |\the\dimen169|), glue registers (e.g.,
%|\the\skip255|), or muglue registers (e.g., |\the\muskip\count2|).
\smallbreak
\textindent\bull |\the|\<register>, 其中 \<register> 是某个 \TeX\ 的整数寄存器%
(比如,|\the\count|\stretch|0|), 尺寸寄存器(比如,|\the\dimen169|),
粘连寄存器(比如,|\the\skip255|), 或 muglue 寄存器(比如,|\the\muskip\count2|)%
的名字。

%\smallbreak
%\textindent\bull |\the|\<codename>\<8-bit number>, where \<codename>
%stands for either ^|\catcode|, ^|\mathcode|, ^|\lccode|, ^|\uccode|,
%^|\sfcode|, or ^|\delcode|. For example, |\the\mathcode`/| produces the
%current (integer) math code value for a slash.
\smallbreak
\textindent\bull |\the|\<codename>\<8-bit number>,
其中 \<codename> 表示 ^|\lccode|、^|\uccode|、|\catcode|、^|\mathcode|、
^|\sfcode| 或 |\delcode|。
例如,|\the\mathcode`/| 得到的是斜线的当前数学码(整数)。

%\smallbreak
%\textindent\bull |\the|\<special register>, where \<special register> is
%one of the integer quantities ^|\prevgraf|, ^|\deadcycles|, ^|\insertpenalties|,
%^|\inputlineno|, ^|\badness|,
%or ^|\parshape| (denoting only the number of lines of\/ |\parshape|); or
%one of the dimensions ^|\pagetotal|, ^|\pagegoal|, ^|\pagestretch|,
%^|\pagefilstretch|, ^|\pagefillstretch|, ^|\pagefilllstretch|, ^|\pageshrink|,
%^|\pagedepth|. In horizontal modes you can also refer to a special integer,
%|\the\spacefactor|; in vertical modes there's a special dimension,
%|\the\prevdepth|.
\smallbreak
\textindent\bull |\the|\<special register>,其中 \<special register> 是
|\prevgraf|、^|\deadcycles|、^|\insertpenalties|、^|\inputlineno|、^|\badness|、
|\parshape|(它只表示 |\parshape| 的行数)中的某个整数量;
或者是 |\pagetotal|、^|\pagegoal|、^|\pagedepth|、^|\pagestretch|、
^|\pagefilstretch|、^|\pagefillstretch|、^|\pagefilllstretch|、^|\pageshrink|、
中的某个尺寸。
在水平模式下还可用于特殊整数 |\the\spacefactor|;
在垂直模式下还可用于特殊尺寸 |\the\prevdepth|。

%\smallbreak
%\textindent\bull |\the|^|\fontdimen|\<parameter number>\<font>.
%This produces a dimension; for example, parameter~6 of a font is its
%``^{em}'' value, so `|\the\fontdimen6\tenrm|' yields `|10.0pt|' (six tokens).
\smallbreak
\textindent\bull |\the|^|\fontdimen|\<parameter number>\<font>,
它得到的是一个尺寸;例如,字体的参数 6 为其``em''的值,
因此,`|\the\fontdimen6\tenrm|'得到的是`|10.0pt|'(6 个记号)。

%\smallbreak
%\textindent\bull |\the|^|\hyphenchar|\<font>, |\the|^|\skewchar|\<font>.
%These produce the corresponding integer values defined for the specified font.
\smallbreak
\textindent\bull |\the|^|\hyphenchar|\<font>, |\the|^|\skewchar|\<font>,
它们得到的是定义给定字体的相应整数值。

%\smallbreak
%\textindent\bull |\the|^|\lastpenalty|, |\the|^|\lastkern|, |\the|^|\lastskip|.
%These yield the amount of penalty, kerning, glue, or
%muglue in the final item on the current list, provided that the item is a
%penalty, kern, or glue, respectively; otherwise they yield `|0|' or `|0.0pt|'.
\smallbreak
\textindent\bull |\the|^|\lastpenalty|, |\the|^|\lastkern|, |\the|^|\lastskip|,
它们得到的是当前列中最后一个项目的惩罚,紧排,粘连或数学粘连的量,
如果此项目分别是惩罚,紧排,或粘连的话;
否则得到的是`|0|'或`|0.0pt|'。

%\smallbreak
%\textindent\bull |\the|\<defined character>, where \<defined character> is
%a control sequence that has been given an integer value with ^|\chardef| or
%^|\mathchardef|; the result is that integer value, in decimal notation.
\smallbreak
\textindent\bull |\the|\<defined character>, 其中 \<defined character> 是一个%
控制系列,它已经由 |\chardef| 或 |\mathchardef| 给定一个整数值;
结果就是此整数值,用小数表示。

%\ddanger In all of the cases listed so far, |\the| produces a result that
%is a sequence of ^{ASCII} character tokens. Category code~12 (``other'') is
%assigned to each token, except that character code~32 gets category~10
%(``space''). The same rule is used to assign ^{category codes} to the
%tokens produced by ^|\number|, ^|\romannumeral|, ^|\string|, ^|\meaning|,
%^|\jobname|, and ^|\fontname|.
\ddanger 在到现在为止的所有情况下,|\the| 得到的结果是一系列 ASCII 字符记号。%
除了字符代码为 32 的记号(``空格'')的类代码为 10 外,
每个记号的类代码都是 12~(``其它字符'')。%
同样的类代码规则也适用于由 |\number|, ^|\romannumeral|, ^|\string|, ^|\meaning|,
^|\jobname| 和 |\fontname| 得到的记号。

%\ddanger There also are cases in which |\the| produces non-character tokens,
%either a font identifier like |\tenrm|, or an arbitrary token list:
\ddanger 在一些情形下,|\the| 得到的是非字符的记号,是象 |\tenrm| 这样的字体标识符,
或者是任意记号列:

%\textindent\bull |\the|\<font> produces a font identifier that selects the
%specified font. For example, `|\the\font|' is a control sequence corresponding
%to the current font.
\textindent\bull |\the|\<font> 得到的是选择给定字体的字体标识符。%
例如,`|\the\font|'是对应于当前字体的控制系列。

%\goodbreak\smallskip
%\textindent\bull |\the|\<token variable> produces a copy of the token list that
%is the current value of the variable. For example, you can expand
%`|\the\everypar|' and `|\the\toks5|'.
\goodbreak\smallskip
\textindent\bull |\the|\<token variable> \1得到的是一个变量当前值的记号列。%
例如,可以展开`|\the\everypar|'和 `|\the\toks5|'。

%\ddanger \TeX's primitive command `^|\showthe|' will display on your
%terminal exactly what `|\the|' would produce in an expanded definition;
%the expansion is preceded by `|> |' and followed by a period. For example,
%`|\showthe\parindent|' will display
%\begintt
%> 20.0pt.
%\endtt
%if the plain \TeX\ paragraph indentation is being used.
\ddanger  \TeX\ 的原始命令`^|\showthe|'将把在展开定义中`|\the|'所得到的东西%
显示在终端上;
展开前面加上 `|> |', 后面跟上句点。%
例如,如果采用 plain \TeX\ 的段落缩进,那么`|\showthe\parindent|'显示的是
\begintt
> 20.0pt.
\endtt

%\ddanger Here now is the promised list of all cases when expandable tokens
%are not expanded. Some of the situations involve primitives that
%haven't been discussed yet, but we'll get to them
%eventually.  Expansion is suppressed at the following times:\enddanger
\ddanger 下面是以前说过的可展开的记号不被展开的所有情形的列表。%
某些情形中含有未讨论过的原始命令,但是我们最后会给出它们的。%
在下列情形下展开被禁止:
\enddanger

%\nobreak\medskip
%\item\bull When tokens are being deleted during ^{error recovery} (see
%Chapter~6).
\nobreak\medskip
\item\bull 当记号在错误修复期间被删除时(见第六章)。

%\smallskip
%\item\bull When tokens are being skipped because conditional text is being
%ignored.
\smallskip
\item\bull 当因为条件文本被忽略而记号被跳过时。

%\smallskip
%\item\bull When \TeX\ is reading the arguments of a macro.
\smallskip
\item\bull 当 \TeX\ 正在读入宏的参量时。

%\smallskip
%\item\bull When \TeX\ is reading a control sequence to be defined by
%^|\let|, ^|\futurelet|, ^|\def|, ^|\gdef|, ^|\edef|, ^|\xdef|, ^|\chardef|,
%^|\mathchardef|, ^|\countdef|, ^|\dimendef|, ^|\skipdef|, ^|\muskipdef|,
%^|\toksdef|, ^|\read|, and ^|\font|.
\smallskip
\item\bull 当 \TeX\ 正在读入由下列定义的控制系列时:
^|\let|, ^|\futurelet|, ^|\def|, ^|\gdef|, ^|\edef|, ^|\xdef|, ^|\chardef|,
^|\mathchardef|, ^|\countdef|, ^|\dimendef|, ^|\skipdef|, ^|\muskipdef|,
^|\toksdef|, ^|\read| 和 |\font|。

%\smallskip
%\item\bull When \TeX\ is reading argument tokens for ^|\expandafter|,
%^|\noexpand|, ^|\string|, ^|\meaning|, ^|\let|, ^|\futurelet|, ^|\ifx|,
%^|\show|, ^|\afterassignment|, ^|\aftergroup|.
\smallskip
\item\bull 当 \TeX\ 正在读入下列控制系列的参量记号时:
|\expandafter|、^|\noexpand|、^|\string|、^|\meaning|、^|\let|、
^|\futurelet|、^|\ifx|、^|\show|、^|\afterassignment|、^|\aftergroup|。

%\smallskip
%\item\bull When \TeX\ is absorbing the parameter text of a |\def|, |\gdef|,
%|\edef|, or |\xdef|.
\smallskip
\item\bull 当 \TeX\ 正在读入的是 |\def|, |\gdef|,
|\edef| 或 |\xdef| 的参数文本时。

%\smallskip
%\item\bull When \TeX\ is absorbing the replacement text of a |\def| or
%|\gdef| or ^|\read|; or the text of a ^{token variable} like ^|\everypar|
%or ^|\toks||0|; or the token list for ^|\uppercase| or ^|\lowercase| or
%^|\write|.  \ (The token list for |\write| will be expanded later, when it
%is actually output to a file.)
\smallskip
\item\bull 当 \TeX\ 正在读入的是 |\def| 或者 |\gdef| 或者 |\read| 的替换文本;
或者是像 |\everypar| 或 |\toks||0| 这样的记号变量的文本;
或者是 |\uppercase| 或 |\lowercase| 或 ^|\write| 的记号列。%
(|\write| 的记号列在稍后实际输出到文件时才被展开。)

%\smallskip
%\item\bull When \TeX\ is reading the preamble of an alignment, except after
%a token for the primitive command |\span| or when reading the \<glue>
%after ^|\tabskip|.
\smallskip
\item\bull 当 \TeX\ 正在读入对齐阵列的导言时,但是除了在原始命令 |\span| 的一个记号后,
或者正在读入 |\tabskip| 的 \<glue> 时。

%\smallskip
%\item\bull Just after a |$|$_3$ token that begins math mode, to see if
%another~|$|$_3$ follows.
\smallskip
\item\bull 在数学模式开始的记号 |$|$_3$ 紧后面时,这是为了看看是否后面%
还跟着一个 |$|$_3$。

%\smallskip
%\item\bull Just after a |`|$_{12}$ token that begins an ^{alphabetic constant}.
\smallskip
\item\bull 在开始字母常数的记号 |`|$_{12}$ 的后面。

%\ddanger Sometimes you will find yourself wanting to define new macros
%whose replacement text has been expanded, based on current conditions,
%instead of simply copying the replacement text verbatim. \TeX\ provides
%the ^|\edef| (expanded definition) command for this purpose, and also
%^|\xdef| (which is equivalent to |\global\edef|). The general format
%is the same as for |\def| and |\gdef|, but \TeX\ blindly expands the tokens
%of the replacement text according to the expansion rules above. For
%example, consider
%\begintt
%\def\double#1{#1#1}
%\edef\a{\double{xy}} \edef\a{\double\a}
%\endtt
%Here the first |\edef| is equivalent to `|\def\a{xyxy}|' and the second is
%equivalent to `|\def\a{xyxyxyxy}|'. All of the other kinds of expansion
%will take place too, including conditionals; for example,
%\begintt
%\edef\b#1#2{\ifmmode#1\else#2\fi}
%\endtt
%gives a result equivalent to `|\def\b#1#2{#1}|' if \TeX\ is in math
%mode at the time of the |\edef|, otherwise the result is equivalent to
%`|\def\b#1#2{#2}|'
\ddanger 有时候你会发现自己要定义一个新宏,它的替换文本由于当前情形而已经被展开了,
而不是简单地逐字复制替换文本。%
为此, \TeX\ 提供了命令 |\edef|~(被展开的定义), 以及 |\xdef|~(它等价于%
~|\global\edef|)。%
其一般格式与 |\def| 和 |\gdef| 一样,但是 \TeX\ 盲目地按照上面的展开规则来%
展开替换文本的记号。%
例如,看看下面这个定义:
\begintt
\def\double#1{#1#1}
\edef\a{\double{xy}} \edef\a{\double\a}
\endtt
这里,第一个 |\edef| 等价于`|\def\a{xyxy}|', 而第二个等价于`|\def\a{xyxyxyxy}|'。%
所有其它类型的展开也要进行,包括条件文本;
例如,在 \TeX\ 给出 |\edef| 时处在数学模式的情况下
\begintt
\edef\b#1#2{\ifmmode#1\else#2\fi}
\endtt
的结果等价于`|\def\b#1#2{#1}|',
否则结果等价于`|\def\b#1#2{#2}|'。

%\ddanger Expanded definitions that are made with |\edef| or |\xdef| continue
%to expand tokens until only unexpandable tokens remain, except that
%token lists produced by `^|\the|' are not expanded further. Furthermore
%a token following `^|\noexpand|' will not be expanded, since its ability
%to expand has been nullified. These two operations can be used to control
%^^{expansion, avoiding} what gets expanded and what doesn't.
\ddanger \1由 |\edef| 和 |\xdef| 给出的被展开的定义要把记号展开到只剩下%
不能展开的记号,但是由`^|\the|'生成的记号列不再进一步展开。%
还有,`^|\noexpand|'后面的记号不展开,因为它的展开能力无效了。%
这两个命令可以用来控制展开哪些,不展开哪些。

%\ddanger Suppose, for example, that you
%want to define |\a| to be equal to~|\b| (expanded) followed by~|\c|
%(not expanded) followed by |\d| (expanded), assuming that |\b| and
%|\d| are simple macros without parameters. There are two easy ways to do it:
%\begintt
%\edef\a{\b\noexpand\c\d}
%\toks0={\c} \edef\a{\b\the\toks0 \d}
%\endtt
%And it's even possible to achieve the same effect without using either
%|\noexpand| or |\the|; a reader who wants to learn more about \TeX's
%expansion mechanism is encouraged to try the next three exercises.
\ddanger 例如,假设你要把 |\a| 定义为 |\b|~(展开的)后面跟 |\c|~(不展开)
再后面跟 |\d|~(展开的), 并且假定 |\b| 和 |\d| 是无参数的简单宏。%
就可以用两种方法来实现:
\begintt
\edef\a{\b\noexpand\c\d}
\toks0={\c} \edef\a{\b\the\toks0 \d}
\endtt
甚至可以不用 |\noexpand| 或 |\the| 也可得到同样的效果;
对想多学习一些 \TeX\ 展开原理的读者,建议做一下下面三个练习。

%\ddangerexercise Figure out a way to define |\a| as in the previous paragraph,
%without using \TeX's primitives `|\noexpand|' and `|\the|'.
%\answer One idea is to say
%\begintt
%\let\save=\c \let\c=0 \edef\a{\b\c\d} \let\c=\save
%\endtt
%because control sequences equivalent to characters are not expandable.
%However, this doesn't expand occurrences of~|\c|
%that might be present in the expansions of\/ |\b| and~|\d|. Another way,
%which is free of this defect, is
%\begintt
%\edef\next#1#2{\def#1{\b#2\d}} \next\a\c
%\endtt
%(and it's worth a close look!).
\ddangerexercise 不用 `|\noexpand|' 和 `|\the|',找出定义前一段中 |\a| 的方法。
\answer 其中一种想法如下:
\begintt
\let\save=\c \let\c=0 \edef\a{\b\c\d} \let\c=\save
\endtt
这是因为等价于字符的控制系列不会被展开。
然而,这导致可能出现在 |\b| 和 |\d| 的展开中的 |\c| 也不会被展开。
另一种可以避免这个缺点的想法是
\begintt
\edef\next#1#2{\def#1{\b#2\d}} \next\a\c
\endtt
(它值得你仔细观察!)。

%\ddangerexercise Continuing the example of expansion avoidance, suppose that
%you want to expand |\b| completely until only unexpandable tokens are left,
%but you don't want to expand |\c| at all, and you want to expand |\d|
%only one level. For example, after |\def\b{\c\c}| and |\def\c{*}| and
%|\def\d{\b\c}| the goal would be to get the effect of\/ |\def\a{**\c\b\c}|.
%How can such a partial expansion be achieved, using ^|\the|?
%\answer |\toks0={\c} \toks2=\expandafter{\d}|\parbreak
%        |\edef\a{\b\the\toks0 \the\toks2 }|
%\smallskip\noindent
%(Notice that ^|\expandafter| expands the token after the left brace here.)
\ddangerexercise 继续讨论避免展开的例子,假设要把 |\b|
完全展开到只剩下不能展开的记号,但是你不希望展开 |\c|,并且对 |\d| 只展开一层。
例如,在定义 |\def\b{\c\c}|,|\def\c{*}| 和 |\def\d{\b\c}| 后,
要得到的是 |\def\a{**\c\b\c}| 这样的结果。怎样才能用 |\the| 来实现这个部分展开?
\answer |\toks0={\c} \toks2=\expandafter{\d}|\parbreak
        |\edef\a{\b\the\toks0 \the\toks2 }|
\smallskip\noindent
(注意这里的 ^|\expandafter| 展开左花括号后面的记号。)

%\ddangerexercise Solve the previous exercise without |\the| or |\noexpand|.
%\ (This is difficult.)
%\answer The following shouldn't be taken too seriously, but it does work:
%^^|\span|
%\begintt
%{\setbox0=\vbox{\halign{#{\c\span\d}\cr
%      \let\next=0\edef\next#1{\gdef\next{\b#1}}\next\cr}}}
%\let\a=\next
%\endtt
\ddangerexercise 不用 |\the| 或 |\noexpand| 解决上一个练习。(这很难。)
\answer 别太严肃看待下面的解答,但它确实有效:
^^|\span|
\begintt
{\setbox0=\vbox{\halign{#{\c\span\d}\cr
      \let\next=0\edef\next#1{\gdef\next{\b#1}}\next\cr}}}
\let\a=\next
\endtt

%\ddanger \TeX's primitive commands ^|\mark||{...}|, ^|\message||{...}|,
%^|\errmessage||{...}|, ^|\special||{...}|, and ^|\write|\<number>|{...}|
%all expand the token lists in braces almost exactly as |\edef| and |\xdef| do.
%However, a macro parameter
%character like~|#| should not be duplicated in such commands; you need
%to say |##| within an |\edef|, but only |#| within a |\mark|. The
%|\write| command is somewhat special, because its token list is first
%read without expansion; expansion occurs later, when the tokens
%are actually being written to a file.
\ddanger 几乎同 |\edef| 和 |\xdef| 一样,\TeX\ 的原始命令 |\mark||{...}|、
^|\message||{...}|、^|\errmessage||{...}|、^|\special||{...}| 和
|\write|\<number>|{...}| 都展开花括号中的记号列。
但是像 |#| 这样的宏参数字符在这样的命令中不用重复;在 |\edef| 中用 |##|,
而在 |\mark| 中只用 |#|。命令 |\write| 有点特殊,因为它的记号列首先读入而不展开;
直到记号实际上被写入一个文件时才进行展开。

%\ddangerexercise Compare the following two definitions:
%\begintt
%\def\a{\iftrue{\else}\fi}
%\edef\b{\iftrue{\else}\fi}
%\endtt
%Which of them yields an ^{unmatched left brace}? (This is tricky.)
%\answer Neither one, although |\a| will behave like an unmatched left
%brace when it is expanded. The definition of\/ |\b| is {\sl not complete},
%because it expands to `|\def\b{{}|'; \TeX\ will continue to read ahead,
%looking for another right brace, possibly discovering a runaway
%definition! It's impossible to define a macro that has unmatched braces.
%But you {\sl can\/} say |\let\a={|; Appendix~D discusses several
%other ^{brace tricks}.
\ddangerexercise 比较下面两个定义:
\begintt
\def\a{\iftrue{\else}\fi}
\edef\b{\iftrue{\else}\fi}
\endtt
哪个得到未匹配的左花括号?(有点技巧。)
\answer 两者都不能得到左花括号,尽管 |\a| 展开后像一个未匹配的左花括号。
|\b| 的定义是{\sl 不完整的},因为它展开为 `|\def\b{{}|';
\TeX\ 将继续往前读取,寻找一个右花括号,也许发现定义失控了!
你不可能定义一个包含未匹配花括号的宏。
但是你 {\sl 可以\/}用 |\let\a={|;附录 D 讨论了几个其他的^{花括号技巧}。

%\def\rhead{Chapter \chapno: Definitions (aka Macros)% my little joke
%  \gdef\rhead{Chapter \chapno: Definitions (also called Macros)}}

%\ddanger \TeX\ has the ability to read individual lines of text from up
%to~16 files at once, in addition to the files that are being |\input|.
%To initiate reading such an auxiliary file, you should say
%\begindisplay
%^|\openin|\<number>|=|\<file name>
%\enddisplay
%where the \<number> is between 0 and 15. \ (Plain \TeX\ allocates
%input stream numbers 0~through~15 with the ^|\newread| command, which
%is analogous to |\newbox|.) \ In most installations
%of \TeX, the extension `^|.tex|' will be appended to the file name,
%as with ^|\input|, if no extension is given explicitly. If the file
%cannot be found, \TeX\ will give no error message; it will simply consider
%that the input stream is not open, and you can test this condition
%with ^|\ifeof|. When you're done with a file, you can say
%\begindisplay
%^|\closein|\<number>
%\enddisplay
%and the file associated with that input stream number will be closed,
%i.e., returned to its initial condition, if such a file was open.
%To get input from an open file, you say
%\begindisplay
%^|\read|\<number>^|to|\<control sequence>
%\enddisplay
%and the control sequence is defined to be a parameterless macro whose
%replacement text is the contents of the next line read from the
%designated file. This line is converted to a token list, using the
%procedure of Chapter~8, based on the current category codes.
%Additional lines are read, if necessary, until an equal number of
%left and right braces has been found. An empty line is implicitly appended
%to the end of a file that is being |\read|. ^^{empty line at end of file}
%If the \<number> is not between 0 and~15, or if no such file is open, or if
%the file has ended, input will be from the terminal; \TeX\ will prompt
%the user unless the \<number> is negative. The macro definition will be
%local unless you say |\global\read|.
\ddanger  \TeX\ 可以同时从大约 16 个文件中读入各个文本行,除了在 |\input| 后的%
文件之外。%
为了开始读入这样一个辅助文件,应该使用
\begindisplay
^|\openin|\<number>=\<file name>
\enddisplay
其中 \<number> 在 0 和 15 之间。%
(Plain \TeX\ 用命令 |\newread| 了分配输入流的数字 0 到 15, 它类似于 |\newbox|。)
在大多数 \TeX\ 的安装时,\1如果没有明确给出扩展名,
那么扩展名`^|.tex|'将添加到文件名之后,就象用 |\input| 一样。%
如果文件找不到, \TeX\ 不给出错误信息;
它只把输入流看作没有打开,并且你可以用 |\ifeof| 来测试这种状态。%
当不再使用某个文件时,
可以使用
\begindisplay
^|\closein|\<number>
\enddisplay
并且如果与输入流数字相对应的文件是打开的,那么它将关闭,
即,返回其初始状态。%
为了从一个打开的文件得到输入,可以使用
\begindisplay
^|\read|\<number>^|to|\<control sequence>
\enddisplay
并且所定义的控制系列是一个无参数的宏,其替换文本是从指定文件读入的下一行的内容。%
此行用第八章的程序按当前类代码转换为一个记号列。%
如果需要,再读入其它行,直到左右大括号的数目相同。%
 \TeX\ 在要读入的文件结尾暗中添加了一个空行。%
如果 \<number> 不在 0 和 15 之间,或者如果这样的文件没有打开,
或者文件结束了,那么输入就来自终端;
如果 \<number> 不是负值,那么 \TeX\ 将给出提示符。%
如果你不使用 |\global\read|, 那么宏的定义将是局部的。

%\ddanger For example, it's easy to have ^{dialogs with the user}, by
%using |\read| together with the ^|\message| command (which
%writes an expanded token list on the terminal and in the log file):
%\begintt
%\message{Please type your name:}
%\read16 to\myname
%\message{Hello, \myname!}
%\endtt
%The |\read| command in this case will print `|\myname=|' and it will wait
%for a response; the response will be echoed on the log file.
%The `|\myname=|' would have been omitted if `|\read16|' had been
%`|\read-1|'.
\ddanger 例如,通过命令 |\read| 以及%
~|\message|~(它把一个展开的记号列写在终端上和 log 文件中),
可以很容易实现与用户对话:
\begintt
\message{Please type your name:}
\read16 to\myname
\message{Hello, \myname!}
\endtt
在这种情况下,命令 |\read| 将写入`|\myname=|'并等待应答;
应答在 log 文件中重复出来。%
如果`|\read16|'变成`|\read-1|', 那么`|\myname=|'就被省略了。

%\ddangerexercise The |\myname| example just given doesn't work quite right,
%because the \<return> at the end of the line gets translated into a
%space. Figure out how to fix that glitch.
%\answer One way is to redefine |\catcode`\^^M=9| (ignored) just before
%the |\read|, so that the \<return> will be ignored. Another solution is
%to redefine ^|\endlinechar||=-1|, so that no character is put at the
%end of the line. Or you could try to be tricky by stripping off the
%space with macro expansion as follows:
%\begintt
%\def\stripspace#1 \next{#1}
%\edef\myname{\expandafter\stripspace\myname\next}
%\endtt
%The latter solution doesn't work if the user types `|%|' at the end of
%his~or her name, or if the name contains control sequences.
\ddangerexercise 刚刚给出的 |\myname| 例子效果并不好,
因为在行尾的 \<return> 被转换为一个空格。看看怎样解决这个小问题?
\answer 其中一种方法就是在 |\read| 之前重新定义 |\catcode`\^^M=9|(可忽略符),
使得 \<return> 被忽略。另一种方法是重新定义 ^|\endlinechar||=-1|,
使得在行尾不加任何字符。或者你也可以试试更有技巧的方法,用下面的宏去掉空格:
\begintt
\def\stripspace#1 \next{#1}
\edef\myname{\expandafter\stripspace\myname\next}
\endtt
如果用户在名字后键入 `|%|',或者用户名字包含控制系列,后面这个方法将无效。

%\ddangerexercise Continuing the previous example, define a macro
%|\MYNAME| that contains the letters of\/ |\myname| all in ^{uppercase
%letters}. For example, if\/ |\myname| expands to |Arthur|, |\MYNAME|
%should expand to |ARTHUR|. Assume that |\myname| contains only
%letters and spaces in its expansion.
%\answer Here are two solutions:
%\begintt
%\def\next#1\endname{\uppercase{\def\MYNAME{#1}}}
%\expandafter\next\myname\endname
%|smallskip\edef\next{\def\noexpand\MYNAME{\myname}}
%\uppercase\expandafter{\next}
%\endtt
\ddangerexercise 继续前一个例子,定义一个叫 |\MYNAME| 的宏,
它包含 |\myname| 的所有字母,但改为^{大写字母}。
例如,如果 |\myname| 展开为 |Arthur|,那么 |\MYNAME| 就展开为 |ARTHUR|。
假定在 |\myname| 的展开中只包含字母和空格。
\answer 这里给出两种解法:
\begintt
\def\next#1\endname{\uppercase{\def\MYNAME{#1}}}
\expandafter\next\myname\endname
|smallskip\edef\next{\def\noexpand\MYNAME{\myname}}
\uppercase\expandafter{\next}
\endtt

%\ddanger Appendices B, D, and E contain numerous examples of how to make
%macros do useful things. Let's close this chapter by presenting a few
%examples that show how \TeX\ can actually be used as a primitive ^{programming}
%language, if you want to achieve special effects, and if you don't care very
%much about computer costs.
\ddanger 附录 B, D, E 包含大量编写的宏的例子,可以做很多事情。%
现在,在本章结尾,我们通过几个例子来说明作为编程语言,
 \TeX\ 实际上可以怎样使用,如果你要得到某些特殊的效果,并且不在意计算机所耗的时间。

%\ddanger Plain \TeX\ contains a |\loop...\repeat| construction, which works
%like this: You say `^|\loop|~$\alpha$~|\if...|~$\beta$~|\repeat|', where
%$\alpha$ and~$\beta$ are any sequences of commands, and where |\if...|\
%is any conditional test (without a matching |\fi|). \TeX\ will first
%do~$\alpha$; then if the condition is true, \TeX\ will do~$\beta$ and
%repeat the whole process again starting with~$\alpha$. If the
%condition ever turns out to be false, the loop will stop. For example,
%here is a program that carries out a little dialog in which \TeX\ waits for
%the user to type `|Yes|' or `|No|': ^^{repeating commands, see :loop}
%\begintt
%\def\yes{Yes } \def\no{No } \newif\ifgarbage
%\loop\message{Are you happy? }
%  \read-1 to\answer
%  \ifx\answer\yes\garbagefalse % the answer is Yes
%    \else\ifx\answer\no\garbagefalse % the answer is No
%      \else\garbagetrue\fi\fi % the answer is garbage
%  \ifgarbage\message{(Please type Yes or No.)}
%\repeat
%\endtt
\ddanger Plain \TeX\ 中含有一个 |\loop...\repeat| 结构,它像这样工作:
假设你给出 `^|\loop|~$\alpha$~|\if...|~$\beta$~|\repeat|',
其中 $\alpha$ 和 $\beta$ 是任意系列的命令,并且 |\if...| 是任意条件测试(无匹配的 |\fi|)。
\TeX\ 就首先执行 $\alpha$;接着如果条件为真,\TeX\ 就执行 $\beta$,
并且再次从 $\alpha$ 开始重复整个过程。如果条件为假,循环就停止。
\1例如,下面有一个小程序,进行一段对话,其中 \TeX\ 等待用户输入 `|Yes|' 或 `|No|':
\begintt
\def\yes{Yes } \def\no{No } \newif\ifgarbage
\loop\message{Are you happy? }
  \read-1 to\answer
  \ifx\answer\yes\garbagefalse % the answer is Yes
    \else\ifx\answer\no\garbagefalse % the answer is No
      \else\garbagetrue\fi\fi % the answer is garbage
  \ifgarbage\message{(Please type Yes or No.)}
\repeat
\endtt

%\ddangerexercise Use the |\loop...\repeat| mechanism to construct a
%general |\punishment| macro that repeats any given paragraph any given number of
%times. For example,
%\begintt
%\punishment{I must not talk in class.}{100}
%\endtt
%should produce the results desired in exercise 20.\punishexno.
%\answer (Here's a solution that also numbers the lines, so that the number of
%repetitions is easily verifiable.
%The only tricky part about this answer is the use of\/ ^|\endgraf|,
%which is a substitute for |\par| because |\loop| is not a ^|\long| macro.)
%\begintt
%\newcount\n
%\def\punishment#1#2{\n=0
%  \loop\ifnum\n<#2 \advance\n by1
%    \item{\number\n.}#1\endgraf\repeat}
%\endtt
\ddangerexercise 用 |\loop...\repeat| 原理编写一个一般的 |\punishment| 宏,
它将任意给定段落重复任意给定次数。例如,
\begintt
\punishment{I must not talk in class.}{100}
\endtt
将得到练习 20.\punishexno 所要的结果。
\answer (这个解答还给各行编号,以让重复次数更容易验证。
唯一有技巧的是使用了 ^|\endgraf|;
因为 |\loop| 不是 ^|\long| 宏,我们用 ^|\endgraf| 代替 |\par|。)
\begintt
\newcount\n
\def\punishment#1#2{\n=0
  \loop\ifnum\n<#2 \advance\n by1
    \item{\number\n.}#1\endgraf\repeat}
\endtt

%\let\plaind=\d
%\newif\ifprime \newif\ifunknown
%\newcount\n \newcount\p \newcount\d \newcount\a
%\def\primes#1{2,~3% assume that #1 is at least 3
%  \n=#1 \advance\n by-2 % n more to go
%  \p=5 % odd primes starting with p
%  \loop\ifnum\n>0 \printifprime\advance\p by2 \repeat}
%\def\printp{, % we will invoke \printp if p is prime
%  \ifnum\n=1 and~\fi % this precedes the last value
%  \number\p \advance\n by -1 }
%\def\printifprime{\testprimality \ifprime\printp\fi}
%\def\testprimality{{\d=3 \global\primetrue
%  \loop\trialdivision \ifunknown\advance\d by2 \repeat}}
%\def\trialdivision{\a=\p \divide\a by\d
%  \ifnum\a>\d \unknowntrue\else\unknownfalse\fi
%  \multiply\a by\d
%  \ifnum\a=\p \global\primefalse\unknownfalse\fi}
%\ddanger The first thirty prime numbers are \primes{30}. You may not
%find this fact very startling; but you may be surprised to learn that
%the previous sentence was typeset by saying
%\begintt
%The first thirty prime numbers are \primes{30}.
%\endtt
%\TeX\ did all of the calculation by expanding the |\primes| macro, so the author
%is pretty sure that the list of ^{prime numbers} given above is quite free
%of typographic errors. Here is the set of macros that did it:
%\begindisplay
%^|\newif||\ifprime \newif\ifunknown % boolean variables|\cr
%^|\newcount||\n \newcount\p \newcount\d \newcount\a % integer variables|\cr
%|\def\primes#1{2,~3% assume that #1 is at least 3|\cr
%|  \n=#1 \advance\n by-2 % n more to go|\cr
%|  \p=5 % odd primes starting with p|\cr
%|  \loop\ifnum\n>0 \printifprime\advance\p by2 \repeat}|\cr
%|\def\printp{, % we will invoke \printp if p is prime|\cr
%|  \ifnum\n=1 and~\fi % `and' precedes the last value|\cr
%|  \number\p \advance\n by -1 }|\cr
%|\def\printifprime{\testprimality \ifprime\printp\fi}|\cr
%|\def\testprimality{{\d=3 \global\primetrue|\cr
%|  \loop\trialdivision \ifunknown\advance\d by2 \repeat}}|\cr
%\noalign{\penalty-500}
%|\def\trialdivision{\a=\p \divide\a by\d|\cr
%|  \ifnum\a>\d \unknowntrue\else\unknownfalse\fi|\cr
%|  \multiply\a by\d|\cr
%|  \ifnum\a=\p \global\primefalse\unknownfalse\fi}|\cr
%\enddisplay
%^^|\multiply|^^|\divide|^^|\advance|^^|\newcount|
%The computation is fairly straightforward, except that it involves
%a loop inside a loop; therefore |\testprimality| introduces an extra
%set of braces, to keep the inner loop control from interfering with
%the outer loop. The braces make it necessary to say `^|\global|' when
%|\ifprime| is being set true or false.  \TeX\ spent more time constructing
%that sentence than it usually spends on an entire page; the
%|\trialdivision| macro was expanded 132 times. % cpu time was 4 sec
\let\plaind=\d
\newif\ifprime \newif\ifunknown
\newcount\n \newcount\p \newcount\d \newcount\a
\def\primes#1{2,~3% assume that #1 is at least 3
  \n=#1 \advance\n by-2 % n more to go
  \p=5 % odd primes starting with p
  \loop\ifnum\n>0 \printifprime\advance\p by2 \repeat}
\def\printp{, % we will invoke \printp if p is prime
  \ifnum\n=1 and~\fi % this precedes the last value
  \number\p \advance\n by -1 }
\def\printifprime{\testprimality \ifprime\printp\fi}
\def\testprimality{{\d=3 \global\primetrue
  \loop\trialdivision \ifunknown\advance\d by2 \repeat}}
\def\trialdivision{\a=\p \divide\a by\d
  \ifnum\a>\d \unknowntrue\else\unknownfalse\fi
  \multiply\a by\d
  \ifnum\a=\p \global\primefalse\unknownfalse\fi}
\ddanger 前 30 个素数为 \primes{30}。%
你可能觉得这没什么;但是你可能吃惊的是上面这句话由下面所排版出:
\begindisplay
前|~|30|~|个素数为|~\primes{30}|。
\enddisplay
 \TeX\ 通过展开宏 |\primes| 执行了所有的计算,因此作者可以确信上面所给%
素数列没什么输入错误。%
下面就是此宏集:
\begindisplay
^|\newif||\ifprime \newif\ifunknown % boolean variables|\cr
^|\newcount||\n \newcount\p \newcount\d \newcount\a % integer variables|\cr
|\def\primes#1{2,~3% assume that #1 is at least 3|\cr
|  \n=#1 \advance\n by-2 % n more to go|\cr
|  \p=5 % odd primes starting with p|\cr
|  \loop\ifnum\n>0 \printifprime\advance\p by2 \repeat}|\cr
|\def\printp{, % we will invoke \printp if p is prime|\cr
|  \ifnum\n=1 and~\fi % `and' precedes the last value|\cr
|  \number\p \advance\n by -1 }|\cr
|\def\printifprime{\testprimality \ifprime\printp\fi}|\cr
|\def\testprimality{{\d=3 \global\primetrue|\cr
|  \loop\trialdivision \ifunknown\advance\d by2 \repeat}}|\cr
\noalign{\penalty-500}
|\def\trialdivision{\a=\p \divide\a by\d|\cr
|  \ifnum\a>\d \unknowntrue\else\unknownfalse\fi|\cr
|  \multiply\a by\d|\cr
|  \ifnum\a=\p \global\primefalse\unknownfalse\fi}|\cr
\enddisplay
这种计算相当直接,除了它有一个循环在另一个循环中;
因此,|\testprimality| 引入了一个额外的大括号组,来保证内部循环控制不受%
外部循环的影响。%
当 |\ifprime| 被设定为真或假时,大括号要求编写使用`^|\global|'。%
 \TeX\ 处理此句所花费的时间比处理一个页面还要多;
\1宏 |\trialdivision| 被展开了 132 次。

%\let\d=\plaind
%\ddanger The |\loop| macro that does all these wonderful things is
%actually quite simple. It puts the code that's supposed to be repeated
%into a control sequence called |\body|, and then another control
%sequence iterates until the condition is false:
%\begintt
%\def\loop#1\repeat{\def\body{#1}\iterate}
%\def\iterate{\body\let\next=\iterate\else\let\next=\relax\fi\next}
%\endtt
%The expansion of\/ |\iterate| ends with the expansion of\/ |\next|; therefore
%\TeX\ is able to remove |\iterate| from its memory before
%invoking |\next|, and the memory does not fill up during a long loop.
%Computer scientists call this ``^{tail recursion}.''
\let\d=\plaind
\ddanger 做这些漂亮工作的宏 |\loop| 实际上非常简单。%
它把假定要重复的代码放在叫做 |\body| 的控制系列中,
并且接着用另一个控制系列重复到条件为假:
\begintt
\def\loop#1\repeat{\def\body{#1}\iterate}
\def\iterate{\body\let\next=\iterate\else\let\next=\relax\fi\next}
\endtt
|\iterate| 的展开的结尾是 |\next| 的展开;
因此 \TeX\ 在调用 |\next| 之前能把 |\iterate| 从内存中清除掉,
这样在长循环中内存就不会被塞满。%
计算机科学家称其为``末端回归''。

%\ddanger The |\hex| macro below, which converts count register |\n| to
%^{hexadecimal notation}, illustrates a {\sl recursive\/} control structure
%in which many copies of\/ |\hex| can be active simultaneously.
%^{Recursion} works better than simple |\loop| ^{iteration} in this
%application because the hexadecimal digits are discovered from right to
%left, while they must be output from left to right.
%\ (The number in |\n| should be $\ge0$.)
%\begintt
%\def\hex{{\count0=\n \divide\n by16
%  \ifnum\n>0 \hex\fi \count2=\n \multiply\count2 by-16
%  \advance\count0 by\count2 \hexdigit}}
%\def\hexdigit{\ifnum\count0<10 \number\count0
%  \else\advance\count0 by-10 \advance\count0 by`A \char\count0 \fi}
%\endtt
\ddanger 下面的宏 |\hex| 把计数寄存器 |\n| 转换为十六进制表示,
它举例说明了一个{\KT{9}递归}控制结构,许多 |\hex| 的副本都可以在其中同时使用。%
在这个应用程序中,递归比简单的 |\loop| 重复更好,
因为十六进制数字是从右到左算出来的,
而它们在输出时是从左到右。
(在 |\n| 中的数字必须 $\ge0$。)
\begintt
\def\hex{{\count0=\n \divide\n by16
  \ifnum\n>0 \hex\fi \count2=\n \multiply\count2 by-16
  \advance\count0 by\count2 \hexdigit}}
\def\hexdigit{\ifnum\count0<10 \number\count0
  \else\advance\count0 by-10 \advance\count0 by`A \char\count0 \fi}
\endtt

%\ddanger Our final example is a macro that computes the number of nonblank
%tokens in its argument; for example, `|\length{argument}|' expands to
%`|8|'. This illustrates yet another aspect of macro technique.
%\begintt
%\def\length#1{{\count0=0 \getlength#1\end \number\count0}}
%\def\getlength#1{\ifx#1\end \let\next=\relax
%  \else\advance\count0 by1 \let\next=\getlength\fi \next}
%\endtt
\ddanger 最后这个例子给出一个宏,它用于计算宏参量中非空记号的数目;
比如 `|\length{argument}|' 展开为 `|8|'。
这个例子展示了宏的另一个方面的技术。
\begintt
\def\length#1{{\count0=0 \getlength#1\end \number\count0}}
\def\getlength#1{\ifx#1\end \let\next=\relax
  \else\advance\count0 by1 \let\next=\getlength\fi \next}
\endtt

\endchapter

By this time [37 A.D.] the influence of ^{Macro} had become supreme.
\author ^{TACITUS}, {\sl Annals\/} (c.\thinspace120 A.D.) % book VI, ch 45

\bigskip

% Oh, you want a definition.
I hate definitions.
\author BENJAMIN ^{DISRAELI}, {\sl Vivian Grey\/} (1826) % Book II, Chapter 6

\vfill\eject\byebye
